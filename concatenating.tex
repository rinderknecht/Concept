\section{Concaténation}

%% \begin{quote}
%% \small[\textsl{Ce redémarrage lent peut être ignoré par les lecteurs impatients.}]
%% \end{quote}
Dans l'introduction, nous n'avons pas expliqué la conception de
\fun{cat/2}, la fonction qui effectue la concaténation de deux
piles. Recommençons et avançons lentement.

Considérons l'écriture d'une fonction \fun{join/2} (\emph{joindre})
qui se comporte comme \fun{cat/2}. Tout d'abord, l'exigence devrait
être exprimée en français comme une fonction qui prend deux piles et
calcule une pile contenant tous les éléments de la première pile,
suivis par tous les éléments de la seconde, tout en conservant l'ordre
relatif des éléments. En d'autres termes, tous les appels de fonction
\(\fun{join}(s,t)\) sont réécrits en une pile contenant tous les
éléments de~\(s\), suivis par tous les éléments de~\(t\). Une exigence
serait incomplète sans quelques exemples. Par exemple,
\begin{equation*}
\fun{join}([3,5],[2]) \twoheadrightarrow [3,5,2].
\end{equation*}
Néanmoins, si cela est encore un peu vague, nous devrions essayer des
cas extrêmes, c'est-à-dire des configurations très particulières des
arguments, de façon à comprendre plus précisément ce qui est attendu
de cette fonction. Il vient naturellement à l'esprit qu'une pile est
nécessairement vide ou non et, puisque les deux arguments sont des
piles, alors nous sommes amenés à distinguer quatre cas: deux piles
vides; la première pile vide et l'autre non; la première pile n'est
pas vide et la seconde l'est; les deux ne sont pas vides:
\begin{equation*}
\begin{array}{r@{\;}l@{\;}l}
\fun{join}(\el,\el) & \phantom{\rightarrow} & \phantom{\el;}\\
\fun{join}(\el,\cons{y}{t}) & \phantom{\rightarrow} & \phantom{\cons{y}{t};}\\
\fun{join}(\cons{x}{s},\el) & \phantom{\rightarrow} & \phantom{\cons{x}{s};}\\
\fun{join}(\cons{x}{s},\cons{y}{t}) & \phantom{\rightarrow} &
  \phantom{\cons{x}{\fun{join}(s,\cons{y}{t})}.}
\end{array}
\end{equation*}
Il est crucial de ne pas se précipiter pour écrire les membres
droits. Est-ce que des cas ont été oubliés? Non, parce qu'il y a
exactement deux arguments qui peuvent chacun être vides ou non, ce qui
fait \(2 \cdot 2 = 4\) cas. Nous pouvons alors préparer le canevas
suivant:
\begin{equation*}
\begin{array}{r@{\;}l@{\;}l}
\fun{join}(\el,\el) & \rightarrow & \fbcode{CCCCCCCCCC}\,;\\
\fun{join}(\el,\cons{y}{t}) & \rightarrow & \fbcode{CCCCCCCCCC}\,;\\
\fun{join}(\cons{x}{s},\el) & \rightarrow & \fbcode{CCCCCCCCCC}\,;\\
\fun{join}(\cons{x}{s},\cons{y}{t}) & \rightarrow & \fbcode{CCCCCCCCCC}\,.
\end{array}
\end{equation*}
Privilégions la règle qui semble la plus aisée, car il n'y a aucune
obligation de les compléter dans l'ordre d'écriture. Quand nous lisons
les filtres, une représentation mentale claire de la situation devrait
surgir. Ici, il semble que la première règle soit la plus facile car
elle ne contient aucune variable du tout: quelle est la pile faite de
tous les éléments de la (première) pile vide, suivis de tous les
éléments de la (seconde) pile vide? Puisque la pile vide, par
définition, ne contient aucun élément, la réponse est la pile vide:
\begin{equation*}
\begin{array}{r@{\;}l@{\;}l}
\fun{join}(\el,\el) & \rightarrow & \el;\\
\fun{join}(\el,\cons{y}{t}) & \rightarrow & \fbcode{CCCCCCCCCC}\,;\\
\fun{join}(\cons{x}{s},\el) & \rightarrow & \fbcode{CCCCCCCCCC}\,;\\
\fun{join}(\cons{x}{s},\cons{y}{t}) & \rightarrow & \fbcode{CCCCCCCCCC}\,.
\end{array}
\end{equation*}
Nous pourrions peut-être nous demander si ce cas ne devrait pas être
supprimé. En d'autres termes, est-ce un cas signifiant? Oui, car la
description en français du comportement de \fun{join/2} impose que le
résultat soit toujours une pile d'éléments d'autres piles (les
arguments), donc, en l'absence d'éléments à «~mettre~» dans le résultat,
la pile finale «~reste~» vide.

Il devrait être évident maintenant que la deuxième et troisième règle
sont symétriques, car il revient au même d'ajouter une pile non-vide à
la droite d'une pile vide que de l'ajouter à gauche: le résultat
est toujours la pile non-vide en question.
\begin{equation*}
\begin{array}{r@{\;}l@{\;}l}
\fun{join}(\el,\el) & \rightarrow & \el;\\
\fun{join}(\el,\cons{y}{t}) & \rightarrow & \cons{y}{t};\\
\fun{join}(\cons{x}{s},\el) & \rightarrow & \cons{x}{s};\\
\fun{join}(\cons{x}{s},\cons{y}{t}) & \rightarrow & \fbcode{CCCCCCCCCC}\,.
\end{array}
\end{equation*}
La dernière règle est la plus difficile. Que révèle le motif sur le
cas présent? Que les deux piles ne sont pas vides et, plus
précisément, que le sommet de la première est dénoté par la
variable~\(x\), le reste (qui peut être vide ou non) est~\(s\), et de
même pour la seconde pile avec~\(y\) et~\(t\). Est-ce que ces briques
sont suffisantes pour construire le membre droit, c'est-à-dire, le pas
suivant vers la valeur finale? Nous savons que la concaténation de
deux piles \(p\)~et~\(q\) préserve dans le résultat l'ordre total des
éléments présents dans les deux, mais aussi l'ordre relatif des
éléments de~\(p\) par rapport à ceux de~\(q\). En d'autres termes, les
éléments de \(p=\cons{x}{s}\) doivent être présents dans le résultat
avant les éléments de \(q=\cons{y}{t}\) et les éléments de~\(p\)
doivent être dans le même ordre que dans~\(p\) (idem pour~\(q\)). Avec
cet ordonnancement en tête, il est peut-être naturel d'esquisser ce
qui suit:
\begin{equation*}
\fun{join}(\cons{x}{s},\cons{y}{t}) \rightarrow \fbcode{II} \; x \;
\fbcode{II} \; s \; \fbcode{II} \; y \; \fbcode{II} \; t \; \fbcode{II}.
\end{equation*}
En une réécriture, l'élément~\(x\) peut-il être placé à sa place
finale, c'est-à-dire, à une position qu'il n'a pas besoin de quitter
ensuite? La réponse est oui: il doit être mis au sommet du résultat.
\begin{equation*}
\fun{join}(\cons{x}{s},\cons{y}{t}) \rightarrow \cons{x}{\fbcode{II}
  \; s \; \fbcode{II} \; y \; \fbcode{II} \; t \; \fbcode{II}}.
\end{equation*}
Qu'en est-il de l'autre sommet, la variable~\(y\)? Il devrait demeurer
sur le sommet de~\(t\):
\begin{equation*}
\fun{join}(\cons{x}{s},\cons{y}{t}) \rightarrow \cons{x}{\fbcode{II}
  \; s \; \fbcode{II} \; \cons{y}{t}}.
\end{equation*}
Quelle est la relation entre \(s\)~et~\(\cons{y}{t}\)? Nous pourrions
être tentés par
\begin{equation*}
\fun{join}(\cons{x}{s},\cons{y}{t}) \rightarrow
  \cons{x}{\cons{s}{\cons{y}{t}}}.
\end{equation*}
qui est erroné. La raison est que, bien qu'une pile puisse être un
élément d'une autre pile (en d'autres termes, les piles peuvent être
arbitrairement imbriquées dans d'autres piles), \(s\)~ne devrait pas
être employée comme un élément ici, comme nous le ferions en écrivant
\(\cons{s}{\cons{y}{t}}\). Considérons l'exemple filé
\(\fun{join}([3,5],[2])\): ici, le membre gauche de la règle en
question lie \(x\)~à~\(3\), \(s\)~à~\([5]\), \(y\)~à~\(2\) et
\(t\)~à~\(\el\), par conséquent, le membre droit supposé
\(\cons{x}{\cons{s}{\cons{y}{t}}}\) est en réalité
\erlcode{[3|[\textbf{[5]}|[2|[]]]]}, ce qui diffère du résultat
\erlcode{[3|[\textbf{5}|[2|[]]]]} en ce que \erlcode{[5]} n'est
pas~\erlcode{5}.

Comment peuvent \(s\)~et~\(\cons{y}{t}\) être joints? Bien entendu,
cela est le but même de la fonction \fun{join/2} en cours de
définition. Donc, ce dont nous avons besoin ici est un appel récursif:
\begin{equation*}
\begin{array}{r@{\;}l@{\;}l}
\fun{join}(\el,\el) & \rightarrow & \el;\\
\fun{join}(\el,\cons{y}{t}) & \rightarrow & \cons{y}{t};\\
\fun{join}(\cons{x}{s},\el) & \rightarrow & \cons{x}{s};\\
\fun{join}(\cons{x}{s},\cons{y}{t}) & \rightarrow & \underline{\cons{x}{\fun{join}(s,\cons{y}{t})}}.
\end{array}
\end{equation*}

\paragraph{Correction et complétude}
\index{correction|(}
\index{complétude|(}

Il est extrêmement fréquent, lors de la conception d'une fonction, que
nous nous concentrions sur un aspect, sur une règle, et puis sur
d'autres parties du code, et la forêt pourrait alors cacher un
arbre. Lorsque nous acquiesçons à une définition de fonction, l'étape
suivante consiste à vérifier si elle est correcte et complète par
rapport à la compréhension que nous avions a priori de son
comportement.

Nous disons qu'une définition est \emph{correcte} si tous les appels
de fonction qui peuvent être réécrits en une étape peuvent être
davantage réécrits en le résultat attendu, \emph{et} si tout appel qui
échoue était bien prévu. Par échec, nous entendons l'obtention d'une
expression contenant un appel de fonction qui ne peut être réécrit
davantage.

Nous disons qu'une définition est \emph{complète} si tout appel de
fonction que nous nous attendons à être calculable est en effet
calculable. En d'autres termes, nous devons aussi nous assurer que la
définition permet de réécrire en une valeur toute entrée que nous
jugeons acceptable.

Comment vérifions-nous que la dernière définition de \fun{join/2} est
correcte et complète? Si le concept de «~résultat attendu~» n'est pas
formellement défini, typiquement par des mathématiques, nous mettons
en place une \emph{revue de code} et des \emph{tests}. Un aspect
important du passage en revue du code consiste à vérifier à nouveau
les membres gauches de la définition et à voir si toutes les entrées
possibles sont acceptées. Dans le cas où certaines entrées ne sont pas
filtrées par les motifs, nous devons justifier ce fait et garder une
trace écrite en commentaire. Les membres gauches de \fun{join/2}
filtrent toutes les combinaisons de deux piles, qu'elles soient vides
ou non, et c'est exactement ce que nous voulions: ni plus, ni
moins. L'étape suivante est l'inspection des membres droits en nous
posant deux questions:
\begin{enumerate}

  \item Est-ce que les membres droits sont réécrits
    en le type attendu de valeur, pour tous les appels de fonctions?

  \item Est-ce que les appels de fonctions reçoivent bien le type attendu
    d'argument?

\end{enumerate}
Ces vérifications procèdent du fait que certains langages
fonctionnels, comme \Erlang, ne comportent pas d'\emph{inférence de
type} à la compilation. D'autres langages fonctionnels, comme \OCaml
et \Haskell, ont des compilateurs qui établissent automatiquement ces
propriétés. L'examen des membres droits dans la définition de
\fun{join/2} confirme que
\begin{itemize}

\item les membres droits des trois premières règles sont des piles
  contenant le même type d'éléments que les arguments;

\item les arguments de l'unique appel récursif dans la dernière règle
  sont des piles faites d'éléments provenant des paramètres;

\item en supposant que l'appel récursif possède le type attendu, nous
  déduisons que le membre droit de la dernière règle est une pile
  constituée d'éléments des arguments.

\end{itemize}
En conclusion, nous avons répondu positivement aux deux questions
ci-dessus. Remarquons comment nous avons dû supposer que l'appel
récursif avait déjà le type que nous essayions d'établir pour la
définition courante. Il n'y a rien d'erroné dans ce raisonnement,
appelé \emph{inductif}, et il est même omniprésent en
mathématiques. Nous revisiterons l'induction dans différents
contextes.

L'étape suivante consiste à tester la définition. Cela signifie
définir un ensemble d'entrées qui mènent à un ensemble de résultats et
d'échecs qui sont tous attendus. Par exemple, il est attendu que
\(\fun{join}(\el,\el) \twoheadrightarrow \el\), donc nous pourrions
sonder la validité de cette assertion en exécutant le code, et l'appel
de fonction en effet passe le test avec succès. Comment choisir les
entrées de façon pertinente? Il n'y a pas de règles générales, mais
des conseils sont utiles. L'un est de considérer le cas vide ou le
plus petit des cas, quelqu'en soit le sens dans le contexte de la
fonction. Par exemple, si un argument est une pile, alors il faudrait
essayer la pile vide. Si un autre argument est un entier naturel,
alors essayons zéro. Un autre conseil est d'avoir des \emph{plans de
  test}, c'est-à-dire des appels de fonction dont les valeurs sont
connues à l'avance et qui font usage de chaque règle. Dans le cas de
\fun{join/2}, il y a quatre règles à couvrir avec le plan de test.
\index{correction|)} \index{complétude|)}

\paragraph{Amélioration}

Une fois que nous sommes convaincus que la fonction qui vient tout
juste d'être définie est correcte et complète, il est souvent utile de
considérer à nouveau le code en vue d'éventuels amendements. Il y a
différentes directions possible pour apporter des améliorations
souvent appelées \emph{optimisations} bien que leur effet ne soit pas
toujours optimal:
\begin{itemize}

\item Pouvons-nous réécrire la définition de telle sorte que tous les
  cas, ou au moins certains d'entre eux, soient plus efficaces?

\item Existe-t-il une définition équivalente qui requiert moins de
  mémoire dans tout ou partie des cas?

\item Pouvons-nous raccourcir la définition en utilisant moins de
  règles (peut-être certaines sont inutiles ou redondantes) ou des
  membres droits plus brefs?

\item Pouvons-nous employer moins de paramètres dans la définition?
  (Ceci est en relation avec l'usage de la mémoire.)

\end{itemize}
Considérons à nouveau \fun{join/2}:
\begin{equation*}
\begin{array}{r@{\;}l@{\;}l}
\fun{join}(\el,\el) & \rightarrow & \el;\\
\fun{join}(\el,\cons{y}{t}) & \rightarrow & \cons{y}{t};\\
\fun{join}(\cons{x}{s},\el) & \rightarrow & \cons{x}{s};\\
\fun{join}(\cons{x}{s},\cons{y}{t}) & \rightarrow & \cons{x}{\fun{join}(s,\cons{y}{t})}.
\end{array}
\end{equation*}
et concentrons notre attention sur les deux premières règles, dont le
point commun est d'avoir la première pile vide. Il est clair
maintenant que les membres droits sont, dans les deux règles, la
seconde pile, quelle soit vide (première règle) ou non (seconde
règle). Par conséquent, nous n'avons pas besoin de discriminer la
structure de la seconde pile lorsque la première est vide et nous
pouvons alors écrire de manière équivalente:
\begin{equation*}
\begin{array}{r@{\;}l@{\;}l}
\fun{join}(\el,t) & \rightarrow & t,\;\text{où \(t\) est une pile};\\
\fun{join}(\cons{x}{s},\el) & \rightarrow & \cons{x}{s};\\
\fun{join}(\cons{x}{s},\cons{y}{t}) & \rightarrow &
  \cons{x}{\fun{join}(s,\cons{y}{t})}.
\end{array}
\end{equation*}
Remarquons comment cette nouvelle définition n'affirme pas
formellement que \(t\)~est une pile ---~d'où le commentaire~--- donc
elle n'est pas strictement équivalente à la définition originelle:
maintenant \(\fun{join}(\el,5) \rightarrow 5\). Acceptons ce compromis
en préférant la concision de la dernière version ou en supposant
l'existence d'une inférence de types.

Considérons ensuite les deux dernières règles et cherchons des motifs
communs. Il se trouve que, dans la pénultième règle, la première pile
est filtrée par \(\cons{x}{s}\), mais rien n'est fait avec
\(x\)~et~\(s\), à part \emph{reconstruire} \(\cons{x}{s}\) dans le
membre droit. Ceci suggère que nous devrions simplifier la règle
comme suit:
\begin{equation*}
\begin{array}{r@{\;}l@{\;}l}
\fun{join}(\el,t) & \rightarrow & t;\\
\fun{join}(s,\el) & \rightarrow & s,\;
  \text{où \(s\) est une pile non-vide};\\
\fun{join}(\cons{x}{s},\cons{y}{t}) & \rightarrow &
  \cons{x}{\fun{join}(s,\cons{y}{t})}.
\end{array}
\end{equation*}
Nous devons vérifier que changer \(\cons{x}{s}\) en~\(s\) n'affecte
pas le filtrage par motifs, c'est-à-dire que les entrées qui étaient
filtrées auparavant le sont toujours maintenant. En effet, il serait
possible en théorie que la nouvelle variable~\(s\) filtrât une pile
vide. Pouvons-nous établir que \(s\)~n'est jamais vide? Le membre
gauche de la pénultième règle filtre seulement si la règle précédente
n'a pas filtré, en d'autres termes, les règles sont essayées dans
l'ordre d'écriture, soit du haut vers le bas. Par conséquent, nous
savons que~\(s\) ne peut être liée à la pile vide, parce que
\(\el\)~est utilisée dans la règle précédente \emph{et} le second
paramètre peut être n'importe quelle pile. Néanmoins, comme cela s'est
produit antérieurement, \(s\)~n'est plus nécessairement une pile, par
exemple, \(\fun{join}(5,\el) \rightarrow 5\). À nouveau, nous
ignorerons cet effet secondaire et choisirons la concisions de la
dernière définition.

Dans la dernière règle, nous observons que le second argument, filtré
par \(\cons{y}{t}\), est simplement retransmis à l'appel récursif,
donc il se révèle inutile de distinguer \(y\)~et~\(t\), et nous
pouvons avancer
\begin{equation*}
\begin{array}{r@{\;}l@{\;}l}
\fun{join}(\el,t) & \rightarrow & t;\\
\fun{join}(s,\el) & \rightarrow & s;\\
\fun{join}(\cons{x}{s},\underline{t}) & \rightarrow &
\cons{x}{\fun{join}(s,\underline{t})}.
\end{array}
\end{equation*}
%% \begin{equation*}
%% \fun{join}(\el,t) \rightarrow t;\quad
%% \fun{join}(s,\el) \rightarrow s;\quad
%% \fun{join}(\cons{x}{s},\underline{t}) \rightarrow
%% \cons{x}{\fun{join}(s,\underline{t})}.
%% \end{equation*}
(Est-ce que~\(t\) peut être vide?) Ici encore, nous devons nous
assurer que~\(t\) ne peut être liée à une pile vide: elle ne peut être
vide parce que, sinon, le motif précédent aurait filtré l'appel. Mais,
tel quel, le pénultième motif est inclus dans le dernier, c'est-à-dire
que toutes les données qui sont filtrées par le pénultième sont aussi
filtrées par le dernier, ce qui nous amène à nous demander si la
définition serait encore correcte si~\(t\) pouvait être liée à la pile
vide, après tout. Étiquetons les règles comme suit:
\begin{equation*}
\begin{array}{r@{\;}l@{\;}l}
\fun{join}(\el,t) & \smashedrightarrow{\alpha} & t;\\
\fun{join}(s,\el) & \smashedrightarrow{\beta} & s;\\
\fun{join}(\cons{x}{s},t) & \smashedrightarrow{\gamma} &
\cons{x}{\fun{join}(s,t)}.
\end{array}
\end{equation*}
%% \begin{equation*}
%% \fun{join}(\el,t) \xrightarrow{\smash{\alpha}} t;\quad
%% \fun{join}(s,\el) \xrightarrow{\smash{\beta}} s;\quad
%% \fun{join}(\cons{x}{s},t) \xrightarrow{\smash{\gamma}}
%% \cons{x}{\fun{join}(s,t)}.
%% \end{equation*}
Soit~\(s\) une pile contenant \(n\)~éléments, ce que nous écrivons de
manière informelle ainsi: \(s = [x_0, x_1, \dots,
x_{n-1}]\). L'indice~\(i\) dans~\(x_i\) est la \emph{position} de
l'élément dans la pile, le sommet se trouvant à la
position~\(0\). Alors nous réécrivons en une étape:
\begin{equation*}
\fun{join}(s,\el) \smashedrightarrow{\beta} s.
\end{equation*}
Si la règle~\clause{\beta} avait été éliminée, nous aurions obtenu à
la place:
\begin{equation*}
\begin{array}{r@{\;}c@{\;}l}
\fun{join}(s,\el)
& \smashedrightarrow{\gamma} &
  \cons{x\sb{0}}{\fun{join}([x\sb{1},\ldots,x\sb{n-1}],\el)}\\
& \smashedrightarrow{\gamma} &
  \cons{x\sb{0}}{\cons{x\sb{1}}{\fun{join}([x\sb{2},\ldots,x\sb{n-1}],\el)}}\\
  & = &
  \cons{x\sb{0},x\sb{1}}{\fun{join}([x\sb{2},\ldots,x\sb{n-1}],\el)}\\
& \smashedrightarrow{\gamma} &
  \cons{x\sb{0},x\sb{1}}{\cons{x\sb{2}}{\fun{join}([x\sb{3},\ldots,x\sb{n-1}],\el)}}\\
  & = &
  \cons{x\sb{0},x\sb{1},x\sb{2}}{\fun{join}([x\sb{3},\ldots,x\sb{n-1}],\el)}\\
& \smash{\vdots} &\\
& \smashedrightarrow{\gamma} &
  \cons{x\sb{0},x\sb{1},\ldots,x\sb{n-1}}{\fun{join}(\el,\el)}\\
& \smashedrightarrow{\alpha} & \cons{x\sb{0},x\sb{1},\ldots,x\sb{n-1}}{\el}\\
  & = &
  [x\sb{0},x\sb{1},\ldots,x\sb{n-1}]\\
& = & s.
\end{array}
\end{equation*}
En bref, nous avons trouvé: \(\fun{join}(s,\el) \twoheadrightarrow s\).
Ceci signifie que la règle~\clause{\beta} est inutile, puisque sa
suppression nous permet d'atteindre le même résultat~\(s\), bien que
plus lentement: \(n\)~pas par la règle~\clause{\gamma} plus~\(1\) par
la règle~\clause{\alpha}, au lieu d'un pas par la
règle~\clause{\beta}. Nous nous trouvons donc dans une situation où la
définition originelle était spécialisée pour plus d'efficacité quand
la seconde pile est vide. Si nous ôtons la règle~\clause{\beta}, le
programme est plus court mais devient plus lent dans ce cas
particulier.

Cette sorte de dilemme est assez fréquent en programmation et il n'y a
pas toujours de choix clair. Peut-être un autre argument peut ici
faire pencher la balance du côté de la suppression. En effet, bien que
l'élimination ralentit certains appels, elle rend le nombre de
réécritures plus facile à retenir: c'est le nombre d'éléments de la
première pile, plus~\(1\); en particulier, la longueur du second
argument n'est pas pertinente. Décidons donc en faveur de la
suppression et rebaptisons la définition finale:
\begin{equation*}
\begin{array}{r@{\;}l@{\;}l}
\fun{cat}(\el,t) & \xrightarrow{\alpha} & t;\\
\fun{cat}(\cons{x}{s},t) & \xrightarrow{\beta} &
\cons{x}{\fun{cat}(s,t)}.
\end{array}
\end{equation*}
%% \begin{equation*}
%% \fun{cat}(\el,t) \xrightarrow{\alpha} t;\quad
%% \fun{cat}(\cons{x}{s},t) \xrightarrow{\beta} \cons{x}{\fun{cat}(s,t)}.
%% \end{equation*}
Remarquons en passant que \(\fun{cat}(5,\el)\) échoue à nouveau, comme
cela se produisait dans la version originelle.

Lorsque l'on programme des applications moyennes ou grandes, il est
recommandé d'employer des variables évocatrices,
comme~\erlcode{PileDeProc}, au lieu de noms énigmatiques ou
génériques, comme~\(s\). Mais, dans ce livre, nous nous intéressons
principalement aux programmes courts, pas à l'ingénierie du logiciel,
donc des variables courtes nous conviendrons tout à fait. Néanmoins,
nous devons établir et suivre une convention pour que nous
reconnaissions aisément le type des variables d'une définition à
l'autre.

\mypar{Forme terminale}

Comme la réécriture de \(\fun{cat}(s,\el)\) le montre, le membre droit
de la règle~\clause{\beta} de \fun{cat/2} contient un appel dont le
\emph{contexte} est \(\cons{x}{\text{\textvisiblespace}}\), où le
symbole~\textvisiblespace{} dénote le lieu de l'appel
\(\fun{cat}(s,t)\). Quand tous les membres droits d'une définition
sont soit des valeurs, ou des expressions arithmétiques, ou des
expressions constituées seulement de constructeurs de données, ou bien
encore des appels dont les arguments sont des valeurs, ou des
expressions arithmétiques ou des constructeurs de données, la
définition est dite en \emph{forme terminale}.

Nous pourrions nous demander si une variante en forme terminale est
nécessaire ou non, et nous envisagerons cette question
ultérieurement. Pour le moment, saisissons cette occasion comme un
exercice de style et, au lieu de présenter une transformation
systématique, envisageons une approche pragmatique. Dans le cas de
\fun{cat/2}, comme nous l'avons déclaré plus haut, le seul contexte
est \(\cons{x}{\text{\textvisiblespace}}\) et
l'opérateur~(\erlcode{|}) n'est pas associatif, c'est-à-dire que
\(\cons{x}{\cons{y}{z}} \not\equiv \cons{\cons{x}{y}}{z}\). L'idée est
d'ajouter un paramètre dans lequel nous conserverons les valeurs du
contexte, et quand les données initiales seront épuisées, nous
reconstruirons ce contexte à partir de ce paramètre, appelé un
\emph{accumulateur}. Ici, nous souhaitons une nouvelle fonction
\fun{cat/3} dont la définition a pour forme
\begin{equation*}
\begin{array}{r@{\;}c@{\;}l}
\fun{cat}(\el,t,\underline{u}) & \rightarrow & \fbcode{CCCCCCC}\,;\\
\fun{cat}(\cons{x}{s},t,\underline{u}) & \rightarrow & \fbcode{CCCCCCC}\,.
\end{array}
\end{equation*}
Le nouveau paramètre~\(u\) est l'accumulateur en question. Puisque
nous voulons y stocker des~\(x\), ce doit être une pile. De plus, sa
valeur initiale doit être la pile vide, sinon des éléments exogènes
seraient mêlés au résultat. Par conséquent, la définition en forme
terminale, équivalente à \fun{cat/2} et nommée \fun{cat\(_0\)/2},
appelle \fun{cat/3} avec le paramètre supplémentaire~\(\el\):
\begin{equation*}
\fun{cat\(_0\)}(s,t) \rightarrow \fun{cat}(s,t,\el).
\end{equation*}
Revenons à \fun{cat/3} et empilons~\(x\) sur~\(u\):
\begin{equation*}
\begin{array}{r@{\;}c@{\;}l}
\fun{cat}(\el,t,u) & \smashedrightarrow{\alpha} & \fbcode{CCCCCCC}\,;\\
\fun{cat}(\cons{x}{s},t,u) & \smashedrightarrow{\beta} &
  \fun{cat}(s,t,\cons{x}{u}).
\end{array}
\end{equation*}
Que contient l'accumulateur du résultat attendu? Nous savons déjà
qu'il ne peut être un résultat partiel, parce que (\erlcode{|})~n'est
pas associatif. Donc nous devons travailler davantage avec
\(u\)~\emph{et}~\(t\), mais, d'abord, nous devons comprendre ce que
\(u\)~contient à ce point en déroulant à plat un appel, avec un
morceau de papier et un crayon. Soit~\(s\) une pile de \(n\)~éléments
\([x_0,x_1,\dots,x_{n-1}]\). Nous avons
\begin{equation*}
\begin{array}{r@{\;}c@{\;}l}
\fun{cat}(s,t,\el) & \smashedrightarrow{\beta} &
  \fun{cat}([x\sb{1},\ldots,x\sb{n-1}],t,[x\sb{0}])\\
              & \smashedrightarrow{\beta} &
  \fun{cat}([x\sb{2},\ldots,x\sb{n-1}],t,[x\sb{1},x\sb{0}])\\
              & \smash{\vdots}\\
              & \smashedrightarrow{\beta} & \fun{cat}(\el,t,[x\sb{n-1},x\sb{n-2},\ldots,x\sb{0}])\\
              & \smashedrightarrow{\alpha} & \fbcode{CCCCCCC}\,.
\end{array}
\end{equation*}
Par conséquent, \(u\)~dans le membre gauche de la
règle~\clause{\alpha} est liée à une pile qui contient les mêmes
éléments que la valeur originelle du paramètre~\(s\), mais en
\emph{ordre inverse}. En d'autres termes, étant donné l'appel
\(\fun{cat}(s,t,\el)\), le paramètre~\(u\) dans le premier motif de
\fun{cat/3} représente~\(s\) renversé. Que pouvons-nous faire avec
\(u\)~et~\(t\) de manière à atteindre le but? La clé est de réaliser
que la réponse dépend du contenu de~\(u\), qui, par conséquent, a
besoin d'être filtré plus précisément: est-ce que~\(u\) est vide ou
non? Ceci nous amène à séparer la règle~\clause{\alpha} en
\(\alpha_0\) et \(\alpha_1\):
\newlength\Split\settowidth\Split{\(_{\alpha_0}\)}
\begin{equation*}
\begin{array}{r@{\;}c@{\;}l}
\fun{cat}(\el,t,\el) & \MyArrow{\Split}{\alpha\sb{0}} &
  \fbcode{CCCCCCC}\,;\\
\fun{cat}(\el,t,\cons{x}{u}) & \MyArrow{\Split}{\alpha\sb{1}} &
  \fbcode{CCCCCCC}\,;\\
\fun{cat}(\cons{x}{s},t,u) & \MyArrow{\Split}{\beta} &
 \fun{cat}(s,t,\cons{x}{u}).
\end{array}
\end{equation*}
Remarquons que les règles \clause{\alpha_0}~et~\clause{\alpha_1}
pourraient être inversées, puisqu'elles filtrent des cas complètement
distincts. Le membre droit de la règle~\clause{\alpha_0} est aisément
deviné: ce doit être~\(t\) puisqu'il correspond au cas où nous voulons
concaténer la pile vide et~\(t\):
\begin{equation*}
\begin{array}{r@{\;}c@{\;}l}
\fun{cat}(\el,t,\el) & \MyArrow{\Split}{\alpha\sb{0}} & t;\\
\fun{cat}(\el,t,\cons{x}{u}) & \MyArrow{\Split}{\alpha\sb{1}} & \fbcode{CCCCCCC}\,;\\
\fun{cat}(\cons{x}{s},t,u) & \MyArrow{\Split}{\beta} &
  \fun{cat}(s,t,\cons{x}{u}).
\end{array}
\end{equation*}
Comment mettons-nous en relation~\(t\), \(x\)~et~\(u\) dans la
règle~\clause{\alpha_1} avec le résultat que nous recherchons? Puisque
\(\fun{cat}(s,t,\el) \twoheadrightarrow
\fun{cat}(\el,t,\cons{x}{u})\), nous savons que \(\cons{x}{u}\)
est~\(s\) retourné, donc l'élément~\(x\) est le dernier dans~\(s\) et
il devrait être au-dessus de~\(t\) dans le résultat. Que devrions-nous
faire avec~\(u\)? La clé est de voir que nous avons besoin de
recommencer le même processus, c'est-à-dire placer un appel récursif:
\begin{equation*}
\begin{array}{r@{\;}c@{\;}l}
\fun{cat}(\el,t,\el) & \MyArrow{\Split}{\alpha\sb{0}} & t;\\
\fun{cat}(\el,t,\cons{x}{u}) & \MyArrow{\Split}{\alpha\sb{1}} &
  \fun{cat}(\el,\cons{x}{t},u);\\
\fun{cat}(\cons{x}{s},t,u) & \MyArrow{\Split}{\beta} &
  \fun{cat}(s,t,\cons{x}{u}).
\end{array}
\end{equation*}
Pour mettre à l'épreuve la correction de cette définition, nous
pouvons essayer un petit exemple:
\begin{equation*}
\begin{array}{r@{\;}c@{\;}l}
\fun{cat}([1,2,3],[4,5],\el)
  & \MyArrow{\Split}{\beta} & \fun{cat}([2,3],[4,5],[1])\\
  & \MyArrow{\Split}{\beta} & \fun{cat}([3],[4,5],[2,1])\\
  & \MyArrow{\Split}{\beta} & \fun{cat}(\el,[4,5],[3,2,1])\\
  & \MyArrow{\Split}{\alpha\sb{1}} &
    \fun{cat}(\el,[3,4,5],[2,1])\\
  & \MyArrow{\Split}{\alpha\sb{1}} &
    \fun{cat}(\el,[2,3,4,5],[1])\\
  & \MyArrow{\Split}{\alpha\sb{1}} &
    \fun{cat}(\el,[1,2,3,4,5],\el)\\
  & \MyArrow{\Split}{\alpha\sb{0}} & [1,2,3,4,5].
\end{array}
\end{equation*}
En conclusion, la version de \fun{cat/2} en forme terminale, appelée
\fun{cat\(_0\)/2}, requiert une fonction auxiliaire \fun{cat/3} avec
un accumulateur dont le but est de retourner le premier argument:
\begin{equation*}
\begin{array}{r@{\;}c@{\;}l}
\fun{cat\(_0\)}(s,t) & \smashedrightarrow{\alpha} & \fun{cat}(s,t,\el).\\
\fun{cat}(\el,t,\el) & \smashedrightarrow{\beta} & t;\\
\fun{cat}(\el,t,\cons{x}{u}) & \smashedrightarrow{\gamma} &
  \fun{cat}(\el,\cons{x}{t},u);\\
\fun{cat}(\cons{x}{s},t,u) & \smashedrightarrow{\delta} &
  \fun{cat}(s,t,\cons{x}{u}).
\end{array}
\end{equation*}
Nous savons aussi que faire quand le contexte n'est pas un appel à un
opérateur associatif: nous empilons les valeurs des variables qu'il
contient et quand la pile d'entrée est vide, nous les dépilons pour
les placer dans le contexte reconstruit que nous évaluons en
dernier. Nous reviendrons sur cette méthode.

\paragraph{Efficacité}

Le nombre d'étapes pour réécrire \(\fun{cat\(_0\)}(s,t)\) en une
valeur est plus élevé qu'avec \(\fun{cat}(s,t)\), comme nous l'avions
deviné lorsque nous avions déroulé l'exemple précédent. En effet,
en supposant que~\(s\) contient \(n\)~éléments, nous avons
\begin{itemize}

  \item un pas pour obtenir \(\fun{cat}(s,t,\el)\) avec
  la règle~\clause{\alpha};

  \item \(n\)~pas pour retourner~\(s\) sur l'accumulateur avec la
  règle~\clause{\delta};

  \item \(n\)~pas pour retourner l'accumulateur sur~\(t\) avec la
    règle~\clause{\gamma};

  \item un pas quand l'accumulateur est finalement vide, avec la
    règle~\clause{\beta}.

\end{itemize}
Donc, le nombre total d'étapes est \(2n+2\), ce qui est le double du
coût avec la version précédente. Pourquoi cette différence entre
\fun{cat/2} et \fun{cat\(_0\)/2}? L'opération effectuée sur
l'accumulateur consiste à empiler un élément sur une pile qui doit
être retournée plus tard: l'accumulateur n'est pas un résultat partiel
mais une \emph{pile temporaire} utilisée pour contenir les éléments de
la première pile en ordre inverse. Nous rencontrerons de nombreuses
occurrences de cette situation. Entre temps, nous devons nous souvenir
qu'une variante en forme terminale d'une fonction opérant sur des
piles peut conduire à un programme plus lent. Par ailleurs, la forme
terminale peut être plus longue, comme l'illustre \fun{cat\(_0\)/2}:
quatre règles au lieu de deux.

L'évaluation \(\fun{cat}_0([1,2,3],[4,5]) \twoheadrightarrow
[1,2,3,4,5]\) vue plus haut peut être conçue abstraitement comme un
produit, ou composition, de règles: \(\alpha \cdot \delta^n \cdot
\gamma^n \cdot \beta\), ou simplement
\(\alpha\delta^n\gamma^n\beta\). Cette expression est appelée une
\emph{trace d'exécution} et sa longueur est le nombre de règles
\(\C{\fun{cat\(_0\)}}{n}\) de \(\fun{cat\(_0\)}(s,t)\), sachant que la
longueur d'une règle est~\(1\), d'où \(\len{\alpha} = \len{\beta} =
\len{\gamma} = \len{\delta} = 1\), et la longueur de la composition de
deux règles est la somme de leur longueur: \(\len{\alpha \cdot \delta}
= \len{\alpha} + \len{\delta}\). Par conséquent:
\begin{align*}
\C{\fun{cat\(_0\)}}{n}
  &= \len{\alpha\delta^n\gamma^n\beta}
   = \len{\alpha} + \len{\delta^n} + \len{\gamma^n} + \len{\beta}
   = 1 + \len{\delta} \cdot n + \len{\gamma} \cdot n + 1\\
  &= 2n + 2.
\end{align*}

\paragraph{Digression}

Reconsidérons la définition~\eqref{def:fact} de
\fun{fact/1}\index{fact@\textsf{fact/1}} \vpageref{def:fact}:
\begin{equation*}
\begin{array}{r@{\;}l@{\;}l}
\fun{fact}(0) & \smashedrightarrow{\alpha} & 1;\\
\fun{fact}(n) & \smashedrightarrow{\beta} & n \cdot \fun{fact}(n-1).
\end{array}
\end{equation*}
Par exemple, nous avons
\begin{equation*}
\begin{array}{r@{\;}c@{\;}l@{\;}c@{\;}l}
\fun{fact}(3) & \xrightarrow{\beta} & 3 \cdot \fun{fact}(3-1)
              & = & 3 \cdot \fun{fact}(2)\\
              & \xrightarrow{\smash[t]{\beta}} &
              3 \cdot (2 \cdot \fun{fact}(2-1))
              & = & 3 \cdot (2 \cdot \fun{fact}(1))\\
              & \xrightarrow{\smash[t]{\alpha}} &
                3 \cdot (2 \cdot (1)) & = & 6 = 3!
\end{array}
\end{equation*}
Il est souvent plus clair de composer implicitement des opérations
arithmétiques intermédiaires~(\(=\)) avec la réécriture en cours et
d'écrire plus simplement:
\begin{equation*}
\fun{fact}(3) \xrightarrow{\beta} 3 \cdot \fun{fact}(2)
\xrightarrow{\smash[t]{\beta}} 3 \cdot (2 \cdot \fun{fact}(1))
\xrightarrow{\smash[t]{\alpha}} 3 \cdot (2 \cdot (1)) = 6.
\end{equation*}
Remarquons que la dernière réécriture
(\(\xrightarrow{\smash[t]{\alpha}}\)) doit être suivie par une série
de multiplications~\(3 \cdot (2 \cdot 1)\) parce que chaque
multiplication doit être différée jusqu'à ce que \(\fun{fact}(1)\)
soit calculé. Ceci aurait pu être anticipé car l'appel à \fun{fact/1}
dans le membre droit de la règle (\(\xrightarrow{\smash[t]{\beta}}\)),
c'est-à-dire, le texte souligné dans
\begin{equation*}
\fun{fact}(n) \xrightarrow{\smash[t]{\beta}} n \cdot \underline{\fun{fact}(n-1)}
\end{equation*}
possède le contexte non-vide
«~\erlcode{\(n\)~*~\textvisiblespace}~». Pour bien comprendre pourquoi
ceci est important, examinons une série légèrement plus
longue:\label{trace_fact_5}
\begin{equation*}
\begin{array}{r@{\;}c@{\;}l}
\fun{fact}(5)
& \smashedrightarrow{\beta} & 5 \cdot \fun{fact}(4)\\
& \smashedrightarrow{\beta} & 5 \cdot (4 \cdot \fun{fact}(3))\\
& \smashedrightarrow{\beta} & 5 \cdot (4 \cdot (3 \cdot \fun{fact}(2)))\\
& \smashedrightarrow{\beta} & 5 \cdot (4 \cdot (3 \cdot (2 \cdot
                              \fun{fact}(1))))\\
& \smashedrightarrow{\alpha} & 5 \cdot (4 \cdot (3 \cdot (2 \cdot
(1)))).
\end{array}
\end{equation*}
Il devient clair que chaque réécriture par
(\(\xrightarrow{\smash[t]{\beta}}\)) engendre une expression plus
longue que la précédente. Concentrons-nous maintenant uniquement sur
les formes que prennent ces expressions:
\begin{equation*}
\setlength\fboxsep{0mm}
\begin{array}{r@{\;}c@{\;}l}
\fun{fact}(5)
& \smashedrightarrow{\beta} & \fbox{\phantom{$5 \cdot \fun{fact}(4)$}}\\
& \smashedrightarrow{\beta} & \fbox{\phantom{$5 \cdot (4 \cdot \fun{fact}(3))$}}\\
& \smashedrightarrow{\beta} & \fbox{\phantom{$5 \cdot (4 \cdot (3 \cdot \fun{fact}(2)))$}}\\
& \smashedrightarrow{\beta} & \fbox{\phantom{$5 \cdot (4 \cdot (3 \cdot (2 \cdot \fun{fact}(1))))$}}\\
& \smashedrightarrow{\alpha} & \fbox{\phantom{$5 \cdot (4 \cdot (3
    \cdot (2 \cdot (1))))$}}.
\end{array}
\end{equation*}
Ce phénomène suggère qu'un grand espace, c'est-à-dire de
\emph{mémoire} d'ordinateur, est nécessaire pour conserver ces
expressions avant la longue série de calculs arithmétiques à la
fin. L'exemple nous incite à induire que le terme le plus grand
apparaissant dans l'évaluation de \(\fun{fact}(n)\) est celui juste
avant (\(\xrightarrow{\smash[t]{\alpha}}\)) et sa taille semble être
proportionnelle à~\(n\), car tous les entiers de \(n\)~à~\(1\) ont dû
être stockés jusqu'au dernier moment.

Une version en forme terminale
\fun{fact\(_0\)/1}\index{fact0@\textsf{fact\(_0\)/1}} est
\begin{equation*}
\begin{array}{r@{\;}c@{\;}l}
\fun{fact}_0(n) & \rightarrow & \fun{fact}_0(n,1),
\;\text{si \(n \geqslant 1\)}.\\
\fun{fact}_0(1,a) & \rightarrow & a;\\
\fun{fact}_0(n,a) & \rightarrow & \fun{fact}_0(n-1,a \cdot n).
\end{array}
\end{equation*}
Ici, par contraste avec \fun{cat/3}, l'opération sur l'accumulateur
est associative (une multiplication) et l'accumulateur est, à tout
instant, un résultat partiel. Au lieu de différer les multiplications,
exactement une multiplication est effectuée à chaque réécriture
élémentaire, donc, au bout du compte, il ne reste plus rien à faire:
il n'y a aucune instance du contexte à réactiver. Ce genre de
définition est donc en forme terminale.\index{langage
  fonctionnel!forme terminale}

Notons que le coût de \fun{fact\(_0\)/1} est~\(n+1\), alors que le
coût de \fun{fact/1} est~\(n\), donc, contrairement à
\fun{cat\(_0\)/2} et \fun{cat/2}, la forme terminale ici n'augmente
pas le coût de façon significative. Ceci est dû à la nature des
opérations sur l'accumulateur, qui ne requièrent pas un retournement
ou, généralement, une inversion.

L'appel de fonction \(\fun{fact}_0(5)\), considéré tantôt, est ainsi
évalué:
\begin{equation*}
\begin{array}{r@{\;}c@{\;}l@{\;}c@{\;}l}
\fun{fact}_0(5)
& \xrightarrow{\smash[t]{\alpha}} & \fun{fact}_0(5,1),
&& \text{car \(5 > 1\)},\\
& \xrightarrow{\smash[t]{\gamma}} & \fun{fact}_0(5-1,1 \cdot 5)
& = & \fun{fact}_0(4,5)\\
& \xrightarrow{\smash[t]{\gamma}} & \fun{fact}_0(4-1,5 \cdot 4)
& = & \fun{fact}_0(3,20)\\
& \xrightarrow{\smash[t]{\gamma}} & \fun{fact}_0(3-1,20 \cdot 3)
& = & \fun{fact}_0(2,60)\\
& \xrightarrow{\smash[t]{\gamma}} & \fun{fact}_0(2-1,60 \cdot 2)
& = & \fun{fact}_0(1,120)\\
& \xrightarrow{\smash[t]{\beta}} & 120.
\end{array}
\end{equation*}
La raison pour laquelle \(\fun{fact}_0(5) \equiv \fun{fact}(5)\) est
que
\begin{equation}
  (((1 \cdot 5) \cdot 4) \cdot 3) \cdot 2
= 5 \cdot (4 \cdot (3 \cdot (2 \cdot 1))).\label{eq:fact5}
\end{equation}
Cette égalité tient parce qu'en général, pour tout nombre~\(x\), \(y\)~et~\(z\),
\begin{enumerate}

\item \label{mult_assoc} la multiplication est associative: \(x \cdot
  (y \cdot z) = (x \cdot y) \cdot z\);

\item \label{mult_one} le nombre~\(1\) est neutre par rapport à
  (\(\cdot\)): \(x \cdot 1 = 1 \cdot x = x\).

\end{enumerate}
Pour voir précisément pourquoi, écrivons~(\(\eqn{\ref{mult_assoc}}\))
et~(\(\eqn{\ref{mult_one}}\)) pour dénoter, respectivement, l'usage de
l'associativité et de la neutralité, puis couchons sur le papier les
égalités suivantes qui relient le membre gauche de
l'égalité~\eqref{eq:fact5} à démontrer à son membre droit:
\begin{align*}
  (((1 \cdot 5) \cdot 4) \cdot 3) \cdot 2
  &\eqn{\smash{\ref{mult_one}}} ((((1 \cdot 5) \cdot 4) \cdot 3) \cdot
  2) \cdot 1\\
  &\eqn{\smash{\ref{mult_assoc}}} (((1 \cdot 5) \cdot 4) \cdot 3)
  \cdot (2 \cdot 1)\\
  &\eqn{\smash{\ref{mult_assoc}}} ((1 \cdot 5) \cdot 4) \cdot (3 \cdot (2
    \cdot 1))\\
  &\eqn{\smash{\ref{mult_assoc}}} (1 \cdot 5) \cdot (4 \cdot (3 \cdot
  (2 \cdot 1))\\
  &\eqn{\smash{\ref{mult_assoc}}} 1 \cdot (5 \cdot (4 \cdot (3 \cdot
  (2 \cdot 1))))\\
  &\eqn{\smash{\ref{mult_one}}} 5 \cdot (4 \cdot (3 \cdot (2 \cdot
  1))).\quad \Box
\end{align*}
De plus, si nous ne voulons pas nous appuyer sur la neutralité
de~\(1\), nous pourrions définir une autre fonction équivalente,
\fun{fact\(_1\)/1}, qui initialise l'accumulateur avec
\(\fun{fact}_1(n-1,n)\), au lieu de \(\fun{fact}_0(n,1)\), et puis
s'arrête lorsque le nombre est~\(0\), au lieu de~\(1\):
\begin{equation*}
\begin{array}{r@{\;}c@{\;}l}
\fun{fact}_1(n) & \smashedrightarrow{\alpha} &
\fun{fact}_1(n-1,n),\;\text{si \(n > 0\)}.\\
\fun{fact}_1(0,a) & \smashedrightarrow{\beta} & a;\\
\fun{fact}_1(n,a) & \smashedrightarrow{\gamma} &
\fun{fact}_1(n-1,a \cdot n).
\end{array}
\end{equation*}
Maintenant, le même exemple se déroule ainsi:
\begin{equation*}
\begin{array}{r@{\;}c@{\;}l@{\;}c@{\;}l}
\fun{fact}_1(5)
& \xrightarrow{\smash[t]{\alpha}} & \fun{fact}_1(5-1,5)
& = & \fun{fact}_1(4,5)\\
& \xrightarrow{\smash[t]{\gamma}} & \fun{fact}_1(4-1,5 \cdot 4)
& = & \fun{fact}_1(3,20)\\
& \xrightarrow{\smash[t]{\gamma}} & \fun{fact}_1(3-1,20 \cdot 3)
& = & \fun{fact}_1(2,60)\\
& \xrightarrow{\smash[t]{\gamma}} & \fun{fact}_1(2-1,60 \cdot 2)
& = & \fun{fact}_1(1,120)\\
& \xrightarrow{\smash[t]{\gamma}} & \fun{fact}_1(1-1,120 \cdot 1)
& = & \fun{fact}_1(0,120)\\
& \xrightarrow{\smash[t]{\beta}} & 120.
\end{array}
\end{equation*}
Cette nouvelle version repose sur l'égalité suivante, qui peut être
démontrée seulement à l'aide de l'associativité: \((((5 \cdot 4) \cdot
3) \cdot 2) \cdot 1 = 5 \cdot (4 \cdot (3 \cdot (2 \cdot 1)))\).

Le nombre de réécritures avec \fun{fact\(_0\)/1} est presque le même
qu'avec \fun{fact/1}, précisément une de plus due à la
règle~\clause{\alpha}. Mais la première présente un avantage en termes
d'usage de la mémoire, tant que l'on suppose que tous les entiers dans
un certain intervalle occupent le même espace. Par exemple, ceci
signifie que la mémoire nécessaire pour conserver le nombre~\(120\)
est la même que pour le nombre~\(5\). Alors les formes des réécritures
précédentes sont:
\begin{equation*}
\setlength\fboxsep{0mm}
\begin{array}{r@{\;}c@{\;}l}
\fun{fact}_0(5)
& \xrightarrow{\smash[t]{\alpha}} &
\fbox{\phantom{$\fun{fact}_0(0,10)$}}\\
& \xrightarrow{\smash[t]{\gamma}} &
\fbox{\phantom{$\fun{fact}_0(0,10)$}}\\
& \xrightarrow{\smash[t]{\gamma}} &
\fbox{\phantom{$\fun{fact}_0(0,10)$}}\\
& \xrightarrow{\smash[t]{\gamma}} &
\fbox{\phantom{$\fun{fact}_0(0,10)$}}\\
& \xrightarrow{\smash[t]{\gamma}} &
\fbox{\phantom{$\fun{fact}_0(0,10)$}}\\
& \xrightarrow{\smash[t]{\beta}} & \fbox{\phantom{CCC}}.
\end{array}
\end{equation*}
Il semble probable que cette version utilise un bout de mémoire
constant, alors que \fun{fact/1} requiert une quantité croissante de
mémoire, proportionnellement à~\(n\) lorsque l'on
détermine~\(n!\). (Dans les sections qui suivent, nous verrons que les
arbres de syntaxe abstraite constituent un modèle plus précis de
l'allocation de mémoire.) Ce phénomène a été prévu par le lecteur
attentif qui a remarqué qu'il n'y a pas de contexte pour les appels
dans les règles définissant \fun{fact\(_0\)/2}, donc il n'y a pas de
calculs différés qui s'accumulent jusqu'à la fin. En conclusion,
\fun{fact\(_0\)/1} est toujours préférable à \fun{fact/1}.

La discussion précédente sur l'obtention de définitions équivalentes
qui sont en forme terminale suppose de considérer les programmes comme
une sorte de donnée. Pour l'instant, il s'agit uniquement là d'un
point de vue méthodologique et nous ne pouvons pas réellement
manipuler les fonctions comme des piles, par exemple. (Nous
reviendrons là-dessus ultérieurement, en discutant des fonctions
d'ordre supérieur et du style par continuations.) Nous voulons ici
dire que ces définitions peuvent être transformées en d'autres
définitions et que ceci est souvent une excellente méthode, par
opposition à essayer de deviner dès le départ à quoi ressemble la
définition finale. Il aurait été probablement plus difficile d'écrire
la version en forme terminale de \fun{cat/2} sans avoir d'abord conçu
la version qui n'était pas en forme terminale.

En général, ce n'est pas une bonne idée de viser bille en tête une
définition en forme terminale parce qu'elle peut être inutile ou la
démarche elle-même peut être entachée d'erreurs, puisque ces
définitions sont souvent plus complexes. Dans les sections et
chapitres qui viennent, nous expliquerons quand une forme terminale
est désirable et comment l'obtenir en suivant une méthode
systématique.

Considérons l'exemple simple qu'est la définition d'une fonction
\fun{last/1} telle que \(\fun{last}(s)\) calcule le dernier élément de
la pile non-vide~\(s\). L'approche correcte est de laisser de côté
toute forme terminale et de viser le c{\oe}ur du sujet. Nous savons
que~\(s\) ne peut être vide, donc commençons par le membre gauche suivant:
\begin{equation*}
\fun{last}(\cons{x}{s}) \rightarrow \fbcode{CCCCCCC}\,.
\end{equation*}
Pouvons-nous atteindre le résultat d'un pas? Non, car nous ne savons
pas si \(x\)~est l'élément recherché: nous devons en savoir plus à
propos de~\(s\). Cette information supplémentaire au sujet de la
structure de~\(s\) est procurée par des motifs plus précis: \(s\)~peut
être vide ou non, c'est-à-dire, \(s = \el\) ou \(s=\cons{y}{t}\). Nous
avons alors:
\begin{equation*}
\begin{array}{r@{\;}c@{\;}l}
\fun{last}(\cons{x}{\el}) & \rightarrow & \fbcode{CCCCCCC};\\
\fun{last}(\cons{x}{\cons{y}{t}}) & \rightarrow & \fbcode{CCCCCCC}.
\end{array}
\end{equation*}
Le premier motif peut être simplifié comme suit:
\begin{equation*}
\begin{array}{r@{\;}c@{\;}l}
\fun{last}([x]) & \rightarrow & \fbcode{CCCCCCC};\\
\fun{last}(\cons{x}{\cons{y}{t}}) & \rightarrow & \fbcode{CCCCCCC}.
\end{array}
\end{equation*}
Le premier membre droit est facile à deviner:
\begin{equation*}
\begin{array}{r@{\;}c@{\;}l}
\fun{last}([x]) & \rightarrow & x;\\
\fun{last}(\cons{x}{\cons{y}{t}}) & \rightarrow & \fbcode{CCCCCCC}.
\end{array}
\end{equation*}
Dans la dernière règle, quelle relation unit \(x\), \(y\), \(t\) et le
résultat? Pouvons-nous l'atteindre d'un pas? Non, bien que nous
sachions que~\(x\) n'est \emph{pas} le résultat, nous ignorons
toujours si~\(y\) l'est, donc nous devons recommencer, ce qui veut
dire qu'un appel récursif s'impose:
\begin{equation*}
\begin{array}{r@{\;}c@{\;}l}
\fun{last}([x]) & \rightarrow & x;\\
\fun{last}(\cons{x}{\cons{y}{t}}) & \rightarrow &
\fun{last}(\fbcode{CCCC}).
\end{array}
\end{equation*}
Remarquons que savoir qu'une partie de la donnée n'est pas utile à la
construction du résultat est une connaissance utile. Nous ne pouvons
appeler récursivement \(\fun{last}(t)\) parce que \(t\)~peut être vide
et l'appel échouerait alors, signifiant que la réponse était en
fait~\(y\). Par conséquent, nous devons appeler avec \(\cons{y}{t}\)
pour conserver la possibilité que~\(y\) soit le dernier:
\begin{equation*}
\begin{array}{r@{\;}c@{\;}l}
\fun{last}([x]) & \rightarrow & x;\\
\fun{last}(\cons{x}{\cons{y}{t}}) & \rightarrow &
\fun{last}(\cons{y}{t}).
\end{array}
\end{equation*}
Comme nous l'avons recommandé précédemment, la phase suivante est de
tester la correction et la complétude de la définition, au moyen
d'ex\-em\-ples significatifs (couvrant des cas extrêmes et toutes les
règles au moins une fois). Pour gagner du temps ici, nous supposerons
néanmoins que \fun{last/1} est correcte et complète. L'étape suivante
est alors de tenter de l'améliorer. Cherchons des motifs communs aux
deux membres d'une même règle et examinons s'ils peuvent être
évités. Par exemple, nous pouvons observer que \(\cons{y}{t}\)~est
utilisé comme un tout, en d'autres termes, \(y\)~et~\(t\) ne sont pas
utilisées séparément dans le second membre droit. Par conséquent, il
vaut la peine de revenir sur nos pas et de remplacer le motif par un
plus général, dans ce cas \(s = \cons{y}{t}\).
\begin{equation*}
\begin{array}{r@{\;}c@{\;}l}
\fun{last}([x]) & \rightarrow & x;\\
\fun{last}(\cons{x}{s}) & \rightarrow & \fun{last}(s).
\end{array}
\end{equation*}
Cette transformation est correcte parce que le cas où \(s\)~est vide a
déjà été filtré par le premier motif. Remarquons d'ailleurs que nous
venons de penser une définition comme une sorte de donnée. (Nous
devrions dire plus pertinemment \emph{métadonnée}, puisque les
définitions ne sont pas des données qui sont l'objet d'un traitement
dans le programme, mais par le programmeur.) En passant, \fun{last/1}
est en forme terminale.

Et si nous avions essayé de trouver directement une définition en
forme terminale? Nous pourrions nous être souvenu que de telles
définitions font souvent usage d'un accumulateur et nous aurions
peut-être esquissé ce qui suit:
\begin{equation*}
\begin{array}{r@{\;}c@{\;}l}
\fun{last}_0(s) & \rightarrow & \fun{last}_1(s,0).\\
\fun{last}_1(\el,y) & \rightarrow & y;\\
\fun{last}_1(\cons{x}{s},y) & \rightarrow & \fun{last}_1(s,x).
\end{array}
\end{equation*}
La première observation pourrait être à propos du nom de fonction
\fun{last\(_1\)}. Pourquoi ne pas écrire comme suit, dans le style
habituel jusqu'à maintenant:
\begin{equation*}
\begin{array}{r@{\;}c@{\;}l}
\fun{last}_0(s) & \rightarrow & \fun{last}_0(s,0).\\
\fun{last}_0(\el,y) & \rightarrow & y;\\
\fun{last}_0(\cons{x}{s},y) & \rightarrow & \fun{last}_0(s,x).
\end{array}
\end{equation*}
Il n'y aurait pas de confusion entre \fun{last\(_0\)/1} et
\fun{last\(_0\)/2} parce que, chacune recevant un nombre différent
d'arguments, elles sont logiquement différentes. La raison pour
laquelle nous recommandons la distinction de nom et en général un nom
unique par fonction, est que cette discipline permet au compilateur de
repérer l'erreur consistant en l'oubli d'un argument. Par exemple, le
programme
\begin{equation*}
\begin{array}{r@{\;}c@{\;}l}
\fun{last}_0(s) & \rightarrow & \fun{last}_0(s,0).\\
\fun{last}_0(\el,y) & \rightarrow & y;\\
\fun{last}_0(\cons{x}{s},y) & \rightarrow &
\underline{\fun{last}_0(s)}.
\end{array}
\end{equation*}
contient une erreur qui n'est pas détectée, alors que
\begin{equation*}
\begin{array}{r@{\;}c@{\;}l}
\fun{last}_0(s) & \rightarrow & \fun{last}_1(s,0).\\
\fun{last}_1(\el,y) & \rightarrow & y;\\
\fun{last}_1(\cons{x}{s},y) & \rightarrow &
\underline{\fun{last}_1(s)}.
\end{array}
\end{equation*}
est identifié comme erroné. Toutefois, dans ce livre, pour des raisons
didactiques, nous ne nous plierons pas toujours à cette recommendation
d'unicité des noms de fonction. La possibilité d'employer le même nom
pour différentes fonctions qui peuvent par ailleurs être distinguées
par leur nombre d'arguments est appelée \emph{surcharge}. La surcharge
de fonctions dans un langage de programmation comme \Cpp est permise,
mais les règles pour distinguer les différentes fonctions de même nom
sont différentes que celles d'\Erlang, car elles tiennent compte du
type des arguments, en plus de leur nombre.

En calculant l'appel \(\fun{last}_0([1,2,3])\) avec la définition
originelle, nous voyons que les trois règles sont couvertes jusqu'à ce
que le résultat, \(3\), soit atteint. Étant donné que nous avions
recommandé auparavant des tests aux limites et que l'argument est une
pile, nous essayons la pile vide et obtenons l'évaluation
\(\fun{last}_0(\el) \rightarrow \fun{last}_1(\el,0) \rightarrow 0\),
ce qui est inattendu, car ce test aurait dû échouer (\fun{last/1}
n'est pas définie pour la pile vide). Pouvons-nous remédier à cela?

Changeons simplement le membre gauche de \fun{last\(_0\)/1} de telle
sorte que seules les piles non-vides soient filtrées. Nous nous
trouvons face à un cas où plus d'information sur la structure de la
donnée est nécessaire et une variable constitue un motif trop
général. Au lieu de cela, nous avons besoin de
\begin{equation*}
\begin{array}{r@{\;}c@{\;}l}
\fun{last}_0(\underline{\cons{x}{s}}) & \rightarrow &
  \fun{last}_0(\underline{\cons{x}{s}},0).\\
\fun{last}_0(\el,y) & \rightarrow & y;\\
\fun{last}_0(\cons{x}{s},y) & \rightarrow & \fun{last}_0(s,x).
\end{array}
\end{equation*}
Cet amendement semble aller contre l'amélioration que nous avions
effectuée tantôt, quand nous avions remplacé \(\cons{y}{t}\)
par~\(s\), mais ce n'est pas le cas: ici, nous voulons exclure des
données; en termes plus généraux, nous ne recherchons pas une fonction
équivalente, alors que précédemment le but était de simplifier et
d'obtenir une fonction équivalente.

La définition de \fun{last\(_0\)/1} est correcte et complète mais un
examen attentif devrait éveiller quelques soupçons quant à sa réelle
simplicité. Par exemple, la valeur initiale de l'accumulateur, donnée
dans l'unique membre droit de \fun{last\(_0\)/1} est~\(0\), mais ce
nombre n'est jamais utilisé et il est immédiatement écarté dans le
second membre droit de \fun{last\(_0\)/2}. En effet, nous pourrions
écrire la définition équivalente suivante:
\begin{equation*}
\begin{array}{r@{\;}c@{\;}l}
\fun{last}_1(\cons{x}{s}) & \rightarrow & \fun{last}_1(\cons{x}{s},\underline{7}).\\
\fun{last}_1(\el,y) & \rightarrow & y;\\
\fun{last}_1(\cons{x}{s},y) & \rightarrow & \fun{last}_1(s,x).
\end{array}
\end{equation*}
La valeur initiale de l'accumulateur ici n'a même pas besoin d'être un
entier, ce pourrait être n'importe quel type de valeur, telle que
\([4,\el]\). Ceci est le signe que nous devrions abandonner cette
inextricable définition, qui est le produit d'une méthode qui ne
considère pas les programmes comme des données et est fondée sur
l'assomption que les définitions en forme terminale requièrent
souvent un accumulateur: en général, ce n'est pas le cas.

Prenons par exemple l'identité polymorphe: \(\fun{id}(x) \rightarrow
x\). Elle est trivialement en forme terminale. En passant, la forme
terminale n'a rien à voir avec la récursivité, malgré l'occurrence
fréquente de la malencontreuse locution «~fonction récursive
terminale~». Une définition récursive peut être en forme terminale,
mais une définition en forme terminale peut ne pas être récursive.
