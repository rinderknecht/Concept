\chapter{Traduction en \Java}
\index{Java@\Java!traduction en $\sim$|(}

Dans ce chapitre, nous montrons comment traduire de simples programmes
fonctionnels en \Java, illustrant ce qu'on pourrait appeler un style
fonctionnel dans un langage à objets. Nous ne ferons pas usage
d'effets, donc à toute variable n'est assignée qu'une seule valeur au
cours de l'exécution. Les objets programmés de cette manière sont
parfois appelés \emph{objets fonctionnels}. Les difficultés que nous
rencontrerons seront dues au système de types de \Java, puisque notre
langage fonctionnel n'est pas typé. En conséquence, certains
programmes ne pourront pas être traduits et d'autres demanderont des
traductions intermédiaires, appelées \emph{raffinements}, avant de
pouvoir produire un équivalent en \Java.

Dans l'introduction, \vpageref{par:java}, nous avons couché le patron
conceptuel pour les piles: une classe abstraite \texttt{Stack<Item>}
paramétrisée par le type \texttt{Item} de ses éléments et deux
extensions, \texttt{EStack<Item>} pour les piles vides et
\texttt{NStack<Item>} pour les autres. Il est important de bien
comprendre les raisons de ce patron, parce qu'il est probablement plus
fréquent de rencontrer une mise en œuvre des piles comme
\begin{alltt}
\public \class Stack<Item> \{\hfill// Stack.java
  \private Item head;
  \private Stack<Item> tail;

  \public Stack() \{ head = \nullX; tail = \nullX; \}

  \public Item pop() \throws EmptyStack \{\hfill// \emph{Dépilage}
    \ifX (empty()) \throw \new EmptyStack();
    \final Item orig = head;
    \ifX (tail.empty()) head = \nullX; \elseX head = tail.pop();
    \return orig; \}

  \public \booleanX empty() \{ \return head == \nullX; \}

  \public \void push(\final Item item) \{\hfill// \emph{Empilage}
    Stack<Item> next = \new Stack<Item>();
    next.head = head;
    next.tail = tail;
    head = item;
    tail = next; \}
\}

// EmptyStack.java
\public \class EmptyStack \extends Exception \{\}
\end{alltt}
Ce codage a plusieurs défauts. Tout d'abord, il est incorrect si le
sommet (\texttt{head}) d'une pile est une référence
\texttt{null}. Nous pouvons remédier à cette confusion en ajoutant un
niveau d'indirection, c'est-à-dire en créant une classe privée
contenant \texttt{head} et \texttt{tail}, ou en modelant la pile vide
avec \texttt{null}, ce qui implique de vérifier la présence de
\texttt{null} avant d'appeler une méthode. Deuxièmement, l'usage
général de références \texttt{null} augmente le risque d'un accès
invalide. Troisièmement, le code de \texttt{pop} et \texttt{push}
suggère déjà que d'autres opérations exigeront de longues
manipulations de références. Quatrièmement, la persistance n'est pas
facile, par exemple conserver des versions successives de la pile.

Une étude approfondie des problèmes que les références nulles créent
dans la conception de langages de programmation a été menée par
\cite{ChalinJames_2007,Cobbe_2008} et \cite{Hoare_2009}. Un
inconvénient pratique de telles références est qu'elles rendent la
composition de méthodes maladroite et, par exemple, au lieu d'écrire
\texttt{s.cat(t).push(x).rev()}, il nous faudrait vérifier si chaque
appel ne retourne pas \texttt{null}.

La conception que nous avons présentée dans l'introduction évite tous
les problèmes et limitations précédemment mentionnés. Bien entendu, il
y a un prix à payer, dont nous allons discuter plus loin. Pour le
moment, souvenons-nous du programme \Java en introduction, qui sera
complété plus tard dans ce chapitre:
\begin{alltt}
\public \abstractX \class Stack<Item> \{\hfill// Stack.java
  \public \final NStack<Item> push(\final Item item) \{
    \return \new NStack<Item>(item,\this); \}
  \public \abstractX Stack<Item> cat(\final Stack<Item> t); \}

// EStack.java
\public \final \class EStack<Item> \extends Stack<Item> \{
  \public Stack<Item> cat(\final Stack<Item> t) \{ \return t; \}\}

// NStack.java
\public \final \class NStack<Item> \extends Stack<Item> \{
  \private \final Item head;
  \private \final Stack<Item> tail;

  \public NStack(\final Item item, \final Stack<Item> stack) \{
    head = item; tail = stack; \}

  \public NStack<Item> cat(\final Stack<Item> t) \{
    \return tail.cat(t).push(head); \}
\}
\end{alltt}
Remarquons que nous avons évité la définition d'une méthode
\texttt{pop}. La raison est qu'elle est intrinsèquement une fonction
partielle, pas définie quand son argument est la pile vide, et c'est
pour cela que nous avions dû ressortir à l'exception
\texttt{EmptyStack} précédemment. De toute façon, en pratique, cette
méthode est peu utile, car elle serait implicitement présente dans un
algorithme, par exemple, dans la définition de \texttt{cat} à la
classe \texttt{NStack<Item>}. Plus fondamentalement, il n'y a pas
besoin d'une méthode \texttt{pop} parce que \texttt{head} et
\texttt{tail} font partie de la structure de donnée elle-même.

\section{Liaison dynamique}
\label{sec:single_dispatch}
% \extends Comparable<? \super Item>

\paragraph{Retournement de pile}
\index{pile!retournement!$\sim$ en \Java|(}

Comme nous l'avons vu à l'équation~\eqref{def:rev} \vpageref{def:rev},
la manière efficace de retourner une pile dans notre langage
fonctionnel consiste à utiliser un accumulateur comme suit:
\begin{equation*}
\fun{rev}(s) \xrightarrow{\smash{\epsilon}}
             \fun{rcat}(s,\el).\quad\;\;
\fun{rcat}(\el,t) \xrightarrow{\smash{\zeta}} t;\quad\;\;
\fun{rcat}(\cons{x}{s},t) \xrightarrow{\smash{\eta}}
                          \fun{rcat}(s,\cons{x}{t}).
\end{equation*}
La fonction \fun{rev/1} est définie par une seule règle qui ne
discrimine pas son paramètre. Par conséquent, sa traduction sera une
méthode dans la classe abstraite \texttt{Stack<Item>}:
\begin{alltt}
// Stack.java
\public \abstractX \class Stack<Item> \{
  ...
  \public Stack<Item> rev() \{
    \return rcat(\new EStack<Item>()); \}
\}
\end{alltt}
Un examen des motifs des règles de la fonction \fun{rcat/2} montre que
seul le premier paramètre est contraint; plus précisément, il doit
être une pile vide ou non-vide. Cette simple sorte de définition
fonctionnelle est facilement traduite en comptant sur un mécanisme de
\Java appelé \emph{liaison dynamique} et qui fait que c'est la classe
d'un objet à l'exécution (et non à la déclaration) qui détermine la
méthode appelée. Dans une règle, quand exactement un paramètre est
contraint, il est alors le candidat naturel pour la liaison
dynamique. Par exemple,
\begin{alltt}
// Stack.java
\public \abstractX \class Stack<Item> \{
  ...
  \public \abstractX Stack<Item> rcat(\final Stack<Item> t);
\}
\end{alltt}
La class \texttt{EStack<Item>} contient la traduction de la règle~\(\zeta\):
\begin{alltt}
// EStack.java
\public \final \class EStack<Item> \extends Stack<Item> \{
  ...
  \public Stack<Item> rcat(\final Stack<Item> t) \{ \return t; \}
\}
\end{alltt}
La classe \texttt{NStack<Item>} contient la traduction de la
règle~\(\eta\) (\texttt{tail} est~\(s\), \texttt{head} est~\(x\)):
\begin{alltt}
// NStack.java
\public \final \class NStack<Item> \extends Stack<Item> \{
  ...
  \public Stack<Item> rcat(\final Stack<Item> t) \{
    \return tail.rcat(t.push(head)); \}
\}
\end{alltt}
\index{pile!retournement!$\sim$ en \Java|)}

\paragraph{Filtrage de pile}
\index{pile!filtrage!$\sim$ en \Java|(}

Souvenons-nous de la fonction \fun{sfst/2} à la
section~\vref{sec:skipping}:
\begin{equation*}
\fun{sfst}(\el,x)          \xrightarrow{\smash{\theta}} \el;\quad\;\;
\fun{sfst}(\cons{x}{s},x)  \xrightarrow{\smash{\iota}}  s;\quad\;\;
\fun{sfst}(\cons{y}{s},x)  \xrightarrow{\smash{\kappa}}
                                         \cons{y}{\fun{sfst}(s,x)}.
\end{equation*}
Cette définition ne peut être traduite telle quelle parce qu'elle
contient un test d'égalité implicite dans ses motifs
\(\iota\)~et~\(\kappa\), donc nous ne pouvons employer la liaison
dynamique. La solution la plus simple est peut-être d'étendre notre
langage fonctionnel avec une expression conditionnelle \(\fun{if}
\dots \fun{then} \dots \fun{else} \dots\) et de raffiner la définition
originelle pour en faire usage. Ici, nous obtenons
\begin{equation*}
\fun{sfst}(\el,x)          \xrightarrow{\smash{\theta}} \el;\quad\;\;
\fun{sfst}(\cons{y}{s},x)  \xrightarrow{\smash{\iota+\kappa}}
\fun{if} \; x = y \; \fun{then} \; s \;
\fun{else} \; \cons{y}{\fun{sfst}(s,x)}.
\end{equation*}
Maintenant que les motifs ne se recoupent plus, nous pouvons traduire
en \Java. Tout d'abord, nous ne devons pas oublier d'étendre la classe
abstraite \texttt{Stack<Item>}:
\begin{alltt}
// Stack.java
\public \abstractX \class Stack<Item> \{
  ...
  \public \abstractX Stack<Item> sfst(\final Item x);
\}
\end{alltt}
La traduction de la règle~\(\theta\) va dans la classe
\texttt{EStack<Item>}:
\begin{alltt}
// EStack.java
\public \final \class EStack<Item> \extends Stack<Item> \{
  ...
  \public EStack<Item> sfst(\final Item x) \{ \return \this; \}
\}
\end{alltt}
Les traductions de \(\iota\)~et~\(\kappa\) vont dans
\begin{alltt}
// NStack.java
\public \final \class NStack<Item> \extends Stack<Item> \{
  ...
  \public Stack<Item> sfst(\final Item x) \{
    \return head.compareTo(x) == 0 ?
           tail : tail.sfst(x).push(head); \}
\}
\end{alltt}
Cette dernière étape révèle une erreur et une limitation de notre
méthode: dans le but de comparer deux valeurs de la classe
\texttt{Item} comme \texttt{head} et~\texttt{x}, nous devons spécifier
que le paramètre \texttt{Item} étend la classe prédéfinie
\texttt{Comparable}. Par conséquent, nous devons réécrire nos
définitions de classes comme suit:
\begin{alltt}
\public \abstractX \class Stack<Item
\hfill\extends Comparable<? \super Item>> \{
  ...
\}
\public \class EStack<Item \extends Comparable<? \super Item>>
       \extends Stack<Item> \{
  ...
\}
\public \class NStack<Item \extends Comparable<? \super Item>>
       \extends Stack<Item> \{
  ...
\}
\end{alltt}
Ceci est une limitation parce que nous devons contraindre
\texttt{Item} pour qu'il soit comparable, même si quelques méthodes
n'en ont pas besoin, comme \texttt{rev}, qui est purement
structurelle. Par ailleurs, la classe abstraite doit être mise à jour
à chaque fois qu'une nouvelle opération est ajoutée. (Pour une
compréhension détaillée des génériques en \Java, voir
\cite{NaftalinWadler_2006}.)  \index{pile!filtrage!$\sim$ en \Java|)}

\paragraph{Tri par insertion}
\index{tri par insertion!$\sim$ en \Java|(}
\index{Java@\Java|see{tri par insertion}}

Le tri par insertion a été défini à la section~\vref{sec:straight_ins}
ainsi:
\begin{equation*}
\begin{array}{@{}r@{\;}l@{\;}lr@{\;}l@{\;}l@{}}
  \fun{ins}(\cons{y}{s},x)
& \xrightarrow{\smash{\kappa}}
& \cons{y}{\fun{ins}(s,x)}, \,\text{si \(x \succ y\)};
& \fun{isrt}(\el)
& \xrightarrow{\smash{\mu}}
& \el;\\
  \fun{ins}(s,x)
& \xrightarrow{\smash{\lambda}}
& \cons{x}{s}.
& \fun{isrt}(\cons{x}{s})
& \xrightarrow{\smash{\nu}}
& \fun{ins}(\fun{isrt}(s),x).
\end{array}
\end{equation*}
La fonction \fun{isrt/1} est facile à traduire car elle n'a besoin que
de la liaison dynamique. La fonction \fun{ins/2} aussi, mais elle doit
être raffinée en faisant apparaître une conditionnelle. La partie
facile d'abord:
\begin{alltt}
// Stack.java
\public \abstractX \class Stack<Item
\hfill\extends Comparable<? \super Item>> \{
  ...
  \public \abstractX Stack<Item> isrt();
\}
// EStack.java
\public \class EStack<Item \extends Comparable<? \super Item>>
       \extends Stack<Item> \{
  ...
  \public EStack<Item> isrt() \{ \return \this; \}
\}
// NStack.java
\public \class NStack<Item \extends Comparable<? \super Item>>
       \extends Stack<Item> \{
  ...
  \public NStack<Item> isrt() \{\return tail.isrt().ins(head);\}
\}
\end{alltt}
Remarquons que \(\fun{ins}(\fun{isrt}(s),x)\) est devenu
\texttt{tail.isrt().ins(head)}. La méthode générale consiste, d'abord,
à traduire \(x\)~et~\(s\) en \texttt{x}~et~\texttt{tail}; ensuite, il
faut trouver un ordre d'évaluation possible (il peut y en avoir plus
d'un) et écrire les traductions de chaque appel de fonction de gauche
à droite, séparés par des points. Dans le cas en cours,
\(\fun{isrt}(s)\) est évalué en premier et devient
\texttt{tail.isrt()}, ensuite le contexte
\(\fun{ins}(\texttt{\textvisiblespace},x)\) donne
\texttt{\textvisiblespace.ins(head)}.

Le raffinement de \fun{ins/2} est
\begin{equation*}
\begin{array}{@{}r@{\;}l@{\;}l@{}}
\fun{ins}(\cons{y}{s},x) & \rightarrow & \fun{if} \; x \succ y \;
\fun{then} \; \cons{y}{\fun{ins}(s,x)} \; \fun{else} \; \cons{x}{s};\\
\fun{ins}(\el,x) & \rightarrow & [x].
\end{array}
\end{equation*}
Remarquons que nous avons séparé la règle~\(\lambda\) en deux cas:
\(\el\) et \(\cons{y}{s}\), donc le dernier peut être fusionné avec la
règle~\(\kappa\) et prendre part à la conditionnelle. Nous pouvons
traduire maintenant en \Java:
\begin{alltt}
// Stack.java
\public \abstractX \class Stack<Item
\hfill\extends Comparable<? \super Item>> \{
  ...
  \protectedX \abstractX NStack<Item> ins(\final Item x);
\}
// EStack.java
\public \class EStack<Item \extends Comparable<? \super Item>>
       \extends Stack<Item> \{
  ...
  \protectedX NStack<Item> ins(\final Item x) \{\return push(x);\}
\}
// NStack.java
\public \class NStack<Item \extends Comparable<? \super Item>>
       \extends Stack<Item> \{
  ...
  \protectedX NStack<Item> ins(\final Item x) \{
    \return head.compareTo(x) < 0 ?
           tail.ins(x).push(head) : push(x); \}
\}
\end{alltt}
La méthode \texttt{ins} est déclarée \texttt{protected}, parce qu'il
est erroné d'insérer un élément dans une pile qui n'est pas ordonnée.
\index{tri par insertion!$\sim$ en \Java|)}

\paragraph{Test}
\index{test|(}

Pour tester ces définitions, nous avons besoin d'une méthode
\texttt{print}:
\begin{alltt}
\public \abstractX \void print();\hfill// Stack.java
\public \void print() \{ System.out.println(); \}\hfill// EStack.java
\public \void print() \{\hfill// NStack.java
  System.out.print(head + " "); tail.print(); \}
\end{alltt}
Finalement, la classe \texttt{Main} ressemblerait à
\begin{alltt}
// Main.java
\public \class Main \{
  \public \static \void \main (String[] args) \{
    Stack<Integer> nil = \new EStack<Integer>();
    Stack<Integer> s = nil.push(5).push(2).push(7);
    s.print();\hfill// 7 2 5
    s.rev().print();\hfill// 5 2 7
    Stack<Integer> t = nil.push(4).push(1);\hfill// 1 4
    s.cat(t).print();\hfill// 7 2 5 1 4
    t.cat(s).isrt().print();\hfill// 1 2 4 5 7
  \}
\}
\end{alltt}
Les propriétés que nous avons prouvées plus tôt sur les programmes
fonctionnels sont transférés sur les traductions en \Java. En
particulier, les coûts sont invariants et ils comptent le nombre
d'appels de méthodes nécessaires pour calculer la valeur d'un appel de
méthode, \emph{en supposant que nous ne comptons pas les appels à
  \texttt{push}}: cette méthode, qui est une commodité, était déclarée
\texttt{final} dans l'espoir que le compilateur copie sa définition
aux sites d'appels.\index{test|)}

\paragraph{Découpage}
\index{pile!découpage!$\sim$ en \Java|(}

À la section~\vref{sec:cutting}, nous avons vu comment découper une
pile en deux en spécifiant la longueur du préfixe:
\begin{equation*}
\begin{array}{@{}r@{\;}l@{\;}lr@{\;}l@{\;}l@{}}
\fun{cut}(s,0) & \rightarrow & \pair{\el}{s};
& \fun{push}(x,\pair{t}{u}) & \rightarrow & \pair{\cons{x}{t}}{u}.\\
\fun{cut}(\cons{x}{s},k) & \rightarrow
& \fun{push}(x,\fun{cut}(s,k-1)).
\end{array}
\end{equation*}
Ici, nous avons un autre exemple de règle qui recouvre deux cas: dans
la première règle de \fun{cut/2}, la pile \(s\)~est vide ou non. Si
elle ne l'est pas, la traduction de la règle doit être fusionnée avec
la traduction de la deuxième règle. Il y a une difficulté
supplémentaire dans le fait que \Java ne fournisse pas de paires
natives pour traduire \(\pair{t}{u}\) immédiatement. Heureusement, il
n'est pas difficile de concevoir une classe pour les paires si nous
nous rendons compte qu'une paire est, abstraitement, une chose avec
deux propriétés: une qui renseigne à propos de sa «première
composante» et une autre à propos de sa «seconde composante.» Nous
avons
\begin{alltt}
// Pair.java
\public \class Pair<Fst,Snd> \{
  \protectedX \final Fst fst;
  \protectedX \final Snd snd;
  \public Pair(\final Fst f, \final Snd s) \{fst = f; snd = s;\}

  \public Fst fst() \{ \return fst; \}
  \public Snd snd() \{ \return snd; \}
\}
\end{alltt}
Nous pouvons procéder comme suit:
\begin{alltt}
// Stack.java
\public \abstractX \class Stack<Item
\hfill\extends Comparable<? \super Item>> \{
  ...
  \public \abstractX
            Pair<Stack<Item>,Stack<Item>> cut(\final int k);
\}
\end{alltt}
Remarquons que, comme d'habitude, une fonction avec \(n\)~paramètres
est traduite en une méthode avec \(n-1\) paramètres parce que l'un
d'entre eux est utilisé pour la liaison dynamique ou, en termes du
système de type, sur lui s'applique le \emph{sous-typage}
\citep{Pierce_2002}.
\begin{alltt}
// EStack.java
\public \class EStack<Item \extends Comparable<? \super Item>>
       \extends Stack<Item> \{
  ...
  \public Pair<Stack<Item>,Stack<Item>> cut(\final int k) \{
    \return \new Pair<Stack<Item>,Stack<Item>>(\this,\this);
  \}
\}

// NStack.java
\public \class NStack<Item \extends Comparable<? \super Item>>
       \extends Stack<Item> \{
  ...
  \public Pair<Stack<Item>,Stack<Item>> cut(\final int k) \{
    \ifX (k == 0)
       \return \new Pair<Stack<Item>,Stack<Item>>
                      (\new EStack<Item>(),\this);
    Pair<Stack<Item>,Stack<Item>> p = tail.cut(k-1);
    \return \new Pair<Stack<Item>,Stack<Item>>
                   (p.fst().push(head),p.snd());
  \}
\}
\end{alltt}
Finalement, la classe~\texttt{Main} pourrait ressembler à ce qui suit:
\begin{alltt}
// Main.java
\public \class Main \{
  \public \static \void \main (String[] args) \{
    Stack<Integer> nil = \new EStack<Integer>();
    Stack<Integer> s = nil.push(5).push(2).push(7);\hfill// 7 2 5
    Stack<Integer> t = nil.push(4).push(1);\hfill// 1 4
    Pair<Stack<Integer>,Stack<Integer>> u = s.cat(t).cut(2);
    u.fst().print();\hfill// 7 2
    u.snd().print();\hfill// 5 1 4
  \}
\}
\end{alltt}
Une traduction peut ou doit satisfaire différentes propriétés
intéressantes. Par exemple, elle doit être correcte en ce sens que le
résultat de l'évaluation d'un appel dans le langage de départ est,
dans un certain sens, égal au résultat de l'évaluation de la
traduction de l'appel. C'est ce que nous souhaitions pour notre
traduction de notre langage fonctionnel vers \Java. Il pourrait être
aussi exigé que les comportements erronés soit aussi traduits en
comportements erronés. Par exemple, il pourrait être intéressant que
les boucles infinies soient aussi traduites en boucles infinies et que
les erreurs à l'exécution dans les évaluations du langage de départ
correspondent à des erreurs similaires dans le langage d'arrivée. (La
traduction de \fun{cut/2} ne préserve pas toutes les
erreurs. Pourquoi?)\index{pile!découpage!$\sim$ en \Java|)}


\section{Méthodes binaires}
\index{Java@\Java|see{tri par interclassement}}

Les fonction définies en filtrant deux piles dans le même motif
correspondent, traduites telles quelles, à des \emph{méthodes
  binaires} dans le langage à objet. Malheureusement, \Java n'a que la
liaison dynamique, c'est-à-dire que seul un paramètre est l'objet de
sous-typage. Dans le cas des méthodes binaires,
\index{Java@\Java!méthode binaire} nous pouvons raffiner la définition
de départ en une autre, équivalente, qui sera traduite avec la liaison
dynamique seule. Cette technique a été proposée il y a longtemps par
\cite{Ingalls_1986} et un résumé des recherches a été publié par
\cite{MuscheviciPotaninTemperoNoble_2008}.

\paragraph{Tri par interclassement}

Par exemple, considérons à nouveau la définition de \fun{mrg/2} à la
\fig~\vref{fig:tms}:
\begin{equation*}
\begin{array}{@{}r@{\;}l@{\;}l@{}}
\fun{mrg}(\el,t)         & \rightarrow & t;\\
\fun{mrg}(s,\el)         & \rightarrow & s;\\
\fun{mrg}(\cons{x}{s},\cons{y}{t}) & \rightarrow
                         & \cons{y}{\fun{mrg}(\cons{x}{s},t)},\;
                           \text{si \(x \succ y\)};\\
\fun{mrg}(\cons{x}{s},t) & \rightarrow & \cons{x}{\fun{mrg}(s,t)}.
\end{array}
\end{equation*}
Nous devons réécrire cette définition de telle sorte qu'une seule pile
est inspectée dans les motifs. Pour cela, nous choisissons
arbitrairement une pile-paramètre, disons la première, et nous la
filtrons avec les motifs vide et non-vide comme suit:
\begin{equation*}
\begin{array}{@{}r@{\;}l@{\;}l@{}}
\fun{mrg}(\el,t)         & \rightarrow & t;\\
\fun{mrg}(\cons{x}{s},t) & \rightarrow & \fun{mrg}_0(x,s,t).\\
\\
\fun{mrg}_0(x,s,\el)     & \rightarrow & \cons{x}{s};\\
\fun{mrg}_0(x,s,\cons{y}{t}) & \rightarrow
& \fun{if} \; x \succ y \; \fun{then} \;
  \cons{y}{\fun{mrg}(\cons{x}{s},t)} \;
  \fun{else} \; \cons{x}{\fun{mrg}(s,\cons{y}{t})}.
\end{array}
\end{equation*}
Remarquons que nous avons dû introduire une fonction auxiliaire
\fun{mrg\(_0\)/3} pour gérer le filtrage du second argument, \(t\), de
\fun{mrg/2}. Nous avons aussi utilisé une conditionnelle dans la
dernière règle de \fun{mrg\(_0\)/3}, où nous pouvons aussi éviter les
empilages \(\cons{x}{s}\) et \(\cons{y}{t}\) en appelant,
respectivement, \(\fun{mrg}_0(x,s,t)\) et \(\fun{mrg}_0(y,t,s)\). Ce
dernier appel est une conséquence du fait que l'interclassement est
une fonction symétrique: \(\fun{mrg}(s,t) \equiv
\fun{mrg}(t,s)\). Nous avons maintenant:
\begin{equation*}
\begin{array}{@{}r@{\;}l@{\;}l@{}}
\fun{mrg}(\el,t)         & \rightarrow & t;\\
\fun{mrg}(\cons{x}{s},t) & \rightarrow & \fun{mrg}_0(x,s,t).\\
\\
\fun{mrg}_0(x,s,\el)     & \rightarrow & \cons{x}{s};\\
\fun{mrg}_0(x,s,\cons{y}{t}) & \rightarrow
& \fun{if} \; x \succ y \; \fun{then} \;
  \cons{y}{\fun{mrg}_0(x,s,t)} \;
  \fun{else} \; \cons{x}{\fun{mrg}_0(y,t,s)}.
\end{array}
\end{equation*}
Nous avons alors deux fonctions qui peuvent être traduites en \Java à
l'aide du sous-typage:\index{tri par interclassement!interclassement!$\sim$ en \Java|(}
\begin{alltt}
// Stack.java
\public \abstractX \class Stack<Item
\hfill\extends Comparable<? \super Item>> \{
  ...
  \public \abstractX Stack<Item> mrg(\final Stack<Item> t);
  \public \abstractX Stack<Item> mrg0(\final Item x,
                                   \final Stack<Item> s);
\}
// EStack.java
\public \class EStack<Item \extends Comparable<? \super Item>>
       \extends Stack<Item> \{
  ...
  \public Stack<Item> mrg(\final Stack<Item> t) \{ \return t; \}
  \public Stack<Item> mrg0(\final\! Item x,\final Stack<Item> s)\{
    \return s.push(x); \}
\}
// NStack.java
\public \class NStack<Item \extends Comparable<? \super Item>>
       \extends Stack<Item> \{
  ...
  \public Stack<Item> mrg(\final Stack<Item> t) \{
    \return t.mrg0(head,tail); \}

  \public Stack<Item> mrg0(\final Item x,
                          \final Stack<Item> s) \{
    \return x.compareTo(head) > 0 ?
           tail.mrg0(x,s).push(head)
         : s.mrg0(head,tail).push(x); \}
\}
\end{alltt}
Gardons à l'esprit que le motif \(\fun{mrg}_0(x,s,\cons{y}{t})\)
signifie que \(\cons{y}{t}\) se traduit par \texttt{this}, donc le
programme cible devrait se loger dans la classe \texttt{NStack<Item>},
où \texttt{head} est la traduction de~\(y\) et \texttt{tail} est
l'image de~\(t\). Finalement,\index{tri par
  interclassement!interclassement!$\sim$ en \Java|)}
\begin{alltt}
// Main.java
\public \class Main \{
  \public \static \void \main (String[] args) \{
    Stack<Integer> nil = \new EStack<Integer>();
    Stack<Integer> s = nil.push(5).push(2).push(7);\hfill// 7 2 5
    Stack<Integer> t = nil.push(4).push(1);\hfill// 1 4
    s.isrt().mrg(t.isrt()).print();\hfill// 1 2 4 5 7
    t.isrt().mrg(s.isrt()).print();\hfill// 1 2 4 5 7
  \}
\}
\end{alltt}
Parvenus à ce point, on pourrait arguer que notre schéma de traduction
mène à des programmes \Java cryptiques. Mais cette critique n'est
valable que parce qu'on ne prend pas en compte leur
spécification. Nous proposons que le programme fonctionnel \emph{est}
la spécification du programme \Java et devrait l'accompagner. Une
traduction en \Erlang pourrait être effectuée d'abord, ce qui est
facile, et celle-ci serait traduite à son tour en \Java, tout en
demeurant en commentaires dans le code cible et sa documentation. Par
exemple, nous écririons:
\begin{alltt}
// NStack.java
\public \class NStack<Item \extends Comparable<? \super Item>>
       \extends Stack<Item> \{
  ...
  // mrg0(X,S,[Y|T]) -> case X > Y of
  //                      true  -> [Y|mrg0(X,S,T)];
  //                      false -> [X|mrg0(Y,T,S)]
  //                    end.
  //
  \public Stack<Item> mrg0(\final Item x,
                          \final Stack<Item> s) \{
    \return x.compareTo(head) > 0 ?
           tail.mrg0(x,s).push(head)
         : s.mrg0(head,tail).push(x); \}
\}
\end{alltt}
(Nous pourrions même aller plus loin et renommer les paramètres
\Erlang \texttt{Y} et~\texttt{T}, respectivement, en \texttt{Head} et
\texttt{Tail}.) Ceci présente l'avantage supplémentaire que la
spécification est exécutable, donc le résultat de l'exécution du
programme \Erlang devrait être égal au résultat de l'évaluation de la
traduction en \Java, ce qui facilite la phase de test.

Finissons maintenant la traduction du tri par interclassement
descendant à la \fig~\vref{fig:tms}. Nous avons
\begin{equation*}
\fun{tms}(\cons{x,y}{t}) \xrightarrow{\smash{\alpha}}
 \fun{cutr}([x],\cons{y}{t},t);\qquad
\fun{tms}(t) \xrightarrow{\smash{\beta}} t.
\end{equation*}
Un raffinement est nécessaire parce que la règle~\(\beta\) couvre le
cas de la pile vide et du singleton. Explicitons tous ces cas et
ordonnons-les ainsi:
\begin{equation*}
\fun{tms}(\el)            \rightarrow  \el;\quad
\fun{tms}([x])            \rightarrow  [x];\quad
\fun{tms}(\cons{x,y}{t})  \rightarrow
                          \fun{cutr}([x],\cons{y}{t},t).
\end{equation*}
Maintenant, nous devons fusionner les deux dernières règles parce
qu'elles sont couvertes par le cas «pile non-vide» et nous créons une
fonction auxiliaire \fun{tms\(_0\)/2} dont le rôle est de distinguer
entre le singleton et les piles plus longues. D'où
\begin{equation*}
\begin{array}{@{}r@{\;}l@{\;}l@{\qquad}r@{\;}l@{\;}l@{}}
  \fun{tms}(\el)             & \xrightarrow{\smash{\gamma}} & \el;
& \fun{tms}_0(x,\el)         & \xrightarrow{\smash{\epsilon}} & [x];\\
\fun{tms}(\cons{x}{t})       & \xrightarrow{\smash{\delta}}
                             & \fun{tms}_0(x,t).
& \fun{tms}_0(x,\cons{y}{t}) & \xrightarrow{\smash{\zeta}}
                             & \fun{cutr}([x],\cons{y}{t},t).
\end{array}
\end{equation*}
À ce point, \fun{tms/1} et \fun{tms\(_0\)/2} peuvent être traduites en
\Java à l'aide du sous-typage mais, avant, nous devons vérifier si le
dernier raffinement est vraiment équivalent au programme
originel. Nous pouvons tester quelques appels dont les évaluations
partielles\index{langage fonctionnel!évaluation!$\sim$ partielle}
utilisent, toutes ensemble, toutes les règles dans le programme
raffiné et ensuite nous les comparons avec les évaluations partielles
dans le programme originel. (Ceci est un cas de \emph{test
  structurel}\index{test!$\sim$ structurel}, plus précisément,
\emph{test de chemins}\index{test!$\sim$ de chemin}.) Ici, nous avons
découvert que trois cas sont distincts: pile vide, singleton et piles
plus longues. Dans le programme raffiné, nous avons les
interprétations
\begin{equation*}
\begin{array}{@{}r@{\;}l@{\;}l@{}}
\fun{tms}(\el) & \xrightarrow{\smash{\gamma}} & \el.\\
\fun{tms}([x]) & \xrightarrow{\smash{\delta}} & \fun{tms}_0(x,\el)
                 \xrightarrow{\smash{\epsilon}} [x].\\
\fun{tms}(\cons{x,y}{t})
  & \xrightarrow{\smash{\delta}} & \fun{tms}_0(x,\cons{y}{t})
    \xrightarrow{\smash{\zeta}} \fun{cutr}([x],\cons{y}{t},t).
\end{array}
\end{equation*}
Nous pouvons maintenant les comparer avec les mêmes appels
partiellement évalués dans le programme originel:
\begin{equation*}
\begin{array}{@{}r@{\;}l@{\;}l@{}}
\fun{tms}(\el) & \xrightarrow{\smash{\beta}} & \el.\\
\fun{tms}([x]) & \xrightarrow{\smash{\beta}} & [x].\\
\fun{tms}(\cons{x,y}{t}) & \xrightarrow{\smash{\alpha}}
                         & \fun{cutr}([x],\cons{y}{t},t).
\end{array}
\end{equation*}
Ces appels concourent et couvrent toutes les flèches dans toutes
définitions. Nous pouvons maintenant traduire le programme
raffiné:\index{tri par interclassement!$\sim$ descendant!$\sim$ en
  \Java|(}
\begin{alltt}
// Stack.java
\public \abstractX \class Stack<Item
\hfill\extends Comparable<? \super Item>> \{
  ...
  \public \abstractX Stack<Item> tms();
  \protectedX \abstractX Stack<Item> tms0(\final Item x);
\}
// EStack.java
\public \class EStack<Item \extends Comparable<? \super Item>>
       \extends Stack<Item> \{
  ...
  \public Stack<Item> tms() \{ \return \this; \}
  \protectedX Stack<Item> tms0(\final Item x) \{\return push(x);\}
\}
// NStack.java
\public \class NStack<Item \extends Comparable<? \super Item>>
       \extends Stack<Item> \{
  ...
  \public Stack<Item> tms() \{ \return tail.tms0(head); \}

  \protectedX Stack<Item> tms0(\final Item x) \{
    \return tail.cutr(\new EStack<Item>().push(x),\this); \}
\}
\end{alltt}
\index{tri par interclassement!$\sim$ descendant!$\sim$ en \Java|)}
Remarquons à nouveau que les fonctions auxiliaires résultent en des
méthodes \texttt{protected} et qu'elles ajoutent au coût total, en
comparaison avec le programme fonctionnel: ceci est une limitation du
transfert de propriétés du programme source au programme
cible. Toutefois, les coûts asymptotiques sont les mêmes. Un examen
plus fouillé de \texttt{tms0} révèle que nous avons traduit
\(\cons{y}{t}\) par \texttt{this} au lieu de \texttt{tail.push(head)},
pour aller plus vite. De plus, nous appelons \texttt{cutr} sur
\texttt{tail} parce que nous savons déjà que c'est le paramètre que
nous allons utiliser pour discriminer lors de la traduction de
\fun{cutr/3}, définie à la \fig~\vref{fig:tms}:
\begin{equation*}
\begin{array}{@{}r@{\;}l@{\;}l@{}}
\fun{cutr}(s,\cons{y}{t},\cons{a,b}{u})
                       & \rightarrow & \fun{cutr}(\cons{y}{s},t,u);\\
\fun{cutr}(s,t,u)      & \rightarrow
                       & \fun{mrg}(\fun{tms}(s),\fun{tms}(t)).
\end{array}
\end{equation*}
Nous remarquons que deux piles sont filtrées par le premier motif,
donc un raffinement est nécessaire et nous devons choisir un paramètre
pour commencer. Un peu d'attention révèle que le troisième est le
meilleur candidat, parce que s'il ne contient pas au moins deux
éléments, il est tout simplement écarté. Nous devons introduire deux
fonctions auxilaires, \fun{cutr\(_0\)/3} et \fun{cutr\(_1\)/3}, et
nous assurer que le troisième paramètre de \fun{cutr\(_0\)/3} contient
au moins deux éléments et que le second paramètre de
\fun{cutr\(_1\)/3} en contient au moins un. Le résultat est montré à
la \fig~\ref{fig:cutr}.
\begin{figure}
\begin{equation*}
\boxed{
\begin{array}{@{}r@{\;}l@{\;}l@{}}
\fun{cutr}(s,t,\el)     & \rightarrow
                        & \fun{mrg}(\fun{tms}(s),\fun{tms}(t));\\
\fun{cutr}(s,t,\cons{a}{u})
                        & \rightarrow & \fun{cutr}_0(s,t,u).\\
\\
\fun{cutr}_0(s,t,\el)   & \rightarrow
                        & \fun{mrg}(\fun{tms}(s),\fun{tms}(t));\\
\fun{cutr}_0(s,t,\cons{b}{u})
                        & \rightarrow & \fun{cutr}_1(s,t,u).\\
\\
\fun{cutr}_1(s,\el,u)   & \rightarrow & \fun{tms}(s);\\
\fun{cutr}_1(s,\cons{y}{t},u)
                        & \rightarrow
                        & \fun{cutr}(\cons{y}{s},t,u).
\end{array}}
\end{equation*}
\caption{Découpage et interclassement descendant}
\label{fig:cutr}
\end{figure}
Notons que \(\fun{cutr}_1(s,\el,u) \rightarrow
\fun{mrg}(\fun{tms}(s),\fun{tms}(\el))\) a été simplifié en usant des théorèmes
\(\fun{tms}(\el) \equiv \el\) et \(\fun{mrg}(s,\el) \equiv
s\). La traduction est directe:\index{tri par
  interclassement!$\sim$ descendant!$\sim$ en \Java|(}
\begin{alltt}
// Stack.java
\public \abstractX \class Stack<Item
\hfill\extends Comparable<? \super Item>> \{
  ...
  \public    \abstractX Stack<Item> cutr(\final Stack<Item> s,
                                      \final Stack<Item> t);
  \protectedX \abstractX Stack<Item> cutr0(\final Stack<Item> s,
                                       \final Stack<Item> t);
  \protectedX \abstractX Stack<Item> cutr1(\final Stack<Item> s,
                                       \final Stack<Item> u);
\}
// EStack.java
\public \class EStack<Item \extends Comparable<? \super Item>>
       \extends Stack<Item> \{
  ...
  \public Stack<Item> cutr(\final Stack<Item> s,
                          \final Stack<Item> t) \{
    \return s.tms().mrg(t.tms()); \}

  \protectedX Stack<Item> cutr0(\final Stack<Item> s,
                              \final Stack<Item> t) \{
    \return s.tms().mrg(t.tms()); \}

  \protectedX Stack<Item> cutr1(\final Stack<Item> s,
                              \final Stack<Item> u) \{
    \return s.tms(); \}
\}
// NStack.java
\public \class NStack<Item \extends Comparable<? \super Item>>
       \extends Stack<Item> \{
  ...
  \public Stack<Item> cutr(\final Stack<Item> s,
                          \final Stack<Item> t) \{
    \return tail.cutr0(s,t); \}

  \protectedX Stack<Item> cutr0(\final Stack<Item> s,
                              \final Stack<Item> t) \{
    \return t.cutr1(s,tail); \}

  \protectedX Stack<Item> cutr1(\final Stack<Item> s,
                              \final Stack<Item> u) \{
    \return u.cutr(s.push(head),tail); \} \}
\end{alltt}
Remarquons qu'il est de la plus haute importance de garder à l'esprit
le paramètre dans la fonction d'origine dont la traduction sera la
base du sous-typage en \Java. Enfin, la classe \texttt{Main} pourrait
ressembler à
\begin{alltt}
\public \class Main \{\hfill// Main.java
  \public \static \void \main (String[] args) \{
    Stack<Integer> nil = \new EStack<Integer>();
    Stack<Integer> s = nil.push(5).push(2).push(7);\hfill// 7 2 5
    Stack<Integer> t = nil.push(4).push(1);\hfill// 1 4
    s.tms().print();\hfill// 2 5 7
    s.cat(t).tms().print(); \} \}\hfill// 1 2 4 5 7
\end{alltt}
\index{tri par interclassement!$\sim$ descendant!$\sim$ en \Java|)}
\index{Java@\Java!traduction en $\sim$|)}
