\section{Retournement}
\label{sec_reversal}

\paragraph{Involution}
\index{pile!retournement!involution}

Il est parfois nécessaire de concevoir un lemme pour faire aboutir une
preuve. Considérons la définition d'une fonction
\fun{rev\(_0\)/1}\label{def_rev0}\index{pile!retournement!définition}\index{rev0@\fun{rev\(_0\)/1}}
qui retourne une pile:
\begin{equation*}
\begin{array}{@{}r@{\;}l@{\;}lr@{\;}l@{\;}l@{}}
  \fun{cat}(\el,t)\index{cat@\fun{cat/2}}
& \xrightarrow{\smash{\alpha}} & t;
& \fun{rev}_0(\el)
& \xrightarrow{\smash{\gamma}} & \el;\\
  \fun{cat}(\cons{x}{s},t)
& \xrightarrow{\smash{\beta}} & \cons{x}{\fun{cat}(s,t)}.
& \fun{rev}_0(\cons{x}{s})
& \xrightarrow{\smash{\delta}} & \fun{cat}(\fun{rev}_0(s),[x]).
\end{array}
\end{equation*}
Une évaluation est donnée en exemple avec des arbres de syntaxe
abstraite à la \fig~\ref{fig_rev0_321}.
\begin{figure}[!t]
\centering
\includegraphics[bb=69 620 328 721]{rev0_321_0}
\includegraphics[bb=64 626 320 721]{rev0_321_1}
\includegraphics[bb=69 627 375 721]{rev0_321_2}
\includegraphics[bb=69 627 375 721]{rev0_321_3}
\caption{\(\fun{rev}_0([3,2,1]) \twoheadrightarrow [1,2,3]\)
\label{fig_rev0_321}\index{pile!retournement!exemple}}
\end{figure}
Soit \(\pred{Inv}{s}\)\index{Inv@\predName{Inv}|(} la propriété
\(\fun{rev}_0(\fun{rev}_0(s)) \equiv s\), c'est-à-dire que la fonction
\fun{rev\(_0\)/1}\index{rev0@\fun{rev\(_0\)/1}} est une
\emph{involution}\index{pile!retournement!involution}. Pour prouver
\(\forall s \in S.\pred{Inv}{s}\), le principe d'induction sur la
structure de~\(s\)\index{induction!exemple|(} exige que nous
établissions
\begin{itemize}

  \item la base \(\pred{Inv}{\el}\);

  \item le pas inductif \(\forall s \in S.\pred{Inv}{s} \Rightarrow
    \forall x \in T.\pred{Inv}{\cons{x}{s}}\).

\end{itemize}
La base est vite trouvée: \(\fun{rev}_0(\fun{rev}_0(\el))
\xrightarrow{\smash{\gamma}} \fun{rev}_0(\el)
\xrightarrow{\smash{\gamma}} \el\). L'hypothèse d'induction est
\(\pred{Inv}{s}\) et nous voulons prouver
\(\pred{Inv}{\cons{x}{s}}\)\index{Inv@\predName{Inv}|)}, pour
tout~\(x\).  Si nous débutons frontalement:
\(\fun{rev}_0(\fun{rev}_0(\cons{x}{s})) \xrightarrow{\smash{\delta}}
\fun{rev}_0(\fun{cat}(\fun{rev}_0(s),[x]))\), nous sommes coincés.
Mais le terme à réécrire implique à la fois \fun{rev$_0$/1} et
\fun{cat/2}, ce qui nous incite à concevoir un lemme où le schéma de
l'obstacle \(\fun{cat}(\fun{rev}_0(\dots),\dots)\) est présent et est
équivalent à un terme plus simple.%\par\vskip\baselineskip

Posons
\(\pred{CatRev}{s,t}\)\index{CatRev@\predName{CatRev}} \label{CatRev}
pour dire \(\fun{cat}(\fun{rev}_0(t),\fun{rev}_0(s)) \equiv
\fun{rev}_0(\fun{cat}(s,t))\)\index{cat@\fun{cat/2}}. De façon à
prouver cette propriété par induction sur la structure de~\(s\), nous
avons besoin d'établir, pour tout~\(t\),
\begin{itemize}

  \item la base \(\pred{CatRev}{\el,t}\);

  \item le pas inductif \(\forall s,t \in S.\pred{CatRev}{s,t}
    \Rightarrow \forall x \in T.\pred{CatRev}{\cons{x}{s},t}\).

\end{itemize}
La base est presque à portée de main:
\begin{equation*}
\fun{rev}_0(\fun{cat}(\el,t)) \xrightarrow{\smash{\alpha}}
\fun{rev}_0(t) \leftrightsquigarrow \fun{cat}(\fun{rev}_0(t),\el)
\xleftarrow{\smash{\gamma}}
\fun{cat}(\fun{rev}_0(t),\fun{rev}_0(\el)).
\end{equation*}
La part manquante est trouvée en montrant que
\((\leftrightsquigarrow)\)~est en
fait~\((\equiv)\).%\par\vskip\baselineskip

Soit \(\pred{CatNil}{s}\)\index{CatNil@\predName{CatNil}|(} la
propriété \(\fun{cat}(s,\el) \equiv s\)\index{cat@\fun{cat/2}}. Pour
la prouver par induction sur la structure de~\(s\), nous devons
prouver
\begin{itemize}

  \item la base \(\pred{CatNil}{\el}\);

  \item le pas inductif \(\forall s \in S.\pred{CatNil}{s}
    \Rightarrow \forall x \in T.\pred{CatNil}{\cons{x}{s}}\).

\end{itemize}
La base est facile, puisque \(\fun{cat}(\el,\el)
\xrightarrow{\smash{\alpha}} \el\)\index{cat@\fun{cat/2}}. Le pas
inductif, quant à lui, n'est pas compliqué non plus:
\(\fun{cat}(\cons{x}{s},\el) \xrightarrow{\smash{\beta}}
\cons{x}{\fun{cat}(s,\el)} \equiv \cons{x}{s}\), où l'équivalence est
l'hypothèse d'induction
\(\pred{CatNil}{s}\)\index{CatNil@\predName{CatNil}|)}.\hfill\(\Box\)

Remarquons qu'à la place nous aurions pu prouver \(\fun{cat}(s,\el)
\twoheadrightarrow s\), pour toute valeur~\(s\). La différence est que
\(\pred{CatNil}{s}\) est vraie même si~\(s\) n'est pas une valeur,
puisque tout ce que nous avons à faire est remplacer
\(\fun{cat}(\cons{x}{s},\el) \xrightarrow{\smash{\beta}}
\cons{x}{\fun{cat}(s,\el)}\) par \(\fun{cat}(\cons{x}{s},\el)
\Rra{\beta} \cons{x}{\fun{cat}(s,\el)}\). Ce principe sera avéré dans
toutes nos preuves: nous supposerons que toute variable dénote une
valeur et, sinon, nous changeons simplement
\((\xrightarrow{\smash{\beta}})\) en \((\Rra{\beta})\), pour toutes
les règles~\(\beta\) dans la preuve. Notre hypothèse améliore
peut-être la lisibilité.\par\vskip\baselineskip

\noindent En supposant
\(\pred{CatRev}{s,t}\)\index{CatRev@\predName{CatRev}}, nous devons
établir \(\forall x \in T.\pred{CatRev}{\cons{x}{s},t}\):
\index{cat@\fun{cat/2}|(} \index{rev0@\fun{rev\(_0\)/1}|(}
\index{CatAssoc@\predName{CatAssoc}}
\begin{equation*}
\begin{array}{@{}r@{\;}l@{\;}l@{\;\;}r@{}}
  \fun{cat}(\fun{rev}_0(t),\fun{rev}_0(\cons{x}{s}))
& \xrightarrow{\smash{\delta}}
& \fun{cat}(\fun{rev}_0(t),\fun{cat}(\fun{rev}_0(s),[x]))\\
& \equiv
& \fun{cat}(\fun{cat}(\fun{rev}_0(t),\fun{rev}_0(s)),[x])
& (\predName{CatAssoc})\\
& \equiv 
& \fun{cat}(\fun{rev}_0(\fun{cat}(s,t)),[x])
& (\pred{CatRev}{s,t})\\
& \xleftarrow{\smash{\delta}}
& \fun{rev}_0(\cons{x}{\fun{cat}(s,t)})\\
& \xleftarrow{\smash{\beta}}
& \fun{rev}_0(\fun{cat}(\cons{x}{s},t)).
& \multicolumn{1}{r@{}}{\Box}
\end{array}
\end{equation*}
Reprenons notre preuve de
\(\pred{Inv}{\cons{x}{s}}\):\index{Inv@\predName{Inv}}
\begin{equation*}
\begin{array}{@{}r@{\;}l@{\;}l@{\quad}r@{}}
  \fun{rev}_0(\fun{rev}_0(\cons{x}{s}))
& \xrightarrow{\smash{\delta}}
& \fun{rev}_0(\fun{cat}(\fun{rev}_0(s),[x]))\\
& \equiv
& \fun{cat}(\fun{rev}_0([x]),\fun{rev}_0(\fun{rev}_0(s)))
& (\pred{CatRev}{\fun{rev}_0(s),[x]})\\
& \equiv
& \fun{cat}(\fun{rev}_0([x]),s)
& (\pred{Inv}{s})\\
& \xrightarrow{\smash{\delta}}
& \fun{cat}(\fun{cat}(\fun{rev}_0(\el),[x]),s)\\
& \xrightarrow{\smash{\gamma}}
& \fun{cat}(\fun{cat}(\el,[x]),s)\\
& \xrightarrow{\smash{\alpha}}
& \fun{cat}([x],s)\\
& \xrightarrow{\smash{\beta}} 
& \cons{x}{\fun{cat}(\el,s)}\\
& \xrightarrow{\smash{\alpha}} &
\cons{x}{s}.
& \hfill\Box
\end{array}
\end{equation*}
\index{rev0@\fun{rev\(_0\)/1}|)}\index{cat@\fun{cat/2}|)}
\index{CatRev@\predName{CatRev}}

\paragraph{Équivalence}

\hspace*{-2pt} Il se pourrait que nous ayons deux définitions d'une
même fonction mais qui diffèrent en termes de complexité ou
d'efficacité. Par exemple, nous avons donné une définition intuitive
de \fun{rev\(_0\)/1}\index{rev0@\fun{rev\(_0\)/1}} où, à la
règle~\(\delta\), l'élément~\(x\), qui est au sommet de la pile
donnée, est placé au bas de la pile construite. Malheureusement, cette
définition est inefficace du point de vue des calculs, c'est-à-dire
qu'elle conduit à un grand nombre de réécritures par rapport à la
taille de l'entrée.

Supposons que nous ayons aussi une définition efficace pour le
retournement de piles, nommée \fun{rev/1}\index{rev@\fun{rev/1}}, qui
dépend d'une fonction auxiliaire
\fun{rcat/2}\index{rcat@\fun{rcat/2}|(}\index{pile!retournement!$\sim$
  et concaténation}\index{pile!retournement!$\sim$ efficace}
(anglais, \emph{reverse and catenate}):
\begin{equation}
\fun{rev}(s) \xrightarrow{\smash{\epsilon}} \fun{rcat}(s,\el).\;\,
\fun{rcat}(\el,t) \xrightarrow{\smash{\zeta}} t;\;\,
\fun{rcat}(\cons{x}{s},t) \xrightarrow{\smash{\eta}} 
                          \fun{rcat}(s,\cons{x}{t}).
\label{def_rev}
\end{equation}
Un paramètre additionnel introduit par \fun{rcat/2} accumule des
résultats partiels, et est donc appelé un
\emph{accumulateur}\index{langage fonctionnel!accumulateur}. Nous
pouvons le voir à l'{\oe}uvre à la \fig~\vref{fig_rev_321}.
\begin{figure}
\begin{equation*}
\boxed{%
\begin{array}{r@{\;}l@{\;}l}
\fun{rev}([3,2,1])
& \xrightarrow{\smash{\epsilon}} & \fun{rcat}([3,2,1],\el)\\
& \xrightarrow{\smash{\eta}}     & \fun{rcat}([2,1],[3])\\
& \xrightarrow{\smash{\eta}}     & \fun{rcat}([1],[2,3])\\
& \xrightarrow{\smash{\eta}}     & \fun{rcat}(\el,[1,2,3])\\
& \xrightarrow{\smash{\zeta}}    & [1,2,3].
\end{array}}
\end{equation*}
\caption{\(\fun{rev}([3,2,1]) \twoheadrightarrow [1,2,3]\)
\label{fig_rev_321}\index{pile!retournement!exemple}}
\end{figure}
\index{rcat@\fun{rcat/2}|)}

Prouvons \label{EqRev} \(\pred{EqRev}{s} \colon \fun{rev}_0(s) \equiv
\fun{rev}(s)\)\index{rev@\fun{rev/1}|(}\index{EqRev@\predName{EqRev}|(}
par induction
structurelle\index{pile!retournement!équivalence}\index{équivalence!preuve
  d'$\sim$} sur~\(s\), c'est-à-dire
\begin{itemize}

  \item la base \(\pred{EqRev}{\el}\),

  \item le pas inductif \(\forall s \in S.\pred{EqRev}{s} \Rightarrow
    \forall x \in T.\pred{EqRev}{\cons{x}{s}}\).

\end{itemize}
La base est aisée: \(\fun{rev}_0(\el) \xrightarrow{\smash{\gamma}} \el
\xleftarrow{\smash{\zeta}} \fun{rcat}(\el,\el)
\xleftarrow{\smash{\epsilon}}
\fun{rev}(\el)\)\index{rev0@\fun{rev\(_0\)/1}}. Pour le pas inductif,
réécrivons \(\fun{rev}_0(\cons{x}{s})\) et \(\fun{rev}(\cons{x}{s})\)
de telle sorte qu'ils convergent:
\begin{equation*}
\begin{array}{@{}r@{\;}l@{\;}lr@{}}
  \fun{rev}_0(\cons{x}{s})
& \xrightarrow{\smash{\delta}}
& \fun{cat}(\fun{rev}_0(s),[x])\\
& \equiv
& \fun{cat}(\fun{rev}(s),[x])
& (\pred{EqRev}{s})\\
& \leftrightsquigarrow
& \fun{rcat}(s,[x])
& (\text{à déterminer})\\
& \xleftarrow{\smash{\eta}}
& \fun{rcat}(\cons{x}{s},\el)\\
& \xleftarrow{\smash{\epsilon}}
& \fun{rev}(\cons{x}{s}).
\end{array}
\end{equation*}
La partie manquante est trouvée en montrant que
\((\leftrightsquigarrow)\) est~\((\equiv)\) de la façon
suivante.\index{rev@\fun{rev/1}|)}

Soit
\(\pred{RevCat}{s,t}\)\index{RevCat@\predName{RevCat}}\label{RevCat}
la propriété \(\fun{rcat}(s,t) \equiv
\fun{cat}(\fun{rev}(s),t)\).\index{rcat@\fun{rcat/2}}\index{cat@\fun{cat/2}|(}\index{rev@\fun{rev/1}}
L'induction sur la structure de~\(s\) demande alors les preuves
\begin{itemize}

  \item de la base \(\forall t \in S.\pred{RevCat}{\el,t}\),

  \item du pas inductif \(\forall s,t \in S.\pred{RevCat}{s,t}
    \Rightarrow \forall x \in T.\pred{RevCat}{\cons{x}{s},t}\).

\end{itemize}
D'abord, \(\fun{rcat}(\el,t) \xrightarrow{\smash{\zeta}} t
\xleftarrow{\smash{\alpha}} \fun{cat}(\el,t)
\xleftarrow{\smash{\zeta}} \fun{cat}(\fun{rcat}(\el,\el),t)
\xleftarrow{\smash{\epsilon}} \fun{cat}(\fun{rev}(\el),t)\).
Supposons \(\forall s,t \in S.\pred{RevCat}{s,t}\) et prouvons
\(\forall x \in T.\pred{RevCat}{\cons{x}{s},t}\):
\begin{equation*}
\begin{array}{@{}r@{\;}l@{\;}l@{}r@{}}
  \fun{rcat}(\cons{x}{s},t)
& \xrightarrow{\smash{\eta}}
& \fun{rcat}(s,\cons{x}{t})\\
& \equiv
& \fun{cat}(\fun{rev}(s),\cons{x}{t})
& (\pred{RevCat}{s,\cons{x}{t}})\\
& \xleftarrow{\smash{\alpha}}
& \fun{cat}(\fun{rev}(s),\cons{x}{\fun{cat}(\el,t)})\\
& \xleftarrow{\smash{\beta}}
& \fun{cat}(\fun{rev}(s),\fun{cat}([x],t))\\
& \equiv
& \fun{cat}(\fun{cat}(\fun{rev}(s),[x]),t)
& (\pred{CatAssoc}{\fun{rev}(s),[x],t})\\
& \equiv
& \fun{cat}(\fun{rcat}(s,[x]),t)
& (\pred{RevCat}{s,[x]})\\
& \xleftarrow{\smash{\eta}}
& \fun{cat}(\fun{rcat}(\cons{x}{s},\el),t)\\
& \xleftarrow{\smash{\epsilon}}
& \fun{cat}(\fun{rev}(\cons{x}{s}),t).
& \multicolumn{1}{r@{}}{\Box}
\end{array}
\end{equation*}
Finalement, nous avons prouvé \(\forall s.\pred{EqRev}{s}\),
c'est-à-dire \(\fun{rev/1} =
\fun{rev\(_0\)/1}\)\index{EqRev@\predName{EqRev}|)}\index{rev@\fun{rev/1}}\index{rev0@\fun{rev\(_0\)/1}}\index{cat@\fun{cat/2}|)}\index{induction!exemple|)}.\hfill\(\Box\)

\paragraph{Coût}
\label{cost_rev0}
\index{pile!retournement!coût}

La définition de \fun{rev\(_0\)/1}\index{rev0@\fun{rev\(_0\)/1}}
mène directement aux récurrences\index{rev0@$\C{\fun{rev}_0}{n}$}
\begin{equation*}
\C{\fun{rev}_0}{0} = 1,\qquad \C{\fun{rev}_0}{k+1} = 1 +
\C{\fun{rev}_0}{k} + \C{\fun{cat}}{k} = \C{\fun{rev}_0}{k} + k + 2,
\end{equation*}
parce que la longueur de \(\fun{rev}_0(s)\) est~\(k\) si la longueur
de~\(s\) est~\(k\), et nous savons déjà que \(\C{\fun{cat}}{k} = k +
1\)\index{cat@$\C{\fun{cat}}{n}$} (page~\pageref{cost_cat}). Nous
avons
\begin{equation*}
\sum_{k=0}^{n-1}(\C{\fun{rev}_0}{k+1} - \C{\fun{rev}_0}{k}) =
\C{\fun{rev}_0}{n} - \C{\fun{rev}_0}{0} = 
\sum_{k=0}^{n-1}(k+2) = 2n + \sum_{k=0}^{n-1}{k}.
\end{equation*}
La somme restante est un classique de l'algèbre:
\begin{equation*}
2 \cdot \sum_{k=0}^{n-1}{k} = 
\sum_{k=0}^{n-1}{k} + \sum_{k=0}^{n-1}{k} = 
\sum_{k=0}^{n-1}{k} + \sum_{k=0}^{n-1}(n-k-1) = n(n-1).
\end{equation*}
Par conséquent,
\begin{equation}
\sum_{k=0}^{n-1}{k} = \frac{n(n-1)}{2},\label{eq_sum_k}
\end{equation}
et nous pouvons finalement conclure
\begin{equation*}
\C{\fun{rev}_0}{n} = \frac{1}{2}n^2 + \frac{3}{2}n + 1 \sim \frac{1}{2}n^2.
\end{equation*}
Une autre manière d'atteindre ce résultat est d'induire une
\emph{trace d'évaluation}\index{langage
  fonctionnel!évaluation!trace}. Une trace est une composition de
règles de réécriture, notée selon l'usage conventionnel de la
multiplication. D'après la \fig~\vref{fig_rev0_321}, nous déduisons la
trace \(\T{\fun{rev}_0}{n}\)\index{trace_rev0@$\T{\fun{rev}_0}{n}$} de
l'évaluation de \(\fun{rev}_0(s)\)\index{rev0@\fun{rev\(_0\)/1}}, où
\(n\)~est la longueur de~\(s\):
\begin{equation*}
\T{\fun{rev}_0}{n} :=
\delta^n\gamma\alpha(\beta\alpha)\dots(\beta^{n-1}\alpha) =
\delta^n\gamma \prod_{k=0}^{n-1}{\beta^k\alpha}.
\end{equation*}
Si nous notons
\(\len{\T{\fun{rev}_0}{n}}\)\index{trace_rev0@$\T{\fun{rev}_0}{n}$} la
longueur de~\(\T{\fun{rev}_0}{n}\), c'est-à-dire, le nombre
d'applications de règles quelle contient, nous nous attendons à
obtenir les équations \(\len{x} = 1\), pour une règle~\(x\), et
\(\len{x \cdot y} = \len{x} + \len{y}\), pour des règles
\(x\)~et~\(y\). Par définition du coût, nous
avons\index{rev0@$\C{\fun{rev}_0}{n}$}
\begin{align*}
\C{\fun{rev}_0}{n}
  &:= \len{\T{\fun{rev}_0}{n}}
    = \left\lvert\delta^n\gamma 
        \prod_{k=0}^{n-1}{\beta^k\alpha}\right\lvert
    = \len{\delta^n\gamma} + \sum_{k=0}^{n-1}\len{\beta^k\alpha}\\
   &= (n+1) + \sum_{k=0}^{n-1}(k+1) = (n+1) + \sum_{k=1}^{n+1}k
    = \frac{1}{2}n^2 + \frac{3}{2}n + 1.
\end{align*}

La raison de cette inefficacité se trouve dans le fait que la
règle~\(\delta\) produit une série d'appels à
\fun{cat/2}\index{cat@\fun{cat/2}} qui se conforment au schéma
suivant:
\begin{equation}
\fun{rev}_0(s) \twoheadrightarrow \fun{cat}(\fun{cat}(\dots
\fun{cat}(\el, [x_n]), \dots, [x_2]), [x_1]),
\label{eq_rev0}
\end{equation}
où \(s = [x_1, x_2, \dots, x_n]\). Le coût de tous ces appels à
\fun{cat/2}\index{cat@\fun{cat/2}} est donc
\begin{equation*}
1 + 2 + \dots + (n-1) = \tfrac{1}{2}n(n-1) \sim
\tfrac{1}{2}n^2,
\end{equation*}
parce que le coût de \(\fun{cat}(s,t)\)\index{cat@\fun{cat/2}} est \(1
+ \fun{len}(s)\), où
\begin{equation}
\fun{len}(\el) \xrightarrow{\smash{a}} 0;
\qquad
\fun{len}(\cons{x}{s}) \xrightarrow{\smash{b}} 1 + \fun{len}(s).
\label{eq_len}
\end{equation}
Le problème n'est pas le recours à \fun{cat/2}, mais le fait que les
appels sont imbriqués de la façon la plus défavorable. En effet, si
l'associativité\index{pile!concaténation!associativité} de
\fun{cat/2}\index{cat@\fun{cat/2}}, prouvée à la
\vpageref{proof_assoc_cat}, dit bien \(\fun{cat}(\fun{cat}(s,t),u)
\equiv \fun{cat}(s,\fun{cat}(t,u))\), les coûts des deux membres
diffèrent. Soit \(\Call{f(x)}\) le coût de l'appel \(f(x)\). Suit
alors \(\Call{\fun{cat}(\fun{cat}(s,t),u)} = (\fun{len}(s) + 1) +
(\fun{len}(\fun{cat}(s,t)) + 1) = (\fun{len}(s) + 1) + (\fun{len}(s) +
\fun{len}(t) + 1) = 2 \cdot \fun{len}(s) + \fun{len}(t) + 2\), en
usant de \(\pred{LenCat}{s,t} \colon \fun{len}(\fun{cat}(s,t)) =
\fun{len}(s) + \fun{len}(t)\); mais
\(\Call{\fun{cat}(s,\fun{cat}(t,u))} = (\fun{len}(t) + 1) +
(\fun{len}(s) + 1) = \fun{len}(s) + \fun{len}(t) + 2\), c'est-à-dire
\begin{equation}
\Call{\fun{cat}(\fun{cat}(s,t),u)} = \fun{len}(s) +
\Call{\fun{cat}(s,\fun{cat}(t,u))}.
\label{ineq_cat_assoc}
\end{equation}
Les éléments de~\(s\) sont donc traversés deux fois, alors qu'une
seule visite suffirait.

Encore une autre manière de déterminer le coût de \fun{rev\(_0\)/1}
consiste à deviner d'abord qu'il est
\emph{quadratique}\index{coût!$\sim$ quadratique}, c'est-à-dire,
\(\C{\fun{rev}_0}{n} = an^2 + bn +
c\)\index{rev0@$\C{\fun{rev}_0}{n}$}, où \(a\), \(b\) et~\(c\) sont
des inconnues. Puisqu'il y a trois coefficients, nous n'avons besoin
que de trois valeurs de~\(\C{\fun{rev}_0}{n}\) pour les déterminer,
par exemple, en produisant quelques traces, nous trouvons
\(\C{\fun{rev}_0}{0} = 1\), \(\C{\fun{rev}_0}{1} = 3\) et
\(\C{\fun{rev}_0}{2} = 6\), donc nous pouvons résoudre le système
d'équations linéaires:
\begin{equation*}
\C{\fun{rev}_0}{0} = c = 1,\quad
\C{\fun{rev}_0}{1} = a + b + c = 3,\quad
\C{\fun{rev}_0}{2} = a \cdot 2^2 + b \cdot 2 + c = 6.
\end{equation*}
Nous tirons \(a = \myfrac1/2\), \(b = \myfrac3/2\) et \(c = 1\),
c'est-à-dire \(\C{\fun{rev}_0}{n} = (n^2 + 3n +
2)/2\)\index{rev0@$\C{\fun{rev}_0}{n}$}. Étant donné que l'hypothèse
du comportement quadratique pourrait être fausse, il est alors crucial
d'essayer d'autre valeurs dans la formule nouvellement obtenue, par
exemple \(\C{\fun{rev}_0}{4} = (4+1)(4+2)/2 = 15\), et ensuite
comparer le résultat avec le coût de
\(\fun{rev}_0([1,2,3,4])\)\index{rev0@$\C{\fun{rev}_0}{n}$}, par
exemple.  Dans ce cas, le contenu de la pile n'est pas significatif,
seule sa longueur importe. Après avoir trouvé la formule pour le coût
en utilisant la méthode empirique ci-dessus, il est nécessaire de
l'établir pour toutes les valeurs de~\(n\). Puisque les équations
originelles sont récurrentes, la méthode de choix est l'induction. Soit
\(\pred{Quad}{n}\)\index{Quad@\predName{Quad}} la propriété
\(\C{\fun{rev}_0}{n} = (n^2 + 3n + 2)/2\). Nous l'avons déjà vérifiée
pour de petites valeurs \(n = 0, 1, 2\). Supposons alors qu'elle est
valide pour une valeur de la variable~\(n\) (hypothèse d'induction) et
prouvons \(\pred{Quad}{n+1}\) (pas inductif). Nous savons déjà que
\(\C{\fun{rev}_0}{n+1} = \C{\fun{rev}_0}{n} + n +
2\)\index{rev0@$\C{\fun{rev}_0}{n}$}. L'hypothèse d'induction implique
alors
\begin{equation*}
\C{\fun{rev}_0}{n+1} = (n^2 + 3n + 2)/2 + n + 2
                     = ((n+1)^2 + 3(n+1) + 2)/2,
\end{equation*}
ce qui n'est autre que \(\pred{Quad}{n+1}\)\index{Quad@\predName{Quad}}. 
Conséquemment, le principe d'induction dit que le coût que nous avons déterminé expérimentalement est toujours correct.

Pour déduire le coût de \fun{rev/1}\index{rev@\fun{rev/1}}, il suffit
de remarquer que le premier argument de
\fun{rcat/2}\index{rcat@\fun{rcat/2}} décroît strictement à chaque
réécriture, donc toute trace d'évaluation a la forme
\(\epsilon\eta^n\zeta\), d'où \(\C{\fun{rev}}{n} =
\len{\epsilon\eta^n\zeta } = n + 2\). Le coût est
\emph{linéaire}\index{coût!$\sim$ linéaire}, donc
\fun{rev/1}\index{rev@\fun{rev/1}} doit être utilisée à la place de
\fun{rev\(_0\)/1}\index{rev0@\fun{rev\(_0\)/1}} en toutes
circonstances.

\paragraph{Exercises}

\begin{enumerate}

  \item Prouvez \(\pred{LenRev}{s} \colon \fun{len}(\fun{rev}_0(s))
    \equiv \fun{len}(s)\).

  \item Prouvez \(\pred{LenCat}{s,t} \colon \fun{len}(\fun{cat}(s,t)) \equiv
  \fun{len}(s) + \fun{len}(t)\).

  \item Qu'est-ce qui ne va pas dans la preuve de l'involution de
  \fun{rev/1}\index{rev@\fun{rev/1}} à la section~3.4.9 du livre de
  \cite{CousineauMauny_1998}?

\end{enumerate}
