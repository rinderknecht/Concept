\section{Persistance}
\label{sec:persistence}

La \emph{persistance}\index{persistance} est une propriété
caractéristique des langages purement fonctionnels. Elle signifie
simplement que toutes les valeurs sont constantes. Les fonctions
mettent à jour une structure des données en en créant une nouvelle
version, au lieu de la modifier sur place et effaçant ainsi son
histoire. Nous avons vu à la section~\vref{sec:functional} que les
sous-arbres communs aux deux membres d'une même règle sont partagés.
Un tel partage est rendu correct grâce à la persistance: il n'y a pas
moyen de distinguer logiquement la copie d'un sous-arbre et
l'original.

% Wrapping figure better declared before a paragraph
%
\setlength{\intextsep}{0pt}
\begin{wrapfigure}[6]{r}[0pt]{0pt}
% [6] vertical lines
% {r} mandatory right placement (better because of a list)
\centering
\includegraphics[bb=71 672 203 721]{red}
\caption{Réduction}
\label{fig:red}
\end{wrapfigure}

\paragraph{Partage maximal}
\index{partage}

Une occasion évidente de partage est l'occurrence d'une variable dans
les deux membres d'une même règle, comme on peut le voir à la
\fig~\vref{fig:cat_dag} par exemple. Mais ceci ne conduit pas
nécessairement à un partage maximal, comme la définition de
\fun{red/1}\index{red@\fun{red/1}} (\emph{réduire}) à la
\fig~\vref{fig:red} le montre. Cette fonction recopie une pile sans
ses éléments répétés consécutivement. Par exemple,
\(\fun{red}([4,1,2,2,2,1,1]) \twoheadrightarrow [4,1,2,1]\). Un graphe
orienté sans circuit représentant la deuxième règle est montré à la
\fig~\vref{fig:red_dag1}.
\begin{figure}[b]
\centering
\subfloat[Partage de variables\label{fig:red_dag1}]{%
  \includegraphics[bb=65 645 163 723]{red_dag1}
}
\qquad\qquad
\subfloat[Partage maximal\label{fig:red_dag2}]{%
  \includegraphics[bb=60 645 163 723]{red_dag2}
}
\caption{Graphe orienté sans circuits de la seconde règle de \fun{red/1}}
\end{figure}
Dans cette figure, le partage est fondé sur des occurrences communes
de variables, mais nous pouvons constater que \(\cons{x}{s}\) n'est
pas complètement partagé. Considérons la même règle à la
\fig~\vref{fig:red_dag2} avec un partage maximal, où un sous-arbre
complet est partagé.

%\setlength{\intextsep}{12pt}

Dans toute discussion sur la gestion de la mémoire, \emph{nous
  supposerons que le partage est maximal pour chaque règle}, donc, par
exemple, \fig~\vref{fig:red_dag2} serait le défaut. Mais cette
propriété n'est pas suffisante pour assurer que le partage est maximal
entre les arguments d'un appel de fonction et sa valeur. Par exemple,
\begin{equation*}
\fun{cp}(\el) \rightarrow \el;\qquad
\fun{cp}(\cons{x}{s}) \rightarrow \cons{x}{\fun{cp}(s)}.
\end{equation*}
fabrique une copie de son argument, mais la valeur de \(\fun{cp}(s)\)
ne partage que ses éléments avec~\(s\), bien que \(\fun{cp}(s) \equiv
s\).

\paragraph{Gestion de versions}

Une idée simple pour réaliser des structures de données qui permettent
de rebrousser chemin (en anglais, \emph{backtracking}), consiste à
conserver toutes les versions\index{persistance!$\sim$ des
  versions}. Une pile peut être utilisée à cet effet, appelée ici
\emph{histoire}\index{persistance!histoire}, et rebrousser chemin se
réduit alors à une recherche linéaire\index{recherche linéaire} dans
l'histoire. Par exemple, nous pourrions vouloir conserver une suite de
piles, chacune étant obtenue de la précédente par un empilage ou un
dépilage, comme \([[4,2,1],[2,1],[3,2,1],[2,1],[1],\el]\), où la pile
initiale était vide; puis \(1\)~a été empilé, suivi de~\(2\) et~\(3\);
ensuite \(3\)~a été dépilé et \(4\)~empilé. De cette façon, la
\emph{dernière version} est le sommet de l'histoire, comme \([4,2,1]\)
dans notre exemple. Par ailleurs, nous souhaitons que deux versions
successives partagent le plus de structure possible. (Nous employons
le terme «~version~» plutôt qu'«~état~» parce que dernier se réfère à
des valeurs qui ne sont pas persistantes.) Ces exigences sont au
c{\oe}ur des logiciels dits de \emph{gestion de versions}, utilisés
par les programmeurs pour conserver une trace de l'évolution de leur
programmes.

Nous continuerons avec notre exemple de l'évolution d'une pile, tout
en gardant à l'esprit que la technique décrite ci-après est applicable
à toute structure de données. Nous devons écrire deux fonctions,
\fun{push/2}\index{push@\fun{push/2}} (différente de celle définie à
la section~\ref{sec:cutting}) et \fun{pop/1}\index{pop@\fun{pop/1}},
qui, au lieu de traiter une pile, traitent une histoire de
piles. Soient les définitions suivantes:
\begin{equation}
\begin{array}{@{}r@{\;}c@{\;}l@{\quad}r@{\;}c@{\;}l@{}}
\fun{push}(x,\el) & \rightarrow & [[x],\el];
& \fun{pop}(\cons{\cons{x}{s}}{h}) & \rightarrow &
                                     \cons{s,\cons{x}{s}}{h}.\\
\fun{push}(x,\cons{s}{h}) & \rightarrow &
\cons{\cons{x}{s},s}{h}. &
\fun{top}(\cons{\cons{x}{s}}{h}) & \rightarrow & x.
\end{array}
\label{eq:push_pop_persistence}
\end{equation}
Les graphes orientés sans circuits (GOSC) associés sont montrés à la
\fig~\vref{fig:push_pop_dag}.
\begin{figure}[t]
\centering
\includegraphics[bb=70 639 294 723]{push_pop_dag}
\caption{GOSC de \fun{push/2} et \fun{pop/1} avec partage maximal}
\label{fig:push_pop_dag}
\end{figure}
L'histoire \([[4,2,1],[2,1],[3,2,1],[2,1],[1],\el]\) est montrée
à la \fig~\ref{fig:history}
\begin{figure}
\centering
\includegraphics[bb=70 575 247 723]{history}
\caption{L'histoire \([[4,2,1],[2,1],[3,2,1],[2,1],[1],\el]\)}
\label{fig:history}
\end{figure}
comme un graphe orienté sans circuits aussi. Il est le résultat de
l'évaluation de
\begin{equation}
\fun{push}(4, \fun{pop}(\fun{push}(3, \fun{push}(2,
\fun{push}(1, \el))))).
\label{ex_history}
\end{equation}

Soit \(\fun{ver}(k,h)\) \index{ver@\fun{ver/2}} dont valeur est la
\(k^\text{e}\) version antérieure dans l'histoire~\(h\), de telle
sorte que \(\fun{ver}(0,h)\) soit la dernière version. Comme on s'y
attendrait, \fun{ver/2} n'est rien d'autre qu'une recherche linéaire
avec un compte à rebours:
\begin{equation*}
%\abovedisplayskip=3pt
%\belowdisplayskip=3pt
\fun{ver}(0,\cons{s}{h}) \rightarrow s;\qquad
\fun{ver}(k,\cons{s}{h}) \rightarrow \fun{ver}(k-1,h).
\end{equation*}
Notre codage de l'histoire permet à la dernière version d'être
aisément copiée avec modification, mais pas les plus anciennes. Quand
toutes les versions d'une structure de donnée sont ainsi modifiables,
on parle de \emph{persistance complète}\index{persistance!$\sim$
  complète}; si seule la dernière version est modifiable, on parle de
\emph{persistance partielle}
\citep{MehlhornTsakalidis_1990}.\index{persistance!$\sim$ partielle}

\paragraph{Revisiter les mises à jour}
\index{persistance!rebroussement}

Dans le but de parvenir à la persistance complète, nous devrions
conserver une histoire des mises à jour\index{persistance!$\sim$ des
  mises à jour}, \(\ufun{push}(x)\)\index{push@\fun{push/1}} et
\(\ufun{pop}()\)\index{pop@\fun{pop/0}}, au lieu de versions
successives partagées le plus possible avec leur successeur. Dans ce
qui suite, le soulignement évite la confusion avec les fonctions
\fun{push/2}\index{push@\fun{push/2}} et
\fun{pop/1}\index{pop@\fun{pop/1}}. Au lieu de
l'équation~\eqref{ex_history}, on a
\begin{equation}
[\ufun{push}(4), \ufun{pop}(), \ufun{push}(3), \ufun{push}(2),
\ufun{push}(1)].
\label{ex_history1}
\end{equation}
Toutes les suites de \(\ufun{push}(x)\) et \(\ufun{pop}()\) ne sont
pas valides, comme, par exemple, \([\ufun{pop}(), \ufun{pop}(),
  \ufun{push}(x)]\) et \([\ufun{pop}()]\). Pour caractériser les
histoires valides, examinons une représentation graphique des mises à
jour à la \fig~\vref{fig:push_pop1}.
\begin{figure}
\centering
\subfloat[$\protect\fun{push}(x)$]{%
  \includegraphics[bb=71 662 130 715,scale=0.8]{push}}
\qquad
\subfloat[$\protect\fun{pop}()$]{%
  \includegraphics[bb=71 662 130 715,scale=0.8]{pop}}
\caption{Mises à jour de piles}
\label{fig:push_pop1}
\end{figure}
Ceci est le même modèle que nous avons employé à la
section~\ref{sec:queueing} où nous avons étudié les files d'attente
fonctionnelles (voir en particulier la \fig~\vref{fig:enq_deq}), sauf
que nous choisissons ici une orientation vers la gauche de manière à
refléter la notation des piles, dont les sommets sont écrits à
gauche. Considérons par exemple l'histoire à la
\fig~\vref{fig:history1}.
\begin{figure}[b]
\centering
\includegraphics[bb=69 577 300 726,scale=0.92]{history1}
\caption{$[\ufun{pop}(), \ufun{pop}(), \ufun{push}(4),\ufun{push}(3),
      \ufun{pop}(), \ufun{push}(2), \ufun{push}(1)]$}
\label{fig:history1}
\end{figure}
Il est clair qu'\emph{une histoire valide est une ligne qui ne coupe
jamais l'axe des abscisses.}

Programmer \fun{top/1}\index{top@\fun{top/1}} avec une histoire de
mises à jour est plus difficile parce que nous devons identifier
l'élément au sommet de la dernière version sans la construire. L'idée
consiste à revisiter la ligne historique et déterminer le dernier
empilage qui a conduit à une version dont la longueur égale celle de
la dernière version. À la \fig~\ref{fig:history1}, la dernière version
est~(\(\bullet\)). Si nous tirons une ligne horizontale à partir de ce
point vers la droite, le premier empilement aboutissant sur la ligne
est \(\ufun{push}(1)\), donc le sommet de la dernière version
est~\(1\).

Cette expérience de pensée est illustrée à la
\fig~\vref{fig:topmost}.
\begin{figure}
\centering
\includegraphics[bb=69 577 300 726,scale=0.92]{topmost}
\caption{Trouver le sommet de la dernière version}
\label{fig:topmost}
\end{figure}
Remarquons que nous n'avons pas besoin de déterminer la longueur de la
dernière version: la \emph{différence} de longueur avec la version courante, alors que nous reculons, est suffisante. Soient
\fun{top\(_0\)/1}\index{top0@\fun{top\(_0\)/1}} et
\fun{pop\(_0\)/1}\index{pop0@\fun{pop\(_0\)/1}} les équivalents de
\fun{top/1} et \fun{pop/1}, opérant sur des mises à jour au lieu de versions. Leur définition se trouve à la \fig~\vref{fig:pop_top}.
Une fonction auxiliaire
\fun{top\(_0\)/2}\index{top0@\fun{top\(_0\)/2}} conserve la différence
entre les longueurs de la dernière version et la version
courante. Nous avons trouvé l'élément quand la différence est nulle et
la mise à jour courante est un empilement.

\begin{center}
\begin{figure}[h]
\begin{equation*}
\boxed{%
\begin{array}{@{}r@{\;}l@{\;}lr@{\;}l@{\;}l@{}}
\fun{pop}_0(h)   & \rightarrow          & \cons{\ufun{pop}()}{h}. &
\fun{top}_0(0,\cons{\ufun{push}(x)}{h}) & \rightarrow             & x;\\
\fun{top}_0(h)   & \rightarrow          & \fun{top}_0(0,h).       &
\fun{top}_0(k,\cons{\ufun{push}(x)}{h}) & \rightarrow             & \fun{top}_0(k-1,h);\\
                 & & &
\fun{top}_0(k,\cons{\ufun{pop}()}{h})   & \rightarrow            & \fun{top}_0(k+1,h).
\end{array}}
\end{equation*}
\caption{Le sommet et le reste d'une histoire de mises à jour}
\label{fig:pop_top}
\end{figure}
\end{center}

Comme précédemment, nous voulons que l'appel
\(\fun{ver}_0(k,h)\)\index{ver0@\fun{ver\(_0\)/2}} construise la
\(k^\text{e}\) version précédente dans~\(h\). Ici, nous devons reculer
de \(k\)~mises à jour dans le passé,
\begin{equation*}
\fun{ver}_0(0,h)           \rightarrow \fun{lst}_0(h);\qquad
\fun{ver}_0(k,\cons{s}{h}) \rightarrow \fun{ver}_0(k-1,h).
\end{equation*}
et construire la dernière version à partir du reste de l'histoire avec
\fun{lst\(_0\)/1}\index{lst0@\fun{lst\(_0\)/1}}:
\begin{mathpar}
\inferrule*{}{\fun{lst}_0(\el) \rightarrow \el};
\quad
\inferrule*{}{\fun{lst}_0(\cons{\ufun{push}(x)}{h})
              \rightarrow \cons{x}{\fun{lst}_0(h)}};
\quad
\inferrule
  {\fun{lst}_0(h)                      \twoheadrightarrow \cons{x}{s}}
  {\fun{lst}_0(\cons{\ufun{pop}()}{h}) \twoheadrightarrow s}.
\end{mathpar}
Soit \(\C{\fun{ver}_0}{k,n}\) le coût de l'appel \(\fun{ver}_0(k,h)\)
et soit \(\C{\fun{lst}_0}{n}\) le coût de~\(\fun{lst}_0(h)\), où
\(n\)~est la longueur de~\(h\):
\begin{equation*}
\C{\fun{lst}_0}{i} = i + 1,\qquad
\C{\fun{ver}_0}{k,n} = (k+1) + \C{\fun{lst}_0}{n-k} = n + 2.
\end{equation*}

Quelle est la quantité totale de mémoire
allouée?\index{partage}\index{mémoire|see{partage}} Plus précisément,
nous souhaitons connaître le nombre d'empilements effectués. La seule
règle de \fun{lst\(_0\)/1} qui use d'un empilement dans son membre
droit est la seconde, donc le nombre de n{\oe}uds d'empilage est le
nombre d'empilages. Mais ceci est un gâchis dans certains cas, par
exemple, quand la version construite est vide, comme avec l'histoire
\([\ufun{pop}(),\ufun{push}(6)]\). La méthode optimale est d'allouer
exactement autant que la valeur finale a besoin.

% Wrapping figure better declared before a paragraph
%
\setlength{\intextsep}{0pt}
\begin{wrapfigure}[]{r}[0pt]{0pt}
% {r} mandatory right placement
\centering
\includegraphics[bb=71 644 257 721]{lst1}
\caption{Dernière version}
\label{fig:lst1}
\end{wrapfigure}
\hspace*{-1pt}Nous pouvons réaliser cette optimalité de la mémoire
avec \fun{lst\(_1\)/1} définie à la \fig~\ref{fig:lst1} en conservant
des caractéristiques à la fois de
\fun{top\(_0\)/1}\index{top0@\fun{top\(_0\)/1}} et de
\fun{lst\(_0\)/1}. Alors\index{lst1@$\C{\fun{lst}_1}{n}$}
\(\C{\fun{lst}_1}{n} = \C{\fun{lst}_0}{n} = n + 1\), et le nombre de
n{\oe}uds d'empilage\index{partage|(} créés est alors la longueur de
la dernière version. Nous avons là encore un exemple de coût qui
dépend de la taille du résultat, comme pour
\fun{flat\(_0\)/1}\index{flat0@\fun{flat$_0$/1}} à la
section~\vref{sec:flattening}.

Une amélioration est encore possible si la ligne historique atteint
l'axe des abscisses, parce qu'il n'y a alors aucune raison de visiter
les mises à jour \emph{antérieures} à un dépilage résultant en une
version vide; comme par exemple à la \fig~\vref{fig:back2zero},
\begin{figure}[b]
\centering
\includegraphics[bb=69 606 300 726,scale=0.9]{back2zero}
\caption{Dernière version $[3]$ trouvée en quatre pas}
\label{fig:back2zero}
\end{figure}
il est inutile d'aller au-delà de
\(\ufun{push}(3)\)\index{push@\fun{push/1}} pour déterminer que la
dernière version est~\([3]\). Mais, pour détecter si la ligne
historique rencontre l'axe des abscisses, nous avons besoin
d'augmenter l'histoire~\(h\) avec la longueur~\(n\) de la dernière
version, c'est-à-dire, de travailler avec \(\pair{n}{h}\), et modifier
\fun{push/2}\index{push@\fun{push/2}}
et~\fun{pop/1}\index{pop@\fun{pop/1}} en conséquence:
\begin{equation*}
\begin{array}{@{}r@{\;}l@{\;}l@{}}
\fun{push}_2(x,\pair{n}{h}) & \rightarrow &
\pair{n\!+\!1}{\cons{\ufun{push}(x)}{h}}.\\
\fun{pop}_2(\pair{n}{h}) & \rightarrow &
\pair{n\!-\!1}{\cons{\ufun{pop}()}{h}}.
\end{array}
\end{equation*}
Nous devons réécrire~\fun{ver/2}\index{ver2@\fun{ver\(_2\)/2}} de
telle sorte qu'elle conserve la longueur de la dernière version:
\begin{equation*}
\begin{array}{@{}r@{\;}l@{\;}l@{}}
\fun{ver}_2(0,\pair{n}{h}) & \rightarrow & \fun{lst}_1(0,h);\\
\fun{ver}_2(k,\pair{n}{\cons{\ufun{pop}()}{h}})
                      & \rightarrow & \fun{ver}_2(k-1,\pair{n+1}{h});\\
\fun{ver}_2(k,\pair{n}{\cons{\ufun{push}(x)}{h}})
                      & \rightarrow & \fun{ver}_2(k-1,\pair{n-1}{h}).
\end{array}
\end{equation*}
Nous pouvons réduire l'emploi de la mémoire\index{partage|)} en
séparant l'histoire courante~\(h\) et la longueur~\(n\) de la dernière
version, de façon à n'allouer aucune paire, et nous pouvons nous
arrêter lorsque la version courante est~\(\el\), comme prévu, à la
\fig~\vref{fig:ver_no_pair}.
\begin{figure}
\begin{equation*}
\boxed{%
\begin{array}{r@{\;}l@{\;}l}
\fun{ver}_3(k,\pair{n}{h}) & \rightarrow & \fun{ver}_3(k,n,h).\\
\fun{ver}_3(0,n,h) & \rightarrow & \fun{lst}_3(0,n,h);\\
\fun{ver}_3(k,n,\cons{\ufun{pop}()}{h})
                      & \rightarrow & \fun{ver}_3(k-1,n+1,h);\\
\fun{ver}_3(k,n,\cons{\ufun{push}(x)}{h})
                      & \rightarrow & \fun{ver}_3(k-1,n-1,h).\\
\fun{lst}_3(\pair{n}{h}) & \rightarrow & \fun{lst}_3(0,n,h).\\
\fun{lst}_3(k,0,h) & \rightarrow & \el;\\
\fun{lst}_3(0,n,\cons{\ufun{push}(x)}{h}) & \rightarrow
                      & \cons{x}{\fun{lst}_3(0,n-1,h)};\\
\fun{lst}_3(k,n,\cons{\ufun{push}(x)}{h}) & \rightarrow
                      & \fun{lst}_3(k-1,n-1,h);\\
\fun{lst}_3(k,n,\cons{\ufun{pop}()}{h}) & \rightarrow
                      & \fun{lst}_3(k+1,n+1,h).
\end{array}}
\end{equation*}
\caption{Requête d'une version sans paires}
\label{fig:ver_no_pair}
\end{figure}

On pourrait se demander si cela vaut la peine d'accoupler~\(n\)
et~\(h\) pour les séparer à nouveau, ce qui va à l'encontre du
principe d'abstraction des données. Cet exemple démontre que
l'abstraction est désirable pour les appelants, mais les fonctions
appelées peuvent la briser à cause du filtrage de motif. Nous
pourrions aussi nous rendre compte que le choix d'une pile pour
conserver les mises à jour n'est pas le meilleur en termes d'usage de
la mémoire. À la place, nous pouvons directement enchaîner les mises à
jour avec l'aide d'un argument supplémentaire qui dénote la mise à
jour précédente, donc, par exemple, au lieu de
l'équation~\eqref{ex_history1}:
\begin{equation*}
\pair{3}{\ufun{push}(4, \ufun{pop}(\ufun{push}(3, \ufun{push}(2, \ufun{push}(1,\el)))))}.
\end{equation*}
Ce nouveau codage\index{pile!codage avec des \(n\)-uplets} reflète
bien l'appel~\eqref{ex_history} à la page~\pageref{ex_history} et
économise \(n\)~n{\oe}uds d'empilage dans une histoire de
longueur~\(n\). Voir la \fig~\ref{fig:ver}.
\begin{figure}
\begin{equation*}
\boxed{%
\begin{array}{r@{\;}l@{\;}l}
\fun{ver}_4(k,\pair{n}{h}) & \rightarrow & \fun{ver}_4(k,n,h).\\
\fun{ver}_4(0,n,h) & \rightarrow & \fun{lst}_4(0,n,h);\\
\fun{ver}_4(k,n,\ufun{pop}(h))
                      & \rightarrow & \fun{ver}_4(k-1,n+1,h);\\
\fun{ver}_4(k,n,\ufun{push}(x,h))
                      & \rightarrow & \fun{ver}_4(k-1,n-1,h).\\
\fun{lst}_4(\pair{n}{h}) & \rightarrow & \fun{lst}_4(0,n,h).\\
\fun{lst}_4(k,0,h) & \rightarrow & \el;\\
\fun{lst}_4(0,n,\ufun{push}(x,h)) & \rightarrow
                      & \cons{x}{\fun{lst}_4(0,n-1,h)};\\
\fun{lst}_4(k,n,\ufun{push}(x,h)) & \rightarrow
                      & \fun{lst}_4(k-1,n-1,h);\\
\fun{lst}_4(k,n,\ufun{pop}(h)) & \rightarrow
                      & \fun{lst}_4(k+1,n+1,h).
\end{array}}
\end{equation*}
\caption{Requête d'une version sans pile}
\label{fig:ver}
\end{figure}
\begin{equation*}
\begin{array}{@{}r@{\;}l@{\;}l@{}}
\fun{push}_4(x,\pair{n}{h}) & \rightarrow &
\pair{n\!+\!1}{\ufun{push}(x,h)}.\\
\fun{pop}_4(\pair{n}{h}) & \rightarrow &
\pair{n\!-\!1}{\ufun{pop}(h)}.
\end{array}
\end{equation*}

Maintenant, il existe un coût minimal et maximal. Le pire des cas est
quand l'élément au fond dans la dernière version est le premier
élément empilé dans l'histoire, donc
\fun{lst\(_4\)/3}\index{lst4@\fun{lst\(_4\)/3}} doit retourner jusqu'à
l'origine. En d'autres termes, la ligne historique n'atteint jamais
l'axe des abscisses après le premier empilage. Nous avons alors le
même coût que précédemment: \(\W{\fun{lst}_4}{n} = n +
1\). \index{lst4@$\W{\fun{lst}_4}{n}$} Le meilleur des cas se produit lorsque la dernière version est vide. Dans ce cas, \(\B{\fun{lst}_4}{n} =
1\)\index{lst4@$\B{\fun{lst}_4}{n}$} et c'est la sorte d'amélioration que nous avions en tête.

%\mypar{Average cost}

\paragraph{Persistance complète}
\index{persistance!$\sim$ complète}

La méthode conservant les mises à jour dans l'histoire est
complètement persistante parce qu'elle permet de modifier une version
passée comme suit: traversons l'histoire jusqu'au moment adéquat,
dépilons la mise à jour, empilons-en une autre et remettons à leur
place les mises à jours précédemment traversées, qui ont été stockées
dans un accumulateur.

Mais changer le passé ne doit pas créer une histoire contenant une
version qui n'est pas constructible, c'est-à-dire que la ligne
historique ne doit pas croiser l'axe des abscisses après la
modification. Si le changement consiste à replacer un dépilage par un
empilage, il n'y aucune raison de s'inquiéter, car cela hissera
de~\(2\) ordonnées le point terminal de la ligne. C'est le changement
inverse qui demande un peu d'attention, car il abaisse de~\(2\) le
point terminal. Cette différence verticale de~\(\pm 2\) niveaux
provient de la différence entre les points terminaux d'un empilement
et d'un dépilement de même origine et peut facilement être visualisée
à la \fig~\vref{fig:push_pop1} en posant l'une sur l'autre les
sous-figures. Par conséquent, à la \fig~\ref{fig:history1}, la
dernière version a pour longueur~\(1\), ce qui implique qu'il est
impossible de remplacer un empilage par un dépilage, à n'importe quel
moment du passé.

Considérons l'histoire à la \fig~\ref{fig:history2}.
\begin{figure}
\centering
\includegraphics[bb=69 578 272 726,scale=0.9]{history2}%[]
\caption{$\ufun{pop}(\ufun{push}(4,\ufun{push}(3,
          \ufun{pop}(\ufun{push}(2,\ufun{push}(1,\el))))))$}
\label{fig:history2}
\end{figure}
Soit \(\fun{chg}(k,u,\pair{n}{h})\)\index{chg@\fun{chg/3}} l'histoire modifiée, où \(k\)~est l'index de la mise à jour que nous voulons changer, en indexant la dernière avec~\(0\); soit~\(u\) la nouvelle mise à jour que nous voulons insérer; finalement, soit~\(n\) la longueur de la dernière version de l'histoire~\(h\). L'appel
\begin{equation*}
\fun{chg}(3,\ufun{push}(5),
  \pair{2}{\ufun{pop}(\ufun{push}(4,\ufun{push}(3,
           \ufun{pop}(\ufun{push}(2,\fun{push}(1,\el))))))})
\end{equation*}
résulte en
\(\pair{4}{\ufun{pop}(\ufun{push}(4,\ufun{push}(3,\ufun{push}(5,
  \ufun{push}(2,\ufun{push}(1,\el))))))}\). Cet appel réussi
parce qu'à la \fig~\ref{fig:pop2push}
\begin{figure}
\centering
\subfloat[Un dépilage devient un empilage\label{fig:pop2push}]{
  \includegraphics[bb=69 521 270 726,scale=0.8]{pop2push}}
\quad
\subfloat[Un empilage devient un dépilage\label{fig:push2pop}]{
  \includegraphics[bb=69 577 270 726,scale=0.8]{push2pop}}
\caption{Changement de mises à jour}
\end{figure}
la nouvelle ligne historique ne croise pas l'axe des abscisses. Nous
pouvons voir à la \fig~\ref{fig:push2pop} le résultat de l'appel
\begin{equation*}
\fun{chg}(2,\ufun{pop}(),
            \pair{2}{\ufun{pop}(\ufun{push}(4,\ufun{push}(3,
                     \ufun{pop}(\ufun{push}(2,\ufun{push}(1,\el))))))}).
\end{equation*}
Tous ces exemples nous aident à deviner la propriété caractéristique
d'un remplacement valide:
\begin{itemize}

  \item le remplacement d'un dépilage par un empilage, d'un dépilage par
  un dépilage, d'un empilage par un empilage est toujours valide;

  \item le remplacement d'un empilage par un dépilage à la position
    \(k>0\) est valide si, et seulement si, la ligne historique entre
    les mises à jour \(0\) et~\(k-1\) se maintient au-dessus ou touche
    sans la traverser la ligne horizontale d'ordonnée~\(2\).

\end{itemize}
Nous pouvons concevoir un algorithme procédant en deux phases. Tout
d'abord, la mise à jour qui doit être à remplacée doit être localisée,
mais, la différence avec \fun{ver\(_4\)/3} est que nous pourrions
avoir besoin de savoir si la ligne historique, avant d'atteindre la
mise à jour, est bien entièrement située au-dessus de la ligne
horizontale d'ordonnée~\(2\). Ceci est facile à vérifier si nous
conservons, à travers les appels récursifs, l'ordonnée la plus petite
atteinte par la ligne. La seconde phase, quant à elle, substitue la mise à jour et vérifie que l'histoire résultant est valide.

Réalisons la première phase. D'abord, nous projetons \(n\)~et~\(h\)
hors de \(\pair{n}{h}\) de manière à économiser un peu de mémoire, et
l'ordonnée la plus basse est~\(n\), que nous passons comme un argument
supplémentaire à une autre fonction
\fun{chg/5}\index{chg@\fun{chg/5}|(}:
\begin{equation*}
\fun{chg}(k,u,\pair{n}{h}) \rightarrow \fun{chg}(k,u,n,h,n).
\end{equation*}
La fonction~\fun{chg/5} traverse~\(h\) tout en décrémentant~\(k\),
jusqu'à ce que \(k=0\), ce qui signifie que la mise à jour a été
trouvée. En même temps, la longueur de la version courante est
calculée (troisième argument) et comparée à la précédente ordonnée la
plus basse (cinquième argument), qui est mise à jour en
conséquence. Nous pourrions essayer le canevas suivant:
\begin{equation*}
\begin{array}{@{}r@{\;}l@{\;}l@{}}
\fun{chg}(0,u,n,h,m) & \rightarrow & \fbcode{CCCCCCC}\,;\\
\fun{chg}(k,u,n,\ufun{pop}(h),m) & \rightarrow
                                 & \fun{chg}(k-1,u,n+1,h,m);\\
\fun{chg}(k,u,n,\ufun{push}(x,h),m) & \rightarrow
                   & \fun{chg}(k-1,u,n-1,h,m),\qquad \text{si \(m < n\)};\\
\fun{chg}(k,u,n,\ufun{push}(x,h),m) & \rightarrow
                                    & \fun{chg}(k-1,u,n-1,h,n-1).
\end{array}
\end{equation*}
\index{chg@\fun{chg/5}|)}

Le problème est que nous oublions l'histoire jusqu'à la mise à jour
recherchée. Deux techniques sont envisageables pour la conserver: soit
nous utilisons un accumulateur\index{langage fonctionnel!accumulateur}
et nous nous conformons à une définition avec des appels
terminaux\index{langage fonctionnel!forme terminale} (conception à
petits pas\index{conception!petits pas}), soit nous remettons à sa
place une mise à jour après la conclusion d'un appel récursif
(conception à grand pas\index{conception!grands pas}). La seconde
option est plus rapide, car il n'y a pas alors besoin de retourner
l'accumulateur quand nous avons fini; la première option nous permet
de partager l'histoire jusqu'à la mise à jour quand l'histoire
résultante est structurellement équivalente, en payant le prix d'un
argument supplémentaire qui est l'histoire originelle. Nous avons déjà
rencontré ce dilemme quand nous avions comparé
\fun{sfst/2}\index{sfst@\fun{sfst/2}} et
\fun{sfst\(_0\)/2}\index{sfst0@\fun{sfst\(_0\)/2}} à la
section~\ref{sec:skipping}, page~\pageref{sec:skipping}. Choisissons
une conception à grands pas, à la \fig~\ref{fig:chg}.
\begin{figure}[b]
\centering
\framebox[\columnwidth]{\vbox{%
\begin{gather*}
\inferrule*{}{\fun{chg}(0,u,n,h,m) \rightarrow \fun{rep}(u,h,m)}
\;\;
\inferrule
  {\fun{chg}(k-1,u,n+1,h,m) \twoheadrightarrow \pair{n'}{h'}}
  {\fun{chg}(k,u,n,\ufun{pop}(h),m) \twoheadrightarrow
                                    \pair{n'}{\ufun{pop}(h')}}
\\
\inferrule
  {\fun{chg}(k-1,u,n-1,h,m) \twoheadrightarrow \pair{n'}{h'}\\
   m < n}
  {\fun{chg}(k,u,n,\ufun{push}(x,h),m) \twoheadrightarrow
                              \pair{n'}{\ufun{push}(x,h')}}
\\
\inferrule
  {\fun{chg}(k-1,u,n-1,h,n-1) \twoheadrightarrow \pair{n'}{h'}}
  {\fun{chg}(k,u,n,\ufun{push}(x,h),m) \twoheadrightarrow
                               \pair{n'}{\ufun{push}(x,h')}}
\end{gather*}
}}
\caption{Changer une version antérieure (grands pas)}
\label{fig:chg}
\end{figure}
Remarquons comment la longueur~\(n'\) de la nouvelle histoire est
simplement passée vers le bas dans les règles d'inférence. Elle peut
facilement être comprise:
\begin{itemize}

  \item remplacer un dépilage par un dépilage ou un empilage par un
    empilage laisse la longueur originelle invariante;

  \item remplacer un dépilage par un empilage accroît la longueur
    originelle de~\(2\);

  \item remplacer un empilage par un dépilage, en supposant que cela
    soit valide, fait décroître la longueur originelle de~\(2\).

\end{itemize}
Cette tâche est dédiée à la
fonction~\fun{rep/3}\index{rep@\fun{rep/3}|(} (anglais,
\emph{replace}), qui réalise la seconde phase. Le dessein est de lui
faire construire une paire faite de la différence de longueur~\(d\) et
de la nouvelle histoire~\(h'\):
\begin{equation*}
\begin{array}{@{}r@{\;}l@{\;}l@{}}
\fun{rep}(\ufun{pop}(),\ufun{pop}(h),m)
     & \rightarrow & \pair{0}{\ufun{pop}(h)};\\
\fun{rep}(\ufun{push}(x),\ufun{push}(y,h),m)
     & \rightarrow & \pair{0}{\ufun{push}(x,h)};\\
\fun{rep}(\ufun{push}(x),\ufun{pop}(h),m)
     & \rightarrow & \pair{2}{\ufun{push}(x,h)};\\
\fun{rep}(\ufun{pop}(),\ufun{push}(y,h),m)
     & \rightarrow & \pair{-2}{\ufun{pop}(h)}.
\end{array}
\end{equation*}
\index{rep@\fun{rep/3}|)} Cette définition implique que nous devons
redéfinir~\fun{chg/3}\index{chg@\fun{chg/3}} comme suit:
\begin{mathpar}
\inferrule{\fun{chg}(k,u,n,h,n) \twoheadrightarrow \pair{d}{h'}}
          {\fun{chg}(k,u,\pair{n}{h}) \twoheadrightarrow \pair{n+d}{h'}}.
\end{mathpar}
