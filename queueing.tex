\section{Files d'attente}
\label{sec:queueing}
\index{file}

Malgré ses qualités didactiques, l'analyse agrégeante (voir
page~\pageref{par:aggregate}) est moins fréquemment appliquée quand
les structures de données ne sont pas en relation avec la
numération. Nous proposons d'étendre sa portée en étudiant le cas des
\emph{files d'attente fonctionnelles}
\citep{Burton_1982,Okasaki_1995,Okasaki_1998b}. Une file d'attente
fonctionnelle est une structure de donnée linéaire définie dans un
langage purement fonctionnel. La sémantique de celui-ci oblige à
représenter la file d'attente à l'aide de deux piles. Des éléments
peuvent être empilés uniquement sur l'une des piles et dépilés
uniquement de l'autre:
\begin{equation*}
\begin{array}{@{}l@{}r@{\;}cc|c|c|c|c|c|cc@{\;}l@{}}
\cline{4-9}
& \text{Empilage, Dépilage}
& \leftrightsquigarrow & & a & b & c & d & e &&&\\
\cline{4-9}
\end{array}
\end{equation*}
Dans la suite, nous appellerons les files d'attente fonctionnelles
plus simplement des files, car le contexte est clair. Une file est
comme une pile où des éléments peuvent être ajoutés, ou
\emph{enfilés}, à une extrémité appelée \emph{arrière}, et retirés, ou
\emph{défilés}, à l'autre extrémité, appelée \emph{front}:
\begin{equation*}
\begin{array}{@{}l@{}r@{\;}cc|c|c|c|c|c|cc@{\;}l@{}}
\cline{4-10}
& \text{Enfilage (arrière)}
                & \rightsquigarrow & & a & b & c & d & e &
& \rightsquigarrow & \text{Défilage (front).}\\
\cline{4-10}
\end{array}
\end{equation*}
Réalisons une file avec deux piles: une pour enfiler, appelée
\emph{pile arrière}\index{file!pile arrière}\index{pile!$\sim$
  arrière|see{file}}, et une pour défiler, appelée \emph{pile
  frontale}\index{file!pile frontale}\index{pile!$\sim$
  frontale|see{file}}. La file précédente est alors équivalente à la
file fonctionnelle suivante:
\begin{equation*}
\begin{array}{r@{\;}cc|c|c|c|c|c|c|cc@{\;}l}
  \cline{3-6}\cline{8-10}
  \text{Enfilage (arrière)} & \rightsquigarrow & & a & b & c & & d & e & &
  \rightsquigarrow & \text{Défilage (front).}\\
  \cline{3-6}\cline{8-10}
\end{array}
\end{equation*}
Enfiler est alors empiler sur la pile arrière et défiler est dépiler de la pile frontale. Dans ce dernier cas, si la pile frontale est vide et la pile arrière ne l'est pas, nous retournons la pile arrière et l'échangeons avec la frontale. Graphiquement, défiler dans la configuration
\begin{equation*}
\begin{array}{c|c|c|c|c|c} \cline{1-4}\cline{6-6} & a & b & c &
&\\ \cline{1-4}\cline{6-6}
\end{array}
\end{equation*}
nécessite au préalable de fabriquer
\begin{equation*}
\begin{array}{c|c|c|c|c|c}
  \cline{1-1}\cline{3-6}
  & & a & b & c &\\
  \cline{1-1}\cline{3-6}
\end{array}
\end{equation*}
et enfin défiler~\(c\).

Modélisons une file comme nous avons modélisé un empilage par le
constructeur \fun{cons/2}\index{cons@\fun{cons/2}}, à la
page~\pageref{par:stacks}. Ici le constructeur de file sera
\fun{q/2}\index{q@\fun{q/2}} et l'appel \(\fun{q}(r,f)\) dénote une
file fonctionnelle dont la pile arrière est~\(r\) et la
frontale~\(f\). Enfiler est réalisé par la fonction
\fun{enq/2}\index{enq@\fun{enq/2}} (anglais: \emph{enqueue}):
\begin{equation}
\fun{enq}(x,\fun{q}(r,f)) \rightarrow \fun{q}(\cons{x}{r},f).
\label{def:enq}
\end{equation}
Défiler requiert que le résultat soit une \emph{paire}\index{langage
  fonctionnel!paire} faite de l'élément défilé et de la nouvelle file
sans lui. En fait, la nouvelle file est la première composante de la
paire, pour mieux correspondre à la façon dont l'opération est
représentée graphiquement, avec une flèche pointant à droite.
%% Nous pourrions dénoter la paire mathématique \((x,y)\) par
%% \(\fun{pair}(x,y)\), mais une notation symbolique est plus commode:
%% \(\pair{x}{y}\). Les chevrons nous permettent d'éviter l'écriture de
%% \(f((x,y))\) quand nous voulons dire \(f(\pair{x}{y})\).
Nommons \fun{deq/1}\index{deq@\fun{deq/1}} (anglais: \emph{dequeue})
la fonction de défilage:
\begin{equation}
\begin{array}{@{}r@{\;}l@{\;}l@{}}
  \fun{deq}(\fun{q}(\cons{x}{r},\el))
& \xrightarrow{\smash{\theta}}
& \fun{deq}(\fun{q}(\el,\fun{rcat}(r,[x])));\\
  \fun{deq}(\fun{q}(r,\cons{x}{f}))
& \xrightarrow{\smash{\iota}}
& \pair{\fun{q}(r,f)}{x}.
\end{array}
\label{def:deq}
\end{equation}
Voir page~\pageref{def:rev} la définition~\eqref{def:rev}
de~\fun{rcat/2}\index{rcat@\fun{rcat/2}}. Nous dirons que la file a
pour taille~\(n\) si le nombre total d'éléments dans les deux piles
est~\(n\).  Le coût de l'enfilage\index{enq@$\C{\fun{enq}}{n}$} est
\(\C{\fun{enq}}{n} = 1\). Le coût
minimal\index{deq@$\B{\fun{deq}}{n}$} du défilage est
\(\B{\fun{deq}}{n} = 1\), dû à la règle~\(\iota\). Le coût maximal est
\begin{equation}
  \W{\fun{deq}}{n} = \len{\theta\eta^{n-1}\zeta\iota} = 1 + (n-1)
+ 1 + 1 = n + 2. \label{eq:Wdeq}
\end{equation}
Soit~\(\A{n}\) le coût d'une série de \(n\)~mises à jour d'une file
fonctionnelle originellement vide. Une première tentative pour
trouver~\(\A{n}\) consiste à ignorer toute dépendance entre opérations
et à cumuler leur coût individuel maximal. Puisque \(\C{\fun{enq}}{k}
\leqslant \C{\fun{deq}}{k}\), nous supposons une série de
\(n\)~défilages dans le pire des cas, c'est-à-dire, avec tous les
éléments situés dans la pile arrière. D'ailleurs, après \(k\)~mises à
jour, il ne peut y avoir plus de \(k\)~éléments dans la file, donc
\begin{equation}
  \A{n} \leqslant \sum_{k=1}^{n-1}{\W{\fun{deq}}{k}} =
  \sum_{k=1}^{n-1}{(k+2)} =
  \frac{1}{2}(n+4)(n-1) \sim \frac{1}{2}{n^2}.
  \label{ineq:Sn_upper_bound}
\end{equation}
d'après les équations~\eqref{eq:Wdeq} et~\eqref{eq:sum_k}.


\paragraph{Analyse agrégeante}
\index{coût!$\sim$ amorti}
\index{file!coût amorti}

En fait, ce qui précède est trop pessimiste et même irréaliste. Tout
d'abord, on ne peut défiler d'une file vide, donc, à tout moment, le
nombre d'enfilages depuis le début est toujours supérieur ou égal au
nombre de défilages et la série doit débuter avec un
enfilage. Ensuite, lorsque l'on défile avec la pile frontale vide, la
pile arrière est retournée et déplacée au front, donc ses éléments ne
peuvent être retournés à nouveau lors du prochain défilage, dont le
coût sera alors~\(1\).

De plus, comme nous l'avons remarqué plus haut, \(\C{\fun{enq}}{k}
\leqslant \C{\fun{deq}}{k}\), donc le pire des cas pour une suite de
\(n\)~opérations se présente lorsque le nombre de défilages est
maximal et vaut donc \(\lfloor{n/2}\rfloor\). Si nous dénotons
par~\(e\) le nombre d'enfilages et par~\(d\) le nombre de défilages,
nous avons la relation triviale \(n = e + d\) et les deux prérequis
pour le pire des cas deviennent \(e=d\) (\(n\)~pair) ou \(e=d+1\)
(\(n\)~impair). Le premier correspond graphiquement à un \emph{chemin
de Dyck}\index{chemin de Dyck} et le dernier à un \emph{méandre de
Dyck}\index{méandre de Dyck}.

\paragraph{Chemin de Dyck}
\index{chemin de Dyck}

Décrivons les mises à jour comme à la \fig~\vref{fig:enq_deq}.
\begin{figure}
\centering
\subfloat[Enfilage\label{fig:enqueue}]{%
  \includegraphics[bb=69 662 132 721,scale=0.75]{enqueue}
}
\qquad
\subfloat[Défilage\label{fig:dequeue}]{%
  \includegraphics[bb=69 662 132 721,scale=0.75]{dequeue}
}
\caption{Représentation graphique des opérations sur les files}
\label{fig:enq_deq}
\end{figure}
Textuellement, nous représentons un enfilage comme une parenthèse
ouvrante et un défilage comme une parenthèse fermante. Par exemple,
\texttt{((()()(()))())} correspond, à la \fig~\ref{fig:dyck_path},
\begin{figure}
\centering
\includegraphics[bb=75 570 508 727,scale=0.77]{dyck_path}
\caption{Chemin de Dyck modélisant des opérations sur une file}
\label{fig:dyck_path}
\end{figure}
à un chemin de Dyck\index{chemin de Dyck}. Pour qu'une ligne brisée
soit un chemin de Dyck de longueur~\(n\), elle doit commencer à
l'origine \((0,0)\) et aboutir au point de coordonnées \((n,0)\). En
termes de \emph{langage de Dyck}, un enfilage est appelé une
\emph{montée}\index{chemin de Dyck!montée} et un défilage est une
\emph{descente}\index{chemin de Dyck!descente}. Une montée suivie
d'une descente, c'est-à-dire \texttt{()}, est appelé un
\emph{sommet}\index{chemin de Dyck!sommet}. Par exemple, à la
\fig~\ref{fig:dyck_path}, nous trouvons quatre sommets. Les nombres
annotant les montées et les descentes sont les coûts encourus par
l'opération. L'axe des abscisses est gradué avec le nombre ordinal de
chaque opération.

Si \(e=d\), la ligne est un chemin de Dyck de longueur
\(n=2e=2d\). Pour déduire le coût total dans ce cas, nous devons
\emph{décomposer}\index{chemin de Dyck!décomposition} le chemin, ce
qui signifie que nous voulons identifier des motifs dont les coûts
sont aisément calculables et qui constituent tout chemin, ou bien cela
peut signifier que nous associons tout chemin à un autre dont le coût
est le même, mais plus simple.

La \fig~\ref{fig:dyck_eq}
\begin{figure}
\centering
\includegraphics[bb=75 570 508 727,scale=0.77]{dyck_eq}
\caption{Chemin de Dyck équivalent à la \fig~\ref{fig:dyck_path}}
\label{fig:dyck_eq}
\end{figure}
montre comment le chemin précédent est associé à un chemin équivalent,
exclusivement fait d'une suite de triangles isocèles dont les bases
reposent sur l'axe des abscisses. Appelons-les
\emph{montagnes}\index{chemin de Dyck!montagne} et leur succession une
\emph{chaîne}\index{chemin de Dyck!chaîne}.

L'association est simple: après la première suite de descentes, si
nous sommes de retour à l'axe des abscisses, nous venons d'identifier
une montagne et nous recommençons avec le reste du chemin. Sinon, la
prochaine opération est une montée et nous l'échangeons avec la
première descente après elle. Cela abaisse le point courant d'un
niveau et le procédé est répété jusqu'à ce que les abscisses soient
atteintes. Nous appelons cette méthode
\emph{réordonnancement}\index{chemin de Dyck!réordonnancement} parce
qu'elle revient, en termes opérationnels, à réordonner des
sous-séquences d'opérations \emph{a posteriori}.

Par exemple, la \fig~\vref{fig:rescheduling}
\begin{figure}
\centering
\subfloat[Initial\label{fig:initial}]{%
  \includegraphics[scale=0.77,bb=48 588 218 726]{mount0}}
\qquad
\subfloat[Échange de \(4\nearrow 5\) et \(5\searrow 6\)]{%
  \includegraphics[scale=0.77,bb=48 588 218 726]{mount1}}\\
\subfloat[Échange de \(5\nearrow 6\) et \(8 \searrow 9\)\label{fig:valley}]{%
  \includegraphics[scale=0.77,bb=48 588 218 726]{mount2}}
\qquad
\subfloat[Dernier]{%
  \includegraphics[bb=13 645 180 782,scale=0.77]{mount3}}
\caption{Réordonnancement de la \fig~\ref{fig:dyck_path} en  \fig~\ref{fig:dyck_eq}}
\label{fig:rescheduling}
\end{figure}
montre le réordonnancement de la \fig~\vref{fig:dyck_path}. Remarquons
que deux chemins différents peuvent être réordonnés en le même
chemin. Ce qui rend la \fig~\ref{fig:valley} équivalente à la
\fig~\ref{fig:initial} est l'invariance du coût parce que toutes les
opérations concernées ont un coût unitaire. En effet, les enfilages
ont toujours un coût unitaire et les défilages impliqués dans un
réordonnancement ont aussi un coût unitaire, parce qu'ils trouvent la
pile frontale non-vide après un sommet. Nous avons prouvé que tous les
chemins sont équivalents à une chaîne de montagnes de même coût, donc
le coût maximal peut être recherché parmi les chaînes uniquement.

Notons \(e_1, e_2, \dots, e_k\) les sous-séquences contiguës maximales
de montées; par exemple, à la \fig~\ref{fig:dyck_eq}, nous avons
\(e_1=3\), \(e_2 = 3\) et \(e_3 = 1\). Bien entendu, \(e = e_1 + e_2 +
\dots + e_k\). La descente qui constitue le \(i^\text{e}\) sommet a
pour coût \index{deq@$\W{\fun{deq}}{n}$} \(\W{\fun{deq}}{e_i} = e_i +
2\), à cause de la vacuité de la pile frontale, puisque nous avons
démarré les montées à partir des abscisses. Les \(e_i-1\) prochaines
descentes ont toutes un coût unitaire parce que la pile frontale n'est
pas vide. Donc, la \(i^\text{e}\) montagne a pour coût
\(e_i+(e_i+2)+(e_i-1) = 3e_i+1\). Alors
\begin{equation*}
  \A{e,k} = \sum_{i=1}^{k}{(3e_i+1)} = 3e + k.
\end{equation*}
Le coût maximal est atteint en maximisant \(\A{e,k}\) pour un~\(e\)
donné:
\begin{equation*}
\W{}{e,e} := \max_{1 \leqslant k \leqslant e}{\A{e,k}} = \A{e,e} = 4e
= 2n,\,\; \text{où \(n=e+d=2e\)},
\end{equation*}
où \(\W{}{e,e}\) est le coût maximal quand il y a \(e\)~enfilages et
\({d=e}\) défilages. En d'autres termes, le pire des cas quand
\(e=d=7\) est le chemin de Dyck en dents de scie montré à la
\fig~\vref{fig:max_dyck}.

Pour conclure, il est crucial de voir qu'il n'y a pas d'autres chemins
de Dyck dont le réordonnancement mène à ce pire des cas, la raison
étant que la transformation inverse, des chaînes de montagnes vers les
chemins généraux, opère sur des défilages de coût unitaire et la
solution que nous avons trouvée est la seule sans défilage de coût
unitaire.
\begin{figure}
\centering
\includegraphics[bb=86 657 507 726,scale=0.7]{max_dyck}
\caption{Pire des cas si \(e=d=7\)}
\label{fig:max_dyck}
\end{figure}

\paragraph{Méandre de Dyck}
\index{méandre de Dyck}

Un autre cas parmi les pires se produit si \(e= d + 1\) et la ligne
est alors un \emph{méandre de Dyck} dont l'extrémité finale est un
point d'ordonnée \(e-d=1\). Un exemple est donné à la
\fig~\ref{fig:meander1},
\begin{figure}
\centering
\includegraphics[bb=75 570 480 727,scale=0.82]{meander1}
\caption{Méandre de Dyck modélisant des opérations sur une file}
\label{fig:meander1}
\end{figure}
où la dernière opération est un défilage. La ligne en pointillés
marque le résultat du réordonnancement que nous avons défini sur les
chemins de Dyck. Ici, la dernière opération devient un enfilage.

Une autre possibilité est montrée à la \fig~\ref{fig:meander2},
\begin{figure}
\centering
\includegraphics[bb=75 570 480 727,scale=0.82]{meander2}
\caption{Méandre de Dyck modélisant des opérations sur une file}
\label{fig:meander2}
\end{figure}
où la dernière opération est inchangée. La différence entre les deux
exemples repose sur le fait que, à l'origine, le dernier défilage a,
dans le premier cas, un coût unitaire (donc est changé) et, dans le
dernier cas, un coût strictement supérieur à~\(1\) (donc inchangé).

La troisième sorte de méandre de Dyck est celle qui termine avec un
enfilage, mais parce que cet enfilage doit partir de l'axe des
abscisses, nous nous trouvons dans la même situation qu'avec le
résultat du réordonnancement d'un méandre conclu avec un défilage de
coût unitaire (voir à nouveau la ligne en pointillés à la
\fig~\ref{fig:meander1}).

Par conséquent, nous n'avons qu'à comparer l'effet du réordonnancement
des méandres se terminant avec un défilage, c'est-à-dire que nous
envisageons les deux cas suivants.
\begin{itemize}

\item Si nous avons une chaîne de \(n-1\)~opérations suivies par un
  enfilage, le coût maximal de la chaîne est le coût d'un chemin de
  Dyck en dents de scie, \(\W{}{e-1,e-1}=4(e-1)=2n-2\), car
  \(n=e+d=2e-1\), plus le coût d'un enfilage, totalisant donc
  \(2n-1\).

  \item Sinon, nous avons affaire à une chaîne de \(n-3\)~opérations
  suivies par deux montées et une descente (de coût~\(6\)). Le coût
  total est alors \(\W{}{e-2,e-2}+6=2n\), ce qui est un petit peu plus
  que le cas précédent.

\end{itemize}

\mypar{Coût amorti}
\index{coût!$\sim$ amorti}
\index{file!coût amorti}

Le coût~\(\A{n}\) d'une suite de \(n\)~mises à jour sur une file
originellement vide est strictement encadré comme
suit:\index{file@$\A{n}$}
\begin{equation*}
n \leqslant \A{n} \leqslant 2n,
\end{equation*}
où la borne inférieure est atteinte si toutes les mises à jour sont
des enfilages, et la borne supérieure est atteinte quand une chaîne en
dents de scie est suivie par un enfilage ou alors deux enfilages et un
défilage. Par définition, le coût amorti d'une opération
est~\(\A{n}/n\), qui se situe donc entre \(1\)~et~\(2\), ce qui est
inférieur à ce qu'aurait pu nous faire craindre la borne supérieure de
l'inégalité~\eqref{ineq:Sn_upper_bound}. \cite{Rinderknecht_2011}
propose une analyse légèrement différente de la précédente, avec les
mêmes exemples.


\paragraph{Aparté}

Nous pouvons gagner un peu plus d'abstraction en utilisant un constructeur dédié pour la file vide, \(\fun{nilq/0}\), et en changeant la définition de
\fun{enq/2} dans~\eqref{def:enq} \vpageref{def:enq} de telle sorte qu'elle
filtre ce cas:
\begin{equation*}
\begin{array}{r@{\;}l@{\;}l}
\fun{enq}(x,\fun{nilq}()) & \rightarrow & \fun{q}([x],\el);\\
\fun{enq}(x,\fun{q}(r,f)) & \rightarrow & \fun{q}(\cons{x}{r},f).
\end{array}
\end{equation*}
Nous pouvons encore gagner un tout petit peu plus de temps en
empilant~\(x\) directement sur la pile frontale:
\begin{equation*}
\begin{array}{r@{\;}l@{\;}l}
\fun{enq}(x,\fun{nilq}()) & \rightarrow & \fun{q}(\el,[x]);\\
\fun{enq}(x,\fun{q}(r,f)) & \rightarrow & \fun{q}(\cons{x}{r},f).
\end{array}
\end{equation*}

\paragraph{Exercices}
\begin{enumerate}

  \item Soit \(\fun{nxt}(q)\)\index{nxt@\fun{nxt/1}} le prochain
    élément à être défilé de~\(q\):
    \begin{equation*}
      \begin{array}{r@{\;}l@{\;}l}
        \fun{nxt}(\fun{q}(\cons{x}{r},\el)) & \rightarrow
        & \fun{nxt}(\fun{q}(\el,\fun{rcat}(r,[x])));\\
        \fun{nxt}(\fun{q}(r,\cons{x}{f})) & \rightarrow & x.
        \index{q@\fun{q/2}}
      \end{array}
    \end{equation*}
    Modifiez \fun{enq/2}\index{enq@\fun{enq/2}},
    \fun{deq/1}\index{deq@\fun{deq/1}} et
    \fun{nxt/1}\index{nxt@\fun{nxt/1}} de telle manière que
    \(\C{\fun{nxt}}{n} = 1\)\index{nxt@$\C{\fun{nxt}}{n}$}, où
    \(n\)~est le nombre d'éléments dans la file.

    \item Trouvez~\(\A{n}\) en utilisant la définition légèrement
      différente suivante:
      \begin{equation*}
        \begin{array}{@{}r@{\;}l@{\;}l@{}}
          \fun{deq}(\fun{q}(\cons{x}{r},\el))
          & \rightarrow
          & \fun{deq}(\fun{q}(\el,\fun{rev}(\cons{x}{r})));\\
          \fun{deq}(\fun{q}(r,\cons{x}{f}))
          & \rightarrow
          & \pair{\fun{q}(r,f)}{x}.
        \end{array}
      \end{equation*}

\end{enumerate}
