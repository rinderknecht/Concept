\chapter{Introduction à \XSLT}

La récursivité est une technique de programmation qui est souvent
négligée dans les cursus universitaires, vue à la va-vite à la fin du
semestre, sauf quand le langage de programmation est fonctionnel,
c'est-à-dire s'il promeut les données constantes et si le flot de
contrôle est principalement défini par la composition de fonctions
mathématiques. Des exemples de tels langages sont \Scheme, \Haskell,
\OCaml, \Erlang et \XSLT~\citep{Kay_2008}. Parmi ceux-ci, \XSLT est
rarement enseigné à l'université, donc les professionnels qui n'ont
pas été exposés au préalable à la programmation fonctionnelle doivent
probablement relever les deux défis que sont l'apprentissage d'un
nouveau paradigme et l'utilisation de \XML pour programmer: alors que
le premier met au premier plan la récursivité, le second l'obscurcit à
cause de la verbosité inhérente à \XML. La difficulté syntaxique est
inévitable avec \XSLT parce sa grammaire \emph{est} \XML, et elle est
aussi celle des données d'entrée et, souvent, du résultat.

C'est pourquoi ce chapitre introduit les fondements de \XSLT en
s'appuyant sur la compréhension de notre langage fonctionnel abstrait,
ou, concrètement, un petit sous-ensemble d'\Erlang, un langage choisi
pour sa syntaxe dépouillée et régulière, de même que l'usage du
filtrage de motifs qui permettent l'écriture de programmes
compacts. Nous espérons que le modèle mental d'un programmeur \Erlang
facilitera la transition pour penser récursivement en \XSLT, en
n'ayant alors qu'à dépasser le seul obstacle de \XML. Dans cette
optique, un minuscule sous-ensemble de \XSLT est présenté et des
exemples précédents en \Erlang sont systématiquement débarrassés de
leurs filtrages pour qu'ils deviennent facilement traduisibles en
\XSLT, qui ne possède pas ce mécanisme. En même temps, nous
travaillons directement en \XSLT de nouveaux exercises sur des arbres
de Catalan, c'est-à-dire des arbres dont les nœuds internes
peuvent avoir un nombre variable d'enfants, dans l'espoir d'un
transfert de compétence à partir d'\Erlang. Le but de ce chapitre
n'est pas d'illustrer le plus de fonctionnalités possible de \XSLT,
mais d'inciter à penser récursivement en \XSLT, qui peut être conçu
comme un langage fonctionnel spécialisé dans le traitement des
documents \XML.


\section{Documents}

Nous commençons notre parcours avec une brève présentation des bases
de \XML, \HTML et \DTD, à partir de nos notes de lectures. Le lecteur
qui est déjà familier avec ces langages peut ignorer cette section et
ne l'utiliser que comme une référence pour les chapitres suivants.

\mypar{\XML}

L'acronyme \XML signifie en anglais \emph{eXtensible Markup Language},
soit «~langage d'annotations extensible~». Il s'agit d'un langage pour
définir des arbres de Catalan (des arbres dont le degré des nœud
n'est pas fixe ou limité) avec simplement du texte, avec un nombre
minimal de constructions syntaxiques. Ces arbres sont employés pour
modéliser des \emph{documents structurés}. Les programmeurs de bases
de données les appelleraient peut-être \emph{données semi-structurées}
parce qu'ils sont alors conçus en opposition aux données qui tiennent
bien dans des tables, la structure fondamentale des bases de données
relationnelles. Ces tables réalisent un modèle mathématique de
\emph{relations} satisfaisant des \emph{schémas}, alors que \XML
représente des \emph{arbres de Catalan} (généraux) et des grammaires
formelles. (Pour aggraver la confusion, on peut adjoindre à \XML des
schémas dédiés.)  De toute façon, pour comprendre ce qu'est \XML et
comment la modélisation fonctionne, il est probablement plus facile de
commencer avec un petit exemple, comme un courriel. Quels en sont les
différents \emph{éléments} et qu'elle en est la \emph{structure},
c'est-à-dire, comment sont reliés les éléments entre eux? En ce qui
concerne les éléments, un courriel contient au moins
\begin{itemize}

  \item l'adresse de l'envoyeur,

  \item un sujet ou titre,

  \item l'adresse du destinataire,

  \item un contenu de simple texte.

\end{itemize}
Les éléments correspondent aux nœuds de l'arbre et la structure
est modélisée par la forme de l'arbre lui-même (sa topologie). Par
exemple:\\
\noindent\rule{\linewidth}{0.5pt}
\begin{alltt}
De: Moi
Sujet: Devoirs
À l'attention de: Vous

  Une \textbf{date-butoir} est la date de remise d'un \emph{devoir}.
\end{alltt}
\rule{\linewidth}{0.5pt}

\bigskip

\noindent Ce courriel peut être modélisé par l'arbre à la
\fig~\vref{fig:mail}.
\begin{figure}[b]
\begin{center}
\includegraphics[scale=1,bb=71 622 403 723]{mail}
\caption{Un courriel vu comme un arbre \XML}
\label{fig:mail}
\end{center}
\end{figure}
Remarquons que les feuilles (sous forme de boîtes aux bords arrondis),
appelées \emph{nœuds textuels}, contiennent du texte, alors que
les nœuds internes contiennent de l'information \emph{au sujet} de
leurs sous-arbres, en particulier les feuilles. (Pour rendre la
différence encore plus évidente, nous avons utilisé l'anglais pour le
contenu des nœuds internes.) Puisque l'information dans les
nœuds internes décrit l'information qui est formatée, elle est
dite \emph{métadonnée} ou \emph{annotation}, ce qui explique une
partie de l'acronyme «~\textsf{XML}~». Le document \XML correspondant
est:
\begin{sverb}
<email>
  <from>Moi</from>
  <subject>Devoirs</subject>
  <to>Vous</to>
  <body>
  Une <definition>date-butoir</definition> est la date de remise
d'un <emphasis>devoir</emphasis>.
  </body>
</email>
\end{sverb}

\paragraph{Éléments}

Chaque sous-arbre est dénoté par une \emph{balise} (anglais,
\emph{tag}) ouvrante et fermante. Une balise ouvrante est un nom situé
entre \texttt{<} et \texttt{>}. Une balise fermante est un nom situé
entre \texttt{</} et \texttt{>}. Un paire de balises, ouvrante et
fermante, constitue un \emph{élément}; en d'autres termes, un
sous-arbre correspond à un élément. En particulier, l'élément qui
inclut tous les autres est appelé l'\emph{élément racine} (ici, il est
nommé \texttt{email}). Le \emph{nom de l'élément} ne fait pas partie
du texte, il est une métadonnée, donc il suggère le sens de la donnée
contenue dans le sous-arbre. Par exemple, l'ensemble du document \XML
est un élément dont le nom est \texttt{email} parce que le document
décrit un courriel (anglais, \emph{email}). Un parcours préfixe de
l'arbre \XML (voir page~\pageref{preorder} pour les arbres binaires)
produit des nœuds dans le même ordre de lecture des éléments
correspondants dans le document \XML. (Nous commenterons cela.) Les
données (par opposition aux métadonnées) sont toujours contenues dans
les feuilles, et il s'agit toujours de texte. En particulier, notons
que le contenu des nœuds textuels qui sont les enfants des
éléments \texttt{definition} et \texttt{emphasis} ont été composés
respectivement en gras et italique, mais d'autres interprétations
auraient été possibles. Il faut bien comprendre que l'interprétation
visuelle des annotations n'est \emph{pas} définie dans \XML, et c'est
pourquoi nous avons écrit plus haut que \XML est purement une
grammaire formelle, dépourvue de sémantique.

En fait, notre exemple n'est pas un document \XML correct parce qu'il
est dépourvu d'un élément spécial qui certifie que le document est
bien \XML, et, plus précisément, quelle est la version de \XML en
usage, par exemple:
\begin{verbatim}
<?xml version="1.0"?>
\end{verbatim}
Cet élément spécial n'est en fait pas un élément à proprement parler,
comme les marqueurs \texttt{<?}~et~\texttt{?>} le montrent. Il s'agit
plutôt d'une déclaration qui contient des informations à propos du
fichier courant à destination du lecteur, que ce soit un analyseur
syntaxique, souvent appelé un \emph{processeur \XML,} ou un être
humain. En tant que tel, il est une \emph{instruction de traitement}.
(Nous y reviendrons.)

Pour le moment, considérons l'élément suivant:
\begin{verbatim}
<axiom>
L'ensemble vide <varnothing/> ne contient aucun élément.
</axiom>
\end{verbatim}
qui pourrait être interprété ainsi:\\
\noindent\rule{\linewidth}{0.5pt}
\textsf{
\textbf{Axiome}: L'ensemble vide \(\varnothing\) ne contient aucun élément.}\\
\noindent\rule{\linewidth}{0.5pt}

Ce \verb|<varnothing/>| est un \emph{élément vide}, il possède un
terminateur de balise, \verb|/>|, qui ne se trouve pas avec les
balises ouvrantes ou fermantes normales. Il est utile pour dénoter des
choses, comme des symboles, qui ne peuvent être écrites avec
l'alphabet roman et doivent être distinguées du texte simple. L'arbre
associé est
\begin{center}
\includegraphics{axiom}
\end{center}
Un élément vide correspond à une feuille dans l'arbre \XML, bien qu'il
soit une annotation et pas une donnée (on peut imaginer la présence
implicite d'un nœud textuel vide comme enfant).

Les nœuds n'ont pas besoin d'être uniques parmi une fratrie
(ensemble des enfants d'un même parent). Par exemple, si nous voulons
envoyer un courriel à plusieurs destinataires, nous écririons:
\begin{alltt}
\small<email>
  <from>Moi</from>
  <subject>Devoirs</subject>
  <to>Vous</to>
  \textbf{<to>Moi</to>}
  <body>
  Une <definition>date-butoir</definition> est la date de remise
d'un <emphasis>devoir</emphasis>.
  </body>
</email>
\end{alltt}
L'arbre \XML associé à ce document \XML est
\begin{center}
\includegraphics[scale=0.93]{mailtoto}
\end{center}
Remarquons qu'il y a deux nœuds \texttt{to} et que leur ordre doit
être le même que dans le document \XML.

\paragraph{Attributs}

Il est possible d'annoter les annotations avec des chaînes
étiquetées, appelées \emph{attributs}. Par exemple, nous pourrions
vouloir spécifier que notre courriel est urgent, ce qui est une
propriété globale du courriel, pas une partie de son contenu
proprement dit:
\begin{alltt}
\small<email \textbf{priority="urgent"}>
  <from>Moi</from>
  <subject>Devoirs</subject>
  <to>Vous</to>
  <body>
  Une <definition>date-butoir</definition> est la date de remise
d'un <emphasis>devoir</emphasis>.
  </body>
</email>
\end{alltt}
Ce document \XML peut être représenté par l'arbre annoté suivant:
\begin{center}
\includegraphics[bb=71 652 322 721]{urgent}
\end{center}
Notons le symbole \textsf{@} qui précède le nom de l'attribut, qui le
distingue des nœuds éléments. Dans une fratrie, les nœuds
attributs sont placés \emph{avant}\label{attr_before} les nœuds
éléments. Nous pouvons attacher plusieurs attributs à un élément
donné:
\begin{alltt}
\small<email \textbf{priority="urgent"} \textbf{ack="oui"}>
  <from>Moi</from>
  <subject>Devoirs</subject>
  <to>Vous</to>
  <body>
  Une <definition>date-butoir</definition> est la date de remise
d'un <emphasis>devoir</emphasis>.
  </body>
</email>
\end{alltt}
\begin{center}
\includegraphics[bb=70 651 367 723]{attr}
\end{center}
L'ordre des attributs est significatif. N'importe quel élément peut
porter des attributs, y compris les éléments vides. Les attributs sont
considérés comme une espèce de nœud spéciale, bien qu'ils ne
soient pas souvent représentés dans l'arbre \XML à cause du manque de
place.

L'instruction de traitement \texttt{xml} peut aussi posséder des
attributs prédéfinis autre que \verb|version|:
\begin{sverb}
<?xml version="1.0" encoding="UTF-8"2554?>
<?xml version='1.1' encoding="US-ASCII"?>
<?xml version='1.0' encoding='iso-8859-1'?>
\end{sverb}
L'encodage (anglais, \emph{encoding}) est l'\emph{encodage
  des caractères} du document \XML, qui est particulièrement utile
pour l'usage d'\Unicode ou de quelque police asiatique, par
exemple. Remarquons que les noms d'attributs doivent être composés en
minuscule et les \emph{valeurs} d'attributs doivent être enfermées
entre apostrophes, simples ou redoublées (guillemets). Dans les cas de
\verb|version| et \verb|encoding|, seules certaines valeurs normatives
sont valides.

\paragraph{Échappement}

La plupart des langages de programmation proposent aux programmeurs
des chaînes de caractères. Par exemple, en \Clang, les chaînes sont
placées entre guillemets, comme \verb|"abc"|. Donc, si la chaîne
contient des guillemets, nous devons prendre soin de les distinguer,
ou \emph{échapper}, pour que le compilateur (ou, plus précisément,
l'analyseur syntaxique) puisse reconnaître le guillemet dans le
contenu et celui qui termine la chaîne. En \Clang, l'échappement d'un
caractère est réalisé en le faisant précéder d'une barre
contre-oblique. Ainsi, la chaîne suivante est valide:
\verb|"Il dit: \"Bonjour!\"."|

En \XML, nous avons le même problème. Une valeur d'attribut peut être
enclose entre apostrophes ou guillemets. Si des guillemets sont
employés, les guillemets dans le contenu doivent être échappés; si des
apostrophes sont choisies, les apostrophes ont besoin d'être
échappées. Il y a aussi des problèmes qui surgissent des caractères
utilisés par les balises. Par exemple, l'élément suivant
\begin{sverb}
<problem>Pour tout entier n, nous avons n < n + 1.</problem>
\end{sverb}
n'est pas valide parce que le texte entre les balises contient le
caractère «~\texttt{<}~», qui est confondu par l'analyseur \XML avec le
début d'une balise:
\begin{alltt}
\small<problem>Pour tout entier n, nous avons n \fbox{<\,} n + 1.\fbox{<\,}/problem>
\end{alltt}
La façon dont \XML échappe ce caractère consiste à le remplacer par la
suite de caractères \verb|&lt;| et, notre élément précédent, une fois
corrigé, devient:
\begin{alltt}
\small<valid>Pour tout entier n, nous avons n \textbf{&lt;} n + 1.</valid>
\end{alltt}


\paragraph{Entités nommées prédéfinies}

La séquence \verb|&lt;| est appelée une \emph{entité nommée
  prédéfinie}. De telles entités
\begin{enumerate}

  \item commencent par une esperluète (\verb|&|),

  \item continuent avec un nom prédéfini (ici, \texttt{lt}),

  \item se concluent par un point-virgule (\texttt{;}).

\end{enumerate}
Bien entendu, l'usage de l'esperluète pour désigner le début d'une
entité entraîne que ce caractère doit lui-même être échappé s'il ne
désigne que lui-même. Dans ce cas, nous devrions employer \verb|&amp;|
à la place. Il existe d'autres caractères qui peuvent \emph{parfois}
causer des soucis aux analyseurs syntaxiques \XML (par opposition à
toujours créer un problème, comme \verb|<| et \verb|&|). Un résumé de
toutes les entités nommées prédéfinies est donné dans la table
suivante.
\begin{center}
\begin{tabular}{cll}
\toprule
\textsf{Caractère} & \textsf{Entité} & \textsf{Obligatoire}\\
\midrule
\verb|&| & \verb|&amp;|  & toujours\\
\verb|<| & \verb|&lt;|   & toujours\\
\verb|>| & \verb|&gt;|   & dans les valeurs d'attributs\\
\verb|"| & \verb|&quot;| & entre guillemets\\
\verb|'| & \verb|&apos;| & entre apostrophes\\
\bottomrule
\end{tabular}
\end{center}

\noindent Le document suivant illustre des usages d'entités:
\footXMLinUnchecked{entities.xml}

\noindent Les deux dernières entités sont des \emph{entités numérotées
  prédéfinies} (anglais, \emph{predefined numbered entity}) parce
qu'elles dénotent des caractères au moyen de leur point \Unicode
(\url{http://www.unicode.org/}). Si le code est donné en décimal, il
est précédé par \verb|&#|, par exemple, \verb|&#100|.  Si le code est
donné est hexadécimal, il est précédé par \verb|&#x|, par exemple,
\verb|&#x00E7|.

\paragraph{Entités internes}

Il est parfois malcommode d'employer des nombres pour se référer à des
caractères, surtout si on se souvient que \Unicode requiert jusqu'à
six chiffres. Pour simplifier la vie, il est possible de lier un nom à
une entité représentant un caractère, et obtenir ainsi une
\emph{entité interne dédiée} (anglais, \emph{user-defined internal
  entity}). Elles sont appelées internes parce que leur définition
doit se trouver dans le même document que leur usage. Par exemple, il
est plus aisé d'utiliser \verb|&n;| plutôt que \verb|&#241|, surtout
si le texte est en espagnol (ceci représente la lettre
\texttt{\~{n}}). Cette sorte d'entité doit être déclarée dans la
\emph{déclaration de type de document} (anglais, \emph{document type
  declaration}), qui est située, si elle existe, juste après la
déclaration \verb|<?xml ... ?>| et avant l'élément racine.
\label{xml_intro_DOCTYPE} Une déclaration de type de document est
constituée des composants suivants:
\begin{enumerate}

  \item l'ouverture \verb|<!DOCTYPE|,

  \item le nom de l'élément racine,

  \item le caractère \verb|[|,

  \item les déclarations d'\emph{entités de caractère nommé} (anglais,
    \emph{named character entities}),

  \item la fermeture \verb|]>|

\end{enumerate}
Une déclaration d'entité de caractère nommé est composée
\begin{enumerate}

  \item de l'ouverture \verb|<!ENTITY|,

  \item d'un nom d'entité,

  \item d'une entité numérotée prédéfinie entre guillemets,

  \item de la fermeture \verb|>|

\end{enumerate}
Par exemple: {\small\verb|<!ENTITY n "&#241;">|}
Voici un exemple complet:
\smallXMLinUnchecked{espana.xml}
\noindent On peut penser une telle entité comme une macro en
\textsf{cpp}, le langage du préprocesseur de \Clang. Il est en effet
possible d'étendre les entités internes dédiées pour qu'elles dénotent
n'importe quelle chaîne de caractères, pas simplement un
caractère. Typiquement, si nous souhaitons répéter un extrait de texte
long ou difficile, comme le nom d'une société étrangère ou la
généalogie des souverains de Mérina, il est préférable de donner un
nom à ce texte et, là ou sa présence est désirée, une entité avec le
même nom est placée. La syntaxe pour la déclaration est
fondamentalement la même. Par exemple,
\begin{sverb}
<!ENTITY univ "Konkuk University">
<!ENTITY motto "<spain>Viva Espa&n;a!</spain>">
<!ENTITY n "&#241;">
\end{sverb}


\paragraph{Entités externes}

Parfois le document \XML a besoin d'inclure d'autres documents \XML,
mais la recopie de ces documents n'est pas une bonne stratégie, car
elle empêche de suivre automatiquement leur mise à jour. Heureusement,
\XML nous permet de spécifier l'inclusion d'autres documents \XML au
moyen d'\emph{entités externes} (anglais, \emph{external
  entities}). La déclaration de ces entités se fait comme suit:
\begin{enumerate}

  \item l'ouverture \verb|<!ENTITY|,

  \item un nom d'entité,

  \item le mot-clé \verb|SYSTEM|,

  \item le nom complet du fichier \XML entre guillemets,

  \item la fermeture \verb|>|

\end{enumerate}
Par exemple,
\begin{sverb}
<?xml version="1.0"?>
<!DOCTYPE longdoc [
  <!ENTITY part1 SYSTEM "p1.xml">
  <!ENTITY part2 SYSTEM "p2.xml">
  <!ENTITY part3 SYSTEM "p3.xml">
]>
<longdoc>
  The included files are:
  &part1;
  &part2;
  &part3;
</longdoc>
\end{sverb}
Lors de l'analyse syntaxique, les entités externes sont obtenues via
le système d'exploitation sous-jacent, puis recopiées dans le fichier
\XML principal à l'emplacement de leur entité associée. Par
conséquent, les parties incluses ne peuvent contenir de prologue,
c'est-à-dire, pas de déclaration \XML \verb|<?xml ... ?>| et pas de
déclaration de type de document \texttt{<!DOCTYPE~...~]>}. Lorsqu'une
entité externe est lue, les processeurs \XML doivent copier
\emph{verbatim} le contenu du document externe référencé, puis
l'analyser comme s'il faisait partie du document principal.

\paragraph{Entités brutes}

Les \emph{entités brutes} (anglais, \emph{unparsed entities}),
permettent la référence à des objets binaires, comme des images, des
vidéos, des sons, ou à du texte qui n'est pas \XML, comme un programme
ou une pièce de Molière. Elles sont déclarées par
\begin{enumerate}

  \item l'ouverture \verb|<!ENTITY|,

  \item un nom d'entité,

  \item le mot-clé \verb|SYSTEM|,

  \item le nom complet du fichier non-\XML entre guillemets,

  \item le mot-clé \verb|NDATA|,

  \item une \emph{notation} (le type de fichier),

  \item la fermeture \verb|>|

\end{enumerate}
Voici un exemple:
\smallXMLinUnchecked{notation.xml}
\noindent Si nous avions employé des entités externes, l'objet aurait
été recopié \emph{in extenso} à la place de la référence et analysé
sous l'hypothèse qu'il s'agit de \XML ---~ce qu'il n'est
pas. Remarquons la notation \texttt{gif}, qui est le type de l'entité
brute. Les notations doivent être définies dans la déclaration de type
de document comme suit:
\begin{enumerate}

  \item l'ouverture \verb|<!NOTATION|,

  \item un nom de notation,

  \item le mot-clé \verb|SYSTEM|,

  \item une description du type d'entité brute que la notation
    référence (ce peut être un type \MIME, une URL, du français...)

  \item la fermeture \verb|>|

\end{enumerate}
Remarquons aussi que les entités brutes doivent être utilisées
\begin{itemize}

  \item comme des valeurs d'attribut (dans
    notre exemple, le nom de l'attribut est \texttt{image}),

  \item ou comme des noms (\texttt{picture}), au lieu de la syntaxe
    d'entité (\texttt{\&picture;}).

\end{itemize}
Par exemple, le document suivant n'est \emph{pas} bien formé:
\begin{sverb}
<?xml version="1.0"?>
<!DOCTYPE doc [
  <!NOTATION jpeg SYSTEM "image/jpeg">
  <!ENTITY pic "pictures/me.jpeg" NDATA jpeg>
]>
<doc>
  &pic;
</doc>
\end{sverb}
\begin{figure}
\begin{center}
\includegraphics[bb=71 627 278 723]{entities}
\end{center}
\caption{Résumé des différents types d'entités}
\end{figure}

\paragraph{Caractères bruts}

Il est parfois ennuyeux d'avoir à échapper des caractères,
c'est-à-dire à employer des entités de caractères. Pour éviter le
besoin d'échappement, il y a une construction spéciale: les
\emph{sections CDATA} (abrégé de l'anglais \emph{Character DATA}, soit
«~donnée de type caractère~»), qui sont faites
\begin{enumerate}

  \item d'une ouverture \verb|<!CDATA[|,

  \item de texte non-échappé et sans la séquence \verb|]]>|,

  \item d'une fermeture \verb|]]>|

\end{enumerate}
Par exemple,
\begin{alltt}
\small<paragraph>Un exemple de conditionnelle en C:
  <c>\textbf{<!CDATA[}if (x < y) return &r;\textbf{]]>}</c>
</paragraph>
\end{alltt}

\paragraph{Liens internes}

Considérons un document représentant un livre technique, comme un
manuel. Il est fréquent d'y trouver des références croisées,
c'est-à-dire des références à d'autres chapitres ou sections, ou bien
des entrées bibliographiques. Une façon simple de les réaliser
consiste à employer certains attributs comme des labels, c'est-à-dire
des noms identifiant sans ambiguïté une position dans la structure, et
d'autres attributs comme des références (vers les labels). Le problème
est que le rédacteur est alors responsable de la vérification
\begin{itemize}

  \item qu'un label donné est unique dans la portée du document, les
    entités externes incluses,

  \item et que chaque référence est bien faite vers un label existant
    (lien).

\end{itemize}
\label{xml_intro_ATTLIST}
\XML offre aux analyseurs le moyen de vérifier ce type de lien
internes automatiquement avec les attributs \texttt{ID} et
\texttt{IDREF}. Les premiers sont la sorte de tous les (attributs)
labels et les derniers la sorte des (attributs) références. Les
attributs employés comme labels ou comme références doivent être
déclarés dans la section \texttt{DOCTYPE} en utilisant
\texttt{ATTLIST} (anglais, \emph{attribute list}).

Pour déclarer des labels, il nous faut
\begin{enumerate}

  \item l'ouverture \verb|<!ATTLIST|,

  \item le nom de l'élément qui porte le label,

  \item les noms des attributs labels séparés par des espaces,

  \item le mot-clé \texttt{ID},

  \item le mot-clé \texttt{\#REQUIRED} si l'élément doit toujours
    porter un label, sinon \texttt{\#IMPLIED},

  \item la fermeture \verb|>|

\end{enumerate}
Pour les références, nous devons écrire
\begin{enumerate}

  \item l'ouverture \verb|<!ATTLIST|,

  \item le nom de l'élément qui porte la référence,

  \item les noms des (attributs) références séparés par des espaces,

  \item le mot-clé \texttt{IDREF},

  \item le mot-clé \texttt{\#REQUIRED} si l'élément doit toujours
    porter la référence, sinon \texttt{\#IMPLIED},

  \item la fermeture \verb|>|

\end{enumerate}
Par exemple,
\label{xml_intro_id_idref}
\smallXMLinUnchecked{id_idref.xml}

\paragraph{Commentaires}

Il est possible d'inclure des commentaires dans un document
\XML. Ceux-ci sont faits de
\begin{enumerate}

  \item l'ouverture \verb|<!--|,

  \item quelque texte sans la séquence \verb|--|,

  \item la fermeture \verb|-->|

\end{enumerate}
Par exemple,
\begin{alltt}
\small<p>Notre boutique est située à</p>
\textbf{<!-- <address>Eunpyeong-gu, Séoul</address> -->}
<address>Gangnam-gu, Séoul</address>
\end{alltt}
Contrairement aux langages de programmation, les commentaires ne sont
\emph{pas} ignorés par les analyseurs syntaxiques et deviennent des
nœuds dans l'arbre \XML.


\paragraph{Espaces de noms}

Chaque document \XML définit ses propres noms d'éléments, que nous
nommons collectivement son \emph{vocabulaire}. Dans le cas où nous
employons des entités externes qui font référence à d'autres documents
\XML utilisant, par coïncidence, des noms identiques, nous avons alors
affaire à une ambiguïté dans le document principal.

Une bonne façon d'éviter ces collisions de noms est d'utiliser des
\emph{espaces des noms}. Un espace de nom est une annotation de chaque
nom d'élément et d'attribut. Par conséquent, si deux documents \XML
usent de deux espaces de noms différents, c'est-à-dire deux
annotations différentes de noms d'éléments, on ne peut mélanger leur
éléments lors de l'importation d'un document par un autre, parce que
chaque nom d'élément transporte une annotation spéciale qui est
différente (idéalement unique au sein du corpus).

La définition d'un espace de noms peut être effectuée au niveau de
chaque élément en utilisant un attribut spécial avec la syntaxe
suivante:
\begin{alltt}
\small{}xmlns:\textit{préfixe} = "\textit{URL}"
\end{alltt}
où \texttt{\emph{préfixe}} est le nom de l'espace et
\texttt{\emph{URL}} (anglais, \emph{Universal Resource Location}) est
l'adresse \HTTP d'une page \HTML décrivant en langue naturelle (par
exemple, en français) l'espace de noms. Considérons l'espace de nom
\texttt{course} (français, \emph{cours}) dans le document suivant:
\begin{alltt}
\small<?xml version="1.0"?>
<\textbf{course:short}
  \textbf{xmlns:course}="http://konkuk.ac.kr/~rinderkn/Mirror/XML">
 <\textbf{course:}date>26 Août 2006</\textbf{course:}date>
 <\textbf{course:}title>Quelques langages autour de XML</\textbf{course:}title>
 <\textbf{course:}topic \textbf{course:}level="approfondi">
   Nous étudions XML, XPath et XSLT.</\textbf{course:}topic>
</\textbf{course:}short>
\end{alltt}
La portée d'un espace de noms, c'est-à-dire la partie du document où
il est utilisable, est le sous-arbre dont la racine est l'élément
déclarant l'espace. Par défaut, si le préfixe est absent, l'élément et
tous ses sous-éléments sans préfixes appartiennent à l'espace de
nom. Donc, l'exemple précédent pourrait être plus simplement écrit ainsi:
\begin{alltt}
\small<?xml version="1.0"?>
<short \textbf{xmlns}="http://konkuk.ac.kr/~rinderkn/Mirror/XML">
 <date>26 Août 2006</date>
 <title>Quelques langages autour de XML</title>
 <topic level="approfondi">
   Nous étudions XML, XPath et XSLT.
 </topic>
</short>
\end{alltt}
Notons l'absence de deux-points dans l'attribut d'espace de noms
lorsque l'on écrit: «~\texttt{xmlns=...}~». Cet exemple illustre le fait
important que ce qui définit ultimement un espace de noms est une
adresse \HTTP, pas un préfixe (comme \texttt{course}).

Comme exemple de collision de noms et comment l'éviter, considérons un
fichier \texttt{fruits.xml} contenant le fragment \HTML suivant:
\begin{sverb}
<table>
  <tr>
    <td>Bananes</td>
    <td>Oranges</td>
  </tr>
</table>
\end{sverb}
\HTML sera ébauché dans une sous-section à venir, mais, pour le
moment, il suffit de dire que les éléments bénéficient d'un sens
implicite si le fichier est bien interprété comme du \HTML. Par
exemple, \texttt{table} fait référence à la composition typographique.

Imaginons maintenant un fichier \texttt{meubles.xml} contenant une
description de meubles, comme
\begin{sverb}
<table>
  <name>Table ronde</name>
  <wood>Chêne</wood>
</table>
\end{sverb}
Le document principal \texttt{principal.xml} inclut les deux
fichiers:
\begin{sverb}
<?xml version="1.0"?>
<!DOCTYPE eclectic [
  <!ENTITY part1 SYSTEM "fruits.xml">
  <!ENTITY part2 SYSTEM "meubles.xml">
]>
<eclectic>
  &part1;
  &part2;
</eclectic>
\end{sverb}
Le problème est que \texttt{table} possède un sens différent dans
chacun des fichiers inclus, donc ils ne devraient pas être confondus:
c'est une collision de noms. La solution consiste à utiliser deux
espaces de noms différents. D'abord:
\begin{sverb}
<html:table xmlns:html="http://www.w3.org/TR/html5/">
  <html:tr>
    <html:td>Bananes</html:td>
    <html:td>Oranges</html:td>
  </html:tr>
</html:table>
\end{sverb}
Ensuite:
\begin{sverb}
<f:table xmlns:f="http://www.e-shop.com/meubles/">
  <f:name>Table ronde</f:name>
  <f:wood>Chêne</f:wood>
</f:table>
\end{sverb}
Mais c'est une solution pesante. Heureusement, les espaces de noms
peuvent aussi exister par défaut:
\begin{sverb}
<table xmlns="http://www.w3.org/TR/html5/">
  <tr>
    <td>Bananes</td>
    <td>Oranges</td>
  </tr>
</table>
\end{sverb}
Ensuite:
\begin{sverb}
<table xmlns="http://www.e-shop.com/meubles/">
  <name>Table ronde</name>
  <wood>Chêne</wood>
</table>
\end{sverb}
Les deux sortes de tables peuvent maintenant être mélangées sans
confusion. Par exemple,
\begin{sverb}
<mix xmlns:html="http://www.w3.org/TR/html5/"
     xmlns:f="http://www.e-shop.com/meubles/">
<html:table>
  ...
  <f:table>
  ...
  </f:table>
  ...
<html:table>
</mix>
\end{sverb}
Notons que l'élément \texttt{mix} n'appartient à aucun espace (ni
\texttt{html} ni~\texttt{f}). Il est possible de délier ou relier un
préfixe d'espace de noms (les exemples suivants sont tirés de
\url{http://www.w3.org/TR/REC-xml-names/}):
\begin{alltt}
\small<?xml version="1.1"?>

<x xmlns:n1="http://www.w3.org">
  <n1:a/> <!-- valide; le préfixe n1 est lié à
               http://www.w3.org -->
    <x \textbf{xmlns:n1=""}>
      <n1:a/> <!-- invalide; le préfixe n1 n'est pas lié ici -->
      <x xmlns:n1="http://www.w3.org">
        <n1:a/> <!-- valide; le préfixe n1 est lié ici -->
      </x>
   </x>
</x>
\end{alltt}

\smallXMLinUnchecked{beers.xml}

\smallXMLinUnchecked{scoping.xml}

\noindent Un élément peut contenir des éléments qui ont le même nom
(comme l'élément~\texttt{to} dans l'exemple du courriel plus haut),
mais un élément ne peut pas avoir d'attributs avec des noms identiques
ou des espaces de noms identiques (adresse \HTTP) et des noms
identiques. Par exemple, tous les éléments vides \texttt{bad} suivants
sont invalides dans
\begin{sverb}
<!-- http://www.w3.org is bound to n1 and n2 -->

<x xmlns:n1="http://www.w3.org" xmlns:n2="http://www.w3.org" >
  <bad a="1"    a="2"/>     <!-- invalide -->
  <bad n1:a="1" n2:a="2"/>  <!-- invalide -->
</x>
\end{sverb}
Toutefois, tous les suivants sont valides, le second parce que
\emph{les espaces par défaut ne s'appliquent pas aux noms
  d'attributs}:
\begin{sverb}
<!-- http://www.w3.org est lié à n1 et est le défaut -->

<x xmlns:n1="http://www.w3.org" xmlns="http://www.w3.org" >
  <good a="1" b="2"/>    <!-- valide -->
  <good a="1" n1:a="2"/> <!-- valide -->
</x>
\end{sverb}
Les espaces de noms se révèleront très important lorsque nous
présenterons \XSLT. Bien que les espaces de noms sont déclarés comme
des attributs, ils sont présents dans l'arbre \XML correspondant au
document comme un nœud spécial, différent des nœuds attributs.

\paragraph{Instructions de traitement}

Dans certain cas exceptionnels, il peut être utile d'inclure dans un
document \XML quelque information à l'intention d'un processeur \XML
spécifique. Ces données sont alors enrobées dans un élément spécial,
et les données sont elles-mêmes appelées \emph{instructions de
  traitement} (anglais, \emph{processing instructions}) parce qu'elles
disent à un processeur particulier, par exemple, \Saxon, que faire à
ce point. La syntaxe est
\begin{alltt}
<?\textit{cible} \textit{données}?>
\end{alltt}
La \emph{cible} est une chaîne destinée à être reconnue par un
processeur et les \emph{données} sont alors passées à ce
processeur. Notons que les données prennent la forme de valeurs
d'attributs, et peuvent être absentes. Par exemple,
\begin{verbatim}
<?xml version="1.0"?>
\end{verbatim}

\paragraph{Validation}

Tous les processeurs \XML doivent vérifier si le document d'entrée
satisfait les exigences \emph{syntaxiques} d'un document \XML bien
formé. En particulier,
\begin{itemize}

  \item les paires de balises délimitant un élément doivent êtres
    présentes, sauf pour les élément vides qui sont clôturés par
    \texttt{/>} (ceci doit être contrasté avec \HTML, qui est très
    relâché dans ce domaine),

  \item les entités prédéfinies doivent réellement être prédéfinies
    (les unicodes sont automatiquement vérifiés),

  \item les entités internes doivent être déclarées dans le prologue,
    etc.

\end{itemize}
\emph{Les processeurs validants} (anglais, \emph{validating
  processors}) doivent aussi s'assurer que les entités externes sont
bien trouvées (leur bonne formation est vérifiée après qu'elles ont
été insérées dans le document principal). Il existe plusieurs
analyseurs \XML disponibles gratuitement à travers internet, réalisés
avec différents langages de programmation. La plupart d'entre eux sont
en fait des bibliothèques d'interfaces (anglais, \emph{Application
  Programming Interface}), donc une application traitant du \XML ne
devrait que s'interfacer avec l'une d'entre elles. Un bon analyseur
syntaxique qui effectue aussi la validation est \texttt{xmllint}.

\mypar{\HTML}

\emph{Hyper-Text Markup Language} (\emph{\textsf{HTML}}) est un
langage employé pour décrire des pages \HTML. Voir la recommandation
du consortium W3C ({\small
  \url{http://www.w3.org/TR/html5/}}). Passons rapidement sur ce vaste
langage et illustrons-le avec quelques petits exemples. Par exemple,
tous les fichiers \HTML contenant du français devraient au moins se
conformer au patron suivant:
\begin{alltt}
\small<!DOCTYPE html>
<html lang="fr">
  <head>
    <title>\textit{le titre de la fenêtre}</title>
  </head>
  <body>
     \emph{...contenu et annotations...}
  </body>
</html>
\end{alltt}
Les éléments \texttt{h1}, \texttt{h2}, ..., \texttt{h6} correspondent
à six sortes d'en-têtes, par tailles décroissantes de police. Ouvrons
dans un navigateur le document suivant: \smallXMLinUnchecked{h.html}
\noindent (La valeur d'attribut \texttt{en-GB} est celle de l'anglais
britannique.) D'autres éléments utiles sont les suivants:
\begin{itemize}

  \item L'élément vide \texttt{<br/>} (anglais, \emph{break}) est
  interprété par les navigateurs comme un \emph{saut de ligne};

  \item l'élément \texttt{em} (anglais, \emph{emphasise}) marque du
  texte qui doit être \emph{mis en valeur} (par exemple, en utilisant
  l'italique);

  \item l'élément \texttt{strong} marque du texte à mettre en exergue
    plus fortement qu'avec \texttt{em} (par exemple, en graissant);

  \item l'élément \texttt{p} délimite un \emph{paragraphe}.

\end{itemize}
Les \emph{listes} permettent de composer ensemble des paragraphes
étroitement reliés entre-eux, comme des énumérations. Il existe trois
espèces de listes:
\begin{enumerate}

  \item les listes non-ordonnées;

  \item les listes ordonnées;

  \item les listes de définitions.

\end{enumerate}
Les paragraphes dans une liste non-ordonnée sont précédés d'un alinéa
et d'un tiret (en typographie anglaise, ce sont des points), comme:
\begin{itemize}

  \item l'élément \texttt{ul} (anglais, \emph{unordered list}) est
  une liste non-ordonnée;

  \item l'élément \texttt{li} (anglais, \emph{list item}) contient une
    phrase de la liste.

\end{itemize}
Essayons la recette suivante:
\begin{sverb}
<h3>Ingrédients</h3>
<ul>
  <li>100g de farine,</li>
  <li>10g de sucre,</li>
  <li>1 verre d'eau,</li>
  <li>2 &oelig;ufs,</li>
  <li>poivre et sel.</li>
</ul>
\end{sverb}
Les paragraphes dans une liste ordonnée sont précédés d'un alinéa et
d'un nombre, rangés par ordre croissant. Ils requièrent
\begin{enumerate}

  \item l'élément \texttt{ol}, contenant la liste ordonnée;

  \item des éléments \texttt{li}, comme dans les listes non-ordonnées.

\end{enumerate}
Par exemple,
\begin{sverb}
<h3>Recette</h3>
<ol>
  <li>Mélanger les ingrédients secs;</li>
  <li>Verser l'eau;</li>
  <li>Mélanger pendant 10 minutes;</li>
  <li>Enfourner pendant une heure à 300 degrés.</li>
</ol>
\end{sverb}
Les paragraphes dans une liste de définitions débutent par un mot en
exergue, composé en gras, suivi par une définition de ce mot. Par
exemple,
\noindent\rule{\linewidth}{0.5pt}
\begin{description}

  \item[hacker]\ \\
    Un programmeur astucieux.

  \item[nerd]

  \item[geek]\ \\
    Un informaticien brillant mais socialement inadapté.

\end{description}
\noindent\rule{\linewidth}{0.5pt}
Les éléments impliqués sont
\begin{itemize}

  \item \texttt{dl} (anglais, \emph{definition list}), qui contient
   tous les mots et leur définition;

  \item \texttt{dt} (anglais, \emph{definition term}), qui contient
   chaque terme à définir;

  \item \texttt{dd} (anglais, \emph{definition description}), qui
    contient chaque définition.

\end{itemize}
L'exemple précédent correspond à l'extrait \HTML suivant:
\begin{sverb}
<dl>
  <dt><strong>hacker</strong></dt>
    <dd>Un programmeur astucieux.</dd>
  <dt><strong>nerd</strong></dt>
  <dt><strong>geek</strong></dt>
    <dd>Un informaticien brillant mais socialement inadapté.</dd>
</dl>
\end{sverb}
Une \emph{table} est un rectangle contenant des rectangles indivis,
appelés \emph{cellules} (anglais, \emph{cell}), qui contiennent du
texte. Toutes les cellules qui sont lues verticalement appartiennent à
une même \emph{colonne} (anglais, \emph{column}), alors
qu'horizontalement, elles forment des \emph{lignes} (anglais,
\emph{row}). Une colonne ou une ligne peuvent avoir un \emph{en-tête}
(anglais, \emph{header}), c'est-à-dire une cellule à leur début qui
contient un nom graissé. Une table peut avoir une \emph{légende}
(anglais, \emph{caption}), qui est un court texte décrivant le contenu
de la table et composé juste au-dessus d'elle, comme un titre. Les
colonnes peuvent être divisées en sous-colonnes, quand cela est
nécessaire. L'exemple suivant est tiré de
\url{http://www.w3.org/TR/html4/struct/tables.html}.
\begin{center}
\emph{A test table with merged cells}\\

\begin{tabular}{|c|c|c|c|}
\hline
                 & \multicolumn{2}{|c|}{\textbf{Average}} & \textbf{Red}\\
\cline{2-3}
                 & \textbf{height}    &   \textbf{weight} & \textbf{eyes}\\
\hline
\textbf{Males}   & 1.9                & 0.003             & 40\%\\
\hline
\textbf{Females} & 1.7                & 0.002             & 43\%\\
\hline
\end{tabular}
\end{center}
\textbf{Males} et \textbf{Females} sont les en-têtes des lignes. Les
en-têtes de colonnes sont \textbf{Average}, \textbf{Red eyes},
\textbf{height} et \textbf{weight}.\! La colonne \textbf{Average}
couvre deux colonnes; autrement dit, elle contient deux sous-colonnes,
\textbf{height} et \textbf{weight}. La légende dit: «~\emph{A test
  table with merged cells}.~» Le code \HTML correspondant est:
\begin{sverb}
<table border="1">
  <caption><em>A test table with merged cells</em></caption>
  <tr>
    <th rowspan="2"/>
    <th colspan="2">Average</th>
    <th rowspan="2">Red<br/>eyes</th>
  </tr>
  <tr><th>height</th><th>weight</th></tr>
  <tr><th>Males</th><td>1.9</td><td>0.003</td><td>40%</td></tr>
  <tr>
    <th>Females</th>
    <td>1.7</td>
    <td>0.002</td>
    <td>43%</td>
  </tr>
</table>
\end{sverb}
Le sens des éléments est le suivant:
\begin{itemize}

\item l'élément \texttt{table} contient la table; son attribut
  \texttt{border} spécifie la largeur des bords de la table, soit les
  lignes séparant les cellules du reste;

\item l'élément \texttt{caption} contient la légende;

\item l'élément \texttt{th} (anglais, \emph{table header}) contient un
  en-tête de ligne ou colonne, c'est-à-dire le titre de la ligne ou
  colonne, en gras;

\item l'élément \texttt{td} (anglais, \emph{table data}) contient
  l'information d'une cellule (si ce n'est pas un en-tête);

\item l'élément \texttt{tr} (anglais, \emph{table row}) contient une
  ligne, soit une succession d'éléments \texttt{td}, avec la
  possibilité de commencer avec un élément \texttt{th}.

\end{itemize}
Remarquons les attributs \texttt{rowspan} et \texttt{colspan} de
l'élément \texttt{th}. L'attribut \texttt{rowspan} nous permet de
spécifier le nombre de lignes couvertes par la cellule courante. Par
exemple, la première ligne, soit celle en haut à gauche, est vide et
couvre deux lignes à cause de \verb|<th rowspan="2"/>|. L'attribut
\texttt{colspan} permet la déclaration du nombre de colonnes couvertes
par la cellule courante. Par exemple, la deuxième cellule, contient le
texte \textbf{Average} et couvre deux colonnes car
\verb|<th colspan="2">Average</th>|. Notons le saut de ligne
\verb|<br/>| dans la troisième cellule (première ligne, dernière
colonne) et aussi le placement correct de \textbf{height} et
\textbf{weight}.

Les \emph{hyperliens} en \HTML sont définis par l'élément «~\texttt{a}~»
avec son attribut obligatoire \texttt{href}
(anglais, \emph{hyper-reference}). Par exemple, considérons
l'hyperlien suivant:
\begin{sverb}
<a href="http://konkuk.ac.kr/~rinderkn/">Voir ma page.</a>
\end{sverb}


\mypar{\XHTML}

Le document de travail actuel de \HTML est \HTML~\textsf{5}. Avant
qu'il ne devienne une norme mise en œuvre par les navigateurs et
les processeurs \XSLT, les débutants devraient plutôt utiliser une
version plus simple d'\HTML, appelée \XHTML (anglais, \emph{eXtensible
  Hyper-Text Markup Language}), dont la recommandation par le
consortium W3C est située à l'adresse
\url{http://www.w3.org/TR/xhtml1/}. À la base, \XHTML est \XML, mais
les éléments qui sont aussi présent en \HTML ont la même
interprétation (au lieu d'aucune). Par exemple, les exemples
précédents sont valides en \XHTML, avec la contrainte supplémentaire
d'un \texttt{DOCTYPE}. Le patron général est le suivant:
\begin{alltt}
\small<?xml version="1.0" encoding="\emph{encodage}"?>

<!DOCTYPE html
    PUBLIC "-//W3C//DTD XHTML 1.0 Strict//EN"
    "http://www.w3.org/TR/xhtml1/DTD/xhtml1-strict.dtd">
<html xmlns="http://www.w3.org/1999/xhtml"
      xml:lang="fr" lang="fr">
  <head>
    <title>\textit{le titre de la fenêtre}</title>
  </head>
  <body>
     \emph{...contenu et annotations...}
  </body>
</html>
\end{alltt}
Tout comme dans les documents \XML, les documents \XHTML peuvent et
devraient être validés avant publication sur la toile, par exemple à
l'aide du site \url{http://validator.w3.org/}.

\mypar{\DTD}

Nous avons vu \vpageref{xml_intro_DOCTYPE} que la déclaration de type
de document (anglais, \emph{Document Type Declaration}) peut comporter
des annotations qui contraignent le document \XML auquel elles
appartiennent (éléments, attributs, etc.) Le contenu d'une telle
déclaration inclut une \emph{définition de type de document} (anglais,
\emph{Document Type Definition}), abrégée en \DTD. Donc la définition
est incluse dans la déclaration. Il est possible que tout ou partie de
la \DTD se trouve dans un fichier séparé, souvent avec l'extension
«~\texttt{.dtd}~». Nous avons déjà rencontré les \emph{listes
  d'attributs} \vpageref{xml_intro_ATTLIST} quand nous mettions en
place des labels et des références à l'intérieur d'un document. En
général, l'élément spécial \texttt{ATTLIST} peut être utilisé pour
déclarer n'importe quelle sorte d'attributs, pas seulement des labels
et des références.

\hspace*{-7pt}Considérons les déclarations d'attributs suivantes pour
l'élément \texttt{memo}:
\begin{sverb}
<!ATTLIST memo ident      CDATA          #REQUIRED
               security   (high | low)   "high"
               keyword    NMTOKEN        #IMPLIED>
\end{sverb}
\texttt{CDATA} veut dire, en anglais, \emph{character data}, soit
«~donnée textuelle~», et représente n'importe quelle chaîne. Un
\emph{lexème nommé} (anglais, \emph{named token}, ou \texttt{NMTOKEN})
est une chaîne commençant par une lettre et qui peut contenir des
lettres, des nombres et certain signes de ponctuation. Pour qu'un
document soit valide, ce qui requiert plus de contraintes que d'être
simplement bien formé, tous les éléments utilisés doivent être
déclarés dans la \DTD. Le nom de chaque élément doit être associé à un
\emph{modèle de contenu} (anglais, \emph{content model}), à savoir,
une description de ce qu'il peut contenir, en termes de données
textuelles et de sous-éléments (annotations). Ceci est réalisé au
moyen des déclarations \texttt{ELEMENT} dans la \DTD. Il y a cinq
modèles de contenu:
\begin{enumerate}

  \item l'\emph{élément vide}:
\begin{sverb}
<!ELEMENT padding EMPTY>
\end{sverb}

  \item les éléments \emph{sans restriction de contenu}:
\begin{sverb}
<!ELEMENT open ALL>
\end{sverb}

  \item les éléments contenant \emph{seulement du texte}:
\begin{sverb}
<!ELEMENT emphasis (#PCDATA)>
\end{sverb}
qui veut dire, en anglais, \emph{parsed-character data};

\bigskip

  \item les éléments contenant \emph{seulement des éléments}:
\begin{sverb}
<!ELEMENT section (title,para+)>
<!ELEMENT chapter (title,section+)>
<!ELEMENT report (title,subtitle?,(section+ | chapter+))>
\end{sverb}
où \texttt{title}, \texttt{subtitle} et~\texttt{para} sont des
éléments;

\bigskip

\item les éléments contenant \emph{à la fois du texte et des
    éléments}:
\begin{sverb}
<!ELEMENT para (#PCDATA | emphasis | ref)+>
\end{sverb}
où \texttt{emphasis} et~\texttt{ref} sont des éléments.
\end{enumerate}
La définition d'un modèle de contenu est semblable aux
\emph{expressions régulières}. De telles expressions sont construites
en combinant les expressions suivantes:
\begin{itemize}

  \item \texttt{(\(e_1\), \(e_2\), \dots, \(e_n\))} représentent les
    éléments représentés par~\(e_1\), suivis par les éléments
    représentés par~\(e_2\) etc. jusqu'à~\(e_n\);

  \item \texttt{\(e_1\) | \(e_2\)} représente les éléments représentés
    par~\(e_1\) ou~\(e_2\);

  \item \texttt{(\(e\))} représente les éléments représentés par~\(e\);

  \item \texttt{\(e\)?} représente les éléments représentés par~\(e\)
    ou aucun;

  \item \texttt{\(e\)+} dénote une répétition non-vide des
    éléments représentés par~\(e\);

  \item \texttt{\(e\)*} représente la répétition des éléments
    représentés par~\(e\).

\end{itemize}
\emph{Attention:} Quand nous mêlons du texte et des éléments, la seule
expression régulière possible est \verb|(#PCDATA)| ou
\verb/(#PCDATA | ...)*/

La partie d'une \DTD qui est incluse dans le même fichier que le
document \XML auquel elle s'applique est appelée le
\emph{sous-ensemble interne} (anglais, \emph{internal subset}). Voir à
nouveau l'exemple \vpageref{xml_intro_id_idref}. La part d'une \DTD
qui est dans un fichier indépendant (\texttt{.dtd}) est le
\emph{sous-ensemble externe} (anglais, \emph{external subset}). S'il
n'y a pas de sous-ensemble interne et que tout se trouve dans le
sous-ensemble externe, nous avons alors une déclaration comme
celle-ci:
\begin{sverb}
<!DOCTYPE some_root_element SYSTEM "some.dtd">
\end{sverb}
Pour valider un document \XML, sa \DTD doit décrire complètement les
éléments et attributs utilisés. Ceci n'est pas obligatoire quand la
bonne formation est requise. Par conséquent, l'exemple
\vpageref{xml_intro_id_idref} est bien formé mais pas valide au sens
ci-dessus, car les éléments \texttt{map} et \texttt{country} ne sont
pas déclarés. Pour valider ce document, il suffirait, par exemple,
d'ajouter
\begin{sverb}
<!ELEMENT map (country*)>
<!ELEMENT country EMPTY>
\end{sverb}


\section{Introduction}

Étant donné un ou plusieurs documents \XML, il peut être utile
\begin{itemize}

  \item de fouiller les documents et afficher ce qui a été trouvé dans
  un format acceptable pour une autre application, en
  particulier, \XML (filtrage inclusif);

  \item de recopier l'entrée, peut-être sans certaines parties
    (filtrage exclusif), et/ou ajouter des données (mise à jour).

\end{itemize}
Quand de telles exigences se présentent, c'est une bonne idée que
d'utiliser les langages de programmation fonctionnelle \XQuery ou
\XSLT (anglais, \emph{eXtensible Stylesheet Language
  Transformations}). Même si les deux langages conviennent à un grand
nombre d'usages communs (au point d'avoir en commun un sous-langage,
\XPath), la première application, qui est plus orientée vers la
gestion de bases de données, est plus communément entreprise avec
\XQuery, alors que le second usage est souvent développé avec \XSLT.

Un processeur \XSLT lit un document \XML et un programme \XSLT,
ensuite applique au document des \emph{transformations} définies en
\XSLT, et le résultat est imprimé, souvent sous forme de texte libre,
\XML ou \HTML. La tournure surprenante est qu'un fichier \XSLT est en
réalité un document \XML, ce qui permet d'utiliser \XSLT pour
transformer des programmes \XSLT. Par exemple, si nous utilisons un
élément \verb|book| dans un document \XML, \XML lui-même n'implique
pas forcément que cet élément modélise un livre, mais une application
utilisant ce document pourrait le faire. On peut concevoir \XML comme
un ensemble de règles syntaxiques, autrement dit, une grammaire
formelle, sans sémantique attachée aux constructions. Un document
\XSLT est donc \XML avec une déclaration performative.

Pour interpréter \XML en tant que \XSLT, les programmes demandent
l'espace de noms prédéfini
{\small\url{http://www.w3.org/1999/XSL/Transform}}, qui est souvent
(mais pas nécessairement) nommé \texttt{xsl}, comme on peut le voir
dans \smallXSLTinUnchecked{empty.xsl} La première ligne dit que ceci
est un document \XML. La deuxième définit l'interprétation comme étant
\XSLT en déclarant l'espace de noms pour \XSLT et en utilisant
l'\emph{élément racine} \texttt{xsl:transform} (l'élément
\texttt{xsl:stylesheet} est valable aussi). La version de \XSLT est
\texttt{2.0}, qui est la version courante au moment de l'écriture de
ces lignes. De plus, l'élément \texttt{xsl:output} dit que le résultat
est du texte brut. À part cela, le programme n'exprime rien d'autre,
donc nous n'attendrions pas grand chose de cette
transformation. Supposons alors le document \XML suivant,
\texttt{cookbook.xml}, à transformer:
\smallXMLinUnchecked{cookbook.xml}
\noindent L'application de la transformation vide à ce document
produit
\smallXMLinUnchecked{cookbook1_out.xml}

Il peut être surprenant de constater que quelque chose s'est bien
produit: les contenus des \emph{nœuds textuels} du document \XML
ont été extraits \emph{dans le même ordre}, mais pas les valeurs
d'attributs. (Notons que s'il manquait
\verb|<xsl:output method="text"/>|, le résultat serait considéré comme
étant du \XML et \verb|<?xml ... ?>| serait produit par défaut.) Plus
précisément, l'ordre correspond à un parcours en préordre de l'arbre
\XML correspondant: ceci est le parcours implicite des processeurs
\XSLT, appelé aussi \emph{ordre documentaire} (anglais, \emph{document
  order}). La raison pour cela est que, puisque le but est souvent de
réécrire un document en un autre, ce parcours correspond à l'ordre
dans lequel un livre est écrit et lu, de la première à la dernière
page. De plus, la raison pour laquelle les nœuds textuels sont
extraits par défaut est due à ce que \XSLT favorise un style avec
filtrage: si une partie de l'entrée devrait être ignorée ou augmentée,
le programmeur doit le dire. Finalement, remarquons qu'il n'y a pas
besoin, en \XSLT, d'instructions explicites pour imprimer: le
programmeur suppose que le résultat est du \XML ou du texte, et
l'environnement d'exécution automatiquement \emph{sérialise} le
résultat.

\paragraph{Filtrage}
\label{par:matching}

Complétons notre transformation vide comme suit:
\smallXSLTinUnchecked{chapter.xsl}
\noindent Notons l'élément prédéfini \texttt{xsl:template} définissant
un patron (anglais, \emph{template}). Il porte l'attribute
\texttt{match}, dont la valeur est le nom de l'élément que nous
voulons transformer. Durant le parcours en préordre, si un élément
\texttt{chapter} est trouvé (c'est-à-dire, filtré), le contenu du
nœud textuel du patron (\texttt{A chapter}) devient le
résultat. (Ne confondons pas le nœud textuel et son contenu.) En
appliquant la transformation précédente au document
dans le fichier nommé \texttt{cookbook.xml} donne
\smallXMLinUnchecked{cookbook2_out.xml}

\noindent Filtrons maintenant l'élément racine avec
\smallXSLTinUnchecked{chapters.xsl}
\noindent Le résultat est alors:
\smallXMLinUnchecked{cookbook3_out.xml}
\noindent La raison est que lorsqu'un patron filtre un nœud,
appelé le \emph{nœud contextuel} (anglais, \emph{context node})
dans le patron, ce nœud est traité (ici, le texte
\texttt{Chapters:} ou \texttt{A chapter} est produit) et le parcours
en préordre reprend \emph{sans visiter les enfants du nœud
  contextuel}. Par conséquent, après que l'élément \texttt{cookbook}
est filtré et traité, le processeur \XSLT ignore tout le reste
puisqu'il s'agit de l'élément racine.

Pour visiter et filtrer les enfants du nœud contextuel, nous
devons informer le processeur en utilisant l'élément vide spécial
\begin{sverb}
<xsl:apply-templates/>
\end{sverb}
Ajoutons cet élément comme enfant du patron qui filtre le nœud
racine: \smallXSLTinUnchecked{cookbook.xsl}
\noindent Le résultat est maintenant
\smallXMLinUnchecked{cookbook4_out.xml}
\noindent Il est frappant de constater que le texte
«~\texttt{Chapters:}~» n'est pas aligné avec le titre. Il est dommage
que le traitement des espaces et des fins de lignes est très compliqué
en \XML et \XSLT, en particulier quand nous souhaitons ou non
certaines espaces si le résultat, comme ici, est du texte brut. Nous
ne discuterons pas de ce sujet épineux ici et nous renvoyons le
lecteur curieux au livre de~\cite{Kay_2008}, page~141. En ce qui
concerne le reste, le titre apparaît après «~\texttt{Chapters:}~», ce
qui est déroutant.

Si nous nous débarrassions simplement du titre, nous pourrions
simplement définir un patron vide qui filtre \texttt{title}:
\smallXSLTinUnchecked{cookbook1bis.xsl}
\noindent Le résultat est maintenant:
\smallXMLinUnchecked{cookbook5_out.xml}

Si nous voulions conserver le titre, nous devrions extraire le texte
du nœud textuel qui est l'enfant de l'élément \texttt{title}, et
le mettre avant «~\texttt{Chapters:}~». Une méthode consiste à appliquer
les patrons à l'élément \texttt{title} seulement en utilisant
l'attribut prédéfini \texttt{select}, dont la valeur est le nom de
l'enfant, puis nous produisons «~\texttt{Chapters:}~», et finalement
nous appliquons les patrons aux chapitres:
\smallXSLTinUnchecked{cookbook1.xsl}
\noindent Le résultat est maintenant
\smallXMLinUnchecked{cookbook6_out.xml}
\noindent Ce n'est pas encore ce que nous voulons, parce que nous ne
devons pas appliquer les patrons à tous les enfants de
\texttt{cookbook}, mais seulement aux chapitres:
\smallXSLTinUnchecked{cookbook2.xsl}
\noindent Le résultat est maintenant:
\smallXMLinUnchecked{cookbook7_out.xml}
\noindent Nous venons de rencontrer une nouvelle bizarrerie avec les
sauts de lignes. Ce qui s'est passé est que la \emph{sélection} (via
l'attribut \texttt{select}) a réuni tous les nœuds
\texttt{chapter} dans une structure linéaire appelée une
\emph{séquence}, les patrons ont été appliqués à tous les nœuds
qu'elle contient et, finalement, un caractère de saut de ligne a été
ajouté. La raison pour laquelle nous avions des sauts de ligne après
chaque «~\texttt{A chapter}~» précédemment est due au fait que chacun de
ces textes constituait une séquence singleton. Pour recouvrer les
sauts de ligne, nous pourrions employer l'élément spécial
\texttt{xsl:text}, dont le but est de produire le contenu de son
unique nœud textuel \emph{tel quel}, sans ajuster les espaces et
les sauts de ligne. Ici, nous pourrions forcer un saut après chaque
«~\texttt{A chapter}~». En \XML, le caractère de saut de ligne est
l'entité numérotée \verb|&#10;|
\smallXSLTinUnchecked{cookbook3.xsl}
\noindent Le résultat est alors:
\smallXMLinUnchecked{cookbook8_out.xml}
\noindent Toujours pas parfait, mais laissons-là cet exemple et
passons un peu de temps à comprendre les séquences.

\section{Transformation de séquences}

Comme nous l'avons déjà vu dans la partie~\ref{part:linear}, la
structure linéaire de prédilection dans les langages fonctionnels est
la pile, appelée aussi liste. En \XSLT, c'est la \emph{séquence}. La
séquence vide est écrite \texttt{()} et la séquence non-vide est
écrite \texttt{(\(x_1\), \(x_2\), \dots, \(x_n\))}, où les~\(x_i\)
sont des items, ou bien \texttt{\(x_1\), \(x_2\), \dots, \(x_n\)}. La
différence avec les piles est double. D'abord, les séquences sont
toujours plates, donc quand un item d'une séquence est lui-même une
séquence, il est remplacé par son contenu, s'il y en a un. Par
exemple, \texttt{(1,(),(2,(3)),4)} est en fait évalué en
\texttt{(1,2,3,4)}. En particulier, une séquence singleton a la même
valeur que l'unique item qu'elle contient: \texttt{((5))}
est~\texttt{5}. Ensuite, aucun coût n'est assigné à la concaténation
de deux séquences (au lieu d'un coût linéaire en fonction de la
longueur de la première pile lorsque l'on met bout à bout deux piles),
donc l'évaluation précédente a pour coût~\(0\). La raison est que la
concaténation de séquences, étant fréquemment utilisée, est une
opération prédéfinie. Par conséquent, en \XSLT, la concaténation est
l'opération de base, pas l'empilement, comme c'est le cas avec les
piles.

Tout comme en \Erlang, les séquences peuvent contenir n'importe
qu'elle sorte d'items, pas seulement des entiers. Nous emploierons les
séquences pour contenir des éléments et attributs \XML, par
exemple. Une autre chose à savoir à propos des séquences est que si
\texttt{seq} est le nom d'une séquence, alors \texttt{\$seq}
représente la séquence: \emph{remarquons le symbole du dollar}. Si
nous écrivons \texttt{seq} dans la valeur d'un attribut \texttt{match}
ou \texttt{select}, il s'agit de l'\emph{élément} \texttt{seq}. De
plus, le premier item dans \texttt{\$seq} est écrit \texttt{\$seq[1]},
le deuxième \texttt{\$seq[2]} etc. où l'entier naturel est la
\emph{position} de l'item. Si nous sélectionnions un item qui n'est
pas dans la séquence, alors le résultat serait la séquence vide, par
exemple, si \texttt{\$seq[2]} est \texttt{()}, cela signifie que
\texttt{\$seq} contient au plus un item. Il est souvent très utile
d'extraire une sous-séquence, par analogie avec la projection de la
sous-pile d'une pile donnée. Ceci est réalisé avec la fonction
prédéfinie \texttt{position}: \texttt{\$seq[position()>1]} ou
\texttt{\$seq[position() != 1]}.

\mypar{Longueur}

Souvenons-nous ici du programme fonctionnel qui calcule la longueur
d'une pile:
\begin{equation*}
\fun{len}_0(\el) \rightarrow 0;\quad
\fun{len}_0(\cons{x}{s}) \rightarrow 1 + \fun{len}_0(s).
\end{equation*}
En vue de la traduction en \XSLT, ajoutons à notre langage une
expression conditionnelle et réécrivons le programme sans filtrage par motifs:
\begin{equation}
\fun{len}_0(s) \rightarrow \fun{if} \; s = \el \; \fun{then} \; 0 \;
\fun{else} \; 1 + \fun{len}_0(\fun{tl}(s)).
\label{def_len0_xslt}
\end{equation}
où \(\fun{tl}(s)\) (anglais, \emph{tail}) calcule la sous-pile
immédiate de~\(s\). Notons que nous ne pouvons définir \fun{tail/1}
sans filtrage par motifs, donc elle doit être traduite en une fonction
prédéfinie. Pour commencer l'écriture du programme \XSLT, nous devons
être plus précis quant aux données. Supposons que nous obtenons une
séquence en sélectionnant les nœuds \texttt{chapter}, enfants de
l'élément racine \texttt{book}. Autrement dit, nous voulons compter le
nombre de chapitres dans un livre. Par exemple,
\smallXMLinUnchecked{toc.xml}
\noindent La \DTD est comme suit:
\smallDTDin{book.dtd}
Le style que nous recommandons en \XSLT consiste à typer explicitement
le plus possible les données et les patrons. Pour ce faire, nous
devons avoir recours à une toute petite partie d'une norme appelée
\XMLSchema~\citep{Walmsley_2002}, au moyen d'un espace de nom, tout
comme nous activons l'interprétation de \XSLT avec un espace de noms.
Ceci explique le canevas de notre programme:
\begin{alltt}
\small<?xml version="1.0" encoding="UTF-8"?>
<xsl:transform version="2.0"
               xmlns:xsl="http://www.w3.org/1999/XSL/Transform"
               \textbf{xmlns:xs="http://www.w3.org/2001/XMLSchema"}>
  <xsl:output method="text" encoding="UTF-8"/>
  ...
</xsl:transform>
\end{alltt}
Dans tous nos programmes \XSLT dont le résultat est du texte, nous
souhaiterions améliorer la lisibilité en ajoutant à la fin un
caractère de saut de ligne. Étant donné que l'élément racine changera
probablement, nous filtrons la \emph{racine du document} (anglais,
\emph{document root}), notée~\texttt{/}, qui est un nœud implicite
dont l'unique enfant est l'élément racine. Nous appliquons alors tout
patron disponible à l'élément racine et mettons un saut de ligne:
\begin{sverb}
  <xsl:template match="/">
    <xsl:apply-templates/>
    <xsl:text>&#10;</xsl:text>
  </xsl:template>
\end{sverb}
Maintenant nous avons besoin de filtrer l'élément racine et d'appeler
la traduction de \fun{len\(_0\)/1}. Pour traduire des fonctions, nous
utiliserons une espèce particulière de patron, appelée \emph{patron
  nommé} (anglais, \emph{named template}), qui diffère des
\emph{patrons filtrants} (anglais, \emph{matching templates}) que nous
avons vus précédemment. Leur usage présente deux aspects: définition
et appel.

Le canevas pour définir un patron nommé est
\begin{alltt}
\small<xsl:template name="\(f\)" as="\(t\)">
  <xsl:param name="\(x\sb{1}\)" as="\(t\sb{1}\)"/>
  ...
  <xsl:param name="\(x\sb{n}\)" as="\(t\sb{n}\)"/>
  ...
</xsl:template>
\end{alltt}
Le nom du patron est~\(f\), chaque~\(x_i\) est un paramètre de
type~\(t_i\), le type de la valeur calculée par~\(f\) est~\(t\).

Le canevas pour appeler un patron nommé est le suivant:
\begin{alltt}
\small<xsl:call-template name="\(f\)">
  <xsl:with-param name="\(x\sb{1}\)" select="\(v\sb{1}\)" as="\(t\sb{1}\)"/>
  ...
  <xsl:with-param name="\(x\sb{n}\)" select="\(v\sb{n}\)" as="\(t\sb{n}\)"/>
</xsl:call-template>
\end{alltt}
Le patron nommé~\(f\) a \(n\)~paramètres \(x_1\), \(x_2\), \ldots,
\(x_n\), tels que le type de~\(x_i\) est~\(t_i\) et sa valeur
est~\(v_i\). Cette façon d'associer des valeurs aux paramètres est
présente dans des langages de programmation comme \Ada (anglais,
\emph{named association}) et \OCaml (\emph{labels}), et elle permet au
programmeur d'oublier l'ordre des paramètres, ce qui est
particulièrement utile lorsqu'ils sont nombreux.

Parvenus à ce point, il faut savoir qu'il existe des fonctions
proprement dites en \XSLT, définies par l'élément
\texttt{xsl:function}, et, bien que nous ayons choisi les patrons
nommés pour la traduction, de «~vraies~» fonctions \XSLT auraient fait
l'affaire aussi.

En reprenant le fil de notre traduction, nous réalisons que nous
devons appeler le patron nommé qui sera la traduction de \fun{len\(0\)/1}:
\begin{sverb}
  <xsl:template match="book" as="xs:integer">
    <xsl:call-template name="len0">
      <xsl:with-param name="chapters" select="chapter"
                      as="element(chapter)*"/>
    </xsl:call-template>
  </xsl:template>
\end{sverb}
Comme nous l'avons mentionné précédemment, nous avons dû nommer le
paramètre pour lui passer une valeur, soit
\texttt{chapters}. Peut-être plus déroutant est le sens de la valeur
d'attribut \texttt{element(chapter)*}: il s'agit du type de la
séquence (peut-être vide) d'éléments \texttt{chapter}. Bien qu'il ne
soit pas en cette occasion nécessaire de fournir ce type parce qu'il
est implicite dans la sélection \texttt{select="chapter"}, nous
recommandons de toujours utiliser l'attribut \texttt{as} avec
\texttt{xsl:with-param}. Par ailleurs, remarquons que le résultat du
patron qui filtre \texttt{book} est aussi typé \texttt{xs:integer}
parce qu'il est aussi le résultat du patron nommé \texttt{len0}. (Dans
le patron filtrant la racine du document (\texttt{/}), nous n'avions
pas spécifié le type du résultat car nous voulions que le patron
fonctionnât avec toutes les transformations.)

Concentrons-nous maintenant sur la définition du patron, c'est-à-dire
la traduction de \fun{len\(_0\)/1} (enfin!). Nous nous attendons au
canevas suivant:
\begin{sverb}
  <xsl:template name="len0" as="xs:integer">
    <xsl:param name="chapters" as="element(chapter)*"/>
    ...
  </xsl:template>
\end{sverb}
Le paramètre nommé \texttt{chapters} correspond à~\(s\) dans la
définition~\eqref{def_len0_xslt} de \fun{len\(_0\)/1}; nous avons
changé le nom pour qu'il convienne mieux au sens spécialisé, limité
aux chapitres ici. Nous devons maintenant traduire l'expression
conditionnelle \(\fun{if} \dots \fun{then} \dots \fun{else} \dots\) en
\XSLT. Malheureusement, les tests en \XSLT sont assez verbeux en
général. Faisons connaissance avec trois éléments qui nous permettront
d'écrire des tests généraux dans le style des constructions
\texttt{switch} de \Java:
\begin{alltt}
\small<xsl:choose>
  <xsl:when test="\(b\sb{1}\)">\(e\sb{1}\)</xsl:when>
  ...
  <xsl:when test="\(b\sb{n}\)">\(e\sb{n}\)</xsl:when>
  <xsl:otherwise>\(e\sb{n+1}\)</xsl:otherwise>
</xsl:choose>
\end{alltt}
Les valeurs~\(b_i\) des attributs \texttt{test} sont évaluées
(L'apparente tautologie est due au vocabulaire de \XSLT: une valeur
d'attribut n'est en fait pas une valeur en général, mais une
expression.) dans l'ordre d'écriture jusqu'à ce qu'une d'entre-elles,
disons~\(b_j\), résulte en le booléen \texttt{true}, causant
l'évaluation de la séquence~\(e_j\); sinon, la séquence~\(e_{n+1}\)
(les enfants de \texttt{xsl:otherwise}) est traitée. En reprenant le
cours de notre exercice, nous comblons un peu plus les ellipses:
\begin{alltt}
\small  <xsl:template name="len0" as="xs:integer">
    <xsl:param name="chapters" as="element(chapter)*"/>
    <xsl:choose>
      <xsl:when test="\textbf{empty(\$chapters)}">
        ... <!-- Traduction de \(0\) -->
      </xsl:when>
      <xsl:otherwise>
        ... <!-- Traduction de \(1 + \fun{len}\sb{0}(s)\) -->
      </xsl:otherwise>
    </xsl:choose>
  </xsl:template>
\end{alltt}
Notons la fonction prédéfinie en \XSLT nommée \texttt{empty}, qui
retourne \texttt{true} si son argument est une séquence vide, et
\texttt{false} sinon. La traduction de~\(0\) n'est pas si simple, ceci
dit! En effet, l'évident~\texttt{0} signifierait que nous produisons
en fait un \emph{texte} contenant le caractère~\texttt{0}, au lieu de
l'entier attendu. Il existe un élément \XSLT très utile en cette
circonstance ---~bien qu'il soit plus versatile qu'il n'y paraît ici,
comme nous le verrons plus loin. Faisons connaissance avec le
\emph{constructeur de séquence}:
\begin{sverb}
<xsl:sequence select="..."/>
\end{sverb}
La sélection doit s'évaluer en une séquence qui est alors substituée à
la place de l'élément \texttt{xsl:sequence}. On pourrait se demander
pourquoi ceci est tellement chantourné, et la raison est que la valeur
de l'attribut \texttt{select} appartient à un sous-langage de \XSLT
appelé \XPath, et \XPath ne peut être employé que pour les sélections
ou les tests (\texttt{empty} est une fonction \XPath). Dans un
attribut \texttt{select}, \texttt{0}~signifie~\(0\), \emph{pas} le
texte constitué d'un unique caractère~\texttt{0}, et
\texttt{xsl:sequence} nous permet d'injecter en \XSLT les valeurs
\XPath, de telle sorte que nous pouvons alors construire une séquence
créée avec \XPath. Bien sûr, nous devons garder présent à l'esprit que
tout item est équivalent à une séquence singleton, en particulier
\texttt{(0)}~est la même chose que~\texttt{0} \emph{en \XPath.} Par
conséquent, la traduction de~\(0\) est
\begin{alltt}
\small      <xsl:when test="empty(\$chapters)">
        \textbf{<xsl:sequence select="0"/>}
      </xsl:when>
\end{alltt}
L'expression \(1 + \fun{len}_0(s)\) est composée de trois parties: la
pile~\(s\), l'appel de fonction \(\fun{len}_0(s)\) et l'addition
de~\(1\) à la valeur de l'appel. Nous savons déjà que les piles sont
traduites par des séquences; nous savons aussi que les appels de
fonctions deviennent des appels à des patrons nommés. En \XPath, on
peut ajouter~\(1\) à un appel de fonction, comme \texttt{1 + f(\$n)},
mais cette syntaxe n'est pas valide en dehors d'une sélection ou d'un
test et, de toute façon, nous avons défini un patron nommé, pas une
fonction en \XSLT (qui doit être appelée en \XPath). Par conséquent,
nous devons contenir temporairement la valeur de l'appel récursif dans
une variable, disons~\texttt{x}, puis employer \texttt{xsl:sequence}
pour calculer (en \XPath) la valeur de \texttt{1 + \$x}. L'élément qui
définit une variable en \XSLT est \texttt{xsl:variable} et il a deux
formes possibles: soit avec un attribut, soit avec des enfants. Dans
le premier cas, nous avons le canevas
\begin{alltt}
\small<xsl:variable name="\(x\)" select="\(v\)" as="\(t\)">
\end{alltt}
et le dernier cas est
\begin{alltt}
\small<xsl:variable name="\(x\)" as="\(t\)">
  ... <!-- Enfants dont la valeur est \(v\) de type \(t\) -->
</xsl:variable>
\end{alltt}
où la valeur de la variable~\(x\) est~\(v\), de type~\(t\). La dualité
de la syntaxe est due encore au territoire délimité par \XPath:
si~\(v\) peut être calculée seulement avec \XPath, nous devrions
utiliser la première forme, sinon la seconde. Dans notre problème,
nous avons besoin de la seconde forme parce que~\(v\) est la valeur
d'un appel récursif qui n'est \emph{pas} exprimé en \XPath, puisque
nous utilisons \texttt{xsl:call-template}. Nous pouvons maintenant
compléter le programme: \smallXSLTinUnchecked{len0.xsl}
\noindent Le résultat obtenu en l'exécutant sur notre table des
matières est, comme nous nous y attendions:
\smallXMLinUnchecked{toc_out.xml}

Après que nous récupérons de l'effort soutenu et de la déception
engendrée par l'incroyable verbosité en comparaison avec \Erlang, nous
pourrions découvrir qu'il existe une fonction prédéfinie en \XPath,
nommée \texttt{count}, qui est, fondamentalement, une traduction de
\(\fun{len}_0/1\). Néanmoins, notre objectif est de nous adresser aux
débutants, donc l'utilité didactique prime toute autre considération.

Travaillons à une traduction d'une meilleure version de
\fun{len\(_0\)/1}:
\begin{equation*}
\begin{array}{@{}r@{\;}l@{\;}lr@{\;}l@{\;}l@{}}
\fun{len}_1(s) & \rightarrow & \fun{len}_1(s,0). &
\fun{len}_1(\el,n) & \rightarrow & n;\\
&&& \fun{len}_1(\cons{x}{s},n) & \rightarrow & \fun{len}_1(s,n+1).
\end{array}
\end{equation*}
La fonction \fun{len\(_1\)/1} est meilleure que \fun{len\(_0\)/1}
parce que son coût est identique à la seconde \emph{et} elle utilise
une quantité de mémoire constante, car elle est en forme terminale.
La forme terminale implique que nous n'avons pas besoin d'une
variable, parce que l'addition est effectuée en \XPath (le paramètre
de l'appel récursif):
\smallXSLTinUnchecked{len1.xsl}
\noindent Notons que nous n'avons pas eu à définir un patron
nommé pour \fun{len\(_0\)/1}.

Considérons une dernière variante où les chapitres, dans l'entrée,
sont tous des enfants d'un élément \texttt{contents} et leur noms sont
présents dans un attribut \texttt{title}, au lieu d'un nœud textuel:
\smallXMLinUnchecked{toc3.xml}
\noindent Bien entendu, la \DTD \texttt{book\_att.dtd} doit être
changée: \smallDTDin{book_att.dtd}
\noindent Pour résoudre ce problème, nous devons modifier une
transformation \XSLT précédente, pas penser à partir de notre langage
fonctionnel abstrait. Tout d'abord, nous devrions modifier l'appel au
patron de façon à sélectionner les chapitres là où ils se trouvent maintenant:
\begin{alltt}
\small  <xsl:template match="book" as="xs:integer">
    <xsl:call-template name="len3">
      <xsl:with-param name="elm" select="\textbf{contents/chapter}"
                                 as="element(chapter)*"/>
      <xsl:with-param name="n" select="0" as="xs:integer"/>
    </xsl:call-template>
  </xsl:template>
\end{alltt}
L'expression \texttt{contents/chapter} est une sélection en \XPath qui
signifie: «~Regrouper tous les enfants \texttt{contents} du nœud
contextuel (\texttt{book}), en préservant leur ordre relatif (ici, il
n'y en a qu'un), puis sélectionner tous les enfants nommés
\texttt{chapter} de tous ces nœuds, en préservant aussi leur ordre
relatif.~» À part cela, il n'y a pas besoin de changer le patron (sauf
son nom, maintenant \texttt{len3}). Notons aussi qu'utiliser des
attributs \texttt{title} n'a fait aucune différence.

Mais profitons de cette occasion pour effectuer de légères variations
et apprendre quelque chose de neuf. Supposons que nous souhaitions
employer le patron avec toutes sortes d'éléments, pas seulement
\texttt{chapter}, et que nous voulions faire usage d'un
\emph{paramètre avec défaut}. En effet, le type du paramètre
\texttt{chapters} du patron est \texttt{element(chapter)*}, donc il
n'est pas assez général. La solution est le type \texttt{element()*},
qui veut dire: «~Une séquence d'éléments, peut-être vide.~» De plus, la
valeur initiale du paramètre~\texttt{n} doit toujours être~\texttt{0},
donc nous pourrions faire de cette valeur un défaut en ajoutant un
attribut \texttt{select} à l'élément \texttt{xsl:param} correspondant:
\begin{alltt}
\small  <xsl:template name="len3" as="xs:integer">
    <xsl:param name="elm" as="\textbf{element()*}"/>
    <xsl:param name="n" as="xs:integer" \textbf{select="0"}/>
    ...
  </xsl:template>
\end{alltt}
En passant, nous avons renommé le paramètre en le neutre
\texttt{elm}. Bien sûr, l'appel du patron est maintenant plus court:
\begin{sverb}
  <xsl:template match="book" as="xs:integer">
    <xsl:call-template name="len3">
      <xsl:with-param name="elm" select="contents/chapter"
                                 as="element(chapter)*"/>
    </xsl:call-template>
  </xsl:template>
\end{sverb}
Notons qu'il est toujours possible d'imposer une valeur initiale
à~\texttt{n} qui ne serait pas~\texttt{0}. Par ailleurs, il est
loisible maintenant de réutiliser le patron \texttt{len3} pour
calculer la longueur de n'importe quelle séquence d'éléments. Au bout
du compte, la nouvelle transformation est
\smallXSLTinUnchecked{len3.xsl}

\mypar{Somme}

Étant donné une pile d'entiers, nous pouvons calculer leur somme comme suit:
\begin{equation*}
\begin{array}{@{}r@{\;}l@{\;}lr@{\;}l@{\;}l@{}}
\fun{sum}(\cons{x}{s}) & \rightarrow & \fun{sum}_0(\cons{x}{s},0). &
\fun{sum}_0(\el,n) & \rightarrow & n;\\
&&& \fun{sum}_0(\cons{x}{s},n) & \rightarrow & \fun{sum}_0(s,n+x).
\end{array}
\end{equation*}
Immédiatement, nous voyons qu'il n'y a qu'une petite différence entre
\fun{sum\(_0\)/2} et \fun{len\(_1\)/2}: au lieu d'ajouter~\(1\), nous
ajoutons~\(x\). Par conséquent, nous devrions nous attendre à une
modification minime du patron \XSLT correspondant. Supposons l'entrée suivante:
\smallXMLinUnchecked{sum.xml}
\noindent avec la \DTD
\smallDTDin{sum.dtd}
\noindent La modification que nous voudrions effectuer sur \texttt{len3} est
\begin{alltt}
\small        <xsl:call-template name="\textbf{sum}">
          <xsl:with-param name="elm" as="element()*"
                          select="\$elm[position()>1]"/>
          <xsl:with-param name="n" as="xs:integer"
                          select="\textbf{\$elm[1]} + \$n"/>
        </xsl:call-template>
\end{alltt}
Malheureusement, le compilateur \Saxon affiche l'avertissement suivant
concernant le changement:
\begin{sverb}
The only value that can pass type-checking is an empty sequence.
Required item type of value of parameter $n is xs:integer;
supplied value has item type xs:double
\end{sverb}
(«~La seule valeur qui peut être typée est une séquence vide. Le type
exigé de la valeur du paramètre \texttt{\$n} est \texttt{xs:integer};
la valeur fournie a le type \texttt{xs:double}.~») et le résultat
erroné \texttt{18135233274}, qui est la concaténation du contenu des
nœuds textuels des éléments \texttt{num}. Qu'est-ce qui s'est
passé? D'après le message, une chose est claire: le problème est lié
au typage, c'est pourquoi nous ne l'avons pas anticipé à partir de
\fun{sum\(_0\)/2}, qui n'est pas typée. Il est clair aussi que le
compilateur comprend que \texttt{\$n} est un entier, donc le coupable
ne peut être que notre modification, \texttt{\$elm[1]}. Nous aimerions
qu'elle soit de type \texttt{xs:integer} aussi, mais l'est-elle
vraiment? Le type de \texttt{\$elm} est \texttt{element()*}, comme
déclaré, ce qui signifie que les items qu'elle contient sont des
éléments, pas des entiers, d'où le problème. Nous devons forcer le
type de \texttt{\$elem[1]} à être \texttt{xs:integer}, c'est-à-dire,
nous le \emph{transtypons} (anglais, \emph{cast}). D'abord, nous
devons sélectionner le nœud textuel de \texttt{\$elem[1]} et
convertir son type en utilisant \texttt{xs:integer} comme une
\emph{fonction} \XPath:
\texttt{xs:integer(\$elm[1]/text()\!)}. L'avertissement disparaît:
\begin{sverb}
18135233274
\end{sverb}
Le résultat est toujours erroné, ceci dit. Il est temps de comprendre
pourquoi! Clairement, il est fait à partir de tous les nœuds
textuels, en ordre documentaire et sérialisés sans séparation. Nous
savons depuis le début que, par défaut, c'est ce que \XSLT fait, donc
nous avons échoué à spécifier ce que \emph{nous} voulions. Un coup
d'œil en arrière au premier appel à \texttt{sum} révèle:
\begin{sverb}
  <xsl:template match="book" as="xs:integer">
    <xsl:call-template name="sum">
      <xsl:with-param name="elm" select="contents/chapter"
                                 as="element(chapter)*"/>
    </xsl:call-template>
  </xsl:template>
\end{sverb}
Parce que nous n'avons pas d'éléments \texttt{chapter} dans l'entrée
maintenant, une séquence vide est sélectionnée par
\texttt{contents/chapter}. Ce devrait être \texttt{num}. Mais ceci ne
change pas le faux résultat. La raison est qu'il n'y a pas de nœud
contextuel car il n'y a pas d'élément \texttt{book} dans le
document. Par conséquent, nous devrions écrire:
\begin{sverb}
  <xsl:template match="numbers" as="xs:integer">
    <xsl:call-template name="sum">
      <xsl:with-param name="elm" select="num"
                                 as="element(num)*"/>
    </xsl:call-template>
  </xsl:template>
\end{sverb}
Cette fois-ci, le résultat correct s'imprime:
\smallXMLinUnchecked{sum_out.xml}
\noindent Il reste encore une erreur subtile, qui devient apparente
quand on fournit la séquence vide en entrée. (Nous recommandons de
toujours tester les programmes avec des valeurs extrêmes de l'entrée.)
En effet, le résultat est alors~\texttt{0}, ce qui n'est pas ce que
nous attendons si nous considérons la fonction abstraite \fun{sum/1}
comme une spécification.
\begin{equation*}
\fun{sum}(\cons{x}{s}) \rightarrow \fun{sum}_0(\cons{x}{s},0).
\end{equation*}
En \XSLT, nous avons oublié d'interdire la séquence vide. Ceci est
réalisé en spécifiant un type «~séquence non-vide d'éléments~»:
\texttt{element()+}.
\begin{alltt}
\small  <xsl:template match="numbers" as="xs:integer">
    <xsl:call-template name="sum">
      <xsl:with-param name="elm" select="num"
                                 as="\textbf{element(num)+}"/>
    </xsl:call-template>
  </xsl:template>
\end{alltt}
Si nous essayons l'entrée
\smallXMLinUnchecked{empty_sum.xml}
\noindent nous obtenons l'erreur attendue
\begin{sverb}
An empty sequence is not allowed as the value of parameter $elm
\end{sverb}
(«~Une séquence vide ne peut être la valeur du paramètre
\texttt{\$elm}~») La transformation est complète maintenant:
\smallXSLTinUnchecked{sum.xsl}

Si nous préférions n'avoir aucun résultat, plutôt qu'un message
d'erreur à l'exécution, nous pourrions vérifier la vacuité avant le
premier appel et ne rien faire. Mais il existe un raccourci, qui peut
être aisément saisi sur le programme abstrait:
\begin{equation*}
\begin{array}{@{}r@{\;}l@{\;}lr@{\;}l@{\;}l@{}}
\fun{sum}_1(s) & \rightarrow & \fun{sum}_2(s,0). &
    \fun{sum}_2(\el,n) & \rightarrow & \fun{nothing}();\\
&&& \fun{sum}_2([x],n) & \rightarrow & x+n;\\
&&& \fun{sum}_2(\cons{x}{s},n) & \rightarrow & \fun{sum}_2(s,x+n).
\end{array}
\end{equation*}
Le constructeur de données quand la pile est vide,
\(\fun{nothing}()\), sera traduit en \XSLT comme un élément vide:
\begin{alltt}
\small      \textbf{<xsl:when test="empty(\$elm)"/>}
      <xsl:when test="empty(\$elm[2])">
        <xsl:sequence select="xs:integer(\$elm[1]/text()) + \$n"/>
      </xsl:when>
\end{alltt}
Le cas de la séquence singleton est \texttt{empty(\$elm[2])}. En
effet, nous savons que \texttt{\$elm} n'est pas vide, parce que cela
est le cas précédent; donc, tout ce que nous avons à faire est
vérifier l'existence de \texttt{\$elm[2]}: si absent, cet sélection
donne la séquence vide et, puisque nous savons que \texttt{\$elm[1]}
existe, la séquence \texttt{\$elm} contient exactement un élément. Il
reste encore un problème avec les types: le patron filtrant
\texttt{numbers} et le patron nommé \texttt{sum} doivent retourner une
valeur de type \texttt{xs:integer}, ce qui n'est pas possible si
\texttt{\$elm} est vide, auquel cas, comme nous venons de le
constater, une séquence vide est retournée (à cause de l'élément vide
\texttt{xsl:when}). On peut exprimer en \XPath le type «~Une séquence
sans item ou exactement un.~» en utilisant l'opérateur «~\texttt{?}~». Si
nous nous souvenons qu'une valeur peut toujours être implicitement
convertie en une séquence contenant ladite valeur, alors
\texttt{xs:integer?} veut dire: «~Un entier ou une séquence vide.~» Par
conséquent, \smallXSLTinUnchecked{sum1.xsl}

\mypar{Filtrage}

Nous voulons copier une pile donnée, sans son dernier item. Une façon
de procéder est de vérifier d'abord si la pile contient zéro, un ou au
moins deux items. Dans les deux premiers cas, le résultat est la pile
vide; dans le dernier, nous savons que le premier item n'est pas le
dernier, donc nous le conservons et continuons récursivement avec le
reste:
\begin{equation*}
\fun{cutl}(\cons{x,y}{s}) \rightarrow \cons{x}{\fun{cutl}(\cons{y}{s})};\quad
\fun{cutl}(s) \rightarrow \el.
\end{equation*}
En \Erlang, ceci serait mis en œuvre comme suit (en-tête omis):
\begin{verbatim}
cutl([X|S=[_|_]]) -> [X|cutl(S)];
cutl(_)           -> [].
\end{verbatim}
Pour voir comment l'exprimer en \XSLT, nous devons d'abord établir le
contexte d'utilisation. Par exemple, disons que nous avons une table
des matières qui se conforme à la \DTD suivante, \texttt{book\_bis.dtd}:
\smallDTDin{book_bis.dtd}
\noindent Par exemple, l'entrée pourrait être
\smallXMLinUnchecked{toc_bis.xml}
\noindent Nous souhaitons une copie de ce document \XML sans le
dernier chapitre:
\smallXMLinUnchecked{toc_bis1_out.xml}
\noindent Ceci est la première fois que nous utilisons \XSLT pour
produire du \XML, jouant le rôle d'un filtre exclusif, c'est-à-dire
excluant une partie de la donnée tout en laissant passer le
reste. Nous pouvons commencer par réutiliser du code de
transformations précédentes et ensuite travailler à la traduction de
la fonction \fun{cutl/1} en \XSLT, qui sera un patron nommé
\texttt{cutl}. Mais, tout d'abord, le menu habituel et une surprise du chef:
\begin{alltt}
\small<?xml version="1.0" encoding="UTF-8"?>

<xsl:transform version="2.0"
               xmlns:xsl="http://www.w3.org/1999/XSL/Transform"
               xmlns:xs="http://www.w3.org/2001/XMLSchema">
               \textbf{exclude-result-prefixes="xs"}>

  \textbf{<xsl:output method="xml" version="1.0"
              encoding="UTF-8" indent="yes"/>}
\end{alltt}
Remarquons que la nature du résultat n'est plus \texttt{text}, mais
\texttt{xml}, puisque que nous voulons produire du \XML. Bien entendu,
nous avons alors besoin de dire quelle version de \XML nous souhaitons
(ici, \texttt{1.0}), ce que l'encodage du fichier sera (ici,
\texttt{UTF-8}), et si nous voulons que le \XML résultant soit indenté
(oui, parce que cela accroît grandement sa lisibilité, mais s'il
devait être lu par un autre programme \XSLT, l'indentation pourrait
être abandonnée). Une autre nouveauté est l'attribution
\texttt{exclude-result-prefixes="xs"} dans
\texttt{xsl:transform}. Pour plus de clarté, nous y reviendrons après
en avoir terminé avec le reste. Nous pouvons continuer maintenant avec
le reste du canevas:
\begin{alltt}
\small  <xsl:template match="/">
    <xsl:apply-templates/>
    <xsl:text>&#10;</xsl:text>
  </xsl:template>

  <xsl:template match="book" as="element(book)">
    \textbf{<xsl:copy>}
      <xsl:sequence select="author"/>
      <xsl:sequence select="title"/>
      <contents>
        <xsl:call-template name="cutl">
          <xsl:with-param name="items" select="contents/chapter"
                          as="element(chapter)*"/>
        </xsl:call-template>
      </contents>
    \textbf{</xsl:copy>}
  </xsl:template>

  <xsl:template name="cutl" as="item()*">
    <xsl:param name="items" as="item()*"/>
    ...
  </xsl:template>

</xsl:transform>
\end{alltt}
Nous avons employé une police grasse pour mettre en avant le nouvel
élément \XSLT \texttt{xsl:copy}. Peut-être nous attendions-nous à voir
l'élément \texttt{<book>...</book>}, et ceci aurait été correct, en
effet. Mais nous pourrions préférer ne pas copier le nom de l'élément
trop souvent, au cas où il changerait dans de futures versions. C'est
alors que \texttt{xsl:copy} se révèle utile: il effectue une
\emph{copie superficielle du nœud contextuel}. Le nœud
contextuel est le dernier nœud qui a été filtré (attribut
\texttt{match}) par un élément \texttt{xsl:template}, donc il s'agit
ici de \texttt{book}, et «~superficielle~» veut dire que les enfants ne
sont pas copiés (nous voulons une copie mais aussi modifier les
descendants).

Par ailleurs, remarquons que nous avons employé \texttt{xsl:sequence}
pour la sélection d'éléments de l'entrée (\texttt{author} et
\texttt{title}). C'est ici que l'élément \texttt{xsl:sequence} montre
sa réelle valeur: ce que cet élément fait est \emph{se référer aux
  éléments sélectionnés}, sans les copier. En ce sens, il se comporte
comme un pointeur, comme on en trouve dans certains langages de
programmation impératifs, tel \Clang, et il économise ainsi de la
mémoire. Enfin, remarquons comment la sortie est construite en
recréant un document \XML; en particulier, la juxtaposition d'éléments
dénote la concaténation des séquences singletons qu'ils sont (par
exemple, les deux \texttt{xsl:sequence} mentionnés plus haut sont
écrits l'un après l'autre).

Maintenant, nous avons besoin de traduire \fun{cutl/1}. Comme nous le
savons déjà, \XSLT n'offre pas le filtrage par motifs, donc nous
devons réécrire notre programme fonctionnel abstrait sans cela:
\begin{equation*}
  \fun{cutl}(t) \rightarrow \fun{if} \; \fun{tl}(t) \neq \el  \;
  \fun{then} \; \cons{\fun{hd}(t)}{\fun{cutl}(\fun{tl}(t))} \; \fun{else} \; \el.
\end{equation*}
où \(\fun{hd}(t)\) (anglais, \emph{head}) \index{head@\fun{hd/1}} a
pour valeur le premier item de la pile~\(t\) et \(\fun{tl}(t)\)
(anglais, \emph{tail})
\index{tail@\fun{tl/1}} la sous-pile immédiate de~\(t\). (Bien sûr,
\(\fun{hd}(\el)\) et \(\fun{tl}(\el)\) échoueraient, donc nous
devons toujours nous assurer que leur argument n'est pas la pile
vide.) Remarquons les deux occurrences de \(\fun{tl}(t)\), donc, en
\XSLT, nous devrions utiliser une variable pour contenir la valeur de
cet appel pour ne pas la recalculer. Nous commençons ainsi:
\begin{alltt}
\small  <xsl:template name="cutl" as="item()*">
    <xsl:param name="items" as="item()*"/>
    \textbf{<xsl:variable name="tail" select="\$items[position()>1]"
                              as="item()*"/>}
    ...
  </xsl:template>
\end{alltt}
Notons que nous n'avons pas spécialisé le patron pour traiter
seulement des éléments \texttt{chapter}, mais toute espèce d'item,
même les types primitifs comme les entiers, mais aussi les nœuds
et, en particulier, les éléments.

Maintenant, nous devons traduire la conditionnelle. Nous avons déjà vu
l'élément \texttt{xsl:choose} et nous pouvons donc poursuivre en
comblant un peu l'ellipse précédente:
\begin{alltt}
\small    <xsl:choose>
      <xsl:when test="\textbf{not(empty(\$tail))}"> ... </xsl:when>
      <xsl:otherwise> ... </xsl:otherwise>
    </xsl:choose>
\end{alltt}
Le but des fonctions \XPath \texttt{empty} et \texttt{not} est
évident. La traduction de la branche \fun{else} est la séquence vide
dans l'élément \texttt{xsl:otherwise}. Ceci est aisément réalisé sans
même l'élément \texttt{xsl:sequence}:
\begin{alltt}
\small    <xsl:choose>
      <xsl:when test="not(empty(\$tail))"> ... </xsl:when>
      \textbf{<xsl:otherwise/>}
    </xsl:choose>
\end{alltt}
En effet, un élément vide peut toujours être considéré comme ayant une
séquence vide d'enfants. En \XSLT, les conditionnelles qui ont la
forme d'un \texttt{xsl:when} et un \texttt{xsl:otherwise} vide sont
mieux exprimées en usant de l'élément \texttt{xsl:if}. Par exemple,
notre code devient:
\begin{alltt}
\small    \textbf{<xsl:if test="not(empty(\$tail))"> ... </xsl:if>}
\end{alltt}
Implicitement, si le test échoue, la valeur de la conditionnelle
\texttt{xsl:if} est la séquence vide. Nous avons besoin ensuite de
traduire
\(\cons{\fun{hd}(t)}{\fun{cutl}(\fun{tl}(t))}\).\index{head@\fun{hd/1}}\index{tail@\fun{tl/1}}
Nous avons déjà à notre disposition la traduction de \(\fun{tl}(t)\),
qui est la variable \XSLT \texttt{tail}. La traduction de
\(\fun{hd}(t)\) est simplement la séquence singleton
\texttt{<xsl:sequence select="\$items[1]"/>}. Au lieu d'empiler, nous
mettons bout à bout deux séquences et cette concaténation est
simplement la juxtaposition textuelle:
\begin{alltt}
\small    <xsl:if test="not(empty(\$tail))">
      <xsl:sequence select="\$items[1]"/>
      <xsl:call-template name="cutl">
        \!<xsl:with-param name="items" select="\$tail" \!as="item()*"/>
      </xsl:call-template>
    </xsl:if>
\end{alltt}
Au bout du compte, la solution est
\smallXSLTinUnchecked{last.xsl}

Arrivés à ce point, nous pourrions nous demander pourquoi nous avons
besoin de l'attribution
\texttt{exclude-result-prefixes="xs"} à l'élément
\texttt{xsl:transform}. Si nous l'ôtions, nous obtiendrions le même
résultat, sauf en ce qui concerne l'élément \texttt{contents}:
\begin{alltt}
\small   <contents \textbf{xmlns:xs="http://www.w3.org/2001/XMLSchema"}>
     ...
   </contents>
\end{alltt}
La raison est que quand un espace de noms est déclaré, tous les
éléments descendants l'héritent, sauf l'espace de noms associé à
\XSLT, ici nommé \texttt{xsl}. Par conséquent, quand nous écrivions
\begin{sverb}
   <contents>
     ...
   </contents>
\end{sverb}
dans la transformation précédente, l'élément \texttt{contents} avait
\emph{implicitement} un enfant qui était le nœud d'espace de
noms~\texttt{xs}. La raison pour laquelle \texttt{author} et
\texttt{title} n'en avaient pas est que nous avions utilisé
\texttt{xsl:sequence} pour référencer l'entrée, où l'espace de noms
est absent. La même chose se produit avec les éléments
\texttt{chapter}, qui sont sélectionnés dans l'entrée. L'élément
\texttt{book} était en fait copié au moyen de \texttt{xsl:copy}, et
nous avons vu que cet élément ne recopie pas les enfants, parmi
lesquels se trouvent les nœuds d'espace de noms. Le comportement
par défaut du processeur \XSLT est de faire figurer les espaces de
noms hérités au cas où ils seraient utiles dans le résultat. Dans
l'exemple présent, \texttt{xs} est inutile, donc il vaut mieux
l'exclure des préfixes (d'espaces de noms) dans le résultat:
\texttt{exclude-result-prefixes="xs"}.

\paragraph{Filtrer en excluant le pénultième item}

Le but de cet exercice est d'écrire une transformation \XSLT qui prend
en entrée une table des matières et produit la même table en \XML où
le pénultième chapitre est absent. S'il n'y a aucun chapitre, ou
seulement un, le résultat est identique à l'entrée. L'entrée doit se
conformer à la \DTD suivante, nommée \texttt{book\_bis.dtd}:
\smallDTDin{book_bis.dtd}
\noindent Par exemple, la donnée pourrait être
\smallXMLinUnchecked{toc_bis.xml}
\noindent Le résultat correspondant est
\smallXMLinUnchecked{toc_bis2_out.xml}
\noindent La ritournelle \XSLT est la même que d'habitude, sauf le
patron, que nous nommons ici \texttt{cutp}. Nous ne débutons pas avec
un programme fonctionnel abstrait, mais avec la transformation
précédente. Nous aurons besoin de plus de cas, donc
\texttt{xsl:choose} est de retour. Peut-être que la première
différence est le cas où \texttt{tail} est vide. Ceci signifie que
nous avons besoin de conserver le premier item, au lieu de l'écarter:
\begin{sverb}
    <xsl:choose>
      <xsl:when test="empty($tail)">
        <xsl:sequence select="$items[1]"/>
      </xsl:when>
      ...
    </xsl:choose>
\end{sverb}
En ce qui concerne le cas complémentaire, quand la sous-pile immédiate
n'est pas vide, c'est-à-dire quand il y a au moins deux items, nous ne
savons pas si le premier est le pénultième ou non, et il en est de
même quant au deuxième. Par conséquent, nous avons besoin de plus
d'information sur la structure de la sous-pile immédiate, en
particulier si sa sous-pile immédiate est, à son tour, vide (la
sous-pile immédiate de la sous-pile immédiate de la séquence
complète), autrement dit, si la séquence contient au moins trois items
ou non. Si c'est le cas, alors nous savons que le premier item n'est
pas le pénultième, mais nous ne pouvons encore rien dire sur les
autres, donc un appel récursif s'impose; si la séquence contient moins
de trois items, elle en contient exactement deux, donc nous plaçons
dans le résultat le second uniquement, et nous ignorons le premier. Au
bout du compte, nous avons
\begin{sverb}
  <xsl:template name="cutp" as="item()*">
    <xsl:param name="items" as="item()*"/>
    <xsl:variable name="tail" select="$items[position()>1]"
                              as="item()*"/>
    <xsl:choose>
      <xsl:when test="empty($tail)">
        <xsl:sequence select="$items[1]"/>
      </xsl:when>
      <xsl:when test="empty($tail[position()>1])">
        <xsl:sequence select="$items[2]"/>
      </xsl:when>
      <xsl:otherwise>
        <xsl:sequence select="$items[1]"/>
        <xsl:call-template name="cutp">
          <xsl:with-param name="items" select="$tail"
                                       as="item()*"/>
        </xsl:call-template>
      </xsl:otherwise>
    </xsl:choose>
  </xsl:template>
\end{sverb}
Remarquons que le cas \texttt{<xsl:when test="empty(\$items)"/>} est
en fait absent parce qu'il n'est pas nécessaire: si \texttt{\$items}
est vide, alors \texttt{\$tail} est vide aussi, et le résultat est
alors \texttt{\$items[1]}, qui n'est autre que la séquence vide.

\mypar{Retournement}

L'objectif de cet exercice est d'écrire une transformation \XSLT qui
prend en entrée une table des matières avec des chapitres et produit
la même table en \XML où l'ordre des chapitres a été inversé par
rapport à l'ordre documentaire (donc, par exemple, l'introduction
apparaît en dernier). À la section~\vref{sec:reversal}, nous avons vu
la définition directe de \fun{rev\(_0\)}:
\begin{equation*}
\begin{array}{@{}r@{\;}l@{\;}lr@{\;}l@{\;}l@{}}
  \fun{cat}(\el,t)\index{cat@\fun{cat/2}}
& \rightarrow & t;
& \fun{rev}_0(\el)
& \rightarrow & \el;\\
  \fun{cat}(\cons{x}{s},t)
& \rightarrow & \cons{x}{\fun{cat}(s,t)}.
& \fun{rev}_0(\cons{x}{s})
& \rightarrow & \fun{cat}(\fun{rev}_0(s),[x]).
\end{array}
\end{equation*}
Nous avons vu que cette définition engendrait un coût quadratique et
donc ne devrait pas être employée pour retourner des piles. En \XSLT,
le coût est linéaire parce que la concaténation a un coût nul. Nous
pourrions donc écrire la traduction suivante:
\begin{sverb}
  <xsl:template name="rev" as="item()*">
    <xsl:param name="items" as="item()*"/>
    <xsl:if test="not(empty($items))">
      <xsl:call-template name="rev">
        <xsl:with-param name="items" as="item()*"
                        select="$items[position()>1]"/>
      </xsl:call-template>
      <xsl:sequence select="$items[1]"/>
    </xsl:if>
  </xsl:template>
\end{sverb}
Au lieu de produire un document \XML, nous pourrions profiter de cette
occasion pour voir comment produire un document \XHTML. Bien que le
but pourrait sembler un peu sot (retourner une table des matières), il
convient bien à l'apprentissage de langages compliqués tels que \XSLT
et \XHTML.

Pour qu'un processeur \XSLT produise du \XHTML, nous devons faire les
attributions suivantes à \texttt{xsl:transform} et
\texttt{xsl:output}:
\begin{sverb}
<xsl:transform version="2.0"
               xmlns:xsl="http://www.w3.org/1999/XSL/Transform"
               xmlns:xs="http://www.w3.org/2001/XMLSchema"
               xmlns:xhtml="http://www.w3.org/1999/xhtml"
               exclude-result-prefixes="xs">

  <xsl:output method="xhtml"
              doctype-public="-//W3C//DTD XHTML 1.0 Strict//EN"
              doctype-system=
              "http://www.w3.org/TR/xhtml1/DTD/xhtml1-strict.dtd"
              indent="yes"
              omit-xml-declaration="yes"/>
\end{sverb}
Notons que nous avons défini un espace de noms \texttt{xhtml} pour les
éléments \XHTML et que la version de \XHTML est~\texttt{1.0} («~strict~»
signifie que cette version de \XHTML est du \XML). Il se peut que la
véritable nouveauté soit l'attribution
\texttt{omit-xml-declaration="yes"}\!. Puisque \XHTML (strict) est
\XML, la déclaration \texttt{<?xml version="1.0"?>} est attendue, mais
certains navigateurs sont déroutés par ceci, donc nous préférons ne
pas inclure cette déclaration par prudence.

Étant donnée la table des matières précédente, nous voudrions
maintenant obtenir
\begin{sverb}
<!DOCTYPE html
  PUBLIC "-//W3C//DTD XHTML 1.0 Strict//EN"
         "http://www.w3.org/TR/xhtml1/DTD/xhtml1-strict.dtd">
<html xmlns="http://www.w3.org/1999/xhtml"
      xml:lang="en" lang="en">
   <head>
      <meta http-equiv="Content-Type"
            content="text/html; charset=UTF-8" />
      <title>Definitive XML Schema</title>
   </head>
   <body>
      <h2>Definitive XML Schema</h2>
      <p>by Priscilla Walmsley</p>
      <h3>Reversed table of contents</h3>
      <ul>
         <li>Instances and schemas</li>
         <li>Schema composition</li>
         <li>Namespaces</li>
         <li>A quick tour of XML Schema</li>
         <li>Schema: An Introduction</li>
      </ul>
   </body>
</html>
\end{sverb}
qui, interprété par un navigateur, est très probablement rendu ainsi:

\noindent\rule{\linewidth}{0.5pt}
\noindent\textbf{\large Definitive XML Schema}

\bigskip

\noindent by Priscilla Walmsley

\bigskip

\noindent\textbf{Reversed table of contents}
\begin{itemize}

  \item Instances and schemas
  \item Schema composition
  \item Namespaces
  \item A quick tour of XML Schema
  \item Schema: An Introduction

\end{itemize}
\noindent\rule{\linewidth}{0.5pt}

\noindent Tout d'abord, voici le patron filtrant \texttt{book}:
\begin{alltt}
\small  <xsl:template match="book" as="element(xhtml:html)">
    <html xmlns="http://www.w3.org/1999/xhtml"
          xml:lang="en" lang="en">
      <head>
        <title><xsl:sequence select="\textbf{title/text()}"/></title>
      </head>
      <body>
        <h2>\textbf{<xsl:value-of select="title"/>}</h2>
        <p>by \textbf{<xsl:value-of select="author"/>}</p>
        <h3>Reversed table of contents</h3>
        <ul>
          <xsl:call-template name="rev">
            <xsl:with-param name="chap"
                            select="contents/chapter"/>
          </xsl:call-template>
        </ul>
      </body>
    </html>
  </xsl:template>
\end{alltt}
Remarquons premièrement la sélection \texttt{title/text()}, qui veut
dire: «~Les nœuds textuels de l'élément \texttt{title}, qui est
l'enfant du nœud contextuel (\texttt{book}).~» Deuxièmement, nous
faisons connaissance avec un nouvel élément \XSLT,
\texttt{xsl:value-of}, dont l'objet est de créer un nœud textuel à
partir des items sélectionnés. Si nous sélectionnions des éléments,
comme ici l'unique élément \texttt{title}, ses nœuds textuels
\emph{descendants} (il n'y a qu'un enfant ici) seraient concaténés en
ordre documentaire et mis dans un nouveau nœud textuel. Par
conséquent, \texttt{<xsl:sequence select="title/text()"/>} produit le
même effet que \texttt{<xsl:value-of select="title"/>}, bien que dans
ce dernier un nouveau nœud textuel soit alloué (au lieu d'être
partagé avec l'entrée). Notons par ailleurs que nous avons décidé de
spécialiser les types pour les faire coller le plus possible aux
éléments à traiter, comme \texttt{element(xhtml:html)} qui signifie:
«~Un élément \texttt{html} dans l'espace de noms \texttt{xhtml}.~» Le
retournement est effectué par le patron nommé \texttt{rev}:
\begin{alltt}
\small  <xsl:template name="rev" as="element(xhtml:li)*">
    <xsl:param name="chap" as="element(chapter)*"/>
    <xsl:if test="not(empty(\$chap))">
      <xsl:call-template name="rev">
        <xsl:with-param name="chap"\!\! select="\$chap[position()>1]"/>
      </xsl:call-template>
      <li \textbf{xmlns="http://www.w3.org/1999/xhtml"}>
        \textbf{<xsl:value-of select="\$chap[1]/@title"/>}
      </li>
    </xsl:if>
  </xsl:template>
\end{alltt}
À nouveau, nous avons spécialisé les types, comme
\texttt{element(xhtml:li)*} dénotant une «~séquence (éventuellement
vide) d'éléments \texttt{li} dans l'espace de noms \texttt{xhtml}.~» Et
\texttt{element(chapter)*} est une séquence de chapitres, sans espace
de noms.

Mais les deux extraits vraiment intéressants sont en gras.

Le premier est la déclaration de l'espace de noms \texttt{xhtml} à
l'élément \texttt{li}:
\texttt{xmlns="http://www.w3.org/1999/xhtml"}. Ceci est tout
simplement nécessaire pour nous conformer au type de la valeur du
patron. En effet, cette valeur devrait être, comme nous l'avons vu,
une séquence d'éléments \texttt{li} dans l'espace de noms
\texttt{xhtml}. Mais \texttt{<li>} est en fait hors de tout espace de
noms, parce qu'il n'y a aucune déclaration par défaut d'espace de noms
dans aucun ascendant, contrairement au patron filtrant \texttt{book},
que nous avons décrit précédemment. Là-bas, nous avions la déclaration
\texttt{<html xmlns="http://www.w3.org/1999/xhtml" xml:lang="en"
  lang="en">}, donc tous les éléments descendants écrits sans un
espace de noms héritaient en réalité l'espace \texttt{xhtml}. Ce n'est
pas le cas avec le patron nommé \texttt{rev}, donc une déclaration
explicite est nécessaire, sinon une erreur de type est notifiée par le
compilateur \XSLT.

Le second extrait digne d'attention est la sélection
\texttt{\$chap[1]/@title} par l'élément \texttt{xsl:value-of} qui veut
dire: «~L'attribut \texttt{title} du premier élément de la séquence
\texttt{\$chap}.~» Ici, nous ne pouvons substituer
\texttt{xsl:sequence}, comme avec le nœud textuel avant. En effet,
si nous essayons à la place \texttt{<xsl:sequence
  select="\$chap[1]/@title"/>}, le résultat est
\begin{sverb}
       ...
         <li title="Instances and schemas"></li>
         <li title="Schema composition"></li>
         <li title="Namespaces"></li>
         <li title="A quick tour of XML Schema"></li>
         <li title="Schema: An Introduction"></li>
       ...
\end{sverb}
Souvenons-nous que \texttt{xsl:sequence} est un alias de l'entrée (une
référence à celle-ci), en ce cas un nœud attribut, donc nous
devrions nous attendre à un \emph{attribut} \texttt{title} dans le
résultat. Mais nous voulions la valeur de l'attribut \texttt{title},
pas l'attribut lui-même, d'où le nécessaire \texttt{xsl:value-of}. Si
nous nous demandions pourquoi nous avons réellement besoin de
fabriquer un nœud textuel, nous devrions comprendre que \emph{la
  valeur d'un attribut n'est pas un nœud textuel}. Ceci peut être
vu en changeant la sélection en \texttt{<xsl:sequence
  select="\$chap[1]/title/text()"/>}, auquel cas, le résultat est
\begin{sverb}
       ...
         <li></li>
         <li></li>
         <li></li>
         <li></li>
         <li></li>
       ...
\end{sverb}
Les attributs sont spéciaux et souvent une source de confusion pour
les débutants.

\mypar{Valeurs ponctuées}
\label{par:CSV}

Le but de cet exercice est d'écrire une transformation \XSLT qui prend
en entrée une séquence d'éléments contenant chacun un nœud textuel
et produit leur contenus dans le même ordre, séparés par des virgules
et terminés par un point final. Si la séquence d'entrée est vide, le
résultat est la séquence vide. Plus précisément, supposons la \DTD suivante:
\smallDTDin{csv.dtd}
\noindent et la donnée conforme
\smallXMLinUnchecked{csv.xml}
\noindent Alors nous voulons le résultat
\smallXMLinUnchecked{csv_out.xml}

L'algorithme est assez simple: si la séquence donnée est vide, le
résultat est la séquence vide; si la donnée est une séquence
singleton, le résultat est l'item qu'elle contient, suivi par un
point; sinon, le premier item du résultat est le premier de la donnée,
suivi par une virgule et la valeur de l'appel récursif sur la
sous-séquence immédiate. La difficulté est plutôt de mettre en
œuvre ce schéma en \XSLT. Voici comment:
\smallXSLTinUnchecked{csv1.xsl}
\noindent Remarquons que nous avons fusionné les deux patrons qui
filtrent la racine du document (\texttt{/}) et l'élément racine
(\texttt{numbers}) parce que nous ne reconstruisons pas un document
\XML. De plus, nous pourrions remarquer le type \texttt{text()*}, qui
veut dire: «~Une séquence (éventuellement vide) de nœuds textuels.~»
Les autres bons morceaux sont les éléments \texttt{xsl:value-of}, en
particulier le nouvel attribut \texttt{separator}. Sa valeur doit être
une chaîne qui est employée pour séparer les items sélectionnés. Par
défaut, cette chaîne est \texttt{'\textvisiblespace'}, c'est pourquoi
nous l'initialisons ici avec la chaîne vide. Sinon, nous obtiendrions
«~\texttt{0\textvisiblespace,1\textvisiblespace,A\textvisiblespace,B\textvisiblespace,C\textvisiblespace.}~». Remarquons
que la valeur de l'attribut \texttt{separator} est le \emph{contenu}
d'une chaîne, donc si nous utilisions \texttt{"''"}, nous ne serions
\emph{pas} en train de spécifier une chaîne vide et nous produirions à
la place: «~\texttt{0'',1'',A'',B'',C''.}~».

Parvenus à ce point, il peut être pertinent de tirer au clair les
relations entre les différents types que nous avons rencontrés, et d'en
apprendre quelques autres encore. Considérons l'arbre dans la \fig~\vref{fig:tests}.
\begin{figure}
\includegraphics{tests}
\caption{Sous-types \XPath}
\label{fig:tests}
\end{figure}
La lecture ascendante d'un arc à partir d'un nœud~\(x\) vers un
nœud~\(y\) est: «~\(x\)~est un [sous-type de] \(y\).~» Par exemple,
un \texttt{element()} est un \texttt{node()}. Ceci modélise une
relation de sous-typage, ce qui signifie que dans tout contexte où un
\texttt{node()} est correct, un \texttt{element()} l'est aussi. Cette
relation est transitive, donc partout où un \texttt{item()} est
attendu, un \texttt{text()} est valide, ce qui peut être constaté avec
le paramètre du patron nommé \texttt{csv}. À des fins d'illustration,
examinons une petite variante de l'entrée, où le contenu qui nous
intéresse est placé dans les valeurs d'attributs, comme il suit:
\smallXMLinUnchecked{csv_att.xml}
\noindent La \DTD \texttt{csv\_att.dtd} est
\smallDTDin{csv_att.dtd}
\noindent Il suffit alors de changer la sélection du paramètre
\texttt{item} ainsi:
\begin{alltt}
\small      <xsl:with-param name="items" select="numbers/hexa/\textbf{@val}"/>
\end{alltt}
Le type \texttt{attribute()*} de la sélection est un sous-type de
\texttt{item()*}, donc une valeur acceptable pour le paramètre du
patron \texttt{csv} et le résultat est le même qu'avec la première
donnée \XML (sans attributs).

Pendant que nous sommes en train d'étudier les types de données,
portons notre attention sur les sélections des éléments
\texttt{xsl:value-of}, par exemple
\texttt{(\$items[1],'.')}. Statiquement, le type de
\texttt{\$items[1]} est \texttt{item()}, bien que nous sachions,
d'après l'appel du patron initial, qu'il est en fait \texttt{text()}.
Le type de \texttt{'.'} est \texttt{xs:string}. La séquence en
question a donc le type \texttt{item()*}, car \texttt{xs:string} est
un sous-type de \texttt{item()}, comme on peut le voir à la
\fig~\vref{fig:tests}. Puisque le résultat du patron a pour type
\texttt{text()*}, les chaînes qu'il contient seront transtypées en
nœuds textuels, impliquant allocation de mémoire. Le sérialiseur
est la partie finale du code produit par le compilateur \XSLT (la
partie frontale est l'analyseur de syntaxe \XML) et sa raison d'être
est d'engendrer du texte à partir de toutes les valeurs obtenues. Dans
ce cas, il va alors déstructurer tous ces nœuds textuels pour la
génération de chaînes (affichées par un terminal ou écrites dans un
fichier). Si nous souhaitions éviter cet emballage de chaînes (sous
la forme de nœuds textuels), puis leur déballage, nous pourrions
anticiper et choisir un type de retour \texttt{xs:string*}, donc
\begin{alltt}
\small  <xsl:template match="/" as="\textbf{xs:string*}">
    ...
  </xsl:template>

  <xsl:template name="csv" as="\textbf{xs:string*}">
    ...
  </xsl:template>
\end{alltt}
L'effet de cette modification est l'obtention de
«~\texttt{0,\textvisiblespace{}1,\textvisiblespace{}A,\textvisiblespace{}B,\textvisiblespace{}C.}~». Les
espaces supplémentaires viennent du fait que nous avons oublié que les
éléments \texttt{xsl:value-of} créent des nœuds textuels et la
sérialisation implicite de ceux-ci (par transtypage en
\texttt{xs:string*}) engendre des séparateurs invisibles, ici rendus
par «~\texttt{\textvisiblespace}~». La morale de cette aventure est
qu'il vaut mieux travailler avec des types non-primitifs, soit
attributs et nœuds, si l'entrée contient des attributs et des
nœuds, et de laisser le sérialiseur d'occuper de la linéarisation
de la sortie. (Il est possible d'utiliser \XSLT pour le traitement de
chaînes de caractères, bien que ceci ne soit pas le principal domaine
d'application du langage, auquel cas travailler avec
\texttt{xs:string*} fait sens.)

Un dernier regard sur cet exercice nous suggère que l'algorithme peut
être conçu comme \emph{opérant en parallèle} sur la séquence d'items,
tant que nous savons distinguer le dernier item parce qu'il doit être
traité différemment (il est suivi par un point au lieu d'une
virgule). Mais «~parallèle~» n'implique pas nécessairement un sens
temporel, comme fils d'exécutions, et peut aussi être pensé comme
traitement en isolement puis fusion des résultats partiels, ce qu'une
\emph{projection} (anglais, \emph{map}) fait. Nous avons rencontré les
projections lors de la présentation des itérateurs fonctionnels en
\Erlang, \vpageref{par:maps}:
\begin{equation*}
\erlcode{map(\(F\),[\(X_1\),\(X_2\),\(\ldots\),\(X_{n}\)])}
\equiv
\erlcode{[\(F\)(\(X_1\)),\(F\)(\(X_2\)),\(\ldots\),\(F\)(\(X_{n}\))]}.
\end{equation*}
Une définition simple est
\begin{verbatim}
map(_,   []) -> [];
map(F,[X|S]) -> [F(X)|map(F,S)].
\end{verbatim}
Notons que \texttt{map} est une fonction d'ordre supérieur et \XSLT
propose seulement des patrons et des fonctions du premier
ordre. Néanmoins, nous pouvons réaliser des projections en utilisant
une sorte de patron que nous avons en fait employée depuis le début de
cette présentation de \XSLT, en commençant à la
page~\pageref{par:matching}: le \emph{patron filtrant}. Le processeur
\XSLT (que nous entendons comme étant l'environnement d'exécution
produit par le compilateur \XSLT) implicitement effectue un parcours
en préordre de l'arbre \XML donné; quand il trouve un élément~\(e\),
il évalue le patron qui le filtre, puis reprend le parcours. Un patron
filtrant fonctionne comme une règle de réécriture dans notre langage
fonctionnel abstrait, chaque règle étant essayée tour à tour quand un
nœud est visité dans l'arbre \XML.

Puisque nous sélectionnons souvent quelques enfants du nœud
contextuel et leur appliquons le même traitement (comme en parallèle),
nous avons besoin d'un mécanisme pour regrouper les résultats pour
chaque enfant en une seule séquence. Dans le contexte de l'exercice
présent, nous souhaitons sélectionner une séquence d'éléments
\texttt{hexa}, enfants de \texttt{numbers}, et nous nous attendons
donc à un patron filtrant \texttt{hexa}. Nous voulons aussi grouper
les résultats, comme une projection le ferait. Ceci est réalisé en
deux temps par la définition d'un patron
\begin{sverb}
  <xsl:template match="hexa" as="text()*">
    ...
  </xsl:template>
\end{sverb}
analogue à la définition du paramètre fonctionnel~\(F\) ci-dessus, et
par
\begin{sverb}
    <xsl:apply-templates select="hexa"/>
\end{sverb}
qui est analogue à l'appel de \texttt{map} en
\Erlang.\label{match_as_map}

Le seul problème qui reste consiste à déterminer si le nœud
contextuel \texttt{hexa} est le dernier dans la séquence sur laquelle
tous les patrons ont été appliqués. C'est ici où \XPath se révèle
utile en fournissant une fonction \texttt{last()}, qui retourne la
position du dernier item filtré dans la même séquence que le nœud
contextuel. Voici ce à quoi la transformation ressemble maintenant:
\smallXSLTinUnchecked{csv3.xsl}
\noindent Ceci nous rappelle que tout item transporte implicitement
des information à son sujet, comme sa position dans une séquence, mais
aussi la position du dernier item dans celle-ci. Remarquons que nous
n'avons besoin d'aucun paramètre dans le patron filtrant
\texttt{hexa}, parce qu'à l'intérieur le nœud contextuel est l'un
des éléments \texttt{hexa} originaux, et nous n'avons pas besoin de
savoir lequel ni quels sont les autres (nous pouvons penser en termes
de traitement en parallèle, si c'est utile). Par exemple,
\begin{sverb}
<xsl:sequence select="text()"/>
\end{sverb}
signifie: «~Référencer le nœud textuel du nœud contextuel,~»
(par opposition à le copier avec \texttt{xsl:value-of}).

Nous emploierons des patrons filtrants dans la section qui vient à
propos de la transformation d'arbres, mais nous devons d'abord
pratiquer un peu plus pour bien comprendre les patrons nommés parce
qu'ils sont plus proches du concept de fonction du premier ordre dans
notre langage fonctionnel abstrait.

\mypar{Entrecoupement}

L'objectif de cet exercice est d'écrire une transformation \XSLT qui
prend en entrée deux séquences d'éléments et les entrecoupe, ou, plus
précisément, le premier item de la séquence résultante est le premier
item de la première séquence, le second est le premier item de la
seconde séquence, le troisième est le second item de la première
séquence, le quatrième est le second item de la seconde séquence etc.
Une analogie éclairante est l'entrecoupement d'une main par une autre
dans un jeu de cartes.

Si les premiers items des deux séquences sont retirés en même temps,
alors vient un moment où les deux séquences sont vides ou l'une
seulement l'est. Le problème est en fait sous-spécifié: rien n'est dit
sur ce qu'il faut faire quand les deux séquences ne sont pas de la
même longueur. Dans ce dernier cas, nous ignorerons les items
restants.

La \DTD que nous avons en tête est la suivante:
\smallDTDin{persons.dtd}
\noindent Alors, cette donnée:
\smallXMLinUnchecked{persons1.xml}
\noindent déterminera la production de
\smallXMLinUnchecked{persons1A_out.xml}
\noindent Cette entrée:
\smallXMLinUnchecked{persons2.xml}
\noindent mènera à:
\smallXMLinUnchecked{persons2A_out.xml}
\noindent Cette entrée:
\smallXMLinUnchecked{persons3.xml}
\noindent engendre
\smallXMLinUnchecked{persons3A_out.xml}
En suivant la stratégie esquissée plus haut, nous nous attendons à ce
que le patron nommé \texttt{shuffle} (anglais pour \emph{entrecouper})
ait deux paramètres, un pour les noms et un pour les notes:
\begin{sverb}
  <xsl:template match="persons" as="element(persons)">
    <xsl:copy>
      <xsl:call-template name="shuffle">
        <xsl:with-param name="names" select="names/name"/>
        <xsl:with-param name="notes" select="notes/note"/>
      </xsl:call-template>
    </xsl:copy>
  </xsl:template>

  <xsl:template name="shuffle" as="element()*">
    <xsl:param name="names" as="element(name)*"/>
    <xsl:param name="notes" as="element(note)*"/>
    ...
  </xsl:template>
\end{sverb}
Notons que nous employons \texttt{xsl:copy} pour copier le nœud
contextuel \texttt{persons} et que nous avons utilisé le type de
retour \texttt{element()*} car nous ne pouvons exprimer en \XPath:
«~Une séquence d'éléments (mêlés) \texttt{name} et \texttt{note}.~» Le
corps de ce patron se conforme à notre plan. Si les paramètres sont
tous deux non-vides, nous faisons quelque chose, sinon une séquence
vide implicite sera produite. Ce test est effectué en \XPath au moyen
du connecteur booléen \texttt{and} comme suit:
\begin{alltt}
\small  <xsl:template name="shuffle" as="element(persons)*">
    <xsl:param name="names" as="element(name)*"/>
    <xsl:param name="notes" as="element(note)*"/>
    \textbf{<xsl:if test="not(empty(\$names)) and not(empty(\$notes))">}
      <xsl:sequence select="\$names[1]"/>
      <xsl:sequence select="\$notes[1]"/>
      <xsl:call-template name="shuffle">
        <xsl:with-param name="names"
                        select="\$names[position()>1]"/>
        <xsl:with-param name="notes"
                        select="\$notes[position()>1]"/>
      </xsl:call-template>
    </xsl:if>
  </xsl:template>
\end{alltt}

Une autre possibilité, quand nous sommes en présence d'items restants,
est de les concaténer à ceux déjà produits. Par exemple, étant donnée
la première entrée ci-dessus, nous avons maintenant:
\smallXMLinUnchecked{persons1B_out.xml}
\noindent La seconde entrée ci-dessus mène à
\smallXMLinUnchecked{persons2B_out.xml}
\noindent Et la dernière entrée ci-dessus résulte en
\smallXMLinUnchecked{persons3B_out.xml}
\noindent Voici une solution:
\begin{sverb}
  <xsl:template name="shuffle" as="element()*">
    <xsl:param name="names" as="element(name)*"/>
    <xsl:param name="notes" as="element(note)*"/>
    <xsl:choose>
      <xsl:when test="empty($notes)">
        <xsl:sequence select="$names"/>
      </xsl:when>
      <xsl:when test="empty($names)">
        <xsl:sequence select="$notes"/>
      </xsl:when>
      <xsl:otherwise>
        <xsl:sequence select="($names[1],$notes[1])"/>
        <xsl:call-template name="shuffle">
          <xsl:with-param name="names"
                          select="$names[position()>1]"/>
          <xsl:with-param name="notes"
                          select="$notes[position()>1]"/>
        </xsl:call-template>
      </xsl:otherwise>
    </xsl:choose>
  </xsl:template>
\end{sverb}
Remarquons comment nous avons placé \texttt{\$notes[1]} après
\texttt{\$names[1]} en \XPath à l'aide de
\begin{sverb}
        <xsl:sequence select="($names[1],$notes[1])"/>
\end{sverb}
au lieu de travailler au niveau de \XSLT, comme précédemment:
\begin{sverb}
      <xsl:sequence select="$names[1]"/>
      <xsl:sequence select="$notes[1]"/>
\end{sverb}

Pour varier un peu les plaisirs, il est possible de parvenir au même
résultat en n'extrayant qu'un élément à la fois, au lieu de deux. Bien
sûr, le programme ira environ deux fois plus lentement, mais il est
néanmoins intéressant:
\begin{sverb}
  <xsl:template name="shuffle" as="element()*">
    <xsl:param name="names" as="element(name)*"/>
    <xsl:param name="notes" as="element(note)*"/>
    <xsl:choose>
      <xsl:when test="empty($names)">
        <xsl:sequence select="$notes"/>
      </xsl:when>
      <xsl:otherwise>
        <xsl:sequence select="$names[1]"/>
        <xsl:call-template name="shuffle">
          <xsl:with-param name="names" select="$notes"/>
          <xsl:with-param name="notes"
                          select="$names[position()>1]"/>
        </xsl:call-template>
      </xsl:otherwise>
    </xsl:choose>
  </xsl:template>
\end{sverb}
L'idée est d'échanger les arguments dans l'appel récursif. Bien
entendu, les noms des paramètres ne sont alors plus pertinents, et un
renommage plus neutre s'imposerait. Dans notre langage fonctionnel
abstrait, nous écririons:
\begin{equation*}
\fun{shuffle}(\el,t) \rightarrow t;\quad
\fun{shuffle}(\cons{x}{s},t) \rightarrow \cons{x}{\fun{shuffle}(t,s)}.
\end{equation*}
En fait, c'est presque la même définition que \fun{cat/1} (concaténation):
\begin{equation*}
\fun{cat}(\el,t) \rightarrow t;\quad
\fun{cat}(\cons{x}{s},t) \rightarrow \cons{x}{\fun{cat}(\underline{s},\underline{t})}.
\end{equation*}

\mypar{Maximum}

Le but de cet exercice est d'écrire une transformation \XSLT qui prend
en entrée une séquence d'entiers et produit le maximum de ces nombres
sous forme de texte libre. Si un item n'est pas transtypable en
\texttt{xs:integer}, une erreur de type dynamique est signalée et
l'exécution est suspendue. Si un item est un nœud textuel vide,
par exemple, \texttt{<num/>}, il est ignoré. Si la séquence ne
contient aucun entier, aucun texte n'est produit, parce que le
maximum n'est pas défini. L'élément racine est \texttt{numbers} et les
éléments contenant les nombres sont nommés \texttt{num}. Tout élément
présent autre que \texttt{num} est ignoré.

La \DTD que nous avons à l'esprit est la suivante:
\smallDTDin{numbers.dtd}
\noindent Si le document donné est
\smallXMLinUnchecked{numbers.xml}
\noindent le résultat produit est
\smallXMLinUnchecked{numbers_out.xml}

\noindent Tout d'abord, fixons les types dans le canevas suivant:
\begin{sverb}
  <xsl:template match="numbers" as="xs:integer?">
    ...
  </xsl:template>

  <xsl:template name="max" as="xs:integer?">
    <xsl:param name="int" as="xs:integer*"/>
    <xsl:param name="cur" as="xs:integer?"/>
    ...
  </xsl:template>
\end{sverb}
Souvenons-nous que l'opérateur de type «~\texttt{?}~» signifie «~Un ou
aucun~», donc \texttt{xs:integer?} dénote une séquence vide ou une
séquence contenant un entier. Cette précaution est nécessaire parce
que nous ne sommes pas certains que l'entrée contienne au moins un
entier (nous avons même laissé la porte ouverte à des éléments
fantaisistes \texttt{foo}, comme nous pouvons le voir en regardant à
nouveau la \DTD). Le paramètre \texttt{int} contient les entiers
restant à examiner, alors que \texttt{cur} est le maximum pour
l'instant, s'il y en a un. C'est pourquoi nous devrions initialiser ce
dernier avec le contenu du premier nombre:
\begin{alltt}
\small  <xsl:template match="numbers" as="xs:integer?">
    <xsl:call-template name="max">
      <xsl:with-param name="int"
                      select="num[position()>1]/text()"/>
      <xsl:with-param name="cur" select="\textbf{num[1]/text()}"/>
    </xsl:call-template>
  </xsl:template>
\end{alltt}
Notons que si nous avions écrit \texttt{\$num[1]/text()}, la sélection
aurait été vide, car il n'y a pas de \emph{variable} \texttt{num},
mais nous voulions dire les \emph{enfants} \texttt{num}. Par ailleurs,
nous avons sélectionné les nœuds textuels, bien que les types
attendus soient \texttt{xs:integer*} et \texttt{xs:integer?}. En fait,
un transtypage sera effectué à l'exécution. Dans le cas de
\texttt{cur}, si le transtypage échoue, la séquence vide en résultera;
sinon, un entier (c'est-à-dire, une séquence contenant un seul
entier). Dans le cas de \texttt{int}, un transtypage est tenté pour
chaque élément dans la séquence et les séquences résultantes sont
concaténées.

Il y a différentes manières de résoudre ce problème. Nous pourrions
distinguer les situations suivantes:
\begin{itemize}

  \item s'il n'y a pas d'entiers à examiner, le résultat est l'entier
    courant, s'il existe;

  \item s'il y a un entier courant, et s'il est plus grand que le
    premier entier à examiner, nous recommençons en écartant ce
    dernier;

  \item sinon, le premier entier à examiner devient le maximum courant
    et nous recommençons et écartons le maximum précédent.

\end{itemize}
Ce schéma est mis en œuvre comme suit:
\begin{sverb}
  <xsl:template name="max" as="xs:integer?">
    <xsl:param name="int" as="xs:integer*"/>
    <xsl:param name="cur" as="xs:integer?"/>
    <xsl:choose>
      <xsl:when test="empty($int)">
        <xsl:sequence select="$cur"/>
      </xsl:when>
      <xsl:when test="not(empty($cur)) and $cur ge $int[1]">
        <xsl:call-template name="max">
          <xsl:with-param name="int"
                          select="$int[position()>1]"/>
          <xsl:with-param name="cur" select="$cur"/>
        </xsl:call-template>
      </xsl:when>
      <xsl:otherwise>
        <xsl:call-template name="max">
          <xsl:with-param name="int"
                          select="$int[position()>1]"/>
          <xsl:with-param name="cur" select="$int[1]"/>
        </xsl:call-template>
      </xsl:otherwise>
    </xsl:choose>
  </xsl:template>
\end{sverb}
Remarquons que nous avons vérifié la présence d'un maximum courant à
l'aide de \texttt{not(empty(\$cur))} et l'opérateur booléen \XPath
«~plus grand ou égal à~» est~\texttt{ge}. Dans les appels récursifs, le
paramètre \texttt{int} vaut \texttt{\$int[position()>1]}, donc nous
pourrions partager le code ainsi:
\begin{alltt}
\small<xsl:template name="max" as="xs:integer?">
  <xsl:param name="int" as="xs:integer*"/>
  <xsl:param name="cur" as="xs:integer?"/>
  <xsl:choose>
    <xsl:when test="empty(\$int)">
      <xsl:sequence select="\$cur"/>
    </xsl:when>
    <xsl:otherwise>
      <xsl:call-template name="max">
        <xsl:with-param name="int" select="\$int[position()>1]"/>
        <xsl:with-param name="cur">
          <xsl:choose>
            <xsl:when test="not(empty(\$cur))\! and\! \$cur ge \$int[1]">
              <xsl:sequence select="\$cur"/>
            </xsl:when>
            <xsl:otherwise>
              <xsl:sequence select="\$int[1]"/>
            </xsl:otherwise>
          </xsl:choose>
        </xsl:with-param>
      </xsl:call-template>
    </xsl:otherwise>
  </xsl:choose>
</xsl:template>
\end{alltt}
Ce patron contient moins de duplications et est structuré plus
logiquement, mais il est plus long, ce qui veut dire que les réflexes
acquis par la pratique d'autres langages de programmation pourraient
se révéler contreproductifs en \XSLT. Remarquons en passant que cette
solution illustre le fait que \texttt{xsl:with-param} peut avoir des
enfants au lieu d'un attribut \texttt{select}.

\mypar{Réduction}

Le but de cet exercice est d'écrire une transformation \XSLT qui prend
en entrée un document dont l'arbre est plat, c'est-à-dire que la
racine du document a des enfants, mais pas de petits-enfants. Les
enfants portent un attribut et le résultat devrait être le même
document sans les enfants qui sont dupliqués consécutivement. Ceci est
réalisé par la fonction \fun{red/1} que nous avons vu à la
\fig~\vref{fig:red}:
\begin{equation*}
\begin{array}{@{}r@{\;}c@{\;}l@{}}
\fun{red}(\el) & \rightarrow & \el;\\
\fun{red}(\cons{x,x}{s}) & \rightarrow & \fun{red}(\cons{x}{s});\\
\fun{red}(\cons{x}{s})   & \rightarrow & \cons{x}{\fun{red}(s)}.
\end{array}
\end{equation*}
En \XSLT, nous avons besoin de plus de contraintes sur l'entrée et
nous devons prendre en compte les types de données. Choisissons la \DTD suivante:
\smallDTDin{numbers_bis.dtd}
\noindent Par exemple, cette entrée
\smallXMLinUnchecked{numbers_bis.xml}
\noindent produit
\smallXMLinUnchecked{numbers_bis_out.xml}
\noindent Si nous partons de \fun{red/1}, nous devons éliminer le
filtrage par motifs et utiliser à la place des
conditionnelles. D'abord, nous séparons les filtres selon le nombre
d'items:
\begin{equation*}
\begin{array}{@{}r@{\;}c@{\;}l@{}}
\fun{red}(\el) & \rightarrow & \el;\\
\fun{red}([x]) & \rightarrow & [x];\\
\fun{red}(\cons{x,x}{s}) & \rightarrow & \fun{red}(\cons{x}{s});\\
\fun{red}(\cons{x,y}{s}) & \rightarrow & \cons{x}{\fun{red}(\cons{y}{s})}.
\end{array}
\end{equation*}
Maintenant, nous pouvons ôter le filtrage par motifs:
\begin{equation*}
\begin{array}{@{}r@{\;}c@{\;}l@{}}
\fun{red}(t) & \rightarrow & \fun{if} \; t=\el\; \fun{or} \;
                             \fun{tl}(t)=\el \; \fun{then} \; t\\
             &             & \fun{else} \; \fun{if} \; \fun{hd}(t) = \fun{hd}(\fun{tl}(t)) \;
\fun{then} \; \fun{red}(\fun{tl}(t)) \; \fun{else} \; \cons{\fun{hd}(t)}{\fun{red}(\fun{tl}(t))}.
\end{array}
\end{equation*}
où \(\fun{hd}(t)\) \index{head@\fun{hd/1}} est le sommet de la
pile~\(t\) et \(\fun{tl}(t)\) \index{tail@\fun{tl/1}} est la sous-pile
immédiate de~\(t\). En \XSLT, nous pouvons définir une variable grâce
à l'élément \texttt{xsl:variable}, donc nous pouvons améliorer la
traduction en calculant seulement une fois la traduction de
\(\fun{red}(\fun{tl}(t))\). Nous savons aussi que \(\fun{hd}(t)\) est
traduit par \texttt{\$t[1]}, où~\texttt{t} est la traduction de~\(t\),
et \(\fun{tl}(t)\) est traduit en \texttt{\$t[position()>1]} ou
\texttt{\$t[position()!=1]}. Voici la transformation complète:
\smallXSLTinUnchecked{red.xsl}
\noindent Nous n'avons pas transtypé la valeur de l'attribut en
\texttt{xs:integer}, par exemple \texttt{xs:integer(\$t[1]/@val) ne
  xs:integer(\$t[2]/@val)}, car nous voulons permettre la comparaison
de toutes sortes de valeurs.


\mypar{Interclassement}
\label{par:merging}

L'objectif de cet exercice est d'écrire une transformation \XSLT qui
prend deux séquences d'éléments triés en l'ordre croissant d'un
attribut entier, et retourne une séquence contenant tous les items
triés en ordre croissant. Nous pouvons réutiliser la fonction
\fun{mrg/2} à la \fig~\vref{fig:mrg}, qui interclasse (anglais,
\emph{merge}) deux piles ordonnées:
\begin{equation*}
\begin{array}{r@{\;}l@{\;}l}
\fun{mrg}(\el,t)         & \rightarrow & t;\\
\fun{mrg}(s,\el)         & \rightarrow & s;\\
\fun{mrg}(\cons{x}{s},\cons{y}{t}) & \rightarrow
& \cons{y}{\fun{mrg}(\cons{x}{s},t)},\;\text{si \(x \succ y\)};\\
\fun{mrg}(\cons{x}{s},t) & \rightarrow & \cons{x}{\fun{mrg}(s,t)}.
\end{array}
\end{equation*}
Nous envisageons la simple \DTD \texttt{list.dtd} suivante:
\smallDTDin{list.dtd}
\noindent Étant donné le document \XML suivant
\smallXMLinUnchecked{ordered_lists.xml}
\noindent le résultat est
\smallXMLinUnchecked{ordered_lists_out.xml}
Ici, nous pouvons traduire \fun{mrg/2} en \XSLT sans nous débarrasser
explicitement du filtrage par motifs:
\smallXSLTinUnchecked{merge1.xsl}
\noindent Si nous voulions produire à la place du texte, nous
n'aurions qu'à changer le code comme suit:
\begin{sverb}
...
  <xsl:output method="text"/>
...
  <xsl:template match="lists" as="xs:integer*">
    <xsl:call-template name="merge">
      <xsl:with-param name="seq1" select="list[1]/item"
                                  as="xs:integer*"/>
      <xsl:with-param name="seq2" select="list[2]/item"
                                  as="xs:integer*"/>
    </xsl:call-template>
  </xsl:template>

  <xsl:template name="merge" as="xs:integer*">
    <xsl:param name="seq1" as="xs:integer*"/>
    <xsl:param name="seq2" as="xs:integer*"/>
    ...
  </xsl:template>
...
\end{sverb}


\section{Transformation d'arbres}

Après un long entrainement à la transformation de séquences, il est
temps d'aborder le cas général, c'est-à-dire les arbres. Comme nous
l'avons mentionné au début de ce chapitre, les arbres \XML sont des
\emph{arbres de Catalan}, ce qui veut dire qu'un nœud élément peut
avoir un nombre variable d'enfants, si ce n'est pas interdit par une
\DTD. Ceci est à contraster avec les arbres binaires, par exemple,
dont les nœuds peuvent n'avoir que deux enfants ou aucun.

\mypar{Taille}

Le but de cet exercice est d'écrire une transformation \XSLT qui prend
une table des matières et compte le nombre de sections. Mais,
contrairement à un exercice précédent, la table n'est pas plate ici,
plus précisément, la \DTD que nous souhaitons est \texttt{book\_deep.dtd}:
\smallDTDin{book_deep.dtd}
\noindent Un exemple de document \XML valide est
\smallXMLinUnchecked{toc_deep.xml}
\noindent Bien sûr, nous nous attendons au résultat suivant:
{\small\verbatiminput{toc_deep_out.txt}}
Au lieu de revenir à notre langage fonctionnel abstrait, ou \Erlang,
puis traduire en \XSLT, essayons d'esquisser l'algorithme en français
et ensuite d'écrire la transformation.

L'idée est de filtrer l'élément racine, puis de sélectionner le
premier niveau des sections, juste en dessous des chapitres. Cette
séquence de nœuds \texttt{section} est passée à un patron nommé
\texttt{count}, dont la tâche est de compter toutes les sections. Si
cette séquence de sections est vide, la réponse est~\texttt{0}. Sinon,
\begin{enumerate}

\item nous appelons récursivement \texttt{count} sur les sous-sections
  de la première section;

\item ce nombre plus~\(1\) est le nombre de sections dans la première
  section (elle-même incluse) de la séquence;

\item finalement, nous appelons récursivement \texttt{count} sur le
  reste de la séquence (soit les sections restantes) et ajoutons ce
  nombre au précédent: le total est le résultat.

\end{enumerate}
Notons que les deux appels récursifs peuvent être interchangés et le
cas décrit (appel d'abord sur les enfants du premier nœud, puis
sur les autres nœuds de même niveau), décrit un parcours \emph{en
  profondeur} de l'arbre, que nous écrivons en premier:
\smallXSLTinUnchecked{count1.xsl}
\noindent Il faut sélectionner \texttt{\$sections[1]/section}, pas
\texttt{sections[1]/section}, qui est vide, parce que \texttt{section}
est un enfant élément du nœud contextuel, alors que
\texttt{\$section} est le contenu de la variable \texttt{section}. Il
est peut être sage d'éviter des variables qui sont aussi des noms
d'éléments de l'entrée. Notons aussi que, puisque nous devons appeler
les patrons nommés au niveau \XSLT, pas en \XPath, nous devons définir
les variables \texttt{subsec} et \texttt{subseq} pour contenir les
résultats des deux appels récursifs. Si nous avions employé des
fonctions \XSLT (\texttt{xsl:function}), nous les aurions appelées en
\XPath. Pour plus d'uniformité, conservons les patrons nommés, même
si, dans certains contextes, ils peuvent augmenter la verbosité d'un
langage déjà assez verbeux.

Si nous voulions visiter les nœuds frères avant les enfants, nous
devrions simplement interchanger les déclarations des variables:
\begin{sverb}
      ...
        <xsl:variable name="subseq" as="xs:integer">
          ...
        </xsl:variable>
        <xsl:variable name="subsec" as="xs:integer">
          ...
        </xsl:variable>
        ...
\end{sverb}
Ceci ne fait aucune différence parce que les parties de l'arbre
parcourues par les deux appels sont complémentaires. Il est néanmoins
instructif de dessiner l'arbre \XML et de suivre les appels
(descendants) avec une couleur sur le côté gauche des nœuds, et
les résultats (ascendants) avec une autre couleur sur le côté droit.

Au lieu de compter les sections d'une manière ascendante, nous pouvons
passer un compteur le long de notre parcours et l'incrémenter à chaque
fois que nous trouvons une section; le résultat final est alors le
compte courant quand nous sommes de retour à la racine. (Le compteur
est une sorte d'\emph{accumulateur}.) Nous avons
\smallXSLTinUnchecked{count3.xsl}
\noindent Remarquons que le cas de la séquence \texttt{\$sections}
vide correspond au compte courant, au lieu de~\texttt{0}, par
opposition aux versions précédentes.

Pour travailler un peu avec la syntaxe de \XSLT, nous pourrions
remarquer que la variable \texttt{subsec} est seulement utilisée pour
initialiser le paramètre \texttt{current} du second appel
récursif. Nous pourrions éviter la création de cette variable si nous
expansions son appel récursif en un enfant du paramètre en question:
\begin{alltt}
\small  <xsl:template name="count" as="xs:integer">
    <xsl:param name="sections" as="element(section)*"/>
    <xsl:param name="current"  as="xs:integer" \textbf{select="0"}/>
    <xsl:choose>
      <xsl:when test="empty(\$sections)">
        <xsl:sequence select="\$current"/>
      </xsl:when>
      <xsl:otherwise>
        <xsl:call-template name="count">
          <xsl:with-param name="sections"
                          select="\$sections[position()>1]"/>
          \textbf{<xsl:with-param name="current" as="xs:integer">
            <xsl:call-template name="count">
              <xsl:with-param name="sections"
                              select="\$sections[1]/section"/>
              <xsl:with-param name="current"
                              select="\$current + 1"/>
            </xsl:call-template>
          </xsl:with-param>}
        </xsl:call-template>
      </xsl:otherwise>
    </xsl:choose>
  </xsl:template>
\end{alltt}
Notons l'usage d'une valeur par défaut pour le paramètre
\texttt{current}, donc en évitant son initialisation dans le premier
appel (dans le patron filtrant l'élément \texttt{book}).


\mypar{Somme}

\hspace*{-3pt} L'objectif de cet exercice est d'écrire une
transformation \XSLT qui prend un document fait d'une sorte d'élément
avec une sorte d'attribut dont la valeur est un entier naturel, et
calcule la somme de tous ces nombres. Plus précisément, nous pensons à
la \DTD suivante: \smallDTDin{numbers_tree.dtd}
\noindent et, par exemple, à l'entrée suivante:
\smallXMLinUnchecked{numbers_tree.xml}
\noindent Le résultat attendu est alors
{\small\verbatiminput{numbers_tree_out.txt}} La clé est de comprendre
la différence entre cet exercice et l'exercice où nous devions compter
le nombre de sections dans une table des matières. Dans le dernier
cas, nous avions compté~\texttt{1} pour chaque section. Dans le
premier, nous prenons simplement la valeur de l'attribut \texttt{val}
au lieu de~\texttt{1}: \smallXSLTinUnchecked{sum2.xsl}

\mypar{Réflexion}

Le but de cet exercice est d'écrire une transformation \XSLT qui prend
une table des matières avec seulement des sections et produit la même
table, en \XML, où les sections sont inversées, niveau par niveau, ce
qui veut dire que l'arbre résultant est l'image de l'arbre d'entrée
dans un miroir. Nous avons déjà défini une fonction abstraite
\fun{mir/1} qui réalise cette opération à la \fig~\vref{fig:mirror}:
\begin{equation*}
\fun{mir}(\fun{ext}()) \rightarrow \fun{ext}();
\quad
\fun{mir}(\fun{int}(x,t_1,t_2)) \rightarrow
\fun{int}(x,\fun{mir}(t_2),\fun{mir}(t_1)).
\end{equation*}
\noindent La \DTD que nous avons en tête ici est la suivante:
\smallDTDin{book_simple.dtd}
\noindent Un exemple d'une entrée valide est la table suivante:
\smallXMLinUnchecked{toc_simple.xml}
\noindent Notons que chaque titre de section a été numéroté en ordre
pour mieux comprendre le résultat:
\smallXMLinUnchecked{toc_simple_out.xml}

La différence avec \fun{mir/1} est que les arbres \XML sont des arbres
de Catalan et qu'ils n'ont pas de nœuds externes. Le cas
\(\fun{mir}(\fun{ext}())\) correspond à la séquence vide de
sous-sections et son membre droit \(\fun{ext}()\) est traduit alors en
une séquence vide aussi, ce qui signifie que la structure du patron
nommé est
\begin{sverb}
  <xsl:template name="mir" as="element(section)*">
    <xsl:param name="sections" as="element(section)*"/>
    <xsl:if test="not(empty($sections))">
      ...
    </xsl:if>
  </xsl:template>
\end{sverb}
Ceci est un cas typique d'usage de \texttt{xsl:if}, au lieu du général
\texttt{xsl:choose}. Ensuite, nous nous concentrons sur la deuxième
règle de réécriture, dont le membre droit est
\(\fun{int}(x,\fun{mir}(t_2),\fun{mir}(t_1))\). En \XSLT, le paramètre
est une \emph{séquence} de sections, c'est-à-dire une forêt, parce que
nous traitons des arbres de Catalan, donc les enfants de la racine
constituent une forêt, pas une paire \((t_1, t_2)\) comme dans les
arbres binaires. Par conséquent, nous devons généraliser la réflexion
d'une pile. Il ne convient pas de simplement la retourner parce que
les enfants doivent être retournés aussi, et aussi les petits-enfants
etc. Autrement dit, nous devons parcourir l'arbre entier, donc nous
devrions nous attendre à effectuer deux appels récursifs: un
horizontalement (pour traiter le niveau courant \texttt{\$sections}),
et un verticalement (pour traiter les enfants d'un nœud dans le
niveau courant, souvent le premier).

Le premier canevas devrait alors être rempli ainsi:
\begin{sverb}
  <xsl:template name="mir" as="element(section)*">
    <xsl:param name="sections" as="element(section)*"/>
    <xsl:if test="not(empty($sections))">
      <xsl:call-template name="mir">
        <xsl:with-param name="sections"
                        select="$sections[position()>1]"/>
      </xsl:call-template>
      <section>
        <xsl:sequence select="$sections[1]/@title"/>
        <xsl:call-template name="mir">
          <xsl:with-param name="sections"
                          select="$sections[1]/section"/>
        </xsl:call-template>
      </section>
    </xsl:if>
  </xsl:template>
\end{sverb}
Ce patron peut être conçu comme l'entrelacement du retournement et de
la réflexion récursive des enfants de la racine. Notons que nous avons
besoin de \texttt{<xsl:sequence select="\$sections[1]/@title"/>} pour
reconstruire l'attribut de l'image réfléchie
\texttt{<section>...</section>} de la première section. Souvenons-nous
que les nœuds attributs doivent être définis \emph{avant} les
autres espèces de nœuds parmi les enfants (voir
page~\pageref{attr_before}), c'est-à-dire immédiatement après la
balise ouvrante \texttt{<section>}.

La transformation complète est
\smallXSLTinUnchecked{flip1.xsl}
\noindent Nous avons à nouveau une illustration de la nécessité de
l'attribution \texttt{exclude-result-prefixes="xs"}, sinon
\texttt{<section>...</section>}, qui est l'élément reconstruit,
hériterait inutilement l'espace de noms~\texttt{xs}.

Répondons maintenant à la question quand la table des matières
contient des chapitres contenant à leur tour des sections, mais
nous ne voulons pas retourner les chapitres, seulement les
sections. Plus précisément, la \DTD est
\smallDTDin{book_deep.dtd}
\noindent et une entrée valide serait
\smallXMLinUnchecked{toc_chap.xml}
\noindent Nous voulons le résultat
\smallXMLinUnchecked{toc_chap_out.xml}
\noindent Nous avons mentionné que nous avons dû écrire
\begin{sverb}
      <section>
        <xsl:sequence select="$sections[1]/@title"/>
        ...
      </section>
\end{sverb}
pour copier l'attribut de la première section. Au lieu de cela, nous
aimerions écrire \texttt{<section title="\$sections[1]/@title">}, mais
la valeur de l'attribut est alors considérée comme du texte, pas comme
une sélection. Par conséquent, le problème se réduit à sélectionner un
attribut qui n'est ni \texttt{test}, ni \texttt{select}. La solution
réside dans un opérateur \XPath \texttt{\{...\}}, qui signifie:
«~Considérer le texte entre accolades comme du \XPath, pas du texte.~»
Autrement dit, nous pourrions écrire
\begin{alltt}
\small      <section title="\textbf{\{}\$sections[1]/@title\textbf{\}}">
        ...
      </section>
\end{alltt}
Clairement, nous n'avons pas besoin de réécrire le patron nommé
\texttt{mir} parce que les sections doivent être traitées de la même
manière que précédemment, bien qu'il pourrait être intéressant
d'utiliser ce nouvel opérateur \XPath à des fins didactiques. Sinon,
tout ce dont nous avons besoin est un nouveau patron nommé pour gérer
les chapitres en les reconstruisant \emph{dans le même ordre}, mais
avec des enfants \texttt{section} réfléchies (s'il y en a). Ceci veut
dire que nous pouvons réutiliser la même structure que celle de
\texttt{mir}, mais sans le retournement:
\begin{sverb}
  <xsl:template match="book" as="element(book)">
    <xsl:copy>
      <xsl:sequence select="@title"/>
      <xsl:sequence select="author"/>
      <xsl:call-template name="mk_chap">
        <xsl:with-param name="chapters" select="chapter"/>
      </xsl:call-template>
    </xsl:copy>
  </xsl:template>

  <xsl:template name="mk_chap" as="element(chapter)*">
    <xsl:param name="chapters" as="element(chapter)*"/>
    <xsl:if test="not(empty($chapters))">
      <chapter title="{$chapters[1]/@title}">
        <xsl:call-template name="mir">
          <xsl:with-param name="sections"
                          select="$chapters[1]/section"/>
        </xsl:call-template>
      </chapter>
      <xsl:call-template name="mk_chap">
        <xsl:with-param name="chapters"
                        select="$chapters[position()>1]"/>
      </xsl:call-template>
    </xsl:if>
  </xsl:template>
\end{sverb}
Notons que l'élément \texttt{xsl:copy} ne peut comporter un attribut
\texttt{select}: il n'effectue qu'une copie superficielle du nœud
contextuel. Ici, il est clair que le nœud contextuel est
\texttt{book} parce que \texttt{xsl:copy} est un enfant du patron
filtrant \texttt{book}. Mais que se passerait-il s'il se trouvait dans
un patron nommé? Comment y connaîtrions-nous le nœud contextuel,
puisque nous ne serions pas dans un patron filtrant? La réponse est
que le nœud contextuel est le dernier nœud filtré dans le flot
de contrôle jusqu'à l'endroit présent. Par exemple, dans le patron
nommé \texttt{mir}, le nœud contextuel est l'élément racine
\texttt{book}.

Parce que l'ordre des chapitres doit rester inchangé, il est
intéressant d'utiliser le patron filtrant \texttt{chapter} pour les
traiter et de l'appeler avec \texttt{<xsl:apply-templates
  select="chapter"/>}, au lieu d'employer le patron nommé
\texttt{mk\_chap}, qui est un peu lourd. Cela signifie:
\begin{enumerate}

\item sélectionner les éléments \texttt{chapter} qui sont des enfants
  du nœud contextuel;

\item pour chaque élément dans la séquence résultante, en parallèle,
  appliquer le premier patron dans la transformation qui filtre
  \texttt{chapter};

\item après en avoir terminé, réunir tous les résultats dans une
  séquence, dans le même ordre que les chapitres originaux.

\end{enumerate}
Comme nous l'avons vu \vpageref{match_as_map}, un patron filtrant est
comme une projection, l'application en parallèle d'un patron aux items
d'une séquence. En d'autres termes, quand le traitement en
\emph{parallèle} des éléments est envisagé, utilisons
\texttt{xsl:apply-templates}, sinon un traitement \emph{séquentiel}
est choisi, soit \texttt{xsl:call-template}. (Gardons à l'esprit que
«~parallèle~» n'implique pas que la réalisation d'un processeur \XSLT
doit être multifils (anglais, \emph{multi-threaded}), seulement que
c'est une possibilité. La fonction \texttt{map} en \Erlang est
clairement séquentielle, par exemple, bien qu'elle pourrait être
programmée en utilisant au moyen de processus concurrents, voir
distribués.) Nous devons réécrire le patron filtrant \texttt{book} et
le patron nommé \texttt{mk\_chap}, qui devient un patron filtrant
\texttt{chapter}:
\begin{sverb}
  <xsl:template match="book" as="element(book)">
    <xsl:copy>
      <xsl:attribute name="title" select="@title"/>
      <xsl:sequence select="author"/>
      <xsl:apply-templates select="chapter"/>
    </xsl:copy>
  </xsl:template>

  <xsl:template match="chapter" as="element(chapter)">
    <xsl:copy>
      <xsl:attribute name="title" select="@title"/>
      <xsl:call-template name="mir">
        <xsl:with-param name="sections" select="section"/>
      </xsl:call-template>
    </xsl:copy>
  </xsl:template>
\end{sverb}
Remarquons comment la structure du nouveau patron n'imite plus celle
du patron nommé \texttt{mir}, donc est plus courte. Par ailleurs, nous
avons fait usage d'un nouvel élément \XSLT:
\begin{alltt}
\small <xsl:template match="book" as="element(book)">
    <xsl:copy>
      \textbf{<xsl:attribute name="title" select="@title"/>}
      ...
\end{alltt}
Ceci est une option à l'usage de \texttt{xsl:sequence} comme
précédemment. Par ailleurs, nous avons maintenant un élément
\texttt{xsl:copy} par patron filtrant, le nœud contextuel étant
\texttt{book} dans un cas, et \texttt{chapter} dans l'autre.

En comparant le contenu du patron filtrant les chapitres avec celui de
l'élément suivant dans le patron nommé \texttt{mir},
\begin{sverb}
      ...
      <section title="{$sections[1]/@title}">
        <xsl:call-template name="flip">
        <xsl:with-param name="sections"
                        select="$sections[1]/section"/>
        </xsl:call-template>
      </section>
      ...
\end{sverb}
il devient apparent que les deux actions sont les mêmes: faire une
copie superficielle d'un élément et réfléchir ses enfants. Par
conséquent, il serait avantageux que le patron filtrant les chapitres
aussi filtrât les sections. À cause de notre emploi de
\texttt{xsl:copy} et \texttt{xsl:attribute}, il devient possible
d'obtenir un patron commun filtrant les chapitres et les sections:
\texttt{<xsl:template match="chapter|section">}, dont l'interprétation
est: «~Filtrer un \texttt{chapter} ou bien une \texttt{section}.~» Voici
maintenant la différence avec la réponse antérieure:
\begin{alltt}
\small  <xsl:template match="chapter|section" as="element()*">
    <xsl:copy>
      <xsl:attribute name="title" select="@title"/>
      <xsl:call-template name="mir">
        <xsl:with-param name="sections" select="section"/>
      </xsl:call-template>
    </xsl:copy>
  </xsl:template>

  <xsl:template name="mir" as="element(section)*">
    <xsl:param name="sections" as="element(section)*"/>
    <xsl:if test="not(empty(\$sections))">
      <xsl:call-template name="mir">
        <xsl:with-param name="sections"
                        select="\$sections[position()>1]"/>
      </xsl:call-template>
      \textbf{<xsl:apply-templates select="\$sections[1]"/>}
    </xsl:if>
  </xsl:template>
\end{alltt}
Notons que nous devons appliquer des patrons à la première section
dans \texttt{mir} (voir le code en gras), au lieu d'appeler
récursivement \texttt{mir} (cet appel est maintenant effectué dans le
patron filtrant les chapitres et sections). Puisque le patron
s'applique à une section, le parallélisme est perdu, mais nous gagnons
néanmoins du partage de code.

Les éléments \texttt{xsl:call-template} et
\texttt{xsl:apply-templates} diffèrent aussi en ce que le premier
résulte toujours en un appel alors que le dernier peut être une
non-opération si l'attribut \texttt{select} s'évalue en une séquence
vide. Autrement dit, \texttt{<xsl:apply-templates select="..."/>} ne
fait rien si \texttt{"..."} est vide, alors que
\texttt{<xsl:call-template name="t">} appelle toujours le patron
nommé~\texttt{t}, même si les paramètres sont des séquences vides. Un
patron filtrant peut être paramétré en plaçant des éléments
\texttt{xsl:param} juste après \texttt{<xsl:template match="...">}
(définition) et \texttt{xsl:with-param} après
\texttt{xsl:apply-templates} (application). C'est la même syntaxe que
\texttt{xsl:call-template}. Changeons alors l'appel du patron
\texttt{mir} en une application de patron avec un paramètre et ôtons
la définition de \texttt{mir} entièrement.

La transformation la plus courte est alors
\smallXSLTinUnchecked{flip4.xsl}

\mypar{Hauteur}

Le but de cet exercice est d'écrire une transformation \XSLT qui prend
une table des matières et calcule sa hauteur.
\begin{itemize}

  \item La hauteur d'une table des matières est la plus grande hauteur
  de ses chapitres.

  \item La hauteur d'un chapitre (respectivement, d'une section)
    est~\texttt{1} plus la plus grande hauteur de ses sections
    (respectivement, de ses sous-sections).

  \item La hauteur d'une séquence vide est~\texttt{0}.

\end{itemize}
Par exemple, un livre sans chapitres a la hauteur~\texttt{0} (il est
vide). Un livre constitué uniquement de chapitres sans sections a pour
hauteur~\texttt{1} (il est plat). Nous utiliserons la même \DTD que
dans l'exercice précédent:
\smallDTDin{book_deep.dtd}
\noindent La même donnée
\smallXMLinUnchecked{toc_chap.xml}
\noindent conduit au résultat
{\small\verbatiminput{toc_chap_out.txt}}

La définition ci-dessus est un algorithme parallèle, parce que les
hauteurs des chapitres et sections peuvent être calculées
séparément. Par conséquent, réécrivons la transformation en employant
des patrons filtrants uniquement et réutilisons le patron nommé
\texttt{max} pour trouver le maximum de deux entiers.
\smallXSLTinUnchecked{height1.xsl}

Même question mais cette fois-ci, au lieu de calculer en parallèle les
hauteurs des enfants d'un nœud donné, calculons-les
séquentiellement à l'aide d'un patron nommé. L'objectif est d'éviter
le calcul d'une séquence de hauteurs et d'en déduire la plus grande:
au lieu de cela, nous pourrions calculer la hauteur courante pendant
le parcours. Deux paramètres sont nécessaires: un paramètre
\texttt{cur} représentant la hauteur d'une séquence jusqu'à présent
(la valeur initiale est~\texttt{0}) et un paramètre \texttt{seq}
contenant le reste de la séquence dont nous voulons connaître la
hauteur. Alors
\begin{enumerate}

  \item nous calculons la hauteur de la séquence des enfants de
    \texttt{\$seq[1]};

  \item nous ajoutons~\texttt{1} pour obtenir la hauteur de
    \texttt{\$seq[1]};

  \item le maximum de cette valeur et \texttt{\$cur} est la valeur de
    \texttt{cur} dans l'appel récursif avec
    \texttt{\$seq[position()>1]}. Si \texttt{\$seq} est vide, la plus
    grande hauteur des nœuds est \texttt{\$cur}. (Ce schéma est
    analogue au comptage du nombre de sections.)

\end{enumerate}
\noindent Ceci est écrit en \XSLT ainsi:
\smallXSLTinUnchecked{height2.xsl}

Dans la question précédente, la hauteur est calculée de façon
\emph{ascendante}, c'est-à-dire que les incréments sur la hauteur sont
effectués juste avant la fin des appels récursifs et de nouveaux
appels initialisent le paramètre de hauteur à~\texttt{0}. Au lieu de
cela, nous pouvons proposer une conception alternative où la hauteur
est incrémentée de manière \emph{descendante}, autrement dit, au
paramètre de hauteur est ajouté~\texttt{1} juste avant le début des
appels récursifs:
\smallXSLTinUnchecked{height3.xsl}

\mypar{Numérotation}

L'objectif de cet exercice est de composer une transformation \XSLT
qui prend une table des matières et la reproduit en \XHTML, d'abord
sans numéroter ses chapitres ni ses sections, puis en les
numérotant. La \DTD est toujours la même:
\smallDTDin{book_deep.dtd}
\noindent La donnée valide est toujours
\smallXMLinUnchecked{toc_chap.xml}
\noindent Le résultat attendu (sans numérotation) est alors
\begin{sverb}
<!DOCTYPE html
  PUBLIC "-//W3C//DTD XHTML 1.0 Strict//EN"
  "http://www.w3.org/TR/xhtml1/DTD/xhtml1-strict.dtd">
<html xmlns:xhtml="http://www.w3.org/1999/xhtml"
      xmlns="http://www.w3.org/1999/xhtml"
      xml:lang="en" lang="en">
  <head>
    <meta http-equiv="Content-Type"
          content="text/html; charset=UTF-8"/>
    <title>Definitive XML Schema</title>
  </head>
  <body>
    <h2>Definitive XML Schema</h2>
    <p>by Priscilla Walmsley</p>
    <h3>Table of contents</h3>
    <ul>
      <li>[I] A quick tour of XML Schema
        <ul>
          <li>[I.1] An example schema</li>
          <li>[I.2] The components of XML Schema
            <ul>
              <li>[I.2.1] Declaration vs. definition</li>
              <li>[I.2.2] Global vs. local components</li>
            </ul>
          </li>
          <li>[I.3] Elements and attributes
            <ul>
              <li>[I.3.1] The tag/type distinction</li>
            </ul>
          </li>
        </ul>
      </li>
      <li>[II] Instances and schemas
        <ul>
          <li>[II.1] Using the instance attributes</li>
          <li>[II.2] Schema processing
            <ul>
              <li>[II.2.1] Validation</li>
              <li>[II.2.2] Augmenting the instance</li>
            </ul>
          </li>
        </ul>
      </li>
    </ul>
  </body>
</html>
\end{sverb}
qui serait probablement interprété par un navigateur ainsi:

\noindent\rule{\linewidth}{0.5pt}
\noindent\textbf{\large Definitive XML Schema}

\bigskip

\noindent by Priscilla Walmsley

\bigskip

\noindent\textbf{Table of contents}
\begin{itemize}
  \item [I] A quick tour of XML Schema
    \begin{itemize}
      \item [I.1] An example schema
      \item [I.2] The components of XML Schema
        \begin{itemize}
          \item [I.2.1] Declaration vs. definition
          \item [I.2.2] Global vs. local components
        \end{itemize}
      \item [I.3] Elements and attributes
        \begin{itemize}
          \item [I.3.1] The tag/type distinction
        \end{itemize}
    \end{itemize}
  \item [II] Instances and schemas
    \begin{itemize}
      \item [II.1] Using the instance attributes
      \item [II.2] Schema processing
        \begin{itemize}
          \item [II.2.1] Validation
          \item [II.2.2] Augmenting the instance
        \end{itemize}
    \end{itemize}
\end{itemize}
\noindent\rule{\linewidth}{0.5pt}

\bigskip

\noindent La solution suivante ne devrait pas présenter de difficultés maintenant:
\smallXSLTinUnchecked{num1.xsl}
\noindent On notera \texttt{<xsl:value-of select="@title"/>}, car les
titres sont des valeurs d'attributs, donc nous avons besoin de
\texttt{xsl:value-of} pour créer des nœuds textuels, tout comme
pour \texttt{"author/@first,author/@last"}, ce qui est le même que
\texttt{"(author/@first,author/@last)"}. Il aurait été possible
d'écrire \texttt{"author/@*"}, ce qui signifie: «~Toutes les valeurs
d'attributs de l'élément \texttt{author}, enfant du nœud
contextuel.~»

À présent, ajoutons un nombre entre crochets après la balise \XHTML
\texttt{<li>}, qui est la position de l'item dans la liste, comme
\begin{alltt}
\small<!DOCTYPE html
  PUBLIC "-//W3C//DTD XHTML 1.0 Strict//EN"
         "http://www.w3.org/TR/xhtml1/DTD/xhtml1-strict.dtd">
<html xmlns:xhtml="http://www.w3.org/1999/xhtml"
      xmlns="http://www.w3.org/1999/xhtml"
      xml:lang="en" lang="en">
  <head>
    <meta http-equiv="Content-Type"
          content="text/html; charset=UTF-8"/>
    <title>Definitive XML Schema</title>
  </head>
  <body>
    <h2>Definitive XML Schema</h2>
    <p>by Priscilla Walmsley</p>
    <h3>Table of contents</h3>
    <ul>
      <li>\textbf{[1]} [I] A quick tour of XML Schema
        <ul>
          <li>\textbf{[1]} [I.1] An example schema</li>
          <li>\textbf{[2]} [I.2] The components of XML Schema
            <ul>
              <li>\textbf{[1]} [I.2.1] Declaration vs. definition</li>
              <li>\textbf{[2]} [I.2.2] Global vs. local components</li>
            </ul>
          </li>
          <li>\textbf{[3]} [I.3] Elements and attributes
            <ul>
              <li>\textbf{[1]} [I.3.1] The tag/type distinction</li>
            </ul>
          </li>
        </ul>
      </li>
      <li>\textbf{[2]} [II] Instances and schemas
        <ul>
          <li>\textbf{[1]} [II.1] Using the instance attributes</li>
          <li>\textbf{[2]} [II.2] Schema processing
            <ul>
              <li>\textbf{[1]} [II.2.1] Validation</li>
              <li>\textbf{[2]} [II.2.2] Augmenting the instance</li>
            </ul>
          </li>
        </ul>
      </li>
    </ul>
  </body>
</html>
\end{alltt}
Les nombres ajoutés ont été graissés. Le seul changement réside dans
le patron filtrant les chapitres et les sections:
\begin{alltt}
\small <xsl:template match="section|chapter" as="element(xhtml:li)">
    <li xmlns="http://www.w3.org/1999/xhtml">
      \textbf{<xsl:value-of select="('[',position(),'] ',@title)"
                    separator=""/>}
      <xsl:if test="not(empty(section))">
        <ul><xsl:apply-templates select="section"/></ul>
      </xsl:if>
    </li>
  </xsl:template>
\end{alltt}
À l'attribut \texttt{separator} doit être affecté la chaîne vide, de
manière à ce que les items dans la sélection (chaînes et entiers)
soient convertis en un nœud textuel sans l'espace séparatrice par
défaut. Par exemple, si nous essayons \texttt{<xsl:value-of
  select="1,2,3">} le résultat est \texttt{'1 2 3'}.

Finalement, nous pouvons compléter la numérotation pour qu'elle
devienne ce que nous attendons dans la table des matières. Reprenons
le fil avec une entrée \emph{sans} nombres:
\smallXMLinUnchecked{toc_deep.xml}
\noindent et, pour rendre les choses un peu plus faciles, la sortie
comportera des chiffres arabes pour les chapitres comme pour les sections:
\begin{sverb}
<!DOCTYPE html
  PUBLIC "-//W3C//DTD XHTML 1.0 Strict//EN"
         "http://www.w3.org/TR/xhtml1/DTD/xhtml1-strict.dtd">
<html xmlns:xhtml="http://www.w3.org/1999/xhtml"
      xmlns="http://www.w3.org/1999/xhtml"
      xml:lang="en" lang="en">
  <head><meta http-equiv="Content-Type"
              content="text/html; charset=UTF-8"/>
        <title>Definitive XML Schema</title>
  </head>
  <body>
    <h2>Definitive XML Schema</h2>
    <p>by Priscilla Walmsley</p>
    <h3>Table of contents</h3>
    <ul>
      <li>[1] A quick tour of XML Schema
        <ul>
          <li>[1.1] An example schema</li>
          <li>[1.2] The components of XML Schema
            <ul>
              <li>[1.2.1] Declarations vs. definitions</li>
              <li>[1.2.2] Global vs. local components</li>
            </ul>
          </li>
          <li>[1.3] Elements and attributes
            <ul>
              <li>[1.3.1] The tag/type distinction</li>
            </ul>
          </li>
        </ul>
      </li>
      <li>[2] Instances and schemas
        <ul>
          <li>[2.1] Using the instance attributes</li>
          <li>[2.2] Schema processing
            <ul>
              <li>[2.2.1] Validation</li>
              <li>[2.2.2] Augmenting the instance</li>
            </ul>
          </li>
        </ul>
      </li>
    </ul>
  </body>
</html>
\end{sverb}
L'idée consiste à ajouter un paramètre \texttt{prefix} au patron
filtrant les chapitres et les sections, qui reçoivent le préfixe
numérotant le parent. Par exemple, lorsque nous filtrons la section
intitulée «~Declarations vs. definitions~», la valeur du paramètre est
la séquence \texttt{(1,'.',2,'.')}, donc nous concaténons simplement
la position de la section parmi les nœuds de même niveau,
soit~\texttt{1}. Ensuite, nous créons un nœud textuel pour
formatter \texttt{[1.2.1]}. Voici la modification:
\begin{sverb}
  <xsl:template match="chapter|section" as="element(xhtml:li)">
    <xsl:param name="prefix" as ="item()*"/>
    <xsl:variable name="current" select="($prefix,position())"/>
    <li xmlns="http://www.w3.org/1999/xhtml">
      <xsl:value-of select="('[',$current,'] ',@title)"
                    separator=""/>
      <xsl:if test="not(empty(section))">
        <ul>
          <xsl:apply-templates select="section">
             <xsl:with-param name="prefix"
                             select="($current,'.')"/>
          </xsl:apply-templates>
        </ul>
      </xsl:if>
    </li>
  </xsl:template>
\end{sverb}
Remarquons que la première application de ce patron n'a pas changé:
\begin{sverb}
        ...
        <ul><xsl:apply-templates select="chapter"/></ul>
        ...
\end{sverb}
parce qu'en \XSLT, une séquence vide est implicitement transmise, ce
qui est pratique en l'occurrence.

\mypar{Tri des feuilles}

L'objectif de cet exercice est de produire une transformation \XSLT
qui prend un document représentant un arbre binaire dont les feuilles
contiennent un entier et les trie en ordre non-décroissant. Les
entiers dans la séquence ordonnée doivent être séparés par des
virgules et conclus par un point dans le texte final. Par exemple, le
document \XML
\smallXMLinUnchecked{num.xml}
\noindent mène au résultat suivant:
{\small\verbatiminput{num_out.txt}}
Le format de la sortie devrait nous rappeler des valeurs ponctuées,
\vpageref{par:CSV}, et l'ordonnancement de l'interclassement de
séquences triées, \vpageref{par:merging}. Alors, une première approche
serait de parcourir l'arbre et de collecter les nombres sous forme de
séquences ordonnées qui sont interclassées à l'aide du patron nommé
\texttt{merge} jusqu'à ce qu'il ne reste qu'une seule séquence à
laquelle nous pouvons appliquer le patron nommé \texttt{csv}. Plus
précisément, ce parcours peut être effectué en parallèle: les
applications récursives fournissent deux séquences triées qui sont
interclassées; si le nœud contextuel est l'élément racine, alors
nous appelons \texttt{csv}. Autrement, les interclassements sont
effectués d'une manière purement ascendante (c'est-à-dire après la fin
des appels récursifs). Par conséquent, nous commençons avec
\begin{alltt}
\small  <xsl:template match="/" as="text()*">
    <xsl:call-template name="csv">
      <!-- The following cast is needed. -->
      <xsl:with-param name="items" \textbf{as="xs:integer*"}>
        <xsl:apply-templates select="num"/>
      </xsl:with-param>
    </xsl:call-template>
    <xsl:text>&#10;</xsl:text>
  </xsl:template>
\end{alltt}
Remarquons que l'annotation de type \texttt{xs:integer*} est
nécessaire quand on invoque le patron \texttt{csv}, dont le type est
\begin{sverb}
  <xsl:template name="csv" as="text()*">
    <xsl:param name="items" as="item()*"/>
    ...
  </xsl:template>
\end{sverb}
Le reste est
\begin{sverb}
  <xsl:template match="num" as="xs:integer*">
    <xsl:choose>
      <xsl:when test="empty(@val)">
        <xsl:call-template name="merge">
          <xsl:with-param name="fst" as="xs:integer*">
            <xsl:apply-templates select="num[1]"/>
          </xsl:with-param>
          <xsl:with-param name="snd" as="xs:integer*">
            <xsl:apply-templates select="num[2]"/>
          </xsl:with-param>
        </xsl:call-template>
      </xsl:when>
      <xsl:otherwise>
        <xsl:value-of select="@val"/>
      </xsl:otherwise>
    </xsl:choose>
  </xsl:template>
\end{sverb}
Le patron \texttt{merge} a besoin d'être simplifié et généralisé parce
qu'il était trop spécialisé:
\begin{sverb}
  <xsl:template name="merge" as="element(item)*">
    <xsl:param name="seq1" as="element(item)*"/>
    <xsl:param name="seq2" as="element(item)*"/>
    ...
  </xsl:template>
\end{sverb}
Il faut aussi qu'il reçoive deux entiers maintenant:
\begin{sverb}
  <xsl:template name="merge" as="xs:integer*">
    <xsl:param name="fst" as="xs:integer*"/>
    <xsl:param name="snd" as="xs:integer*"/>
    <xsl:choose>
      <xsl:when test="empty($fst)">
        <xsl:sequence select="$snd"/>
      </xsl:when>
      <xsl:when test="empty($snd)">
        <xsl:sequence select="$fst"/>
      </xsl:when>
      <xsl:when test="$fst[1] lt $snd[1]">
        <xsl:sequence select="$fst[1]"/>
        <xsl:call-template name="merge">
          <xsl:with-param name="fst"
                          select="$fst[position()>1]"/>
          <xsl:with-param name="snd" select="$snd"/>
        </xsl:call-template>
      </xsl:when>
      <xsl:otherwise>
        <xsl:sequence select="$snd[1]"/>
        <xsl:call-template name="merge">
          <xsl:with-param name="fst" select="$fst"/>
          <xsl:with-param name="snd"
                          select="$snd[position()>1]"/>
        </xsl:call-template>
      </xsl:otherwise>
    </xsl:choose>
  </xsl:template>
\end{sverb}

\paragraph{Exercice}

Est-ce que l'exemple~2.2 dans le livre de~\cite{Mangano_2006},
page~39, est vraiment en forme terminale?
