\section{Tri optimal}
\label{sec_opt_sort}
\index{tri}

Trier consiste à arranger une suite donnée d'objets, appelés
\emph{clés}\index{tri!clé}, pour qu'ils satisfassent un ordre
prédéfini. D'après \cite{Knuth_1998}, Les premiers algorithmes de tri
ont été inventés et automatisés dans des tabulateurs à la fin du
dix-neuvième siècle pour l'établissement du recensement des États-Unis
d'Amérique.

\paragraph{Permutations}
\label{par_permutations}
\index{permutation}

\hspace*{-10pt} Nous avons vu \vpageref{par_mean_sort} que le coût
moyen d'un algorithme de tri opérant par comparaisons est défini comme
étant la moyen\-ne arithmétique des coûts du tri de toutes les
permutations d'une taille fixe. Une permutation de \((1,2,\dots,n)\)
est un \(n\)-uplet \((a_1,a_2,\dots,a_n)\) tel que \(a_i \in
\{1,\dots,n\}\) et \({a_i \neq a_j}\) pour tout \({i \neq j}\). Par
exemple, toutes les permutations de \((1,2,3)\) sont
\begin{equation*}
(1,2,3) \quad (1,3,2) \quad (2,1,3) \quad (2,3,1) \quad (3,1,2) \quad
(3,2,1).
\end{equation*}
Étant donné toutes les permutations de \((1,2,\dots,n-1)\),
construisons inductivement toutes les permutations de
\((1,2,\dots,n)\). Si \((a_1,a_2,\dots,a_{n-1})\) est une permutation
de \((1,2,\dots,n-1)\), alors nous pouvons construire
\(n\)~permutations de \((1,2,\dots,n)\) en insérant~\(n\) entre toutes
les paires d'objets consécutifs dans \((a_1,a_2,\dots,a_{n-1})\):
\begin{equation*}
(\boldsymbol{n},a_1,a_2,\dots,a_{n-1})\quad
(a_1,\boldsymbol{n},a_2,\dots,a_{n-1})\quad \ldots \quad
(a_1,a_2,\dots,a_{n-1},\boldsymbol{n}).
\end{equation*}
Par exemple, il est clair que toutes les permutations de \((1,2)\)
sont \((1,2)\) et \((2,1)\). La méthode nous conduit de \((1,2)\) à
\((\boldsymbol{3},1,2)\), \((1,\boldsymbol{3},2)\) et
\((1,2,\boldsymbol{3})\); et de \((2,1)\) à \((\boldsymbol{3},2,1)\),
\((2,\boldsymbol{3},1)\) et \((2,1,\boldsymbol{3})\). Si nous
nommons~\(p_n\) le nombre de permutations de \(n\)~éléments, nous
tirons de ce qui précède la récurrence \(p_n = n \cdot p_{n-1}\), qui,
avec l'ajout de l'évidente égalité \(p_1 = 1\), produit \(p_n = n!\),
pour tout \({n > 0}\), exactement comme nous nous y attendions. Si les
\(n\)~objets à permuter ne sont pas \((1,2,\dots,n)\) mais, par
exemple, \((\textsf{b},\textsf{d},\textsf{a},\textsf{c})\), il suffit
alors simplement d'associer à chacun d'eux leur index dans le
\(n\)-uplet, par exemple \textsf{b}~est représenté par~\(1\),
\textsf{d}~par~\(2\), \textsf{a}~par~\(3\) et \textsf{c}~par~\(4\),
donc le \(n\)-uplet est associé à \((1,2,3,4)\) et, par exemple, la
permutation \((4,1,2,3)\) est alors le codage de
\((\textsf{c},\textsf{b},\textsf{d},\textsf{a})\).

\paragraph{Factorielle}\index{factorielle}

Nous avons rencontré la fonction factorielle dans l'introduction et
ici encore. Voici un simple calcul, proposé par
\cite{GrahamKnuthPatashnik_1994}, pour caractériser sa croissance
asymptotique:
\begin{equation*}
n!^2 = (1 \cdot 2 \cdot \ldots \cdot n) (n \cdot \ldots \cdot 2 \cdot
1) = \prod_{k=1}^{n}{k(n+1-k)}.
\end{equation*}
La parabole \(P(k) := k(n+1-k) = -k^2 + (n+1)k\) atteint son maximum
là où sa dérivée s'annule: \(P'(k_{\max}) = 0 \Leftrightarrow
k_{\max}=(n+1)/2\). L'ordonnée correspondante est \(P(k_{\max}) =
((n+1)/2)^2 = k_{\max}^2\). Quand \(k\)~varie de~\(1\) à~\(n\),
l'ordonnée minimum, \(n\), est atteinte aux points d'abscisses
\(1\)~et~\(n\), comme le montre la \fig~\vref{fig_parabola}.
\begin{figure}
\centering
\includegraphics[bb=55 563 205 730]{parabola}
\caption{Parabole \(P(k) := k(n+1-k)\)\label{fig_parabola}}
\end{figure}
Donc, \(1 \leqslant k \leqslant k_{\max}\) implique
\begin{equation*}
P(1) \leqslant P(k) \leqslant P(k_{\max}),\quad \text{c'est-à-dire,}
\quad n \leqslant k(n+1-k) \leqslant \left(\frac{n+1}{2}\right)^2.
\end{equation*}
En multipliant les côtés pour des valeurs de la variable~\(k\)
décrivant l'intervalle discret \([1..n]\) produit
\begin{gather*}
n^n = \prod_{k=1}^{n}{n} \leqslant n!^2 
\leqslant
\prod_{k=1}^{n}{\left(\!\frac{n+1}{2}\!\right)^2} \!\!=
\left(\!\frac{n+1}{2}\!\right)^{2n}
\!\!\!\Rightarrow\!
n^{n/2} \leqslant n! \leqslant \left(\!\frac{n+1}{2}\!\right)^n.
\end{gather*}
Il clair maintenant que \(n!\)~est \emph{exponentielle}, donc sa
croissance asymptotique domine celle de tout polynôme. Concrètement,
une fonction dont le coût est proportionnel à la factorielle est
inutile même pour de petites données. Pour les cas où une équivalence
est préférée, la formule de Stirling\index{Stirling (formule de)}
affirme
\begin{equation}
n! \sim n^n e^{-n} \sqrt{2\pi n}.\label{eq_Stirling}
\end{equation}

\paragraph{Dénombrer les permutations}

Nous voulons écrire un programme qui forme toutes les permutations
d'une pile donnée. Nous définissons la
fonction~\fun{perm/1}\index{perm@\fun{perm/1}} telle que
\(\fun{perm}(s)\) est la pile de toutes les permutations des éléments
de la pile~\(s\). Nous allons employer la méthode inductive vue
ci-dessus et qui opérait en insérant un nouvel objet dans tous les
intervalles possibles d'une permutation plus courte.
\begin{equation*}
\fun{perm}(\el)         \xrightarrow{\smash{\alpha}} \el;\quad
\fun{perm}([x])         \xrightarrow{\smash{\beta}} [[x]];\quad
\fun{perm}(\cons{x}{s}) \xrightarrow{\gamma}
                          \fun{dist}(x,\fun{perm}(s)).
\end{equation*}
La fonction~\fun{dist/2}\index{dist@\fun{dist/2}} (\emph{distribuer})
est telle que \(\fun{dist}(x,s)\) est la pile de toutes les piles
obtenues en insérant l'élément~\(x\) à toutes les positions dans la
pile~\(s\). Puisqu'une telle insertion dans une permutation de
longueur~\(n\) résulte en une permutation de longueur~\(n+1\), nous
devons ajouter ces nouvelles permutations à celles de même longueur
déjà formées:
\begin{equation*}
\fun{dist}(x,\el)         \xrightarrow{\smash{\delta}} \el;\quad
\fun{dist}(x,\cons{p}{t}) \xrightarrow{\smash{\epsilon}}
                            \fun{cat}(\fun{ins}(x,p),\fun{dist}(x,t)).
\end{equation*}
L'appel~\(\fun{ins}(x,p)\)\index{ins@\fun{ins/2}} calcule la pile de
permutations qui résultent de l'insertion de~\(x\) à toutes les
positions dans la permutation~\(p\). Par conséquent, il vient
\begin{equation*}
\fun{ins}(x,\el) \xrightarrow{\smash{\zeta}} [[x]];\quad
\fun{ins}(x,\cons{j}{s}) \xrightarrow{\smash{\eta}}
 \cons{\cons{x,j}{s}}{\fun{push}(j,\fun{ins}(x,s))}.
\end{equation*}
où la fonction~\fun{push/2}\index{push@\fun{push/2}} (à ne pas
confondre avec la fonction de même nom et arité à la
section~\ref{sec_persistence}) est telle que tout
appel~\(\fun{push}(x,t)\) empile l'élément~\(x\) sur toutes les
permutations de la pile de permutation~\(t\). Leur ordre reste
invariant:
\begin{equation*}
\fun{push}(x,\el) \xrightarrow{\smash{\theta}} \el;\quad
\fun{push}(x,\cons{p}{t}) \xrightarrow{\smash{\iota}}
 \cons{\cons{x}{p}}{\fun{push}(x,t)}.
\end{equation*}
Maintenant, nous pouvons produire toutes les permutations de
\((4,1,2,3)\) ou \((\fun{c}, \fun{b}, \fun{d}, \fun{a})\) en appelant
\(\fun{perm}([4,1,2,3])\) ou \(\fun{perm}([\fun{c}, \fun{b}, \fun{d},
\fun{a}])\). Nous pouvons remarquer qu'après la formation des
permutations de longueur \(n+1\), les permutations de longueur~\(n\)
ne sont plus nécessaires, ce qui permet à un environnement d'exécution
de réutiliser la mémoire correspondante pour d'autres usages (un
processus appelé \emph{glanage de cellules}\index{glanage de
  cellules}). En ce qui concerne les coûts, la définition de
\fun{push/2} implique
\begin{equation*}
\C{\fun{push}}{0} \eqn{\theta} 1;\qquad
\C{\fun{push}}{n+1} \eqn{\iota} 1 + \C{\fun{push}}{n},\quad
\text{avec \(n \geqslant 0\)}.
\end{equation*}
Nous déduisons aisément \(\C{\fun{push}}{n} = n +
1\).\index{push@$\C{\fun{push}}{n}$} Nous savons que le résultat de
\(\fun{ins}(x,p)\) est une pile de longueur~\(n+1\) si \(p\)~est une
permutation de \(n\)~objets dans laquelle nous insérons un objet
supplémentaire~\(x\). Par conséquent, la définition de \fun{ins/2}
donne lieu aux récurrences suivantes:
\begin{equation*}
\C{\fun{ins}}{0}   \eqn{\zeta} 1;\quad
\C{\fun{ins}}{k+1} \eqn{\eta} 1 + \C{\fun{push}}{k+1} +
\C{\fun{ins}}{k} = 3 + k + \C{\fun{ins}}{k}, \quad \text{où \(k
                    \geqslant 0\)},
\end{equation*}
et~\(\C{\fun{ins}}{k}\) est le coût de \(\fun{ins}(x,p)\) avec~\(p\)
de longueur~\(k\). En sommant membre à membre pour toutes les valeurs
de~\(k\) allant de~\(0\) à~\(n-1\), avec \({n>0}\):
\begin{equation*}
\sum_{k=0}^{n-1}{\C{\fun{ins}}{k+1}}
  = \sum_{k=0}^{n-1}{3} + \sum_{k=0}^{n-1}{k}
     + \sum_{k=0}^{n-1}{\C{\fun{ins}}{k}}.
\end{equation*}
En éliminant les termes identiques dans
\(\sum_{k=0}^{n-1}{\C{\fun{ins}}{k+1}}\) et
\(\sum_{k=0}^{n-1}{\C{\fun{ins}}{k}}\), nous
déduisons\index{ins@$\C{\fun{ins}}{n}$}
\begin{equation*}
\C{\fun{ins}}{n}
  = 3n + \tfrac{1}{2}n(n-1) + \C{\fun{ins}}{0}
  = \tfrac{1}{2}{n^2} + \tfrac{5}{2}{n} + 1.
\end{equation*}
Cette dernière équation est en fait valide même si \({n = 0}\).
Soit~\(\C{\fun{dist}}{n!}\) le coût de la distribution d'un élément
parmi \(n!\)~permutations de longueur~\(n\). La définition de
\fun{dist/2}\index{dist@\fun{dist/2}} montre qu'elle itère des appels
à \fun{cat/2}\index{cat@\fun{cat/2}} et
\fun{ins/2}\index{ins@\fun{ins/2}} dont les argument sont toujours de
longueur \(n+1\)~et~\(n\), respectivement, parce que toutes les
permutations traitées ici ont la même longueur. Nous déduisons,
pour~\(k \geqslant 0\), les récurrences suivantes:
\begin{equation*}
\C{\fun{dist}}{0} \eqn{\delta} 1;\quad
\C{\fun{dist}}{k+1}
  \eqn{\epsilon} 1 + \C{\fun{cat}}{n+1} + \C{\fun{ins}}{n}
                    + \C{\fun{dist}}{k}
  = \tfrac{1}{2}{n^2} + \tfrac{7}{2}{n} + 4 + \C{\fun{dist}}{k},
\end{equation*}
puisque nous savons déjà que \(\C{\fun{cat}}{n} = n +
1\)\index{cat@$\C{\fun{cat}}{n}$} et nous connaissons la valeur
de~\(\C{\fun{ins}}{n}\). En sommant membre à membre la dernière
équation pour toutes les valeurs de~\(k\) allant de~\(0\) à~\(n!-1\),
nous pouvons éliminer la plupart des termes et obtenir une définition
non-récurrente de
\(\C{\fun{dist}}{n!}\)\index{dist@$\C{\fun{dist}}{n}$}:
\begin{align*}
\sum_{k=0}^{n!-1}{\C{\fun{dist}}{k+1}}
   &= \sum_{k=0}^{n!-1}{\left(\frac{1}{2}{n^2}
      + \frac{7}{2}{n} + 4\right)}
      + \sum_{k=0}^{n!-1}{\C{\fun{dist}}{k}},\\
\C{\fun{dist}}{n!}
  &= \left(\frac{1}{2}{n^2}+\frac{7}{2}{n}+4\right)n! +
\C{\fun{dist}}{0} = \frac{1}{2}(n^2 + 7n + 8)n! + 1.
\end{align*}
Évaluons finalement le coût de \(\fun{perm}(s)\), noté
\(\C{\fun{perm}}{k}\)\index{perm@$\C{\fun{perm}}{n}$|(}, où \(k\)~est
la longueur de la pile~\(s\). D'après les règles de~\clause{\alpha}
à~\clause{\gamma}, nous déduisons les équations récurrentes suivantes,
où \(k > 0\):
\begin{align*}
\C{\fun{perm}}{0}   &\eqn{\alpha} 1;\qquad
\C{\fun{perm}}{1}   \eqn{\beta} 1;\\
\C{\fun{perm}}{k+1}
  &\eqn{\gamma} 1 + \C{\fun{perm}}{k} + \C{\fun{dist}}{k!}
   = \tfrac{1}{2}(k^2 + 7k + 8)k! + 2 + \C{\fun{perm}}{k}.
\intertext{Sommant membre à membre, la plupart des termes s'annulent:}
\sum_{k=1}^{n-1}{\C{\fun{perm}}{k+1}}
  &= \frac{1}{2}\sum_{k=1}^{n-1}{(k^2+7k+8)k!} + \sum_{k=1}^{n-1}{2}
     + \sum_{k=1}^{n-1}{\C{\fun{perm}}{k}},\\
\C{\fun{perm}}{n}
  &= \frac{1}{2}\sum_{k=1}^{n-1}{(k^2 + 7k + 8)k!}
     + 2(n-1) + \C{\fun{perm}}{1}\\
  &= \frac{1}{2}\sum_{k=1}^{n-1}{((k+2)(k+1)+6+4k)k!} + 2n - 1\\
  &= \frac{1}{2}\sum_{k=1}^{n-1}{(k+2)(k+1)k!}
     + 3 \sum_{k=1}^{n-1}{k!} + 2 \sum_{k=1}^{n-1}{kk!} + 2n - 1\\
  &= \frac{1}{2}\sum_{k=1}^{n-1}{(k+2)!}
     + 3 \sum_{k=1}^{n-1}{k!} + 2 \sum_{k=1}^{n-1}{kk!} + 2n - 1\\
  &= \frac{1}{2}\sum_{k=3}^{n+1}{k!}
     + 3 \sum_{k=1}^{n-1}{k!} + 2 \sum_{k=1}^{n-1}{kk!} + 2n - 1\\
%  &= \left(\frac{1}{2}((n+1)! + n! - 2! - 1!)
%     + \frac{1}{2}\sum_{k=1}^{n-1}{k!}\right)
%     + 3 \sum_{k=1}^{n-1}{k!}\\
%  &\phantom{= x} + 2 \sum_{k=1}^{n-1}{kk!} + 2n - 1.\\
  &= \frac{1}{2}{(n+2)n!} + \frac{7}{2}\sum_{k=1}^{n-1}{k!}
     + 2 \sum_{k=1}^{n-1}{kk!} + 2n - \frac{5}{2}.
\end{align*}
Cette dernière équation est en fait valide même si~\(n = 1\). Une des
sommes a une forme close simple:
\begin{equation*}
\sum_{k=1}^{n-1}{kk!} = \sum_{k=1}^{n-1}((k+1)! - k!) =
\sum_{k=2}^{n}{k!} - \sum_{k=1}^{n-1}{k!} = n! - 1.
\end{equation*}
Précisons alors notre dérivation précédente:
\begin{align*}
\C{\fun{perm}}{n}
  &= \frac{1}{2}{nn!} + n! + \frac{7}{2}\sum_{k=1}^{n-1}{k!}
     + 2(n! - 1) + 2n - \frac{5}{2}\\
  &= \frac{1}{2}{nn!} + 3n! 
     + 2n - \frac{9}{2} + \frac{7}{2}\sum_{k=1}^{n-1}{k!},\quad
     \text{où \(n > 0\)}.
\end{align*}
La somme restante est appelée la \emph{factorielle gauche}
\citep{Kurepa_1971}\index{factorielle!$\sim$ gauche} et elle est notée
habituellement comme suit:
\begin{equation*}
!n := \sum_{k=1}^{n-1}{k!},\quad \text{avec \(n > 0\)}.
\end{equation*}
Malheureusement, on ne connaît pas de forme close pour la factorielle
gauche.  C'est en fait une situation courante lorsque l'on détermine
le coût de fonctions relativement complexes. La meilleure suite à
adopter est alors de trouver une approximation asymptotique du
coût. D'abord, \(n!  \leqslant !(n+1)\). Ensuite,
\begin{equation*}
!(n+1) - n! \leqslant (n-2) \cdot (n-2)! + (n-1)! \leqslant
(n-1) \cdot (n-2)! + (n-1)! = 2 (n-1)!
\end{equation*}
Par conséquent,
\begin{equation*}
1 \leqslant \frac{!(n+1)}{n!} \leqslant \frac{n! + 2(n-1)!}{n!} = 1 +
\frac{2}{n} \;\Rightarrow\; !n \sim (n-1)!
\end{equation*}
Nous avons \((n+1)! = (n+1)n!\), d'où \((n+1)!/(nn!) = 1 + 1/n\), et
\(nn!  \sim (n+1)!\). Donc,
\begin{equation*}
\abovedisplayshortskip=0pt
\abovedisplayskip=0pt
%\belowdisplayskip=0pt
\C{\fun{perm}}{n} \sim \frac{1}{2}(n+1)! + 3n! + \frac{7}{2}(n-1)!
+ 2n - \frac{9}{2} \sim \frac{1}{2}(n+1)!
\end{equation*}
C'est un programme à la lenteur insupportable, comme on s'y
attendrait. Nous ne devrions même pas espérer calculer facilement
\(\C{\fun{perm}}{11}\) \index{perm@$\C{\fun{perm}}{n}$|)} et il n'y a
pas moyen d'améliorer significativement le coût parce le nombre de
permutations qu'il forme est intrinsèquement exponentiel, donc il
suffirait ne serait-ce que d'un appel de fonction par permutation pour
tout de même engendrer un coût exponentiel. En d'autres termes, la
mémoire nécessaire pour contenir le résultat a une taille qui est
exponentielle en fonction de la taille de la donnée, parce qu'au moins
un appel de fonction par permutation est nécessaire. Pour une étude
approfondie du calcul de toutes les permutations d'une taille donnée,
se référer au travail encyclopédique de \cite{Knuth_2011}.

\paragraph{Permutations et tri}

Les permutations\index{permutation|(} sont un sujet digne d'être
étudier en détail à cause de leur lien intime avec le tri. Une
permutation peut être conçue comme le bouleversement de clés
originellement ordonnées et une permutation de tri les remet à leur
place. Une notation légèrement différente pour les permutations se
révèle utile ici, une qui montre les index et les clés ensemble. Par
exemple, au lieu d'écrire \(\pi_1 = (2,4,1,5,3)\), nous écrivons
\begin{equation*}
\pi_1 =
\begin{pmatrix}
1 & 2 & 3 & 4 & 5\\
2 & 4 & 1 & 5 & 3
\end{pmatrix}.
\end{equation*}
La première ligne est faite des index ordonnés et la seconde ligne
contient les clés. En général, une permutation \(\pi =
(a_1,a_2,\dots,a_n)\) équivaut à
\begin{equation*}
\pi =
\begin{pmatrix}
     1 &      2 & \dots &     n\\
\pi(1) & \pi(2) & \dots & \pi(n)
\end{pmatrix},
\end{equation*}
où~\(a_i = \pi(i)\), pour tout~\(i\) allant de~\(1\) à~\(n\). La
permutation suivante~\(\pi_s\) trie les clés de~\(\pi_1\):
\begin{equation*}
\pi_s =
\begin{pmatrix}
1 & 2 & 3 & 4 & 5\\
3 & 1 & 5 & 2 & 4
\end{pmatrix}.
\end{equation*}
Pour voir comment, nous définissons la composition
\index{permutation!composition} de deux
permutations\index{permutation} \(\pi_a\)~et~\(\pi_b\) ainsi:
\begin{equation*}
\pi_b \circ \pi_a :=
\begin{pmatrix}
              1 &               2 & \dots & n\\
\pi_b(\pi_a(1)) & \pi_b(\pi_a(2)) & \dots & \pi_b(\pi_a(n)) 
\end{pmatrix}.
\end{equation*}
Alors \(\pi_s~\circ~\pi_1 = \mathcal{I}\), où la \emph{permutation
  identité}~\(\mathcal{I}\)\index{permutation!identité} est telle que
\(\mathcal{I}(k) = k\), pour tous les index~\(k\). En d'autres termes,
\(\pi_s = \pi_1^{-1}\), c'est-à-dire que \emph{trier une permutation
  consiste à construire son inverse}\index{permutation!inverse}:
\begin{equation*}
\begin{pmatrix}
1 & 2 & 3 & 4 & 5\\
3 & 1 & 5 & 2 & 4
\end{pmatrix}
\circ
\begin{pmatrix}
1 & 2 & 3 & 4 & 5\\
2 & 4 & 1 & 5 & 3
\end{pmatrix}
=
\begin{pmatrix}
1 & 2 & 3 & 4 & 5\\
1 & 2 & 3 & 4 & 5
\end{pmatrix}.
\end{equation*}

Une autre représentation graphique des permutations et de leur
composition\index{permutation!composition} se fonde sur la vue
qu'elles sont des bijections d'un intervalle discret sur lui-même. On
parle alors de \emph{graphes bipartis}\index{graphe!$\sim$
  biparti}\index{permutation!bigraphe}, aussi appelés
\emph{bigraphes}\index{bigraphe|see{graphe, biparti}}. De tels
graphes sont faits de deux ensembles disjoints de n{\oe}uds de même
cardinal, les index et les clés, et les arcs joignent un index et une
clé, sans partager des n{\oe}uds avec d'autres arcs. Par exemple, la
permutation ~\(\pi_1\) est montrée à la \fig~\vref{fig_perm1}
\begin{figure}
\centering
\subfloat[Bigraphe de \(\pi_1 = (2,4,1,5,3)\)\label{fig_perm1}]{%
\includegraphics[bb=79 662 210 721]{perm1}
}
\qquad\qquad
\subfloat[Bigraphe de \(\pi_1^{-1} = (3,1,5,2,4)\)\label{fig_perm3}]{%
\includegraphics[bb=77 662 214 721]{perm3}
}
\caption{Permutation \(\pi_1\) et son inverse \(\pi_1^{-1}\)
\label{fig_pi_1}\index{permutation!inverse}}
\end{figure}
et son inverse \(\pi_1^{-1}\) est visible à la
\fig~\vref{fig_perm3}. La composition\index{permutation!composition}
de~\(\pi_1^{-1}\) et~\(\pi_1\) est obtenue en identifiant les clés
de~\(\pi_1\) avec les index de~\(\pi_1^{-1}\), comme montré à la
\fig~\vref{fig_perm4}.
\begin{figure}[t]
\centering
\subfloat[\(\pi_1^{-1} \circ \pi_1\)\label{fig_perm4}]{%
\includegraphics[bb=81 634 204 721]{perm4}
}%
\qquad\qquad
\subfloat[\(\mathcal{I} = \pi_1^{-1} \circ \pi_1\)\label{fig_perm2}]{%
\includegraphics[bb=81 634 204 721]{perm2}
}
\caption{Application de \(\pi_1\) à \(\pi_1^{-1}\).}
\end{figure}
La permutation identité\index{permutation!identité} est obtenue en
remplaçant deux arcs adjacents par leur clôture transitive
\index{clôture transitive} et en effaçant le n{\oe}ud intermédiaire,
comme montré à la \fig~\vref{fig_perm2}. Notons qu'une permutation
peut être son propre inverse, comme
\begin{equation*}
\pi_3 =
\begin{pmatrix}
1 & 2 & 3 & 4 & 5\\
3 & 4 & 1 & 2 & 5
\end{pmatrix}.
\end{equation*}
À la \fig~\ref{fig_involution}
\begin{figure}
\centering
\subfloat[\(\pi_3 \circ \pi_3\)\label{fig_inv1}]{%
\includegraphics[bb=81 634 204 721]{inv1}
}
\qquad\qquad
\subfloat[\(\pi_3 \circ \pi_3\)]{%
\includegraphics[bb=81 634 204 721]{inv2}
}
\caption{L'involution \(\pi_3\) se trie elle-même\label{fig_involution}}
\end{figure}
est montré que \(\pi_3~\circ~\pi_3 = \pi_3\), donc \(\pi_3\) est une
\emph{involution}.\index{permutation!involution}\index{permutation!inverse}

L'étude des permutations et de leur propriétés fondamentales permet de
comprendre les algorithmes de tri et, en particulier, leur coût
moyen. Elle permet aussi de quantifier le désordre. Étant donné
\((1,3,5,2,4)\), nous voyons que seules les paires de clés \((3,2)\),
\((5,2)\) et~\((5,4)\) ne sont pas ordonnées. En général, étant donné
\((a_1, a_2, \dots, a_n)\), les paires \((a_i,a_j)\) telles que~\(i <
j\) et~\(a_i > a_j\) sont appelées
\emph{inversions}\index{permutation!inversion}. Plus il y a
d'inversions, plus le désordre est grand. Sans surprise, la
permutation identité\index{permutation!identité} ne possède aucune
inversion et la permutation \(\pi_1 = (2,4,1,5,3)\), que nous avons
étudiée précédemment, en a~\(4\). Lorsque nous considérons les
bigraphes de permutations\index{graphe!$\sim$
  biparti}\index{permutation!bigraphe}, une
inversion\index{permutation!inversion} correspond à l'intersection de
deux arcs, plus précisément, elle est la paire constituée des clés
pointées par deux arcs. Par conséquent, le nombre d'inversions
\index{permutation!inversion} est le nombre de croisement d'arcs,
donc, par exemple, \(\pi_1^{-1}\) a~\(4\) inversions. En fait,
l'\emph{inverse\index{permutation!inverse} d'une permutation a le même
  nombre d'inversions\index{permutation!inversion} que la
  permutation\index{permutation} elle-même}. Cela est clairement vu
par la comparaison des bigraphes\index{permutation!bigraphe}
de~\(\pi_1\) et~\(\pi_1^{-1}\) à la \fig~\vref{fig_pi_1}: pour déduire
le bigraphe de~\(\pi_1^{-1}\) de celui correspondant à~\(\pi_1\),
inversons l'orientation de chaque arc, puis échangeons les index et
les clés, c'est-à-dire les deux lignes de n{\oe}uds. Une autre façon
est d'imaginer que nous plions vers les bas la feuille de papier le
long de la ligne des clés, puis regardons à travers et inversons les
arcs. De toute façon, les intersections sont invariantes. La symétrie
horizontale est évidente aux \figs~\ref{fig_perm4} et~\ref{fig_inv1}.

\paragraph{Minimiser le coût maximum}
\label{par_minimax}

Après l'analyse du coût d'un algorithme de tri opérant par
comparaisons, nous aurons besoin de savoir dans quelle mesure son
efficacité est proche de celle d'un algorithme
optimal.\index{tri!optimalité|(} Le premier problème théorique que
nous examinerons est celui de minimiser le coût maximum. La
\fig~\vref{fig_cmp_tree}
\begin{figure}
\centering
\includegraphics[bb=71 636 327 723]{cmp_tree}
\caption{Un arbre de comparaisons pour trois clés\label{fig_cmp_tree}}
\end{figure}
montre l'arbre de toutes les comparaisons possibles
\index{arbre!$\sim$ de comparaison|see{tri}} pour trier trois
nombres. Les \emph{n{\oe}uds externes}\index{arbre
  binaire!n{\oe}ud externe}
\label{def_external_node} sont toutes les
permutations\index{permutation|)} de \((a_1,a_2,a_3)\). Les
\emph{n{\oe}uds internes}\index{arbre binaire!n{\oe}ud interne} sont
des comparaisons entre deux clés \(a_i\)~et~\(a_j\), notée
\(a_i?a_j\). Remarquons que les feuilles, dans ce cadre, sont les
n{\oe}uds internes avec deux enfants qui sont des n{\oe}uds externes.
Si \({a_i < a_j}\), alors cette propriété est vraie partout dans le
sous-arbre gauche, sinon \({a_i > a_j}\) est avérée dans le sous-arbre
droit. Cet arbre n'est qu'un arbre parmi d'autres: il correspond à un
algorithme qui commence par comparer \(a_1\)~et~\(a_2\) et il y a,
bien sûr, bien d'autres stratégies. Mais il ne contient pas de
comparaisons redondantes: si un chemin de la racine à une feuille
inclut \({a_i < a_j}\) et \({a_j < a_k}\), nous n'avons pas besoin d'y
trouver la comparaison \({a_i < a_k}\). La \fig~\vref{fig_red_cmp}
\begin{figure}
\centering
\includegraphics[bb=71 638 217 723]{red_cmp}
\caption{Inutilité de \(a_1 > a_3\)\label{fig_red_cmp}}
\end{figure}
%\end{wrapfigure}
montre un arbre de comparaisons avec une telle comparaison
redondante. Le n{\oe}ud externe spécial~\(\bot\) ne correspond à
aucune permutation parce qu'une comparaison \({a_1 < a_3}\) ne peut
échouer car elle est impliquée par transitivité par les comparaisons
précédentes sur le chemin depuis la racine.
\begin{quote}
  \emph{Un arbre de comparaisons non-redondantes pour \(n\)~clés
    a \(n!\)~n{\oe}uds externes.}
\end{quote}
Étant donné que nous étudions ici comment trier \(n\)~clés avec un
nombre minimum de comparaisons, nous considérerons par la suite les
arbres de comparaisons optimaux avec \(n!\)~n{\oe}uds externes. De
plus, parmi eux, nous voulons déterminer les arbres pour lesquels le
nombre maximum de comparaisons est minimum.

Un \emph{chemin externe}\index{arbre binaire!chemin externe} est un
chemin de la racine à un n{\oe}ud externe. Nous dirons que la
\emph{hauteur}\index{arbre binaire!hauteur} d'un arbre est la
longueur, comptée en nombre d'arcs, de son chemin externe le plus
long. À la \fig~\ref{fig_cmp_tree}, la hauteur est~\(3\) et il y
a~\(4\) chemins externes de longueur~\(3\), par exemple \((a_1 < a_2)
\rightarrow (a_2 > a_3) \rightarrow (a_1 > a_3) \rightarrow
(a_3,a_1,a_2)\).

La contrainte de maximalité signifie que nous devons porter notre
attention sur la hauteur de l'arbre de comparaisons parce que le
nombre de n{\oe}uds internes (comparaisons) le long des chemins
externes maximaux est un majorant du nombre de comparaisons
nécessaires pour trier \emph{toutes} les
permutations.\index{permutation}\index{tri}

La contrainte de minimalité dans l'énoncé du problème signifie alors
que \emph{nous voulons un minorant de la hauteur d'un arbre de
  comparaisons avec \(n!\)~n{\oe}uds externes}.

% Wrapping figure better declared before a paragraph
%
%\setlength{\intextsep}{0pt}
Un \emph{arbre binaire parfait}\index{arbre binaire!$\sim$ parfait}
est un arbre binaire dont les n{\oe}uds internes ont des enfants qui
sont deux n{\oe}uds internes ou bien deux n{\oe}uds
\begin{wrapfigure}[]{r}[0pt]{0pt}
% {r} mandatory right placement (better because of a list)
\centering
\includegraphics{comp_tree}
\caption{Arbre binaire parfait de hauteur~\(3\)\label{fig_comp_tree}}
\end{wrapfigure}
externes. Si un tel arbre a hauteur~\(h\), alors il a
\(2^h\)~n{\oe}uds externes. Par exemple, la \fig~\vref{fig_comp_tree}
illustre le cas où la hauteur~\(h\) est~\(3\) et nous constatons qu'il
y a en effet \(2^h=8\) n{\oe}uds externes, figurés par des
carrés. Puisque, par définition, les arbres minimisant les
comparaisons ont \(n!\)~n{\oe}uds externes et hauteur~\(S(n)\), ils
contiennent nécessairement moins de n{\oe}uds externes qu'un arbre
binaire parfait de même hauteur, c'est-à-dire que \(n! \leqslant
2^{S(n)}\), donc
\begin{equation*}
S(n) \geqslant \ceiling{\lg n!},
\end{equation*}
où \(\ceiling{x}\) (\textsl{partie entière par excès de~\(x\)})
\index{partie entière par excès@$\ceiling{x}$|see{partie entière par
    excès}} \index{partie entière par excès} est le plus petit entier
qui est plus grand ou égal à~\(x\). Pour obtenir un bon minorant
de~\(S(n)\), nous avons besoin du théorème suivant, admis sans
démonstration.  \setlength{\intextsep}{12pt}
\begin{thm}[Somme et intégrale]
\label{thm:integral_bounds}
Soit \(f\colon [a,b] \rightarrow \mathbb{R}\) une fonction croissante,
monotone et intégrable au sens de Riemann. Alors
\begin{equation}
\abovedisplayskip=0pt
\sum_{k=a}^{b-1}f(k) \leqslant \int_{a}^{b}{\!\!f(x)} \,dx
                   \leqslant \sum_{k=a+1}^{b}{\!\!\!f(k)}.
\tag*{\(\blacksquare\)}
\end{equation}
\end{thm}
\noindent Prenons \(a := 1\), \(b := n\) et \(f(x) := \lg x\). Le
théorème implique donc
\begin{equation*}
n\lg n - \frac{n}{\ln 2} + \frac{1}{\ln 2}
= \int_{1}^{n}{\!\!\lg x} \,dx \leqslant \sum_{k=2}^{n}{\lg k}
= \lg n! \leqslant S(n).
\end{equation*}
Une approche plus puissante mais bien plus compliquée, connue sous le
nom de sommation d'Euler-Maclaurin, produit la formule de Stirling
\citep[chap.~4]{SedgewickFlajolet_1996}\index{Stirling (formule de)},
un grand minorant de \(\lg n!\):
\begin{equation}
\Big(n + \frac{1}{2}\Big)\lg n - \frac{n}{\ln 2} + \lg\sqrt{2\pi}
< \lg n! \leqslant S(n).
\label{ineq_S_lower}
\end{equation}

\paragraph{Minimiser le coût moyen}
\label{par_opt_sort_minimean}
\index{arbre binaire!longueur externe|(}

Nous examinons ici la tâche consistant à minimiser le coût
moyen. Nommons \emph{longueur
  externe}\label{sorting__external_path_length} \index{arbre
  binaire!longueur externe} la somme des longueurs de tous les chemins
externes d'un arbre donné. Le nombre moyen de comparaisons est la
longueur externe moyenne. À la \fig~\vref{fig_cmp_tree}, elle vaut
\((2+3+3+3+3+2)/3!=8/3\). Notre problème ici est donc de déterminer la
forme de l'arbre de comparaisons optimal de longueur externe minimum.

Ces arbres ont leurs n{\oe}uds externes sur un ou deux niveaux
successifs et sont donc \emph{presque parfaits}.\index{arbre
  binaire!$\sim$ presque parfait} Pour comprendre pourquoi,
considérons un arbre binaire ne vérifiant pas cette propriété, donc
ses n{\oe}uds externes les plus près de la racine sont situés au
niveau~\(l\) et ceux les plus près des feuilles sont au niveau~\(L\),
avec \(L \geqslant l + 2\). Si nous échangeons une feuille au
niveau~\(L-1\) avec un n{\oe}ud externe au niveau~\(l\), la longueur
externe est diminuée de la quantité \((l+2L) - (2(l+1) + (L-1)) = L -
l - 1 \geqslant 1\). En répétant ces échanges, on abouti à la forme
annoncée.

Les chemins externes sont faits de \(p\)~chemins se terminant au
pénultième niveau~\(h-1\) et de \(q\)~chemins finissant au niveau le
plus bas~\(h\). (La racine a pour niveau~\(0\).) Cherchons deux
équations dont les solutions sont \(p\)~et~\(q\).
\begin{itemize}

  \item D'après le problème ci-dessus de la minimisation du coût
    maximum, nous savons qu'un arbre de comparaison optimal de
    \(n\)~clés a \(n!\)~n{\oe}uds externes: \(p+q=n!\).

  \item Si nous remplaçons les n{\oe}uds externes au niveau \(h-1\)
    par des feuilles, le niveau~\(h\) devient complet, avec~\(2^h\)
    n{\oe}uds externes, donc \(2p+q=2^h\).

\end{itemize}
Nous avons maintenant deux équations linéaires satisfaites par~\(p\)
et~\(q\), dont les solutions sont \(p=2^h-n!\) et \(q=2n!-2^h\). De
plus, nous pouvons maintenant exprimer la longueur externe minimum
comme suit: \((h-1)p + hq = (h+1)n! - 2^h\).

Finalement, nous devons déterminer la hauteur~\(h\) de l'arbre en
fonction de~\(n!\). Ceci peut être accompli en remarquant que, par
construction, le dernier niveau peut être incomplet, donc \(0 < q
\leqslant 2^h\), ou, de manière équivalente, \(h = \ceiling{\lg
  n!}\). Nous concluons que la longueur externe minimum est
\begin{equation*}
(\ceiling{\lg n!} + 1) n! - 2^{\ceiling{\lg n!}}.
\end{equation*}
Soit \(M(n)\) \label{def_Mn} le plus petit nombre moyen de
comparaisons effectuées par un algorithme de tri optimal. Nous avons
alors
\begin{equation*}
M(n) = \ceiling{\lg n!} + 1 - \frac{1}{n!}2^{\ceiling{\lg n!}}.
\end{equation*}
Puisque \(\ceiling{\lg n!} = \lg n! + x\), avec \(0 \leqslant x < 1\),
si nous posons la fonction \(\theta(x) := 1 + x - 2^x\), nous
déduisons
\begin{equation*}
M(n) = \lg n! + \theta(x).
\end{equation*}
Nous avons \(\max_{0\leqslant x < 1}\theta(x) = 1 - (1 + \ln\ln
2)/\!\ln 2 \simeq 0.08607\), donc
\begin{equation}
\lg n! \leqslant M(n) < \lg n! + 0.09.
\label{ineq_Mn}
\end{equation}
\index{tri!optimalité|)}
\index{arbre binaire!longueur externe|)}
