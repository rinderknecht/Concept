Soit~\(h_n\) la hauteur moyenne des arbres de Catalan avec \(n\)~arcs
et soit \(H_{n,h}\) le nombre d'arbres de Catalan avec \(n\)~arcs et
de hauteur~\(h\). Nous avons alors \(h_n = S_n/C_{n}\), où \(S_n :=
\sum_{h \geqslant 1} h \cdot H_{n,h}\). Nous devons saisir cette
somme. Par exemple, nous pourrions définir~\(A_{n,h}\) comme étant le
nombre d'arbres avec \(n\)~arcs et de hauteur inférieure ou égale
à~\(h\). Bien sûr, \(A_{n,h} = A_{n,n+1} = C_{n}\), si \(h >
n\). Alors \(H_{n,h} = A_{n,h}-A_{n,h-1}\). Les formules peuvent être
davantage simplifiées en définissant \(B_{n,h}\) comme étant le nombre
d'arbres avec \(n\)~arcs et de hauteur supérieure à~\(h\):
\begin{equation}
S_n = \sum_{h \geqslant 1}h(A_{n,h}-A_{n,h-1})
    = \sum_{h \geqslant 1}h(B_{n,h-1}-B_{n,h}) = \sum_{h\geqslant 0} B_{n,h}.
\label{eq:Sn}
\end{equation}

Avec la détermination de~\(A_{n,h}\) en tête, considérons à la
\fig~\ref{fig:height} un chemin de Dyck de longueur~\(2n\) et
hauteur~\(h-1\).
\begin{figure}[!b]
\centering
\subfloat[Chemin de Dyck de longueur~\(2n\) et
de hauteur \(h-1\) \label{fig:height}]{
\includegraphics[scale=0.9,bb=71 565 247 721]{height}}
\quad
\subfloat[Chemin de \(A\) à \(\Omega\) évitant \(y=x+s\) et
\(y=x-t\)\label{fig:mohanty}]{
\includegraphics[scale=0.9,bb=71 621 219 721]{mohanty}}
\caption{Chemins évitant des frontières diagonales}
\label{fig:boundaries}
\end{figure}
Les lignes doubles sont des frontières qui ne peuvent être atteintes
par le chemin. Ceci est en fait un cas particulier d'un chemin
monotone entre deux frontières diagonales, comme on peut le voir à la
\fig~\ref{fig:mohanty}, où \(s\)~dénote la distance verticale à partir
de~\(A\), et \(t\)~est la distance horizontale à partir de~\(A\). Il
est bien connu que le nombre de chemins monotones s'étendant
de~\(A(0,0)\) à~\(\Omega(a,b)\) et évitant les frontières est
\begin{equation}
\left\lvert\mathcal{L}(a,b;t,s)\right\rvert = \sum_{k \in \mathbb{Z}}\left[\binom{a+b}{b+k(t+s)} - \binom{a+b}{b+k(t+s)+t}\right].
\label{eq:mohanty}
\end{equation}

La preuve par \citet[p.~6]{Mohanty_1979} de cette formule est fondée
sur le principe de réflexion et le principe d'inclusion et
d'exclusion, par lequel un sur-ensemble est pris et un sous-ensemble
est soustrait parce que ces deux ensembles sont plus aisément
énumérables que le tout (nous avons employé ce principe implicitement
pour parvenir à la formule~\eqref{eq:Cn}). Nous reproduisons verbatim
sa preuve ici parce qu'elle n'est pas facile à trouver de nos jours.
\begin{proof}
  (\textbf{Mohanty}) Pour faire bref, nommons les frontières \(x=y+t\)
  et \(x=y-s\), \(\mathcal{L}^{+}\) et \(\mathcal{L}^{-}\),
  respectivement. Dénotons par~\(A_1\) l'ensemble des chemins qui
  atteignent \(\mathcal{L}^{+}\), par~\(A_2\) l'ensemble des chemins
  qui atteignent \(\mathcal{L}^{+}\), \(\mathcal{L}^{-}\) dans cet
  ordre, et en général par~\(A_i\) l'ensemble des chemins qui
  atteignent \(\mathcal{L}^{+}\), \(\mathcal{L}^{-}\),
  \(\mathcal{L}^{+}\), \ldots (\(i\)~fois) dans l'ordre indiqué. De la
  même manière, soit~\(B_i\) l'ensemble des chemins atteignant
  \(\mathcal{L}^{-}\), \(\mathcal{L}^{+}\), \(\mathcal{L}^{-}\),
  \ldots (\(i\)~fois) dans l'ordre spécifié. Une application de la
  méthode usuelle d'inclusion\--exclusion donne
  \begin{equation}
    \left\lvert\mathcal{L}(a,b;t,s)\right\rvert = \binom{a+b}{b} +
    \sum_{i \leqslant 1}(-1)^{i}(\lvert{A_i}\rvert +
    \lvert{B_i}\rvert),\label{eq:L}
  \end{equation}
  où~\(\lvert{A_i}\rvert\) et~\(\lvert{B_i}\rvert\) sont évalués
  répétitivement au moyen du principe de réflexion. Par exemple,
  considérez~\(A_3\). Puisque chaque chemin dans~\(A_3\) doit
  atteindre~\(\mathcal{L}^{+}\), \(A_3\) lorsqu'il est réfléchi par
  rapport à \(\mathcal{L}^{+}\) devient l'ensemble des chemins de
  \((t,-t)\) à \((a,b)\), chacun atteignant \(\mathcal{L}^{+}\) après
  avoir atteint \(\mathcal{L}^{-}\). Une autre réflexion par rapport à
  \(\mathcal{L}^{-}\) rendrait~\(A_3\) équivalent à l'ensemble des
  chemins de \((-s-t,s+t)\) à \((a,b)\) qui
  atteignent~\(\mathcal{L}^{+}\), qui à son tour peut s'écrire
  \(R(a+s+t,b-s-t; 2s+3t)\). [Note: \(R(a,b;t)\) est l'ensemble des
    chemins de \((0,0)\) à \((a,b)\) réfléchis par rapport
    à~\(\mathcal{L}^{+}\).] Par conséquent, puisque
  \(\lvert{R(a,b;t)}\rvert = \binom{a+b}{a-t}\),
  \begin{equation*}
    \left\lvert{A_3}\right\rvert = \binom{a+b}{a-s-2t},
  \end{equation*}
  et, plus généralement,
  \begin{equation*}
    \left\lvert{A_{2j}}\right\rvert = \binom{a+b}{a+j(s+t)}
    \quad\text{et}\quad \left\lvert{A_{2j+1}}\right\rvert =
    \binom{a+b}{a-j(s+t)-t}.
  \end{equation*}
  Les expressions pour \(\lvert{B_{2j}}\rvert\),
  \(\lvert{B_{2j+1}}\rvert\), \(j=0, 1, 2, \dots\), avec
  \(\lvert{A_0}\rvert\), \(\lvert{B_0}\rvert\) étant
  \(\binom{a+b}{b}\), sont obtenues en interchangeant \(a\)~et~\(b\)
  et \(s\)~avec~\(t\). La substitution de ces valeurs
  dans~\eqref{eq:L} donne~\eqref{eq:mohanty} après quelques
  simplifications.
\end{proof}

En reprenant le fil de notre argumentation, si nous faisons se
correspondre les figures dans la \fig~\ref{fig:boundaries}, nous
déduisons que \(s=h\), \(t=1\), \(a=b=n\), d'où \(a+b=2n\) et
\(b+k(s+t)=n+k(h+1)\), que nous remplaçons dans la
formule~\eqref{eq:mohanty} où nous changeons \(h\)~en \(h-1\):
\begin{equation*}
A_{n,h-1} = \sum_{k \in \mathbb{Z}}\left[\binom{2n}{n+kh} -
           \binom{2n}{n+1+kh}\right].
\end{equation*}
Après la partition de la somme en les cas \(k<0\), \(k=0\) et \(k>0\),
puis le changement de signe de~\(k\) dans le premier cas, ensuite en
utilisant \(\binom{p}{q} = \binom{p}{p-q}\) dans le second et dernier
cas, puis enfin en regroupant les sommes restantes indexées par \(k
\geqslant 1\), nous obtenons
\begin{align*}
A_{n,h-1}
  &= - \sum_{k \geqslant 1}\left[\binom{2n}{n+1-kh} -
    2\binom{2n}{n-kh} + \binom{2n}{n-1-kh}\right]\\
  &\phantom{=}\; + \binom{2n}{n} - \binom{2n}{n-1}.
\end{align*}
Nous reconnaissons \(C_n\) de la page~\pageref{eq:Ann}, et nous
réécrivons
\begin{equation*}
C_n - A_{n,h-1}
  = \sum_{k \geqslant 1}\left[\binom{2n}{n+1-kh} -
    2\binom{2n}{n-kh} + \binom{2n}{n-1-kh}\right].
\end{equation*}
Finalement, nous souvenant que \(B_{n,h} = A_{nn} - A_{n,h}\) et \(C_n =
A_{n,n+1}\), nous parvenons à la formule
\begin{equation}
B_{n,h-1} = \sum_{k \geqslant 1}
            \left[\binom{2n}{n+1-kh} - 2\binom{2n}{n-kh}
            + \binom{2n}{n-1-kh}\right].
\label{eq:Bn}
\end{equation}

\citet*{KnuthdeBruijnRice_1972} ont publié un article majeur où ils
obtiennent le même résultat en employant des mathématiques bien plus
compliquées. Ils commencent par modéliser le problème avec une
fonction génératrice \citep{Wilf_1990} qui satisfait une équation de
récurrence dont la solution exprime la fonction génératrice en termes
de fractions continuées de polynômes de Fibonacci. Ils utilisent alors
l'intégration sur le plan complexe pour trouver la
formule~\eqref{eq:Bn}. Alternativement, les fonctions génératrices
peuvent être employées sur les chemins monotones dans des treillis, au
lieu des arbres de Catalan \citep[page~64]{Kemp_1984}
\citep{FlajoletNebelProdinger_2006}.

\citet*{SedgewickFlajolet_1996} \citep{FlajoletSedgewick_2009} utilisent l'analyse combinatoire et l'analyse réelle pour obtenir l'approximation asymptotique de~\(B_{n,h}\). Ils écrivent~\cite[p.~260]{SedgewickFlajolet_1996}: «~Cette analyse est la plus difficile de celles que nous menons dans ce livre. Elle combine des techniques pour résoudre des récurrences linéaires et des fractions continuées, des développements limités de fonctions génératrices, en particulier au moyen du théorème d'inversion de Lagrange, les approximations binomiales et les sommes d'Euler\--Maclaurin.~» Il n'est pas possible d'entrer dans les détails ici, mais nous pouvons esquisser comment une approximation asymptotique peut être menée à bien.

L'équation~\eqref{eq:Sn} implique \(S_{n} = \sum_{h \geqslant 1}
B_{n,h-1}\), d'où
\begin{equation*}
S_{n} = \sum_{k' \geqslant 1}d(k') \cdot
         \left[\binom{2n}{n+1-k'} - 2\binom{2n}{n-k'}
         + \binom{2n}{n-1-k'}\right],
\end{equation*}
où~\(d(k')\) est le nombre de diviseurs positifs de~\(k'\), mais une
analyse sur les nombres complexes est nécessaire
\citep{KnuthdeBruijnRice_1972,FlajoletGourdonDumas_1995}. Une autre
approche consiste à exprimer les coefficients binomiaux en termes de
\(\binom{2n}{n-kh}\) comme suit:
\begin{align*}
\binom{2n}{n-m+1} &= \frac{(2n)!}{(n-m+1)!\,(n+m-1)!}\\
                  &= \frac{(2n)!\,(n+m)}{(n-m)!\,(n-m+1)(n+m)!}
                   = \frac{n+m}{n-m+1}\binom{2n}{n-m},\\
\binom{2n}{n-m-1} &= \frac{(2n)!}{(n-m-1)!\,(n+m+1)!}\\
                  &= \frac{(2n)!\,(n-m)}{(n-m)!\,(n+m)!\,(n+m+1)}
                   = \frac{n-m}{n+m+1}\binom{2n}{n-m}.
\end{align*}
Par conséquent,
\begin{equation*}
\binom{2n}{n-m+1} - 2\binom{2n}{n-m} + \binom{2n}{n-m-1}
= 2 \cdot \frac{2m^2-(n+1)}{(n+1)^2-m^2}\binom{2n}{n-m}.
\end{equation*}
Posons \(F_n(m) = (2m^2-n)/(n^2-m^2)\). Nous avons
\begin{equation*}
S_{n} = 2 \cdot \sum_{h \geqslant 1}\sum_{k \geqslant 1} F_{n+1}(kh)
\cdot \binom{2n}{n-kh},
\end{equation*}
De l'équation~\eqref{eq:Cn} et de \(h_n = S_n/C_n\), nous tirons
\(h_{n} = (n+1)S_{n}{\binom{2n}{n}}^{-1}\), donc nous devons approcher
\((n+1)F_{n+1}(m)\) et \(\binom{2n}{n-m}\binom{2n}{n}^{-1}\). D'une
part, nous avons
\begin{equation*}
F_{n+1}(m) \sim \frac{2m^2-n}{n^2} \sim \frac{2m^2-n}{n(n+1)},
\end{equation*}
donc \((n+1)F_{n+1}(kh) \sim 2k^2h^2\!/n-1\). D'une autre part,
\citet*[4.6, 4.8]{SedgewickFlajolet_1996} montrent que
\begin{equation*}
\binom{2n}{n-m}{\binom{2n}{n}}^{-1} \sim e^{-m^2\!/n}.
\end{equation*}
En supposant que les termes d'erreur implicites des deux
approximations précédentes diminuent exponentiellement, nous avons
\begin{equation*}
h_{n} \sim \sum_{h \geqslant 1}\sum_{k \geqslant 1}
(4k^2h^2\!/n - 2)e^{-k^2h^2\!/n}
= \sum_{h \geqslant 1}H(h/\!\sqrt{n}),
\end{equation*}
avec \(H(x) := \sum_{k \geqslant
  1}(4k^2x^2-2)e^{-k^2x^2}\). Finalement,
\citet*[5.9]{SedgewickFlajolet_1996}, tout comme
\citet*[9.6]{GrahamKnuthPatashnik_1994}, font usage d'analyse réelle
pour conclure
\begin{equation*}
h_{n} \sim \sum_{h \geqslant 1}H(h/\!\sqrt{n})
\sim \sqrt{n} \int_0^{\infty}\!\!H(x) dx \sim \sqrt{\pi n}.
\end{equation*}
Après que \(B_{n,h-1}\) a été obtenu, la fin de notre dérivation est
difficile et même pas entièrement formelle parce que les termes
d'erreur dans l'approximation asymptotique bivaluée devraient être
prudemment vérifiés, comme les auteurs référencés le
font. Malheureusement, ceci aussi veut dire que cette partie a peu de
chance d'être simplifiée davantage.
