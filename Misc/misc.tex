\addcontentsline{toc}{subsection}{Coût}
\mypar{Coût minimum}
\index{tri par interclassement!$\sim$ ascendant!coût minimum|(}

Soit \(\OB{\fun{bms}}{n}\)\index{bms@$\OB{\fun{bms}}{n}$|(} le coût
minimum pour trier \(n\)~clés de façon ascendante et recherchons une
définition de \(\OB{\fun{bms}}{n+1}\) en termes de
\(\OB{\fun{bms}}{n}\). Étant donné que les arbres
d'interclassement\index{arbre!$\sim$ d'interclassement} dépendent de
la notation binaire de~\(n\), et, en gardant présent à l'esprit le
fonctionnement de l'addition en binaire, nous envisagerons les cas où
\(n\)~est pair et impair. Tous les arbres d'interclassement minimisant
les comparaisons\index{arbre!$\sim$ d'interclassement} sont isomorphes
à l'arbre montré à la \fig~\ref{fig_b2p_mrg},
\begin{figure}[t]
\centering
\subfloat[Longueurs des piles\label{fig_b2p_mrg}]{
\includegraphics[bb=71 636 166 723]{b2p_mrg}}
\qquad
\subfloat[La somme est $\protect\OB{\protect\fun{bms}}{n}$\label{fig_b2p}]{
\includegraphics[bb=71 631 168 721]{b2p}}
\caption{Arbre de coût minimum pour $n := \sum_{j=0}^{r}{2^{e_j}}$
\label{fig_B2p}}
\end{figure}
dont les n{\oe}uds contiennent les longueurs des piles et est
simplement une version compacte de la \fig~\ref{fig_msort_gen}. Ils
sont aussi isomorphes à l'arbre à la \fig~\ref{fig_b2p}, dont les
n{\oe}uds contiennent le nombre minimum de comparaisons pour
l'interclassement associé et est ainsi nommé \emph{arbre de coût
  minimum}. Le nombre minimum de comparaisons \(\OB{\fun{bms}}{n}\)
est alors la somme de tous les n{\oe}uds. Nous avons deux cas de
figure.

\medskip

\begin{itemize}

\item \emph{\(n\)~est pair}. Nous avons seulement besoin de supposer
  \(e_0 > 0\) à la \fig~\ref{fig_B2p}. L'arbre d'interclassement
  abstrait pour \(n+1\) est montré à la \fig~\vref{fig_b2p_mrg_succ}
  et son arbre de coût minimum est à la \fig~\ref{fig_b2p_succ}. Les
  expressions encadrées dans ce dernier ne sont pas présentes dans
  l'arbre de coût minimum pour~\(n\) (\fig~\vref{fig_b2p} avec \(e_0 >
  0\)), donc \(\OB{\fun{bms}}{n+1} = \OB{\fun{bms}}{n} + (r + 1 +
  \OB{\fun{bms}}{2^0}) = \OB{\fun{bms}}{n} + \nu_n\).
\begin{figure}[b]
\centering
\subfloat[Longueurs des piles\label{fig_b2p_mrg_succ}]{
\includegraphics{b2p_mrg_succ}
}
\quad
\subfloat[La somme est \(\protect\OB{\protect\fun{bms}}{n+1}\) avec \(n\)~pair
\label{fig_b2p_succ}]{\includegraphics{b2p_succ}}
\caption{Coût minimum pour \(n+1\) impair\label{fig_B2p_succ}}
\end{figure}

\medskip

  \item \emph{\(n\)~est impair}. Alors \(n = (\Xi 01^q)_2\), où
  \(\Xi\)~est une chaîne de bits et~\(q\) un entier strictement
  positif. En termes de la décomposition en puissances de~\(2\), ceci
  revient à dire qu'il existe un nombre positif~\(q\) tel que
  \(e_q>q\) et, pour tout \(i < q\), nous avons \(e_i=i\). Ceci
  mène aux arbres de la \fig~\ref{fig_B2p1}.
\begin{figure}[t]
\centering
\subfloat[Longueurs des piles\label{fig_b2p1_mrg}]{
\includegraphics{b2p1_mrg}
}
\qquad
\subfloat[La somme est \(\protect\OB{\protect\fun{bms}}{n}\) avec \(n\)~impair
\label{fig_b2p1}]{\includegraphics[bb=68 590 218 721]{b2p1}}
\caption{Coût minimum si \(n:=2^{e_r} + \dots + 2^{e_q} + 2^{q-1} +
  \dots + 2^1 + 2^0\), où \(e_q > q > 0\)
\label{fig_B2p1}}
\end{figure}
Les arbres de coût minimum pour \(n+1\) sont montrés aux \figs{}
\ref{fig_b2p1} \&~\ref{fig_b2p1_succ}.
\begin{figure}
\centering
\subfloat[Longueurs des piles\label{fig_b2p1_mrg_succ}]{
\includegraphics[bb=60 661 153 721]{b2p1_mrg_succ}}
\qquad
\subfloat[\(\protect\OB{\protect\fun{bms}}{n+1}\) avec \(n\)~impair
\label{fig_b2p1_succ}]{
\includegraphics[bb=66 653 156 721]{b2p1_succ}}
\caption{Coût minimum pour \(n+1\) pair
(voir \textsc{Fig}.~\ref{fig_B2p1}, \(q>0\))}
\end{figure}
\index{tri par interclassement!$\sim$ ascendant!coût minimum|)}
\index{bms@$\OB{\fun{bms}}{n}$|)}
Les expressions encadrées sont celles qui sont absentes dans l'arbre
d'interclassement correspondant. Par conséquent, 
\(\OB{\fun{bms}}{n+1} -
\OB{\fun{bms}}{2^q} = \OB{\fun{bms}}{n+1} - q2^{q-1}\) est la somme
des termes qui ne sont pas encadrés dans l'arbre de \(n+1\), alors que
la somme des termes non-encadrés dans l'arbre de~\(n\) est
\(\OB{\fun{bms}}{n} - \sum_{k=0}^{q-1}\OB{\fun{bms}}{2^k} -
\sum_{k=1}^{q-1}2^k + ((q-1)+(r-q+1)) = \OB{\fun{bms}}{n} -
\sum_{k=0}^{q-1}{k2^{k-1}} - (2^q - 2) + (\nu_n - 1)\), avec
l'équation~\eqref{eq_best_power}, car \(\B{\Join}{k} =
\OB{\fun{bms}}{2^k}\). En égalant les expressions non-encadrées dans
les deux arbres, nous déduisons
\begin{equation}
\abovedisplayskip=4pt
\belowdisplayskip=4pt
\OB{\fun{bms}}{n+1} = \OB{\fun{bms}}{n} + q2^{q-1} -
\sum_{k=0}^{q-1}{i2^{k-1}} - 2^q + \nu_n + 1.
\label{eq_B1}
\end{equation}
Une forme close de la somme restante est obtenue par la \emph{méthode
  des perturbations}\index{méthode des perturbations}
\citep[\S~2.3]{GrahamKnuthPatashnik_1994}, qui consiste à ajouter
un terme de plus et à réécrire la nouvelle somme en termes de
l'originelle. Posons \(S_{j} := \sum_{k=1}^{j}{k2^{k-1}}\). Alors
\begin{align}
\abovedisplayskip=2pt
\belowdisplayskip=0pt
S_{j} + (j+1)2^j &= \smash[t]{\sum_{k=1}^{j+1}{k2^{k-1}}} = \smash[t]{\sum_{k=0}^{j}{(k+1)2^{k}}}
= \smash[t]{\sum_{k=0}^{j}{k2^{k}}} + \smash[t]{\sum_{k=0}^{j}{2^{k}}}\notag\\
&= 2 \cdot S_{j} + 2^{j+1} - 1,\notag\\
\smash[t]{\sum_{k=1}^{j}{k2^{k-1}}} &= (j-1)2^j + 1.\label{eq_Sj}
\end{align}
La substitution de cette forme close dans l'équation~\eqref{eq_B1}
donne, pour tout \(n \geqslant 0\), \(\OB{\fun{bms}}{n+1} =
\OB{\fun{bms}}{n} + r + 1 = \OB{\fun{bms}}{n} + \nu_n\).
\end{itemize}
En rassemblant les résultats pour \(n\)~pair et impair, et en faisant
appel à l'équation~\eqref{eq_OB_tms} \vpageref{eq_OB_tms}, nous
concluons finalement avec élégance:
\begin{equation}
\OB{\fun{bms}}{n} = \sum_{k=0}^{n-1}{\nu_k} = \OB{\fun{tms}}{n}.
\label{eq_OBbms}\index{bms@$\OB{\fun{bms}}{n}$}
\end{equation}








Si nous choisissons de ne pas accepter ces fonctions comme des formes
closes, nous pouvons néanmoins en déduire des encadrements du coût
maximum.

Posons \(U_n := \sum_{j=1}^{n-1}2^{\rho_j}\). Nous avons \(U_1 = 0\) et
en réutilisant les définitions~\eqref{eq_ruler}:
\begin{align*}
U_{2p}   &= \sum_{k=0}^{p-1}2^{\rho_{2k+1}} + \sum_{k=1}^{p-1}2^{\rho_{2k}}
         = p + 2 \cdot U_p,\\
U_{2p+1} &= \sum_{k=0}^{p-1}2^{\rho_{2k+1}} + \sum_{k=1}^{p}2^{\rho_{2k}} 
        = p + 2 \cdot U_{p+1}.
\end{align*}
En d'autres termes, \(U_n = 2 \cdot U_{\ceiling{n/2}} + \floor{n/2} =
2 \cdot U_{\ceiling{n/2}} + n - \ceiling{n/2}\), puisque \(\floor{n/2}
+ \ceiling{n/2} = n\). En déroulant la récurrence apparaît l'équation
\begin{equation*}
2 \cdot U_n = 2n + \sum_{k=1}^{\floor{\lg
    n}}\left\lceil{\frac{n}{2^k}}\right\rceil 2^k - 2^{\floor{\lg
    n}+1},
\end{equation*}
si nous gardons à l'esprit que \(2^{\floor{\lg n}}\) est la plus
grande puissance de~\(2\) dans la notation binaire de~\(n\), donc \(0
\leqslant n/2^{\floor{\lg n}+1} < 1\), et si nous utilisons le lemme suivant:
\begin{thm}[Parties entières par excès et fractions]
\label{thm_ceilings}
\textsl{Soit \(x\)~un nom\-bre réel et \(q\)~un entier naturel. Alors
  \(\ceiling{\ceiling{x}/q} = \ceiling{x/q}\).}
\end{thm}
\begin{proof}
  L'égalité est équivalente à la conjonction des deux inégalités
  complémentaires \(\ceiling{\ceiling{x}/q} \geqslant \ceiling{x/q}\)
  et \(\ceiling{\ceiling{x}/q} \leqslant \ceiling{x/q}\). La première
  est simple car \(\ceiling{x} \geqslant x \Rightarrow \ceiling{x}/q
  \geqslant x/q \Rightarrow \ceiling{\ceiling{x}/q} \geqslant
  \ceiling{x/q}\). Ensuite, puisque les deux membres de l'inégalité
  sont des entiers, \(\ceiling{\ceiling{x}/q} \leqslant
  \ceiling{x/q}\) équivaut à \(p \leqslant \ceiling{\ceiling{x}/q}
  \Rightarrow p \leqslant \ceiling{x/q}\), pour tout entier~\(p\). Un
  lemme évident est que si \(i\)~est un entier et \(y\)~un réel, \(i
  \leqslant \ceiling{y} \Leftrightarrow i \leqslant y\), donc
  l'inéquation originelle est équivalente à \(p \leqslant
  \ceiling{x}/q \Rightarrow p \leqslant x/q\), pour tout entier~\(p\),
  c'est-à-dire \(pq \leqslant \ceiling{x} \Rightarrow pq \leqslant
  x\). Le lemme valide cette implication et achève la preuve.
\end{proof}
\noindent En employant les inégalités classiques \(x \leqslant
\ceiling{x} < x + 1\), nous déduisons
\begin{equation*}
n\floor{\lg n} + 2n - 2^{\floor{\lg n}+1}
\leqslant 2 \cdot U_n < n\floor{\lg n} + 2n - 2.
\end{equation*}
L'utilisation de \(\floor{x} = x - \{x\}\) nous permet d'avancer un
peu plus:
\begin{equation*}
n\lg n + 2n - n\cdot\theta_L(\{\lg n\})
\leqslant 2 \cdot U_n < n\lg n + 2n - 2,
\end{equation*}
où \(\theta_L(x) := x + 2^{1 - x}\). Puisque \(\max_{0 \leqslant x <
  1}\theta_L(x) = \theta_L(0) = 2\), nous pouvons affaiblir
l'inéquation précédente en
\begin{equation}
n\lg n \leqslant 2 \cdot U_n < n\lg n + 2n - 2.
\label{ineq_Un}
\end{equation}
Le minorant est atteint si \(n=2^p\).

Posons \(T_n := \sum_{j=1}^{n}{2^{\rho_{j}}}\). Nous avons alors \(T_n
= U_n + 2^{\rho_n}\) et l'encadrement~\eqref{ineq_Un} conduit à \(n\lg
n + 2 \leqslant 2 \cdot T_n < n\lg n + 4n - 2\), car \(1 \leqslant
2^{\rho_n} \leqslant n\). Mais le majorant est mauvais, parce que \(n
= 2^{\rho_n}\) lorsque le \emph{minorant} de~\(T_n\) est atteint. Par
conséquent, nous devrions appliquer à~\(T_n\) la même technique que
nous avons employée pour encadrer~\(U_n\). Les récurrences satisfaites
par~\(\rho_n\) \eqref{eq_ruler} nous permettent de trouver une
récurrence pour~\(T_n\) comme suit:
\begin{align*}
T_{2p} &= \sum_{k=0}^{\smash[t]{p-1}}{2^{\rho_{2k+1}}} +
\sum_{k=1}^{p}{2^{\rho_{2k}}} = p + 2 \cdot T_{p},\\
 T_{2p+1}
&= \sum_{j=1}^{\smash[t]{2p+1}}{2^{\rho_{j}}} = 1 + T_{2p} = (p + 1) +
2 \cdot T_{p}.
\end{align*}
De manière équivalente, \(T_{n} = 2 \cdot T_{\floor{n/2}} +
\ceiling{n/2} = 2 \cdot T_{\floor{n/2}} + n - \floor{n/2}\). Donc,
dérouler quelques termes de la récurrence révèle rapidement
\begin{equation*}
2 \cdot T_n = 2n + \sum_{j=1}^{\floor{\lg n}}
            {\left\lfloor{\frac{n}{2^j}}\right\rfloor 2^j}.
\end{equation*}
si nous gardons à l'esprit le lemme~\ref{thm_floors}
\vpageref{thm_floors}. Par définition, \(\{x\} := x - \floor{x}\),
donc
\begin{equation*}
2 \cdot T_n = n\floor{\lg n} + 2n - \sum_{j=1}^{\floor{\lg
    n}}\left\lbrace\frac{n}{2^j}\right\rbrace 2^j.
\end{equation*}
En usant de \(0 \leqslant \{x\} < 1\), nous obtenons l'encadrement
\begin{equation*}
n\floor{\lg n} + 2n - 2^{\floor{\lg n}+1} + 2 < 2 \cdot T_n \leqslant
n\floor{\lg n} + 2n.
\end{equation*}
De plus, \(x - 1 < \floor{x} \leqslant x\) et \(\floor{x} = x -
\{x\}\), par conséquent
\begin{align*}
n(\lg n - \{\lg n\}) + 2n - 2^{\lg n - \{\lg n\} +
  1} + 2 < 2 \cdot T_n &\leqslant n\lg n + 2n,\\
n\lg n + 2n + 2 - n \cdot \theta_L(\{\lg n\}) < 2 \cdot T_n 
&\leqslant n\lg n + 2n,
\end{align*}
où \(\theta_L(x) := x + 2^{1 - x}\). Puisque \(\max_{0 \leqslant x <
  1}\theta_L(x) = \theta_L(0) = 2\), nous pouvons affaiblir
l''inéquation précédente en
\begin{equation}
n\lg n + 2 < 2 \cdot T_n \leqslant n\lg n + 2n.
\label{ineq_Tn}
\end{equation}
Le majorant est atteint si \(n=2^p\).

Nous pouvons maintenant encadrer
\(\OW{\fun{bms}}{2p}\) et
\(\OW{\fun{bms}}{2p+1}\) grâce aux équations~\eqref{eq_OWbms_2p}
et~\eqref{eq_OWbms_2p_1}, en privilégiant la borne supérieure. Tout
d'abord, nous avons
\begin{align*}
\OW{\fun{bms}}{2p+1} = 2 \cdot T_p + 2 \cdot \OB{\fun{bms}}{p} +
\nu_p,
\quad
\OW{\fun{bms}}{2p} = 2 \cdot U_p + 2 \cdot \OB{\fun{bms}}{p} + 1.
\end{align*}
Ensuite, nous faisons appel à~\eqref{ineq_Tn}, ci-dessus,
et~\eqref{ineq_bounds_Btms}, \vpageref{ineq_bounds_Btms}:
\begin{equation*}
\OW{\fun{bms}}{2p+1}
  \leqslant (p\lg p + 2p) + p\lg p + \nu_p
   \leqslant (2p+1)\lg p + 2p + 1.
\end{equation*}
En posant \(n=2p+1\), nous obtenons
\begin{equation*}
\OW{\fun{bms}}{n}
  \leqslant n\lg(\tfrac{1}{2}(n-1)) + n = n\lg(n-1).
\end{equation*}
Si nous rappelons maintenant les inégalités~\eqref{ineq_Un}
et~\eqref{ineq_bounds_Btms}, nous avons
\begin{equation*}
\OW{\fun{bms}}{2p} < 1 + (p\lg p + 2p - 2) + (p\lg p)
             = 2p\lg p + 2p - 1,
\end{equation*}
qui n'est autre que \(\OW{\fun{bms}}{n} < n\lg n - 1\), si nous
posons \(n = 2p\).

Nous retenons alors le plus grand des majorants de
\(\OW{\fun{tms}}{2p}\) et \(\OW{\fun{tms}}{2p+1}\), puis l'utilisons
pour encadrer \(\OW{\fun{bms}}{n}\) pour tout~\(n > 2\):
\begin{equation}
\OW{\fun{bms}}{n} < n\lg n - 1.
\label{ineq_OWbms_up}
\end{equation}
\index{bms@$\OW{\fun{bms}}{n}$|)}

\paragraph{Comparaison avec le tri descendant}

Il nous manque encore un minorant de \(\OW{\fun{bms}}{n}\). 

