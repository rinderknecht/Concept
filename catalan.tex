\chapter{Arbres de Catalan}
\label{chap:Catalan}

Dans la partie~\ref{part:linear} de ce livre, nous traitons des
structures linéaires, comme les piles et les files d'attente, mais,
pour vraiment comprendre les programmes qui opèrent sur ces
structures, nous avons besoin du concept d'arbre. C'est pourquoi nous
avons présenté très tôt les arbres de syntaxe
abstraite\index{arbre!$\sim$ de syntaxe abstraite}, les graphes
orientés sans circuits (\fig~\vref{fig:cat_dag}), les arbres de
comparaison (\fig~\vref{fig:cmp_tree}), les arbres binaires
(\fig~\vref{fig:comp_tree}), les arbres de preuve
(\fig~\vref{fig:perm_proof}), les arbres d'évaluation
(\fig~\vref{fig:2way_unbal}) et les arbres d'interclassement
(\fig~\vref{fig:bot_up1}). Ces arbres étaient des méta-objets, ou
concepts, employés pour comprendre la structure linéaire à l'étude.

Dans ce chapitre, nous prenons un point de vue plus abstrait et nous
considérons la classe générale des \emph{arbres de Catalan}, ou arbres
généraux. Nous les étudierons comme des objets mathématiques dans le
but de transférer nos résultats aux arbres utilisés comme structures
de données pour réaliser des algorithmes. En particulier, nous nous
intéresserons à les mesurer, à les compter, et à déterminer quelques
paramètres moyens relatifs à leur taille, la raison étant que la
connaissance de ce à quoi ressemble un arbre moyen nous informera sur
le coût de le traverser de différentes façons.

% Wrapping figure better declared before a paragraph
%
\begin{wrapfigure}[9]{r}[0pt]{0pt}
% [9] vertical lines
% {r} mandatory right placement
% [0pt] of margin overhang
\centering
\includegraphics[bb=71 646 160 716]{catalan_tree}
\caption{Arbre de Catalan de hauteur~5}
\label{fig:catalan_tree}
\end{wrapfigure}
Les arbres de Catalan sont un type spécial de graphe, soit un objet
constitué de \emph{n{\oe}uds} (appelés aussi \emph{sommets}) reliés
par des \emph{arcs}, sans orientation (seule la connexion importe). Ce
qui fait un arbre de Catalan est la distinction d'un n{\oe}ud, appelé
la \emph{racine}, et l'absence de cycles, c'est-à-dire de chemins
fermés contenant des n{\oe}uds reliés successivement. Les arbres de
Catalan sont souvent appelés \emph{arbres ordonnés} (anglais,
\emph{ordered trees} ou \emph{planted plane trees}), en théorie des
graphes, et \emph{arbres généraux} (anglais, \emph{unranked trees}, ou
\emph{\(n\)-ary trees}) en théorie de la programmation. Un exemple est
donné à la \fig~\ref{fig:catalan_tree}. Remarquons que la racine est
le n{\oe}ud situé en haut de la figure et qu'elle possède quatre
sous-arbres ordonnés, dont les racines sont nommées
\emph{enfants}. Les n{\oe}uds dessinés comme des disques blancs
\((\circ)\) constituent un chemin maximal commençant à la racone (le
nombre de n{\oe}uds le long de celui-ci est maximal). Le n{\oe}ud
terminal n'a pas d'enfants; il y a précisément~\(8\) n{\oe}uds de ce
genre au total, appelés \emph{feuilles}. Le nombre de disques blancs
est la \emph{hauteur} de l'arbre de Catalan (il peut y avoir plusieurs
chemins maximaux de même longueur), donc l'exemple donné a pour
hauteur~\(5\).

Les programmeurs mettent en {\oe}uvre les arbres de Catalan comme une
structure de donnée, par exemple via l'usage de \textsf{XML}, auquel
cas, des informations sont logées dans les n{\oe}uds et leur recherche
peut exiger l'atteinte d'une feuille, dans la circonstance la plus
défavorable. Le coût maximum d'une recherche est donc proportionnel à
la hauteur de l'arbre et la détermination de la hauteur moyenne
devient pertinente lorsque l'on réalise une série de recherches
aléatoires \citep{VitterFlajolet_1990}. Pour mener à bien ce type
d'analyse, nous devons d'abord déterminer le nombre d'arbres de
Catalan d'une taille donnée. Il y a deux mesures usuelles de la
taille: soit nous quantifions les arbres par le nombre de leurs
n{\oe}uds, ou bien nous comptons le nombre d'arcs. En fait, le choix
de l'une ou l'autre mesure n'est dicté que par des considérations de commodité ou de style: il y a \(n\)~arcs s'il y a \(n+1\) n{\oe}uds, tout simplement parce que chaque n{\oe}ud, sauf la racine, possède un parent. Il arrive fréquemment que les formules à propos des arbres de Catalan soient un peu plus simples lorsqu'elles font référence au nombre d'arcs, donc ce sera là notre mesure de la taille dans ce chapitre.

\section{Énumeration}
\label{sec:Catalan_enumeration}

Un grand nombre de manuels \cite[\S~5.1
\&~5.2]{SedgewickFlajolet_1996} montre comment déterminer le nombre
d'arbres de Catalan avec \(n\)~arcs au moyen d'outils mathématiques
puissants connus sous le nom de \emph{fonctions génératrices}
\index{dénombrement combinatoire!fonction génératrice}
\cite[chap.~7]{GrahamKnuthPatashnik_1994}. À leur place, pour des
raisons didactiques, nous choisissons une technique plus intuitive
issue du dénombrement combinatoire\index{dénombrement combinatoire}
qui consiste à construire une bijection entre deux ensembles finis, le
cardinal de l'un étant donc le cardinal de l'autre. Pour une taille
donnée, nous allons associer chaque arbre de Catalan à un objet de
manière exclusive, et aucun objet ne sera laissé de côté. De plus, ces
objets doivent être aisément dénombrables.

% Wrapping figure better declared before a paragraph
%
\begin{wrapfigure}[11]{r}[0pt]{0pt}
% [11] vertical lines
% {r} mandatory right placement
% [0pt] of margin overhang
\centering
\subfloat[Chemin de Dyck\label{fig:dyck}]{%
  \includegraphics[bb=71 634 158 712]{dyck}}
\quad
\subfloat[Arbre de Catalan\label{fig:ctree}]{%
\includegraphics[bb=65 663 127 721]{ctree}
}
\caption{Bijection}
\label{fig:bijection}
\end{wrapfigure}
Les objets qui conviennent le mieux à notre objectif sont ici les
\emph{chemins monotones dans un trellis}, \index{chemin monotone dans
  un treillis|see{chemin de Dyck}} c'est-à-dire une grille carrée
\citep{Mohanty_1979,Humphreys_2010}. Ces chemins sont constitués d'une
suite de pas orientés vers le haut (\(\uparrow\)), appelés
\emph{montées}, et de pas vers la droite (\(\rightarrow\)), appelés
\emph{descentes}, et ils commencent au coin en bas à gauche
\((0,0)\). Les \emph{chemins de Dyck}\index{chemin de Dyck} de
longueur~\(2n\) sont les chemins se aboutissant à \((n,n)\) et qui
restent au-dessus de la diagonale, ou la touchent. Nous souhaitons
montrer qu'\emph{il existe une bijection entre les chemins de Dyck de
  longueur~\(2n\) et les arbres de Catalan avec \(n\)~arcs.}

\begin{wrapfigure}[7]{r}[0pt]{0pt}
\centering
\includegraphics[bb=71 661 128 720]{ctree_pre}
\caption{}
\label{fig:ctree_pre}
\end{wrapfigure}
Pour comprendre cette bijection, nous devons d'abord présenter un type
particulier de \emph{parcours}\index{arbre!parcours}, ou
\emph{traversée}\index{arbre!traversée|see{parcours}}, de l'arbre de
Catalan. Imaginons qu'un arbre est une carte routière où les n{\oe}uds
dénotent des villes et les arcs des routes. Un parcours complet de
l'arbre consiste alors à commencer notre voyage à la racine et, en
suivant les arcs, à visiter tous les n{\oe}uds. (Il est permis de
passer plusieurs fois par le même n{\oe}ud, puisqu'il n'y a pas de
cycles.) Bien entendu, il existe de nombreuses façons de réaliser
cette tournée et celle que nous envisagerons ici est dite un
\emph{parcours en préordre}.

À chaque n{\oe}ud, nous prenons l'arc le plus à gauche qui n'a pas
encore été emprunté et nous parcourons le sous-arbre correspondant en
préordre; lorsque nous sommes de retour, nous itérons le choix avec
les enfants qui restent. Pour plus de clarté, nous montrons à la
\fig~\ref{fig:ctree_pre} la \emph{numérotation en
  préordre}\index{arbre!$\sim$ de Catalan!numérotation en préordre} de
la \fig~\ref{fig:ctree}, où l'ordre dans lequel un n{\oe}ud est
rencontré la première fois remplace le disque noir \((\bullet)\).

La première partie de la bijection est une injection des arbres de
Catalan avec \(n\)~arcs vers les chemins de Dyck de
longueur~\(2n\). En parcourant l'arbre en préordre, nous associons une
montée à chaque arc en descendant, et une descente sur le même arc en
montant. Clairement, il y a \(2n\)~pas dans le chemin de Dyck. La
surjection consiste simplement à inverser le procédé en lisant le
chemin de Dyck pas à pas, de la gauche vers la droite, et en
construisant l'arbre correspondant.

Nous devons maintenant compter le nombre de chemins de Dyck de
longueur~\(2n\), dons nous savons maintenant qu'il est aussi le nombre
d'arbres de Catalan avec \(n\)~arcs.

\begin{wrapfigure}[14]{r}[0pt]{0pt}
% [14] vertical lines
% {r} mandatory right placement
% [0pt] of margin overhang
\centering
\includegraphics[bb=71 591 186 721]{reflection}
\caption{Réflexion d'un préfixe par rapport à \(y = x - 1\)}
\label{fig:reflection}
\end{wrapfigure}
Le nombre total de chemins monotones de longueur~\(2n\) est le nombre
de choix de \(n\)~montées parmi \(2n\)~pas, soit
\(\binom{2n}{n}\). Nous devons ensuite soustraire le nombre de chemins
qui débutent par une montée et qui traversent la diagonale. Un tel
chemin est montré à la \fig~\ref{fig:reflection}, dessiné avec un
trait continu et gras. Le premier point atteint sous la diagonale est
utilisé pour tracer une ligne en pointillés parallèle à la
diagonale. Tous les pas depuis ce point vers \((0,0)\) sont alors
changés en leur contrepartie: une montée par une descente et
vice-versa. Le segment résultant est figuré par une ligne
hachurée. Cette opération est nommée une \emph{réflexion}
\citep{Renault_2008}. Le point crucial est que nous pouvons faire se
réfléchir tout chemin monotone qui traverse la diagonale et obtenir un
chemin distinct de \((1,-1)\) à \((n,n)\). De plus, tout chemin
réfléchi peut être réfléchi lorsqu'il atteint la ligne en pointillés
vers leur contrepartie originelle. En d'autres termes, la réflexion
est bijective. (Une autre approche visuelle et intuitive de ce même
fait a été publiée par \cite{Callan_1995}.) Par conséquent, il y a
autant de chemins monotones de \((0,0)\) à \((n,n)\) qui coupent la
diagonale qu'il y a de chemins monotones de \((1,-1)\) à \((n,n)\). Ces derniers sont aisément énumérés: \(\binom{2n}{n-1}\). En conclusion, le nombre de chemins de Dyck de longueur~\(2n\)
est\label{eq:Ann}
\begin{align*}
C_n &= \binom{2n}{n} - \binom{2n}{n-1}
= \binom{2n}{n} - \frac{(2n)!}{(n-1)!(n+1)!}\\
&= \binom{2n}{n} - \frac{n}{n+1} \cdot \frac{(2n)!}{n!n!}
 = \binom{2n}{n} - \frac{n}{n+1} \binom{2n}{n} = \frac{1}{n+1}\binom{2n}{n}.
\end{align*}
Les nombres \(C_n\) sont les \emph{nombres de Catalan}\index{nombres
  de Catalan@$C_n$|see{nombres de Catalan}}\index{nombres de Catalan}.
En employant la formule de Stirling \index{formule de Stirling}, vue à
l'équation~\eqref{eq:Stirling} à la page~\pageref{eq:Stirling}, nous trouvons que le nombre d'arbres de Catalan avec \(n\)~arcs est
\begin{equation}
  C_n = \frac{1}{n+1}\binom{2n}{n} \sim \frac{4^n}{n\sqrt{\pi n}}.
\label{eq:Cn}
\end{equation}

\section{Hauteur moyenne}
\label{sec:Catalan_height}

Comme mentionné plus haut, la \emph{hauteur} d'un arbre est le nombre
de n{\oe}uds sur un chemin maximal de la racine à une feuille,
c'est-à-dire un n{\oe}ud sans sous-arbres; par exemple, nous pouvons
suivre vers le bas et compter les n{\oe}uds~\((\circ)\) dans la
\fig~\ref{fig:catalan_tree}. Un arbre réduit à une unique feuille a
pour hauteur~\(0\).

Nous commençons par l'observation clé qu'un arbre de Catalan avec
\(n\)~arcs et de hauteur~\(h\) est en bijection avec un chemin de Dyck
de longueur~\(2n\) \emph{et de hauteur \(h-1\)} (voir
\fig~\vref{fig:catalan_tree} et \fig~\ref{fig:bijection}). Ce simple
fait nous permet de raisonner sur la hauteur des chemins de Dyck et de
transférer le résultat vers les arbres de Catalan de départ.

Soit~\(h_n\) la hauteur moyenne des arbres de Catalan avec \(n\)~arcs
et soit \(H_{n,h}\) le nombre d'arbres de Catalan avec \(n\)~arcs et
de hauteur~\(h\). Nous avons alors \(h_n = S_n/C_{n}\), où \(S_n :=
\sum_{h \geqslant 1} h \cdot H_{n,h}\). Nous devons saisir cette
somme. Par exemple, nous pourrions définir~\(A_{n,h}\) comme étant le
nombre d'arbres avec \(n\)~arcs et de hauteur inférieure ou égale
à~\(h\). Bien sûr, \(A_{n,h} = A_{n,n+1} = C_{n}\), si \(h >
n\). Alors \(H_{n,h} = A_{n,h}-A_{n,h-1}\). Les formules peuvent être
davantage simplifiées en définissant \(B_{n,h}\) comme étant le nombre
d'arbres avec \(n\)~arcs et de hauteur supérieure à~\(h\):
\begin{equation}
S_n = \sum_{h \geqslant 1}h(A_{n,h}-A_{n,h-1})
    = \sum_{h \geqslant 1}h(B_{n,h-1}-B_{n,h}) = \sum_{h\geqslant 0} B_{n,h}.
\label{eq:Sn}
\end{equation}

Avec la détermination de~\(A_{n,h}\) en tête, considérons à la
\fig~\ref{fig:height} un chemin de Dyck de longueur~\(2n\) et
hauteur~\(h-1\).
\begin{figure}[!b]
\centering
\subfloat[Chemin de Dyck de longueur~\(2n\) et
de hauteur \(h-1\) \label{fig:height}]{
\includegraphics[scale=0.9,bb=71 565 247 721]{height}}
\quad
\subfloat[Chemin de \(A\) à \(\Omega\) évitant \(y=x+s\) et
\(y=x-t\)\label{fig:mohanty}]{
\includegraphics[scale=0.9,bb=71 621 219 721]{mohanty}}
\caption{Chemins évitant des frontières diagonales}
\label{fig:boundaries}
\end{figure}
Les lignes doubles sont des frontières qui ne peuvent être atteintes
par le chemin. Ceci est en fait un cas particulier d'un chemin
monotone entre deux frontières diagonales, comme on peut le voir à la
\fig~\ref{fig:mohanty}, où \(s\)~dénote la distance verticale à partir
de~\(A\), et \(t\)~est la distance horizontale à partir de~\(A\). Il
est bien connu que le nombre de chemins monotones s'étendant
de~\(A(0,0)\) à~\(\Omega(a,b)\) et évitant les frontières est
\begin{equation}
\left\lvert\mathcal{L}(a,b;t,s)\right\rvert = \sum_{k \in \mathbb{Z}}\left[\binom{a+b}{b+k(t+s)} - \binom{a+b}{b+k(t+s)+t}\right].
\label{eq:mohanty}
\end{equation}

La preuve par \citet[p.~6]{Mohanty_1979} de cette formule est fondée
sur le principe de réflexion et le principe d'inclusion et
d'exclusion, par lequel un sur-ensemble est pris et un sous-ensemble
est soustrait parce que ces deux ensembles sont plus aisément
énumérables que le tout (nous avons employé ce principe implicitement
pour parvenir à la formule~\eqref{eq:Cn}). Nous reproduisons verbatim
sa preuve ici parce qu'elle n'est pas facile à trouver de nos jours.
\begin{proof}
  (\textbf{Mohanty}) Pour faire bref, nommons les frontières \(x=y+t\)
  et \(x=y-s\), \(\mathcal{L}^{+}\) et \(\mathcal{L}^{-}\),
  respectivement. Dénotons par~\(A_1\) l'ensemble des chemins qui
  atteignent \(\mathcal{L}^{+}\), par~\(A_2\) l'ensemble des chemins
  qui atteignent \(\mathcal{L}^{+}\), \(\mathcal{L}^{-}\) dans cet
  ordre, et en général par~\(A_i\) l'ensemble des chemins qui
  atteignent \(\mathcal{L}^{+}\), \(\mathcal{L}^{-}\),
  \(\mathcal{L}^{+}\), \ldots (\(i\)~fois) dans l'ordre indiqué. De la
  même manière, soit~\(B_i\) l'ensemble des chemins atteignant
  \(\mathcal{L}^{-}\), \(\mathcal{L}^{+}\), \(\mathcal{L}^{-}\),
  \ldots (\(i\)~fois) dans l'ordre spécifié. Une application de la
  méthode usuelle d'inclusion\--exclusion donne
  \begin{equation}
    \left\lvert\mathcal{L}(a,b;t,s)\right\rvert = \binom{a+b}{b} +
    \sum_{i \leqslant 1}(-1)^{i}(\lvert{A_i}\rvert +
    \lvert{B_i}\rvert),\label{eq:L}
  \end{equation}
  où~\(\lvert{A_i}\rvert\) et~\(\lvert{B_i}\rvert\) sont évalués
  répétitivement au moyen du principe de réflexion. Par exemple,
  considérez~\(A_3\). Puisque chaque chemin dans~\(A_3\) doit
  atteindre~\(\mathcal{L}^{+}\), \(A_3\) lorsqu'il est réfléchi par
  rapport à \(\mathcal{L}^{+}\) devient l'ensemble des chemins de
  \((t,-t)\) à \((a,b)\), chacun atteignant \(\mathcal{L}^{+}\) après
  avoir atteint \(\mathcal{L}^{-}\). Une autre réflexion par rapport à
  \(\mathcal{L}^{-}\) rendrait~\(A_3\) équivalent à l'ensemble des
  chemins de \((-s-t,s+t)\) à \((a,b)\) qui
  atteignent~\(\mathcal{L}^{+}\), qui à son tour peut s'écrire
  \(R(a+s+t,b-s-t; 2s+3t)\). [Note: \(R(a,b;t)\) est l'ensemble des
    chemins de \((0,0)\) à \((a,b)\) réfléchis par rapport
    à~\(\mathcal{L}^{+}\).] Par conséquent, puisque
  \(\lvert{R(a,b;t)}\rvert = \binom{a+b}{a-t}\),
  \begin{equation*}
    \left\lvert{A_3}\right\rvert = \binom{a+b}{a-s-2t},
  \end{equation*}
  et, plus généralement,
  \begin{equation*}
    \left\lvert{A_{2j}}\right\rvert = \binom{a+b}{a+j(s+t)}
    \quad\text{et}\quad \left\lvert{A_{2j+1}}\right\rvert =
    \binom{a+b}{a-j(s+t)-t}.
  \end{equation*}
  Les expressions pour \(\lvert{B_{2j}}\rvert\),
  \(\lvert{B_{2j+1}}\rvert\), \(j=0, 1, 2, \dots\), avec
  \(\lvert{A_0}\rvert\), \(\lvert{B_0}\rvert\) étant
  \(\binom{a+b}{b}\), sont obtenues en interchangeant \(a\)~et~\(b\)
  et \(s\)~avec~\(t\). La substitution de ces valeurs
  dans~\eqref{eq:L} donne~\eqref{eq:mohanty} après quelques
  simplifications.
\end{proof}

En reprenant le fil de notre argumentation, si nous faisons se
correspondre les figures dans la \fig~\ref{fig:boundaries}, nous
déduisons que \(s=h\), \(t=1\), \(a=b=n\), d'où \(a+b=2n\) et
\(b+k(s+t)=n+k(h+1)\), que nous remplaçons dans la
formule~\eqref{eq:mohanty} où nous changeons \(h\)~en \(h-1\):
\begin{equation*}
A_{n,h-1} = \sum_{k \in \mathbb{Z}}\left[\binom{2n}{n+kh} -
           \binom{2n}{n+1+kh}\right].
\end{equation*}
Après la partition de la somme en les cas \(k<0\), \(k=0\) et \(k>0\),
puis le changement de signe de~\(k\) dans le premier cas, ensuite en
utilisant \(\binom{p}{q} = \binom{p}{p-q}\) dans le second et dernier
cas, puis enfin en regroupant les sommes restantes indexées par \(k
\geqslant 1\), nous obtenons
\begin{align*}
A_{n,h-1}
  &= - \sum_{k \geqslant 1}\left[\binom{2n}{n+1-kh} -
    2\binom{2n}{n-kh} + \binom{2n}{n-1-kh}\right]\\
  &\phantom{=}\; + \binom{2n}{n} - \binom{2n}{n-1}.
\end{align*}
Nous reconnaissons \(C_n\) de la page~\pageref{eq:Ann}, et nous
réécrivons
\begin{equation*}
C_n - A_{n,h-1}
  = \sum_{k \geqslant 1}\left[\binom{2n}{n+1-kh} -
    2\binom{2n}{n-kh} + \binom{2n}{n-1-kh}\right].
\end{equation*}
Finalement, nous souvenant que \(B_{n,h} = A_{nn} - A_{n,h}\) et \(C_n =
A_{n,n+1}\), nous parvenons à la formule
\begin{equation}
B_{n,h-1} = \sum_{k \geqslant 1}
            \left[\binom{2n}{n+1-kh} - 2\binom{2n}{n-kh}
            + \binom{2n}{n-1-kh}\right].
\label{eq:Bn}
\end{equation}

\citet*{KnuthdeBruijnRice_1972} ont publié un article majeur où ils
obtiennent le même résultat en employant des mathématiques bien plus
compliquées. Ils commencent par modéliser le problème avec une
fonction génératrice \citep{Wilf_1990} qui satisfait une équation de
récurrence dont la solution exprime la fonction génératrice en termes
de fractions continuées de polynômes de Fibonacci. Ils utilisent alors
l'intégration sur le plan complexe pour trouver la
formule~\eqref{eq:Bn}. Alternativement, les fonctions génératrices
peuvent être employées sur les chemins monotones dans des treillis, au
lieu des arbres de Catalan \citep[page~64]{Kemp_1984}
\citep{FlajoletNebelProdinger_2006}.

\citet*{SedgewickFlajolet_1996} \citep{FlajoletSedgewick_2009} utilisent l'analyse combinatoire et l'analyse réelle pour obtenir l'approximation asymptotique de~\(B_{n,h}\). Ils écrivent~\cite[p.~260]{SedgewickFlajolet_1996}: «~Cette analyse est la plus difficile de celles que nous menons dans ce livre. Elle combine des techniques pour résoudre des récurrences linéaires et des fractions continuées, des développements limités de fonctions génératrices, en particulier au moyen du théorème d'inversion de Lagrange, les approximations binomiales et les sommes d'Euler\--Maclaurin.~» Il n'est pas possible d'entrer dans les détails ici, mais nous pouvons esquisser comment une approximation asymptotique peut être menée à bien.

L'équation~\eqref{eq:Sn} implique \(S_{n} = \sum_{h \geqslant 1}
B_{n,h-1}\), d'où
\begin{equation*}
S_{n} = \sum_{k' \geqslant 1}d(k') \cdot
         \left[\binom{2n}{n+1-k'} - 2\binom{2n}{n-k'}
         + \binom{2n}{n-1-k'}\right],
\end{equation*}
où~\(d(k')\) est le nombre de diviseurs positifs de~\(k'\), mais une
analyse sur les nombres complexes est nécessaire
\citep{KnuthdeBruijnRice_1972,FlajoletGourdonDumas_1995}. Une autre
approche consiste à exprimer les coefficients binomiaux en termes de
\(\binom{2n}{n-kh}\) comme suit:
\begin{align*}
\binom{2n}{n-m+1} &= \frac{(2n)!}{(n-m+1)!\,(n+m-1)!}\\
                  &= \frac{(2n)!\,(n+m)}{(n-m)!\,(n-m+1)(n+m)!}
                   = \frac{n+m}{n-m+1}\binom{2n}{n-m},\\
\binom{2n}{n-m-1} &= \frac{(2n)!}{(n-m-1)!\,(n+m+1)!}\\
                  &= \frac{(2n)!\,(n-m)}{(n-m)!\,(n+m)!\,(n+m+1)}
                   = \frac{n-m}{n+m+1}\binom{2n}{n-m}.
\end{align*}
Par conséquent,
\begin{equation*}
\binom{2n}{n-m+1} - 2\binom{2n}{n-m} + \binom{2n}{n-m-1}
= 2 \cdot \frac{2m^2-(n+1)}{(n+1)^2-m^2}\binom{2n}{n-m}.
\end{equation*}
Posons \(F_n(m) = (2m^2-n)/(n^2-m^2)\). Nous avons
\begin{equation*}
S_{n} = 2 \cdot \sum_{h \geqslant 1}\sum_{k \geqslant 1} F_{n+1}(kh)
\cdot \binom{2n}{n-kh},
\end{equation*}
De l'équation~\eqref{eq:Cn} et de \(h_n = S_n/C_n\), nous tirons
\(h_{n} = (n+1)S_{n}{\binom{2n}{n}}^{-1}\), donc nous devons approcher
\((n+1)F_{n+1}(m)\) et \(\binom{2n}{n-m}\binom{2n}{n}^{-1}\). D'une
part, nous avons
\begin{equation*}
F_{n+1}(m) \sim \frac{2m^2-n}{n^2} \sim \frac{2m^2-n}{n(n+1)},
\end{equation*}
donc \((n+1)F_{n+1}(kh) \sim 2k^2h^2\!/n-1\). D'une autre part,
\citet*[4.6, 4.8]{SedgewickFlajolet_1996} montrent que
\begin{equation*}
\binom{2n}{n-m}{\binom{2n}{n}}^{-1} \sim e^{-m^2\!/n}.
\end{equation*}
En supposant que les termes d'erreur implicites des deux
approximations précédentes diminuent exponentiellement, nous avons
\begin{equation*}
h_{n} \sim \sum_{h \geqslant 1}\sum_{k \geqslant 1}
(4k^2h^2\!/n - 2)e^{-k^2h^2\!/n}
= \sum_{h \geqslant 1}H(h/\!\sqrt{n}),
\end{equation*}
avec \(H(x) := \sum_{k \geqslant
  1}(4k^2x^2-2)e^{-k^2x^2}\). Finalement,
\citet*[5.9]{SedgewickFlajolet_1996}, tout comme
\citet*[9.6]{GrahamKnuthPatashnik_1994}, font usage d'analyse réelle
pour conclure
\begin{equation*}
h_{n} \sim \sum_{h \geqslant 1}H(h/\!\sqrt{n})
\sim \sqrt{n} \int_0^{\infty}\!\!H(x) dx \sim \sqrt{\pi n}.
\end{equation*}
Après que \(B_{n,h-1}\) a été obtenu, la fin de notre dérivation est
difficile et même pas entièrement formelle parce que les termes
d'erreur dans l'approximation asymptotique bivaluée devraient être
prudemment vérifiés, comme les auteurs référencés le
font. Malheureusement, ceci aussi veut dire que cette partie a peu de
chance d'être simplifiée davantage.

\section{Longueur moyenne des chemins}

La \emph{longueur des chemins} d'un arbre de Catalan est la somme des
longueurs des chemins depuis la racine. Nous avons vu ce concept dans
le contexte des arbres binaires, où il était décliné en deux
variantes, la \emph{longueur interne}
(page~\pageref{insertion__internal_path_length}) et la \emph{longueur
  externe} (page~\pageref{sorting__external_path_length}), selon que
le n{\oe}ud final était interne ou externe. Dans le cas des arbres de
Catalan, la distinction pertinente entre n{\oe}uds est d'être une
\emph{feuille} (c'est-à-dire un n{\oe}ud sans sous-arbre) ou non, mais
certains auteurs parlent néanmoins de longueur externe lorsqu'ils se
réfèrent aux distances des feuilles, et de longueur interne pour les
non-feuilles, donc nous devons garder présent à l'esprit si le
contexte est les arbres de Catalan ou les arbres binaires.

Dans le but d'étudier la longueur moyenne (des chemins) des arbres de
Catalan, et quelques paramètres voisins, nous pourrions suivre
\cite{DershowitzZaks_1981} en recherchant d'abord le nombre moyen de
n{\oe}uds de degré~\(d\) au niveau~\(l\) dans un arbre de Catalan
avec \(n\)~arcs. Le \emph{degré d'un n{\oe}ud} est le nombre de ses
enfants et son \emph{niveau} est sa distance jusqu'à la racine,
comptée en arcs, sachant que la racine se trouve au niveau~\(0\). (À
contraster avec la définition de la hauteur, où ce sont les n{\oe}uds
qui sont comptés.)

Le premier pas de notre méthode pour obtenir la longueur moyenne (des
chemins) nécessite la définition d'une nouvelle bijection entre les
arbres de Catalan et les chemins de Dyck. À la \fig~\vref{fig:ctree},
nous voyons un arbre de Catalan équivalent au chemin de Dyck à la
\fig~\ref{fig:dyck}, construit à partir du parcours en préordre de cet
arbre. La \fig~\ref{fig:ctree_deg}
\begin{figure}
\centering
\subfloat[Chemin de Dyck\label{fig:dyck_deg}]{%
  \includegraphics[bb=71 634 158 721]{dyck_deg}}
\qquad
\subfloat[Arbre de Catalan\label{fig:ctree_deg}]{%
\includegraphics[bb=58 660 140 721]{ctree_deg}
}
\caption{Bijection fondée sur les degrés}
\label{fig:bijection_deg}
\end{figure}
montre le même arbre, où le contenu des n{\oe}uds est leur degré. Le
parcours en préordre (des degrés) est \([3,0,0,2,1,0,0]\). Puisque le
dernier degré est toujours~\(0\) (une feuille), nous l'éliminons et
conservons \([3,0,0,2,1,0]\). Un autre chemin de Dyck équivalent est
obtenu en traduisant les degrés de cette liste en suites de montées
(\(\uparrow\)) suivies d'une descente (\(\rightarrow\)), donc, par
exemple, \(3\)~est traduit en (\(\uparrow, \uparrow, \uparrow,
\rightarrow\)) et \(0\)~en (\(\rightarrow\)). À la fin,
\([3,0,0,2,1,0]\) est traduit en \([\uparrow, \uparrow, \uparrow,
\rightarrow, \rightarrow, \rightarrow, \uparrow, \uparrow,
\rightarrow, \uparrow, \rightarrow, \rightarrow]\), ce qui correspond
au chemin de Dyck à la \fig~\ref{fig:dyck_deg}. Il est aisé de se
convaincre soi-même que nous pouvons reconstruire l'arbre à partir du
chemin de Dyck, par conséquent, nous avons bien là une bijection.

La raison d'être de cette nouvelle bijection est la nécessité de
trouver le nombre moyen de n{\oe}uds dans un arbre de Catalan dont la
racine possède un degré donné. Ce nombre nous aidera à parvenir à la
longueur moyenne des chemins, en appliquant une idée de
\cite{Ruskey_1983}. D'après la bijection, il est clair que le nombre
d'arbres dont la racine a pour degré~\(r=3\) est le nombre de chemins
de Dyck contenant le segment de \((0,0)\) à \((0,r)\), suivi d'une
descente (voir le point \((1,r)\) à la \fig~\ref{fig:dyck_deg}), et
ensuite tous les chemins monotones au-dessus de la diagonale jusqu'au
coin en haut à droite \((n,n)\).  Par conséquent, nous devons
déterminer le nombre de ces chemins.

Nous avons vu à la section~\vref{sec:Catalan_height} la réflexion
bijective de chemins et l'énumération à l'aide du principe d'inclusion
et d'exclusion. Ajoutons maintenant à notre arsenal une bijection de
plus qui se révèle souvent utile: le \emph{retournement}. Elle
consiste simplement à renverser l'ordre des pas constitutifs d'un
chemin. Considérons par exemple la \fig~\ref{fig:path_reversal}.
\begin{figure}
\centering
\subfloat[Retournement de la
\fig~\ref{fig:reflection} \label{fig:path_reversal}]{
\includegraphics[bb=71 606 188 721]{path_reversal}}
\qquad
\subfloat[Retournement et réflexion de la \fig~\ref{fig:dyck_deg} après \((1,3)\)\label{fig:path_deg}]{
\includegraphics[bb=62 634 168 721]{path_deg}}
\caption{Retournements et réflexions}
\end{figure}
Bien entendu, la composition de deux bijections étant une bijection,
la composition d'un retournement et d'une réflexion est bijective,
donc les chemins monotones au-dessus de la diagonale de \((1,r)\) à
\((n,n)\) sont en bijection avec les chemins monotones au-dessus de la
diagonale de \((0,0)\) à \((n-r,n-1)\). Par exemple, la
\fig~\ref{fig:path_deg} montre le retournement et la réflexion du
chemin de Dyck de la \fig~\ref{fig:dyck_deg} après le point \((1,3)\),
distingué par un disque noir (\(\bullet\)).

En nous souvenant que les arbres de Catalan avec \(n\)~arcs sont en
bijection avec les chemins de Dyck de longueur~\(2n\)
(section~\vref{sec:Catalan_enumeration}), nous savons à présent que le
nombre d'arbres de Catalan avec \(n\)~arcs et dont la racine a pour
degré~\(r\) est le nombre de chemins monotones au-dessus de la
diagonale du point \((0,0)\) à \((n-r,n-1)\). Nous pouvons trouver ce
nombre en utilisant la même technique que pour le nombre total~\(C_n\)
de chemins de Dyck. Le principe d'inclusion et d'exclusion dit que
nous devrions compter le nombre total de chemins avec les mêmes
extrémités et soustraire le nombre de chemins qui traversent la
diagonale. Le premier nombre est \(\binom{2n-r-1}{n-1}\), qui énumère
les façons d'interclasser \(n-1\)~montées (\(\uparrow\)) avec \(n-r\)
descentes (\(\rightarrow\)). Le dernier nombre est le même que le
nombre de chemins monotones de \((1,-1)\) à \((n-r,n-1)\), comme on
peut le voir en faisant se réfléchir les chemins jusqu'à leur première
traversée de la diagonale, soit \(\binom{2n-r-1}{n}\); en d'autres
termes, c'est le nombre d'interclassements de \(n\)~montées avec
\(n-r-1\) descentes. Finalement, en imitant la dérivation de
l'équation~\eqref{eq:Cn}, le nombre \(\mathcal{R}_n(r)\) d'arbres avec
\(n\)~arcs et racine de degré~\(r\) est
\begin{equation*}
\mathcal{R}_n(r) = \binom{2n-r-1}{n-1} - \binom{2n-r-1}{n}
                 = \frac{r}{2n-r} \binom{2n-r}{n}.
\end{equation*}

Soit \(\mathcal{N}_n(l,d)\) le nombre d'arbres de Catalan avec
\(n\)~arcs au niveau~\(l\) et de degré~\(d\). Ce nombre constitue
l'étape suivante dans la détermination de la longueur moyenne des
chemins parce que \cite{Ruskey_1983} a trouvé une jolie bijection qui
le met en relation avec \(\mathcal{R}_n(r)\) dans l'équation suivante:
\begin{equation*}
\mathcal{N}_n(l,d) = \mathcal{R}_{n+l}(2l+d).
\end{equation*}
À la \fig~\ref{fig:level_deg}
\begin{figure}
\centering
\subfloat[\(n\)~arcs, (\(\bullet\)) de degré~\(d\) et niveau~\(l\) \label{fig:level_deg}]{%
  \includegraphics[bb=71 594 150 721]{level_deg}}
\quad
\subfloat[\(n+l\) arcs, racine de degré~\(2l+d\)
\label{fig:lifted_node}]{%
\includegraphics[bb=71 663 295 721]{lifted_node}
}
\caption{Bijection}
\label{fig:bij_root_level}
\end{figure}
se trouve le motif général d'un arbre de Catalan avec un n{\oe}ud
(\(\bullet\)) au niveau~\(d\) et de degré~\(d\). Les arcs doubles
dénotent un ensemble d'arcs, donc \(\mathcal{L}_i\), \(\mathcal{R}_i\)
et \(\mathcal{B}_i\) représentent en fait des forêts. À la
\fig~\ref{fig:lifted_node}, nous voyons un arbre de Catalan en
bijection avec le premier, obtenu de celui-ci en élevant le n{\oe}ud
qui nous intéresse (\(\bullet\)) pour en faire la racine, les forêts
\(\mathcal{L}_i\) avec leurs parents respectifs sont attachées
en-dessous, ensuite les \(\mathcal{B}_i\), et, finalement, les
\(\mathcal{R}_i\) pour lesquelles de nouveaux parents sont nécessaires
(dans un cadre hachuré dans la figure). Il est clair que la nouvelle
racine a pour degré \(2l+d\) et il y a \(n+l\) arcs. Il est essentiel
de comprendre que la transformation peut être inversée pour tout arbre
(elle est injective et surjective), donc est elle bien bijective. Nous
en déduisons
\begin{equation*}
\mathcal{N}_n(l,d) = \frac{2l+d}{2n-d}\binom{2n-d}{n+l}
= \binom{2n-d-1}{n+l-1} - \binom{2n-d-1}{n+l},
\end{equation*}
où la dernière étape résulte de l'expression du coefficient binomial en termes de la fonction factorielle. En particulier, ceci entraîne que le nombre total de n{\oe}uds au niveau~\(l\) dans tous les arbres de Catalan avec \(n\)~arcs est
\begin{equation*}
\sum_{d=0}^{n}\mathcal{N}_n(l,d)
  = \sum_{d=0}^{n}\binom{2n-d-1}{n+l-1}
    - \sum_{d=0}^{n}\binom{2n-d-1}{n+l}.
\end{equation*}
Tournons notre attention vers la première somme:
\begin{equation*}
\sum_{d=0}^{n}\binom{2n-d-1}{n+l-1}
  = \sum_{i=n-1}^{2n-1}\binom{i}{n+l-1}
  = \sum_{i=n+l-1}^{2n-1}\binom{i}{n+l-1}.
\end{equation*}
Nous pouvons maintenant employer l'identité \eqref{eq:binom_sum}
\vpageref{eq:binom_sum}, qui équivaut à \(\sum_{i=j}^{k}\binom{i}{j} =
\binom{k+1}{j+1}\), donc \(j=n+l-1\) et \(k=2n-1\) entraînent
\begin{equation*}
\sum_{d=0}^{n}\binom{2n-d-1}{n+l-1} = \binom{2n}{n+l}.
\end{equation*}
De plus, en remplaçant \(l\)~par \(l+1\) donne
\(\sum_{d=0}^{n}\binom{2n-d-1}{n+l} = \binom{2n}{n+l+1}\), donc le
nombre total de n{\oe}uds au niveau~\(l\) dans tous les arbres de
Catalan avec \(n\)~arcs est
\begin{equation}
\sum_{d=0}^{n}\mathcal{N}_n(l,d)
 = \binom{2n}{n+l} - \binom{2n}{n+l+1}
 = \frac{2l+1}{2n+1}\binom{2n+1}{n-l}.
\label{eq:N_n_l_d}
\end{equation}

Soit \(\Expected{P_n}\) la \emph{longueur moyenne des chemins} d'un
arbre de Catalan avec \(n\)~arcs. Nous avons
\begin{equation*}
  \Expected{P_n} = \frac{1}{C_{n}} \cdot \sum_{l=0}^{n}l\sum_{d=0}^{n}\mathcal{N}_n(l,d),
\end{equation*}
parce qu'il y a \(C_n\)~arbres et la double somme est la somme des
longueurs des chemins de tous les arbres. Si nous moyennons encore par
le nombre de n{\oe}uds, soit \(n+1\), nous obtenons le niveau moyen
d'un n{\oe}ud dans un arbre de Catalan aléatoire, et nous devons
prendre garde car certains auteurs choisissent cette quantité comme la
définition de la longueur moyenne. Alternativement, si nous
choisissons au hasard des arbres de Catalan distincts avec \(n\)~arcs,
puis choisissons des n{\oe}uds distincts dans ceux-ci,
\(\Expected{P_n}/(n+1)\) est la limite du coût moyen pour les
atteindre depuis la racine. En usant des équations \eqref{eq:N_n_l_d}
et~\eqref{eq:Cn} \vpageref{eq:Cn}, alors
\begin{align*}
\Expected{P_n} \cdot C_n
 &= \sum_{l=0}^{n}l\left[\binom{2n}{n+l} - \binom{2n}{n+l+1}\right]\\
 &= \sum_{l=1}^{n}l \binom{2n}{n+l} - \sum_{l=0}^{n-1}l \binom{2n}{n+l+1}\\
 &= \sum_{l=1}^{n}l \binom{2n}{n+l} - \sum_{l=1}^{n}(l-1)\binom{2n}{n+l}\\
 &= \sum_{l=1}^{n}\binom{2n}{n+l}
  = \sum_{i=n+1}^{2n}\binom{2n}{i}.
\end{align*}
La somme restante est facile à débloquer parce qu'elle est la somme de
la moitié d'une ligne paire dans le triangle de Pascal. Nous voyons à
la \fig~\vref{fig:pascal_triangle} que la première moitié égale la
seconde, seul l'élément central restant (il y a un nombre impair
d'éléments dans une ligne paire). Ceci est vite vu:
\(\sum_{j=0}^{n-1}\binom{2n}{j} = \sum_{j=0}^{n-1}\binom{2n}{2n-j} =
\sum_{i=n+1}^{2n}\binom{2n}{i}\). Par conséquent,
\begin{equation*}
\sum_{i=0}^{2n}\binom{2n}{i} = 2 \cdot \!\! \sum_{i=n+1}^{2n}\binom{2n}{i}
+ \binom{2n}{n},
\end{equation*}
et nous pouvons poursuivre comme suit:
\begin{equation*}
\frac{\Expected{P_n}}{n+1}
  = \frac{1}{2}\binom{2n}{n}^{-1}\left[\sum_{i=0}^{2n}\binom{2n}{i} -
    \binom{2n}{n}\right]
  = \frac{1}{2}\left[\binom{2n}{n}^{-1}\sum_{i=0}^{2n}\binom{2n}{i} - 1\right].
\end{equation*}
La somme restante est peut-être l'identité combinatoire la plus
fameuse parce qu'elle est un corollaire du vénérable \emph{théorème
  binomial}, qui établit que, pour tous nombres réels \(x\)~et~\(y\),
et tout entier positif~\(n\), nous avons l'égalité suivante:
\begin{equation*}
(x+y)^n = \sum_{k=0}^{n}\binom{n}{k}x^{n-k}y^k.
\end{equation*}
La vérité de cette proposition apparaît grâce au raisonnement
suivant. Puisque, par définition, \((x+y)^n =
\underline{(x+y)(x+y)\ldots(x+y)}_{\text{\(n\) fois}}\), chaque terme
dans le développement de \((x+y)^{n}\) a la forme \(x^{n-k}y^k\), pour
un certain~\(k\) variant de~\(0\) à~\(n\), inclus. Le coefficient de
\(x^{n-k}y^{k}\) pour un~\(k\) donné est simplement le nombre de choix
de~\(k\) variables~\(y\) parmi les \(n\)~facteurs de \((x+y)^{n}\),
les variables~\(x\) provenant des \(n-k\) facteurs restants.

Poser \(x=y=1\) entraîne l'identité \(2^n =
\sum_{k=0}^{n}\binom{n}{k}\), qui, finalement, débloque notre dernière
étape:
\begin{equation}
\Expected{P_n} = \frac{n+1}{2}\left[4^{n}\binom{2n}{n}^{-1} - 1\right].
\label{eq:EPn}
\end{equation}
En nous souvenant de~\eqref{eq:Cn} \vpageref{eq:Cn}, nous obtenons le
développement asymptotique suivant:
\begin{equation*}
\Expected{P_n} \sim \frac{1}{2}n\sqrt{\pi n}.
\end{equation*}
Remarquons que cette équivalence est vraie aussi si~\(n\) dénote un
nombre de n{\oe}uds, au lieu d'un nombre d'arcs. La formule exacte
donnant la longueur moyenne des chemins d'un arbre de Catalan avec
\(n\)~n{\oe}uds est \(\Expected{P_{n-1}}\) car il possède alors
\(n-1\) arcs.

Pour certaines applications, il est utile de connaître les longueurs
externes et internes, qui sont, respectivement, les longueurs des
chemins jusqu'aux feuilles et aux n{\oe}uds internes (à ne pas
confondre avec les longueurs externes et internes des arbres
binaires). Soit \(\Expected{E_n}\) la première et \(\Expected{I_n}\)
la seconde. Nous avons:
\begin{align*}
\Expected{E_n} \cdot C_n
  &= \sum_{l=0}^{n} l \cdot \mathcal{N}_n(l,0)
   = \sum_{l=0}^{n}l\left[\binom{2n-1}{n+l-1} -
     \binom{2n-1}{n+l}\right]\\
  &= \sum_{l=0}^{n-1}(l+1)\binom{2n-1}{n+l} -
     \sum_{l=0}^{n-1}l \binom{2n-1}{n+l}
   = \sum_{l=0}^{n-1}\binom{2n-1}{n+l},\\
\Expected{E_n} \cdot C_n
  &= \sum_{i=n}^{2n-1}\binom{2n-1}{i}
   = \frac{1}{2}\sum_{j=0}^{2n-1}\binom{2n-1}{j} = 4^{n-1},
\end{align*}
où la dernière égalité procède du fait qu'une ligne impaire dans le
triangle de Pascal contient un nombre pair de coefficients et les deux
moitiés ont des sommes égales. Nous concluons:
\begin{equation}
\Expected{E_n} = (n+1) 4^{n-1}\binom{2n}{n}^{-1} \sim
\frac{1}{4}n\sqrt{\pi n}.
\label{eq:EEn}
\end{equation}
La dérivation de \(\Expected{I_n}\) est:
\begin{align}
  \Expected{I_n} \cdot C_n
  &= \sum_{l=0}^{n}l\sum_{d=1}^{n}\mathcal{N}_n(l,d)
   = \sum_{l=0}^{n}l\left[\binom{2n-1}{n+l} -
     \binom{2n-1}{n+l+1}\right]\notag\\
  &= \sum_{l=1}^{n-1}\binom{2n-1}{n+l} = \Expected{E_n} \cdot C_n -
  \binom{2n-1}{n}\notag\\
  &= 4^{n-1} - \frac{1}{2}\binom{2n}{n}
   = 4^{n-1} - \frac{1}{2}(n+1)C_n.\notag
\intertext{Par conséquent,}
\Expected{I_n}
  &= (n+1)4^{n-1}\binom{2n}{n}^{-1} - \frac{n+1}{2} \sim
  \frac{1}{4}n\sqrt{\pi n}.
\label{eq:EIn}
\end{align}
Des formules \eqref{eq:EEn} et~\eqref{eq:EIn}, nous tirons
\begin{equation*}
\Expected{E_n} = \Expected{I_n} + \frac{n+1}{2}.
\end{equation*}
On peut montrer que \((n+1)/2\) est le nombre moyen de feuilles dans
un arbre Catalan avec \(n\)~arcs, c'est-à-dire que la moitié des
n{\oe}uds sont des feuilles en moyenne.

À l'asymptote, les formules \eqref{eq:EPn}, \eqref{eq:EEn}
et~\eqref{eq:EIn} entraînent
\begin{equation*}
\Expected{I_n} \sim \Expected{E_n} \sim \frac{1}{2}\Expected{P_n}.
\end{equation*}
Pour appronfondir le sujet, nous recommandons les articles de
\citet*{DershwowitzZaks_1980,DershowitzZaks_1981,DershowitzZaks_1990}.
