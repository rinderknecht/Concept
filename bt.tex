\chapter{Arbres binaires}

Dans ce chapitre, nous plaçons l'accent sur l'arbre binaire
comme structure de donnée, en la redéfinissant au passage et en
présentant quelques algorithmes et mesures classiques qui s'y
rapportent.

La \fig~\vref{fig:bt_shape} montre un arbre binaire en exemple.
\begin{figure}
\centering
\includegraphics{bt_shape}
\caption{Un arbre binaire}
\label{fig:bt_shape}
\end{figure}
Les n{\oe}uds sont de deux sortes: internes\index{arbre
  binaire!n{\oe}ud interne|(} (\(\circ\) et \(\bullet\)) ou
externes\index{arbre binaire!n{\oe}ud externe|(} (\(\scriptstyle
\Box\)). La caractéristique d'un arbre binaire est que les n{\oe}uds
internes sont reliés vers le bas à deux autres n{\oe}uds, appelé
\emph{enfants}\index{arbre!n{\oe}ud!enfant}, alors que les n{\oe}uds
externes n'ont pas de tels liens. La
racine\index{arbre!n{\oe}ud!racine} est le n{\oe}ud interne le plus
haut, représenté par un cercle et un diamètre. Les
\emph{feuilles}\index{arbre binaire!feuille} sont les n{\oe}uds
internes dont les enfants sont deux n{\oe}uds externes; elles sont
figurées par des disques (\(\bullet\)).

% Wrapping figure better declared before a paragraph
%
%\begin{wrapfigure}[9]{r}[0pt]{0pt}
\begin{figure}
% [9] vertical lines
% {r} mandatory right placement
% [0pt] of margin overhang
\centering
\includegraphics[bb=71 648 177 723]{leaf_tree}
\caption{Un arbre feuillu}
\label{fig:leaf_tree}
\end{figure}
%\end{wrapfigure}
Les n{\oe}uds internes sont d'habitude associés avec quelque
information, alors que les n{\oe}uds externes ne le sont pas, comme
celui montré à la \fig~\vref{fig:bt_ex1}:
\begin{figure}
\centering
\subfloat[Arbre binaire étendu\label{fig:bt_ex1}]{
\includegraphics{bt_ex1}}
\qquad
\subfloat[Arbre binaire taillé\label{fig:bt_ex2}]{
\includegraphics[bb=67 645 169 721]{bt_ex2}}
\caption{Deux représentations d'un arbre binaire}
\label{fig:bt_ex}
\end{figure}
ceci sera la représentation par défaut dans ce livre. Parfois, pour
attirer plus d'attention sur les n{\oe}uds internes, les n{\oe}uds
externes peuvent être omis, comme à la \fig~\vref{fig:bt_ex2}. De
plus, on pourrait alors restreindre l'information aux seules feuilles,
auquel cas l'arbre est dit \emph{feuillu}\index{arbre binaire!$\sim$
  feuillu}, comme à la \fig~\vref{fig:leaf_tree}. Dans une autre
variante, tous les n{\oe}uds comportent quelque donnée, comme c'était
le cas avec les arbres de comparaison (voir par exemple la
\fig~\vref{fig:cmp_tree}, où les n{\oe}uds externes contiennent des
permutations et les n{\oe}uds internes ont des comparaisons de clés).

Comme nous l'avons vu avec l'étude des algorithmes de tri optimaux en
moyenne, \vpageref{par:minimax}, un \emph{chemin
  externe}\index{arbre!chemin externe} est un chemin de la racine à un
n{\oe}ud externe\index{arbre!n{\oe}ud!$\sim$ externe} et la
\emph{hauteur}\index{arbre!hauteur} d'un arbre est la longueur des
chemins externes maximaux. Par exemple, la hauteur de l'arbre binaire
à la \fig~\vref{fig:bt_shape} est~\(5\) et il y a deux chemins
externes de longueur maximale. Un chemin de la racine à un n{\oe}ud
interne est un \emph{chemin interne}\index{arbre!chemin interne}. La
\emph{longueur interne}\index{arbre!longueur interne} d'un arbre est
la somme des longueurs de tous ses chemins internes. Nous avons déjà
rencontré la \emph{longueur externe} \vpageref{external_path_length},
qui est la somme des longueurs de tous les chemins externes. (La
longueur d'un chemin est le nombre de ses arcs.)

\paragraph{Avertissement}

Certains auteurs utilisent une nomenclature différente pour la
définition des feuilles et de la hauteur. Il n'est pas rare de
rencontrer le concept de \emph{profondeur}\index{arbre
  binaire!profondeur} d'un arbre, qui peut être confondu par erreur
avec sa hauteur, la profondeur comptant le nombre de n{\oe}uds sur un
chemin maximal.
\begin{thm}%[Internal and external nodes]
\label{thm_int_ext}
\textsl{Un arbre binaire avec \(n\)~n{\oe}uds internes a
  \(n+1\)~n{\oe}uds externes.}
\end{thm}
\begin{proof}
  Soit \(e\)~le nombre de n{\oe}uds externes à déterminer. Nous
  pouvons compter les arcs de deux manières complémentaires. Du haut
  vers le bas, nous voyons que chaque n{\oe}ud interne a exactement
  deux enfants, donc \(l=2n\), où \(l\)~est le nombre d'arcs. Du bas
  vers le haut, nous voyons que chaque n{\oe}ud a exactement un
  parent, sauf la racine, qui n'en a pas. Par conséquent,
  \(l=(n+e)-1\). En identifiant les deux valeurs de~\(l\), nous
  obtenons \(e=n+1\).
\end{proof}

\paragraph{Structure de donnée}

Il y a de nombreuses manières de représenter un arbre binaire comme
une structure de donnée. D'abord, nous pouvons remarquer que, tout
comme une pile peut être vide ou non, il y a deux sortes de n{\oe}uds,
internes ou externes, et l'arbre vide peut être identifié à un
n{\oe}ud externe. Donc, nous n'avons besoin que de deux constructeurs
de donnée, disons \fun{ext/0}\index{ext@\fun{ext/0}} pour les
n{\oe}uds externes et \fun{int/3}\index{int@\fun{int/3}} pour les
n{\oe}uds internes. Ce dernier a trois arguments parce que deux
enfants sont attendus, ainsi que quelque information. L'ordre de ces
arguments pourrait varier. Si nous voyons un n{\oe}ud interne se
trouver, horizontalement, entre ses sous-arbres \(t_1\)~et~\(t_2\),
nous pourrions préférer l'écriture \(\fun{int}(t_1,x,t_2)\). (Les
lecteurs sémites pourraient souhaiter échanger \(t_1\)~et~\(t_2\).)
Alternativement, nous pourrions considérer qu'un n{\oe}ud interne se
trouve, verticalement, avant ses sous-arbres, auquel cas nous
préférerions écrire \(\fun{int}(x,t_1,t_2)\). Ce choix rend la
dactylographie et l'écriture manuscrite de petits arbres plus aisée,
en particulier pour tester\index{test} nos programmes. Par exemple,
l'arbre binaire de la \fig~\vref{fig:bt_ex1} correspond à
\(\fun{int}(8, t_1, \fun{int}(3, \fun{ext}(), \fun{ext}(2,
\fun{ext}(), \fun{ext}())))\), avec le sous-arbre \(t_1=\fun{int}(1,
\fun{ext}(3, \fun{ext}(), \fun{int}(5, \fun{ext}(), \fun{ext}())),
\fun{ext}(9, \fun{ext}(), \fun{ext}()))\). Nous adopterons dorénavant
la convention \(\fun{int}(x,t_1,t_2)\), parfois appelée \emph{notation
  préfixe}.\index{arbre binaire!notation préfixe}\index{arbre
  binaire!n{\oe}ud interne|)}\index{arbre binaire!n{\oe}ud externe|)}

La \emph{taille}\index{arbre binaire!taille} d'un arbre binaire est le
nombre de ses n{\oe}uds internes. C'est là la mesure la plus fréquente
des arbres quand on exprime le coût de fonctions. En guise de
réchauffement, écrivons un programme qui calcule la taille d'un arbre
binaire:
\begin{equation}
\fun{size}(\fun{ext}()) \rightarrow 0;\quad
\fun{size}(\fun{int}(x,t_1,t_2))
  \rightarrow 1 + \fun{size}(t_1) + \fun{size}(t_2).
\index{size@\fun{size/1}}
\label{eq:size}
\end{equation}
Notons la similitude avec le calcul de la longueur d'une pile:
\begin{equation*}
\fun{len}(\fun{nil}()) \rightarrow 0;\quad
\fun{len}(\fun{cons}(x,s)) \rightarrow 1 + \fun{len}(s).
\index{len@\fun{len/1}}
\index{nil@\fun{nil/0}}
\index{cons@\fun{cons/2}}
\end{equation*}
La différence est que deux appels récursifs sont nécessaires pour
visiter tous les n{\oe}uds d'un arbre binaire, au lieu d'un pour une
pile. Cette topologie bidimensionnelle rend possible de nombreux types
de visites, appelés \emph{marches} ou \emph{parcours}.

\section{Parcours}
\index{arbre!parcours}
\index{arbre!marche|see{parcours}}
\label{sec:traversals}

Dans cette section, nous présentons les parcours classiques des arbres
binaires, qui sont distingués selon l'ordre dans lequel l'information
emmagasinée aux n{\oe}uds internes est empilée sur une pile.

\mypar{Préfixe}
\index{arbre binaire!préfixe|(}
\label{preorder}

Un parcours \emph{préfixe}\index{arbre binaire!préfixe} d'un arbre
binaire non-vide est une pile dans laquelle on trouve d'abord la
racine (récursivement, le n{\oe}ud interne courant), suivie par les
n{\oe}uds en préfixe du sous-arbre gauche et, finalement, les
n{\oe}uds en préfixe du sous-arbre droit. (Par abus de langage, nous
identifierons le contenu des n{\oe}uds aux n{\oe}uds eux-mêmes
lorsqu'il n'y a pas d'ambiguïté.) Par exemple, les n{\oe}uds en
préfixe de l'arbre à la \fig~\vref{fig:bt_ex1} sont
\([8,1,3,5,9,3,2]\). Étant donné que cette méthode visite les enfants
d'un n{\oe}ud avant ses frères\index{arbre!n{\oe}ud!frère} (deux
n{\oe}uds internes sont frères s'ils ont le même parent), elle est un
\emph{parcours en profondeur}.\index{arbre binaire!parcours en
  profondeur} Une simple fonction le réalisant est \fun{pre\(_0\)/1}:
\begin{equation}
\fun{pre}_0(\fun{ext}()) \xrightarrow{\smash{\gamma}} \el;
\;\;
\fun{pre}_0(\fun{int}(x,t_1,t_2)) \xrightarrow{\smash{\delta}}
 \cons{x}{\fun{cat}(\fun{pre}_0(t_1),\fun{pre}_0(t_2))}.
\label{eq:pre0}
\index{pre0@\fun{pre\(_0\)/1}}
\end{equation}
Nous avons utilisé la concaténation\index{pile!concaténation} de piles
réalisée par \fun{cat/2}\index{cat@\fun{cat/2}}, définie
en~\eqref{def:cat} \vpageref{def:cat}, pour ordonner les valeurs des
sous-arbres. Nous savons que le coût de \fun{cat/2} est linéaire en la
taille de son premier argument: \(\C{\fun{cat}}{p} :=
\Call{\fun{cat}(s,t)} = p+1\), où \(p\)~est la longueur de~\(s\). Soit
\(\C{\fun{pre}_0}{n}\)\index{pre0@$\C{\fun{pre}_0}{n}$} le coût de
\(\fun{pre}_0(t)\), où \(n\)~est le nombre de n{\oe}uds internes
de~\(t\). D'après la définition de \fun{pre\(_0\)/1}, nous déduisons
\begin{equation*}
\C{\fun{pre}_0}{0} = 1;\quad
\C{\fun{pre}_0}{n+1} =
  1 + \C{\fun{pre}_0}{p} + \C{\fun{pre}_0}{n-p} + \C{\fun{cat}}{p},
\index{pre0@$\C{\fun{pre}_0}{n}$}
\end{equation*}
où \(p\)~est la taille de~\(t_1\). Donc \(\C{\fun{pre}_0}{n+1} =
\C{\fun{pre}_0}{p} + \C{\fun{pre}_0}{n-p} + p + 2\). Cette récurrence
appartient à une classe dite «~diviser pour
régner~»\index{conception!grands pas}\index{diviser pour
  régner|see{conception, grands pas}} parce qu'elle provient de
stratégies qui divisent la donnée (ici, de taille \(n+1\)), puis
résolvent séparément les parties ainsi obtenues (ici, de tailles~\(p\)
et \(n-p\)) et, finalement, combinent les solutions des parties en une
solution de la partition. Le coût supplémentaire dû à la composition
des solutions partielles (ici, \(p+2\)) est appelé le
\emph{péage}\index{péage} (anglais, \emph{toll}) et l'existence d'une
forme close pour la solution, ainsi que son comportement asymptotique,
dépendent de manière cruciale de la sorte de péage.

Dans d'autres contextes (voir page~\pageref{big-step}) et de manière
idiosyncratique, nous appelions cette stratégie \emph{conception à
  grands pas}\index{conception!grands pas} parce que nous voulions une
façon commode de la contraster avec une autre sorte de modélisation
que nous avons appelée \emph{conception à petits
  pas}\index{conception!petits pas}. Conséquemment, nous avons déjà
rencontré des cas de «~diviser pour régner~», par exemple, avec le tri
par interclassement\index{tri par interclassement} au
chapitre~\vref{chap:merge_sort}, qui souvent symbolise le concept
lui-même.

Le coût maximum
\(\W{\fun{pre}_0}{k}\)\index{pre0@$\W{\fun{pre}_0}{n}$} satisfait la
récurrence
\begin{equation}
\W{\fun{pre}_0}{0} = 1;\quad
\W{\fun{pre}_0}{k+1} =
  2 + \max_{0 \leqslant p \leqslant k}\{\W{\fun{pre}_0}{p}
                                  + \W{\fun{pre}_0}{k-p} + p\}.
\index{pre0@$\W{\fun{pre}_0}{n}$}
\label{eq:pre0_max}
\end{equation}
Au lieu d'attaquer frontalement ces équations, nous pouvons deviner
une solution possible et la vérifier. Ici, nous pourrions essayer de
choisir à chaque appel récursif \(p=k\), en suivant l'idée que
maximiser le péage à chaque n{\oe}ud de l'arbre conduira peut-être à
un total qui est maximum. Donc, nous envisageons
\begin{equation}
\W{\fun{pre}_0}{0} = 1;\quad
\W{\fun{pre}_0}{k+1} = \W{\fun{pre}_0}{k} + k + 3.
\label{eq:max_pre0_rec}
\end{equation}
En sommant les membres pour des valeurs croissantes de la variable
\(k=0\) à \(k=n-1\) et en simplifiant, nous obtenons
\begin{equation*}
\W{\fun{pre}_0}{n} = \frac{1}{2}(n^2 + 5n + 2) \sim \frac{1}{2}n^2.
\end{equation*}
Maintenant, nous vérifions si cette forme close satisfait
l'équation~\eqref{eq:pre0_max}. Nous avons \(2(\W{\fun{pre}_0}{p} +
\W{\fun{pre}_0}{n-p} + p + 2) = 2p^2 + 2(1-n)p + n^2 + 5n + 8\). Ceci
est l'équation d'une parabole dont le minimum se produit pour \(p =
(n-1)/2\) et dont le maximum pour \(p = n\), sur l'intervalle
\([0,n]\). Le maximum, dont la valeur est \(n^2 + 7n + 8\), égale \(2
\cdot \W{\fun{pre}_0}{n+1}\), donc la forme close satisfait l'équation
de récurrence.

À quoi ressemble un arbre binaire qui maximise le coût de
\fun{pre\(_0\)/1}\index{pre0@\fun{pre\(_0\)/1}}? Nous obtenons le
péage \(k+3\) dans~\eqref{eq:max_pre0_rec} parce que nous prenons le
coût maximum de \fun{cat/2}\index{cat@\fun{cat/2}} à chaque n{\oe}ud,
ce qui signifie que tous les n{\oe}uds internes qui sont concaténés
proviennent du sous-arbre gauche, du sous-arbre gauche du sous-arbre
gauche etc. donc ces n{\oe}uds sont concaténés encore et encore en
chemin vers la racine (c'est-à-dire, au retour de chaque appel
récursif), entraînant un coût quadratique. La forme d'un tel arbre est
montrée à la \fig~\vref{fig:max_pre0}.
\begin{figure}
\centering
\subfloat[Maximum\label{fig:max_pre0}]{
  \includegraphics{max_pre0}}
\qquad
\subfloat[Minimum\label{fig:min_pre0}]{
  \includegraphics{min_pre0}}
\caption{Arbres extrêmes pour \(\C{\fun{pre}_0}{n}\)}
\label{fig:tree_stack}
\end{figure}

Par dualité, le coût minimum
\(\B{\fun{pre}_0}{k}\)\index{pre0@$\B{\fun{pre}_0}{n}$} satisfait la
récurrence
\begin{equation*}
\B{\fun{pre}_0}{0} = 1;\quad
\B{\fun{pre}_0}{k+1} =
  2 + \min_{0 \leqslant p \leqslant k}\{\B{\fun{pre}_0}{p}
                                  + \B{\fun{pre}_0}{k-p} + p\}.
\end{equation*}
En reprenant le raisonnement précédent, mais dans la direction
opposée, nous essayons de minimiser le péage en choisissant \(p=0\),
ce qui signifie que tous les n{\oe}uds externes, sauf un, sont des
sous-arbres gauches. En conséquence, nous avons
\begin{equation}
\B{\fun{pre}_0}{0} = 1;\quad
\B{\fun{pre}_0}{k+1} = \B{\fun{pre}_0}{k} + 3.
\label{eq:min_pre0_rec}
\end{equation}
En sommant les membres pour des valeurs croissantes de la variable de
\(k=0\) à \(k=n-1\) et en simplifiant, nous obtenons
\begin{equation*}
\B{\fun{pre}_0}{n} = 3n + 1 \sim 3n.
\end{equation*}
Il est aisé de s'assurer que ceci est en fait une solution
de~\eqref{eq:min_pre0_rec}. La forme de l'arbre correspondant est
montrée à la \fig~\vref{fig:min_pre0}. Notez que les deux arbres
extrêmes sont isomorphes à une pile (c'est-à-dire à l'arbre de syntaxe
abstraite\index{arbre!$\sim$ de syntaxe abstraite} d'une pile) et, en
tant que tels, il sont des \emph{arbres dégénérés}.\index{arbre
  binaire!$\sim$ dégénéré} Par ailleurs, le coût maximum de
\fun{pre\(_0\)/1}\index{pre0@\fun{pre\(_0\)/1}} est quadratique, ce
qui nécessite une amélioration.

Nous pouvons parvenir à une autre conception à grands
pas\index{conception!grands pas} en n'utilisant pas
\fun{cat/2}\index{cat@\fun{cat/2}} et en appelant à la place
\fun{flat/1}\index{stack!flattening}\index{flat@\fun{flat/1}}, définie
à la \fig~\vref{fig:flat}, \emph{une seule fois à la fin}. La
\fig~\vref{fig:pre1} montre qu'une nouvelle version du parcours
préfixe, appelée \fun{pre\(_1\)/1}\index{pre1@\fun{pre\(_1\)/1}}.
\begin{figure}[b]
\begin{equation*}
\boxed{
\begin{array}{r@{\;}c@{\;}l}
\fun{pre}_1(t) & \rightarrow & \fun{flat}(\fun{pre}_2(t)).\\
\\
\fun{pre}_2(\fun{ext}()) & \rightarrow & \el;\\
\fun{pre}_2(\fun{int}(x,t_1,t_2)) & \rightarrow &
  \cons{x,\fun{pre}_2(t_1)}{\fun{pre}_2(t_2)}.
\end{array}}
\end{equation*}
\caption{Parcours préfixe avec \fun{flat/1}}
\label{fig:pre1}
\end{figure}
Le coût de \(\fun{pre}_2(t)\)\index{pre2@\fun{pre\(_2\)/1}} est
\(2n+1\) (voir théorème~\ref{thm_int_ext} \vpageref{thm_int_ext}). À
la page~\pageref{cost_flat}, nous avons trouvé que le coût de
\(\fun{flat}(s)\)\index{flat@\fun{flat/1}} est \(1 + n + \Omega +
\Gamma + L\), où \(n\)~est la longueur de
\(\fun{flat}(s)\)\index{flat@\fun{flat/1}}, \(\Omega\)~est le nombre
de pile vides dans~\(s\), \(\Gamma\)~est le nombre de piles non-vides
et \(L\)~est la somme des longueurs des piles imbriquées. La valeur
de~\(\Omega\) est \(n+1\) car c'est le nombre de n{\oe}uds
externes. La valeur de~\(\Gamma\) est \(n-1\), parce que chaque
n{\oe}ud interne donne lieu à une pile non-vide via la deuxième règle
de \fun{pre\(_2\)/1}\index{pre2@\fun{pre\(_2\)/1}} et la racine est
exclue car nous ne comptons que les piles imbriquées. La valeur
de~\(L\) est \(3(n-1)\) car ces piles ont pour longueur~\(3\), par la
même règle. Au total, \(\Call{\fun{flat}(s)} = 6n - 2\), où
\(\fun{pre}_2(t) \twoheadrightarrow s\) et \(n\)~est la taille
de~\(t\). Finalement, nous devons tenir compte de la règle définissant
\fun{pre\(_1\)/1}\index{pre1@\fun{pre\(_1\)/1}} et évaluer à nouveau
le coût dû à l'arbre vide:
\begin{equation*}
\C{\fun{pre}_1}{0} = 3;\index{pre1@$\C{\fun{pre}_1}{n}$}
\quad
\C{\fun{pre}_1}{n} = 1 + (2n+1) + (6n - 2) = 8n,\;
\text{si \(n > 0\).}
\end{equation*}

Malgré une amélioration significative du coût et l'absence de cas
extrêmes, nous devrions tenter une conception à petits
pas\index{conception!petits pas} avant d'abandonner la partie. Le
principe qui soutend ce type d'approche est d'en faire le moins
possible dans chaque règle. Il est clair que la racine est
correctement placée par
\fun{pre\(_0\)/1}\index{pre0@\fun{pre\(_0\)/1}}, mais sans
\(\fun{pre}_3(t_1)\)\index{pre3@\fun{pre\(_3\)/1}} et
\(\fun{pre}_3(t_2)\) dans le schéma suivant, que pouvons-nous faire?
\begin{equation*}
\fun{pre}_3(\fun{ext}()) \rightarrow \el;
\quad
\fun{pre}_3(\fun{int}(x,t_1,t_2)) \rightarrow
  \cons{x}{\fbcode{CCCCCCC}\,}.
\end{equation*}
Il faut alors penser en termes de forêts\index{arbre!forêt}, au lieu
d'arbres isolés, parce qu'une forêt est une pile d'arbres et, en tant
que telle, elle peut être utilisée aussi pour accumuler des
arbres.\index{langage fonctionnel!accumulateur} Ceci est une
technique fréquente lorsque l'on calcule avec des arbres. Voir
la \fig~\vref{fig:pre3}.
\begin{figure}
\begin{equation*}
\boxed{
\begin{array}{r@{\;}c@{\;}l}
\fun{pre}_3(t) & \xrightarrow{\smash{\alpha}} & \fun{pre}_4([t]).\\
\\
\fun{pre}_4(\el) & \xrightarrow{\smash{\beta}} & \el;\\
\fun{pre}_4(\cons{\fun{ext}()}{f})
  & \xrightarrow{\smash{\gamma}} & \fun{pre}_4(f);\\
\fun{pre}_4(\cons{\fun{int}(x,t_1,t_2)}{f})
  & \xrightarrow{\smash{\delta}} &
  \cons{x}{\fun{pre}_4(\cons{t_1,t_2}{f})}.
\end{array}}
\end{equation*}
\caption{Parcours préfixe efficace avec une forêt}
\label{fig:pre3}
\end{figure}
Les arbres vides de la forêt sont ignorés par la
règle~\(\gamma\). Dans la règle~\(\delta\), les sous-arbres
\(t_1\)~et~\(t_2\) sont maintenant simplement empilés à nouveau sur la
forêt~\(f\), pour être examinés ultérieurement. De cette manière, il
n'y a pas besoin de calculer
\(\fun{pre}_4(t_1)\)\index{pre4@\fun{pre\(_4\)/1}} ou
\(\fun{pre}_4(t_2)\) immédiatement. Cette méthode est légèrement
différente de celle qui use d'un accumulateur qui contient, à chaque
instant, un résultat partiel ou un résultat partiel retourné. Ici,
aucun paramètre n'est ajouté, mais, une pile remplace l'arbre originel
dans lequel on choisit des n{\oe}uds internes facilement (la racine du
premier arbre) dans l'ordre attendu. Nous donnons un exemple à la \fig~\vref{fig:pre3_abcde},
\begin{figure}
\centering
\includegraphics[bb=71 650 374 723]{pre3_abcde_0}
\includegraphics[bb=71 668 306 721]{pre3_abcde_1}
\includegraphics{pre3_abcde_2}
\caption{Parcours préfixe avec \fun{pre\(_4\)/1}}
\label{fig:pre3_abcde}
\end{figure}
où la forêt est l'argument de \fun{pre\(_4\)/1} et les n{\oe}uds
cerclés sont la valeur courante de~\(x\) dans la règle~\(\delta\) de
la \fig~\ref{fig:pre3}. Le coût de
\(\fun{pre}_3(t)\)\index{pre3@\fun{pre\(_3\)/1}}, où \(n\)~est la taille
de~\(t\), est facilement calculable:
\begin{itemize}

  \item la règle~\(\alpha\) est utilisée une fois;

  \item la règle~\(\beta\) est utilisée une fois;

  \item la règle~\(\gamma\) est utilisée une fois pour chaque n{\oe}ud
    externe, c'est-à-dire \(n+1\) fois (voir le
    théorème~\ref{thm_int_ext} \vpageref{thm_int_ext});

  \item la règle~\(\delta\) est utilisée une fois pour chaque n{\oe}ud
    interne, donc \(n\)~fois, par définition.

\end{itemize}
Au total, \(\C{\fun{pre}_3}{n} = 2n + 3 \sim
2n\)\index{pre3@$\C{\fun{pre}}{n}$}, ce qui est une amélioration
notable. Le lecteur attentif pourrait remarquer que nous pourrions
réduire davantage le coût en ne visitant pas les n{\oe}uds externes,
comme montré à la \fig~\ref{fig:pre5}.
\begin{figure}[b]
\begin{equation*}
\boxed{
\begin{array}{r@{\;}c@{\;}l}
\fun{pre}_5(t) & \rightarrow & \fun{pre}_6([t]).\\
\\
\fun{pre}_6(\el) & \rightarrow & \el;\\
\fun{pre}_6(\cons{\fun{ext}()}{f})
  & \rightarrow & \fun{pre}_6(f);\\
\fun{pre}_6(\cons{\fun{int}(x,\fun{ext}(),\fun{ext}())}{f})
  & \rightarrow & \cons{x}{\fun{pre}_6(f)};\\
\fun{pre}_6(\cons{\fun{int}(x,\fun{ext}(),t_2)}{f})
  & \rightarrow & \cons{x}{\fun{pre}_6(\cons{t_2}{f})};\\
\fun{pre}_6(\cons{\fun{int}(x, t_1, \fun{ext}())}{f})
  & \rightarrow & \cons{x}{\fun{pre}_6(\cons{t_1}{f})};\\
\fun{pre}_6(\cons{\fun{int}(x,t_1,t_2)}{f})
  & \rightarrow & \cons{x}{\fun{pre}_6(\cons{t_1,t_2}{f})}.
\end{array}}
\end{equation*}
\caption{Longue définition d'un parcours préfixe}
\label{fig:pre5}
\end{figure}
Nous avons alors \(\C{\fun{pre}_5}{n} = \C{\fun{pre}_3}{n} - (n+1) = n
+
2\).\index{pre6@\fun{pre\(_6\)/1}}\index{pre5@\fun{pre\(_5\)/1}}\index{pre5@$\C{\fun{pre}_5}{n}$}
Malgré le gain, le nouveau programme est significativement plus long
et les membres droits sont des évaluations partielles\index{langage
  fonctionnel!évaluation!$\sim$ partielle} de la règle~\(\delta\). La
mesure de la donnée que nous utilisons pour évaluer les coûts n'inclut
pas le temps abstrait pour sélectionner la règle à appliquer, mais il
est probable que, plus il y a de motifs, plus cette pénalité cachée
est élevée. Dans ce livre, nous préférons les programmes concis et
donc visiter les n{\oe}uds externes, à moins qu'il n'y ait une raison
logique de ne pas le faire.

Au total, le nombre de n{\oe}uds d'empilage\index{n{\oe}ud d'empilage}
créés par les règles~\(\alpha\) et~\(\delta\) est le nombre total de
n{\oe}uds, soit~\(2n+1\), mais si nous voulons savoir combien il peut
y avoir de n{\oe}uds à un moment donné, il faut considérer comment la
forme de l'arbre originel influe sur les règles~\(\gamma\)
et~\(\delta\). Dans le meilleur des cas, \(t_1\)~dans~\(\delta\) est
\(\fun{ext}()\) et sera éliminé à la réécriture suivante par la
règle~\(\gamma\) sans création de n{\oe}uds supplémentaires. Dans le
pire des cas, \(t_1\)~maximise le nombre de n{\oe}uds internes de sa
branche gauche. Par conséquent, ces deux configurations correspondent
aux cas extrêmes pour le coût de
\fun{pre\(_0\)/1}\index{pre0@\fun{pre\(_0\)/1}} à la
\fig~\vref{fig:tree_stack}. Dans le pire des cas, tous les \(2n+1\)
n{\oe}uds de l'arbre se trouveront dans la pile à un moment donné,
alors que, dans le meilleur des cas, seulement deux s'y trouveront. La
question de la taille moyenne de la pile sera étudiée plus loin, en
relation avec la hauteur moyenne.

La différence entre conception à grands pas\index{conception!grands
  pas} (ou «~diviser pour régner~»\index{diviser pour régner}) et à
petits pas\index{conception!petits pas} n'est pas toujours nette et
est principalement un moyen didactique. En particulier, nous ne
devrions pas supposer qu'il n'y a que deux types possibles de
modélisation pour chaque tâche donnée. Pour clarifier davantage ce
sujet, utilisons une approche hétérogène pour concevoir une autre
version, \fun{pre/1}\index{pre@\fun{pre/1}}, qui calcule efficacement
le parcours préfixe d'un arbre binaire. En considérant
\fun{pre\(_0\)/1}\index{pre0@\fun{pre\(_0\)/1}}, nous pouvons
identifier la source de l'inefficacité dans le fait que, dans le pire
des cas,
\begin{equation*}
\fun{pre}_0(t) \twoheadrightarrow
\cons{x_1}{\fun{cat}(\cons{x_2}{\fun{cat}(\dots
    \fun{cat}(\cons{x_n}{\fun{cat}(\el, \el)},\el) \dots}}
\index{cat@\fun{cat/2}}
\end{equation*}
où \(t = \fun{int}(x_1,\fun{int}(x_2,\dots, \fun{int}(x_n,
\fun{ext}(), \fun{ext}()), \dots, \fun{ext}()),\fun{ext}())\) est
l'arbre de la \fig~\vref{fig:max_pre0}. Nous avons rencontré ce genre
de réécriture partielle dans la
formule~\eqref{eq:rev0}\index{rev0@\fun{rev\(_0\)/1}}
\vpageref{eq:rev0} et~\eqref{inv_isrt}
\vpageref{inv_isrt}\index{isrt@\fun{isrt/1}}\index{ins@\fun{ins/2}},
et nous avons trouvé qu'elle mène à un coût quadratique. Bien que
l'emploi de \fun{cat/2}\index{cat@\fun{cat/2}} en lui-même n'est pas
problématique, mais, plutôt, l'accumulation d'appels à
\fun{cat/2}\index{cat@\fun{cat/2}} dans le premier argument, cherchons
néanmoins une définition qui n'utilise pas du tout la concaténation.
Cela signifie que nous voulons construire le parcours préfixe
exclusivement à l'aide d'empilages. Par conséquent, nous devons
ajouter un paramètre auxiliaire, égal à la pile vide au départ, sur
lequel le contenu des n{\oe}uds est empilé dans l'ordre attendu:
\(\fun{pre}(t) \rightarrow \fun{pre}(t,\el)\).\index{pre@\fun{pre/2}}
Maintenant, nous devrions nous demander quelle est l'interprétation de
cet accumulateur\index{langage fonctionnel!accumulateur} en
considérant le motif \(\fun{pre}(t,s)\). Examinons le n{\oe}ud interne
\(t=\fun{int}(x,t_1,t_2)\) à la \fig~\vref{fig:pre_tree}.
\begin{figure}
\centering
\subfloat[Préfixe avec \fun{pre/1}\label{fig:pre_tree}]{
  \includegraphics[bb=71 672 166 721]{pre_tree}}
\subfloat[Infixe avec \fun{in/1}\label{fig:in_tree}]{
  \includegraphics[bb=71 672 183 721]{in_tree}}
\subfloat[Postfixe avec \fun{post/1}\label{fig:post_tree}]{
  \includegraphics[bb=71 672 183 721]{post_tree}}
\caption{Parcours classiques efficaces}
\label{fig:classic_walks}
\end{figure}
Les flèches rendent le parcours dans l'arbre plus évident et
connectent les différentes étapes de la pile en préfixe: une flèche
vers le bas pointe l'argument d'un appel récursif sur l'enfant
correspondant; une flèche vers le haut pointe le résultat d'un appel
sur le parent. Par exemple, le sous-arbre~\(t_2\) correspond à l'appel
récursif \(\fun{pre}(t_2,s)\)\index{pre@\fun{pre/2}} dont la valeur
est~\(s_1\).  De même, nous avons \(\fun{pre}(t_1,s_1)
\twoheadrightarrow s_2\), ce qui équivaut donc à
\(\fun{pre}(t_1,\fun{pre}(t_2,s)) \twoheadrightarrow
s_2\). Finalement, la racine est associée à l'évaluation
\(\fun{pre}(t,s) \twoheadrightarrow \cons{x}{s_2}\), c'est-à-dire
\(\fun{pre}(t,s) \equiv
\cons{x}{\fun{pre}(t_1,\fun{pre}(t_2,s))}\).\index{pre@\fun{pre/2}} La
règle pour les n{\oe}uds externes n'est pas montrée et consiste
simplement à laisser la pile invariante. Nous pouvons finalement
rédiger le programme fonctionnel à la \fig~\vref{fig:pre}.
\begin{figure}[b]
\begin{equation*}
\boxed{
\begin{array}{r@{\;}c@{\;}l}
\fun{pre}(t) & \xrightarrow{\smash{\theta}} & \fun{pre}(t,\el).\\
\\
\fun{pre}(\fun{ext}(),s) & \xrightarrow{\smash{\iota}} & s;\\
\fun{pre}(\fun{int}(x,t_1,t_2),s)
  & \xrightarrow{\smash{\kappa}}
  & \cons{x}{\fun{pre}(t_1,\fun{pre}(t_2,s))}.
\end{array}}
\end{equation*}
\caption{Parcours préfixe efficace (coût et mémoire)}
\label{fig:pre}
\end{figure}
Nous comprenons à présent qu'étant donné \(\fun{pre}(t,s)\), les
n{\oe}uds dans la pile~\(s\) sont les n{\oe}uds qui suivent, en
préfixe, les n{\oe}uds dans le sous-arbre~\(t\). Le coût est
facilement déterminé:
\begin{equation}
\C{\fun{pre}}{n} = 1 + (n+1) + n = 2n+2.\label{eq:pre}
\index{pre@$\C{\fun{pre}}{n}$}
\end{equation}
(Il y a \(n+1\) n{\oe}uds externes quand il y a \(n\)~n{\oe}uds
internes.)

Nous devrions privilégier cette variante, plutôt que
\fun{pre\(_3\)/1}\index{pre3@\fun{pre\(_3\)/1}}, parce qu'elle a
besoin de moins de mémoire\index{mémoire}: la règle~\(\delta\) à la
\fig~\vref{fig:pre3} empile \(t_1\) et~\(t_2\), donc alloue deux
n{\oe}uds d'empilage\index{n{\oe}ud d'empilage} pour chaque n{\oe}ud
interne, faisant un total de \(2n\)~n{\oe}uds supplémentaires. Par
contraste, \fun{pre/1}\index{pre@\fun{pre/1}} n'en créé aucun, mais
alloue \(n\)~n{\oe}uds d'appel à \fun{pre/2}\index{pre@\fun{pre/2}}
(un pour chaque n{\oe}ud interne), donc l'avantage tient, même s'il
est nuancé. Notons en passant que
\fun{pre\(_3\)/1}\index{pre3@\fun{pre\(_3\)/1}} est en forme
terminale\index{langage fonctionnel!forme terminale}, mais pas
\fun{pre/2}.

\paragraph{Numérotation préfixe}

La \fig~\vref{fig:preorder}
\begin{figure}
\centering
\subfloat[Préfixe\label{fig:preorder}]{\includegraphics{preorder}}
\;
\subfloat[Postfixe\label{fig:postorder}]{\includegraphics{postorder}}
\;
\subfloat[Infixe\label{fig:inorder}]{\includegraphics{inorder}}
\;
\subfloat[Par niveau\label{fig:lorder}]{\includegraphics{lorder}}
\caption{Numérotations classiques d'un arbre binaire taillé}
\label{fig:orders}
\end{figure}
montre un arbre binaire dont les n{\oe}uds internes ont été remplacés
par leur rang dans le parcours préfixe, \(0\)~étant le plus petit
rang, assigné à la racine. En particulier, le parcours préfixe de cet
arbre est \([0,1,2,3,4,5,6]\). Ce type d'arbre est une
\emph{numérotation préfixe}\index{arbre binaire!numérotation préfixe}
d'un arbre donné. Un exemple complet est montré à la
\fig~\vref{fig:bt_ex3},
\begin{figure}[b]
\centering
\includegraphics{bt_ex3}
\caption{Rangs préfixes en exposant}
\label{fig:bt_ex3}
\end{figure}
où les rangs préfixes sont les exposants des n{\oe}uds
internes. Remarquons comment ces nombres augmentent le long de chemins
descendants.

Nous produisons cette numérotation (des rangs) en deux temps: d'abord,
nous avons besoin de comprendre comment calculer le rang correct pour
un n{\oe}ud donné; ensuite, nous devons utiliser les rangs pour
construire un arbre. Le plan pour la première partie est montré à
l'{\oe}uvre sur les n{\oe}uds internes à la \fig~\vref{fig:prenum1}.
\begin{figure}[t]
\centering
\subfloat[Rangs seuls\label{fig:prenum1}]{%
\includegraphics[bb=71 668 170 721]{prenum1}}
\qquad
\subfloat[Rangs et arbre\label{fig:prenum0}]{%
\includegraphics[bb=71 668 217 721]{prenum0}}
\caption{Numérotation préfixe en deux temps}
\label{fig:prenum}
\end{figure}
Un nombre à la gauche d'un n{\oe}ud est son rang, par exemple,
\(i\)~est le range du n{\oe}ud~\(x\): ces rangs descendent dans
l'arbre. Un nombre à la droite d'un n{\oe}ud est le plus petit rang
qui n'est pas utilisé dans la numérotation des sous-arbres de ce
n{\oe}ud: ces rangs montent et peuvent être employés pour la
numérotation d'un autre sous-arbre. Par exemple, \(j\)~est le plus
petit rang qui ne numérote pas les n{\oe}uds du
sous-arbre~\(t_1\). Les n{\oe}uds externes ne changent pas le rang qui
leur parvient et ne sont pas montrés.

La seconde et dernière étape consiste à construire l'arbre contenant
la numérotation préfixe, et elle schématisée à la
\fig~\vref{fig:prenum0}. Conceptuellement, il s'agit-là de la
complétion de la première phase en ce sens que les flèches
ascendantes, qui dénotent des valeurs des appels récursifs sur des
sous-arbres, maintenant pointent des paires dont la première
composante est le rang trouvé à la première phase et la seconde
composante est un arbre numéroté. Comme d'habitude avec les
définitions récursives, nous supposons que les appels récursifs sur
les sous-structures sont corrects (c'est-à-dire qu'ils sont réécrits
en la valeur attendue) et nous inférons la valeur de l'appel sur la
totalité de la structure courante, ici
\(\pair{k}{\fun{int}(i,t'_1,t'_2)}\).

La fonction \fun{npre/1}\index{npre@\fun{npre/1}} (anglais,
\emph{number in preorder}) à la \fig~\vref{fig:npre}
\begin{figure}[t]
\abovedisplayskip=0pt
\belowdisplayskip=0pt
\centering
\framebox[0.75\columnwidth]{\vbox{%
\begin{gather*}
\inferrule
  {\fun{npre}(0,t) \twoheadrightarrow \pair{i}{t'}}
  {\fun{npre}(t) \twoheadrightarrow t'}\,.
\qquad
\fun{npre}(i,\fun{ext}()) \rightarrow \pair{i}{\fun{ext}()};
\\
\inferrule
  {\fun{npre}(i+1,t_1) \twoheadrightarrow \pair{j}{t'_1}
   \and
   \fun{npre}(j,t_2) \twoheadrightarrow \pair{k}{t'_2}}
  {\fun{npre}(i,\fun{int}(x,t_1,t_2)) \twoheadrightarrow
   \pair{k}{\fun{int}(i,t'_1,t'_2)}}\,.
\end{gather*}
}}
\caption{Numérotation préfixe}
\label{fig:npre}
\end{figure}
réalise cet algorithme. Nous utilisons une fonction auxiliaire
\fun{npre/2}\index{npre@\fun{npre/2}} telle que \(\fun{npre}(i,t)
\twoheadrightarrow \pair{j}{t'}\), où \(t'\)~est la numérotation
préfixe de~\(t\), de racine~\(i\), et \(j\)~est le plus petit rang qui
manque dans~\(t'\) (en d'autres termes, \(j-i\) est la taille de~\(t\)
et~\(t'\)). Cette fonction est celle illustrée à la
\fig~\vref{fig:prenum0}.

En passant, nous devrions peut-être nous souvenir que les systèmes
d'inférence\index{induction!système d'inférence}, vus d'abord
\vpageref{par:infsys}, peuvent être éliminés en introduisant des
fonctions auxiliaires (une par prémisse). Dans le cas présent, nous
écririons\index{snd@\fun{snd/1}} le programme équivalent de la
\fig~\vref{fig:npre_bis}.
\begin{figure}
\begin{equation*}
%\belowdisplayskip=0pt
\boxed{
\begin{array}{r@{\;}c@{\;}l}
\fun{npre}(t) & \rightarrow & \fun{snd}(\fun{npre}(0,t)).\\
\\
\fun{snd}(\pair{x}{y}) & \rightarrow & y.\\
\\
\fun{npre}(i,\fun{ext}()) & \rightarrow & \pair{i}{\fun{ext}()};\\
\fun{npre}(i,\fun{int}(x,t_1,t_2)) & \rightarrow &
\fun{t}_1(\fun{npre}(i+1,t_1),i,t_2).\\
\\
\fun{t}_1(\pair{j}{t'_1},i,t_2) & \rightarrow &
\fun{t}_2(\fun{npre}(j,t_2),i,t'_1).\\
\\
\fun{t}_2(\pair{k}{t'_2},i,t'_1) & \rightarrow & \pair{k}{\fun{int}(i,t'_1,t'_2)}.
\end{array}
}
\end{equation*}
\caption{Version de \fun{npre/1} sans système d'inférence}
\label{fig:npre_bis}
\end{figure}
\index{arbre binaire!préfixe|)}

\paragraph{Terminaison}
\index{terminaison!parcours préfixe|(}
\index{arbre binaire!préfixe!terminaison|(}

Il est facile de démontrer la terminaison de
\fun{pre/2}\index{pre@\fun{pre/2}} car la technique que nous avons
utilisée pour établier la terminaison de la fonction de Ackermann à la
page~\pageref{par:ackermann} est pertinente ici aussi. Nous
définissons un ordre lexicographique\index{induction!ordre
  lexicographique} sur les appels à \fun{pre/2}\index{pre@\fun{pre/2}}
(paires de dépendance\index{terminaison!paire de dépendance}):
\begin{equation}
\fun{pre}(t,s) \succ \fun{pre}(t',s') :\Leftrightarrow \text{\(t
  \succ_{B} t'\) ou (\(t = t'\) et \(s \succ_{S} s'\))},
\label{eq:BS_order}
\end{equation}
où \(B\)~est l'ensemble des arbres binaires, \(S\)~est l'ensemble de
toutes les piles, \(t \succ_{B} t'\) signifie que l'arbre~\(t'\) est
un sous-arbre immédiat\index{arbre!sous-arbre!$\sim$ immédiat}
de~\(t\), et \(s \succ_{S} s'\) veut dire que la pile \(s'\)~est la
sous-pile immédiate\index{induction!ordre de la sous-pile immédiate}
de~\(s\) (page~\pageref{par:well-founded}). D'après la définition à la
\fig~\vref{fig:pre}, nous voyons que la règle~\(\theta\) conserve la
terminaison si \fun{pre/2}\index{pre@\fun{pre/2}} termine; la
règle~\(\iota\) termine trivialement; finalement, la règle~\(\kappa\)
réécrit un appel en des appels plus petits selon \((\succ)\):
\(\fun{pre}(\fun{int}(x,t_1,t_2),s) \succ \fun{pre}(t_2,s)\) et
\(\fun{pre}(\fun{int}(x,t_1,t_2),s) \succ \fun{pre}(t_1,u)\), pour
tout~\(u\), en particulier si \(\fun{pre}(t_2,s) \twoheadrightarrow
u\). En conséquence, \fun{pre/1}\index{pre@\fun{pre/1}} termine pour
toutes les données.\index{arbre
  binaire!préfixe!terminaison|)}\index{terminaison!parcours
  préfixe|)}\hfill\(\Box\)

\paragraph{Équivalence}

Pour voir comment l'induction structurelle peut être utilisée pour
prouver des propriétés portant sur des arbres binaires, nous
considérerons une proposition simple que nous avons énoncée tantôt, et
que nous exprimons ici comme \(\pred{Pre}{t} \colon \fun{pre}_0(t)
\equiv
\fun{pre}(t)\).\index{Pre@\predName{Pre}}\index{pre0@\fun{pre\(_0\)/1}}\index{pre@\fun{pre/1}}
Nous devons donc établir
\begin{itemize}

  \item la base \(\pred{Pre}{\fun{ext}()}\);

  \item le pas inductif \(\forall t_1.\pred{Pre}{t_1} \Rightarrow
    \forall t_2.\pred{Pre}{t_2} \Rightarrow \! \forall
    x.\pred{Pre}{\fun{int}(x,t_1,t_2)}\).

\end{itemize}
La base est aisée car \(\fun{pre}_0(\fun{ext}())
\xrightarrow{\smash{\gamma}} \el \xleftarrow{\smash{\iota}}
\fun{pre}(\fun{ext}(),\el) \xleftarrow{\smash{\theta}}
\fun{pre}(\fun{ext}())\). Voir la définition de \fun{pre\(_0\)/1} à
l'équation~\eqref{eq:pre0} \vpageref{eq:pre0}, et celle de
\fun{cat/2}\index{cat@\fun{cat/2}}:
\begin{equation*}
\fun{cat}(\el,t) \xrightarrow{\smash{\alpha}} t;
\qquad
\fun{cat}(\cons{x}{s},t) \xrightarrow{\smash{\beta}}
\cons{x}{\fun{cat}(s,t)}.
\end{equation*}
Dans le but de découvrir comment utiliser les deux hypothèses
d'induction \(\pred{Pre}{t_1}\) et
\(\pred{Pre}{t_2}\)\index{Pre@\predName{Pre}}, commençons avec un côté
de l'équivalence que nous souhaitons établir, par exemple le membre
gauche, et réécrivons-le jusqu'à ce que nous atteignions l'autre
membre ou bien nous soyons arrêtés. Posons \(t :=
\fun{int}(x,t_1,t_2)\), alors
\begin{equation*}
\begin{array}{@{}r@{\;}c@{\;}l@{\qquad}r@{}}
\fun{pre}_0(t)
& = & \fun{pre}_0(\fun{int}(x,t_1,t_2))\\
& \xrightarrow{\smash{\delta}}
& \cons{x}{\fun{cat}(\fun{pre}_0(t_1), \fun{pre}_0(t_2))}\\
& \equiv & \cons{x}{\fun{cat}(\fun{pre}(t_1), \fun{pre}_0(t_2))}
& (\pred{Pre}{t_1})\\
& \equiv & \cons{x}{\fun{cat}(\fun{pre}(t_1), \fun{pre}(t_2))}
& (\pred{Pre}{t_2})\\
& \xrightarrow{\smash{\theta}}
& \cons{x}{\fun{cat}(\fun{pre}(t_1,\el), \fun{pre}(t_2))}\\
& \xrightarrow{\smash{\theta}}
& \cons{x}{\fun{cat}(\fun{pre}(t_1,\el), \fun{pre}(t_2,\el))}.
\end{array}
\end{equation*}
À l'arrêt, nous réécrivons alors le membre opposé, le plus loin
possible: \(\fun{pre}(t) = \fun{pre}(\fun{int}(x,t_1,t_2))
\xrightarrow{\smash{\theta}} \fun{pre}(\fun{int}(x,t_1,t_2),\el)
\xrightarrow{\smash{\kappa}}
\cons{x}{\fun{pre}(t_1,\fun{pre}(t_2,\el))}\)\index{pre@\fun{pre/1}}\index{pre@\fun{pre/2}}. La
comparaison des deux expressions sur lesquelles nous nous sommes
arrêtés suggère un sous-but à atteindre.

Soit \(\pred{CatPre}{t,s} \colon \fun{cat}(\fun{pre}(t,\el),s) \equiv
\fun{pre}(t,s)\)\index{CatPre@\predName{CatPre}}\index{cat@\fun{cat/2}}\index{pre@\fun{pre/2}}. Quand
un prédicat dépend de deux paramètres, différentes possibilités
s'ouvrent à nous: soit nous usons d'un ordre lexicographique, soit une
simple induction sur l'une seulement des variables suffit. Il est
généralement préférable d'utiliser un ordre
lexicographique\index{induction!ordre lexicographique} sur les paires
de paramètres et, si nous nous rendons compte plus tard qu'une seule
composante est réellement nécessaire, nous pouvons toujours réécrire
la preuve avec une induction portant sur cette unique
composante. Définissons
\begin{equation*}
(t,s) \succ_{B \times S} (t',s') :\Leftrightarrow \text{\(t \succ_{B}
    t'\) ou (\(t = t'\) et \(s \succ_{S} s'\))}.
\end{equation*}
Conceptuellement, il s'agit du même ordre que celui sur les appels à
\fun{pre/1}\index{pre@\fun{pre/1}} dans la
définition~\eqref{eq:BS_order}. Si nous réalisons plus tard que des
relations de sous-termes immédiats sont trop restrictives, nous
choisirions ici des relations de sous-termes générales, ce qui
signifie, dans le cas des arbres binaires, qu'un arbre est un
sous-arbre d'un autre. L'élément minimum de l'ordre lexicographique
que nous venons de définir est \((\fun{ext}(),\el)\). Le principe
d'induction bien fondée requiert alors que nous établissions
\begin{itemize}

  \item la base \(\pred{CatPre}{\fun{ext}(),\el}\);

  \item \(\forall t,s. (\forall t',s'.(t,s) \succ_{B \times S} (t',s')
    \Rightarrow \pred{CatPre}{t',s'}) \Rightarrow
    \pred{CatPre}{t,s}\).

\end{itemize}
La base est aisée: \(\fun{cat}(\fun{pre}(\fun{ext}(),\el),\el)
\xrightarrow{\smash{\iota}} \fun{cat}(\el,\el)
\xrightarrow{\smash{\alpha}} \el \xleftarrow{\smash{\iota}}
\fun{pre}(\fun{ext}(),\el)\).\index{cat@\fun{cat/2}} Nous supposons
alors \(\forall t',s'.(t,s) \succ_{B \times S} (t',s') \Rightarrow
\pred{CatPre}{t',s'}\), ce qui est l'hypothèse d'induction, et nous
poursuivons en réécrivant le membre gauche après avoir posé \(t :=
\fun{int}(x,t_1,t_2)\):
%. Le résultat est montré à la
%\fig~\vref{fig:CatPre},\index{CatPre@\predName{CatPre}}
%\begin{figure}[t]
\begin{equation*}
%\boxed{%
\begin{array}{r@{\;}c@{\;}l}
  \fun{cat}(\fun{pre}(t,\el),s)
& =
& \fun{cat}(\fun{pre}(\fun{int}(x,t_1,t_2),\el),s)\\
& \xrightarrow{\smash{\kappa}}
& \fun{cat}(\cons{x}{\fun{pre}(t_1,\fun{pre}(t_2,\el))},s)\\
& \xrightarrow{\smash{\beta}}
& \cons{x}{\fun{cat}(\fun{pre}(t_1,\fun{pre}(t_2,\el)),s)}\\
& \equiv_0
& \cons{x}{\fun{cat}(\fun{cat}(\fun{pre}(t_1,\el),\fun{pre}(t_2,\el)),s)}\\
& \equiv_1
& \cons{x}{\fun{cat}(\fun{pre}(t_1,\el),\fun{cat}(\fun{pre}(t_2,\el),s))}\\
& \equiv_2
& \cons{x}{\fun{cat}(\fun{pre}(t_1,\el),\fun{pre}(t_2,s))}\\
& \equiv_3
& \cons{x}{\fun{pre}(t_1,\fun{pre}(t_2,s))}\\
& \xleftarrow{\smash{\kappa}}
& \fun{pre}(\fun{int}(x,t_1,t_2),s)\\
& =
& \fun{pre}(t,s).\hfill\Box
\end{array}
%}
\end{equation*}
%\caption{Preuve de \(\pred{CatPre}{t} \colon
%  \fun{cat}(\fun{pre}(t,\el),s) \equiv
%  \fun{pre}(t,s)\)\label{fig:CatPre}}
%\end{figure}
où
\begin{itemize}

  \item (\(\equiv_0\)) est l'instance
  \(\pred{CatPre}{t_1,\fun{pre}(t_2,\el)}\) de l'hypothèse d'induction
  car \((t,s) \succ_{B \times S} (t_1,s')\), pour toutes les
  piles~\(s'\), en particulier si \(\fun{pre}(t_2,\el)
  \twoheadrightarrow s'\);\index{pre@\fun{pre/2}}

  \item (\(\equiv_1\)) est une application de l'associativité de la
  concaténation de piles\index{pile!concaténation}
  (page~\pageref{proof_assoc_cat}), à savoir
  \index{pre@\fun{pre/2}}\index{CatAssoc@\predName{CatAssoc}}
  \(\pred{CatAssoc}{\fun{pre}(t_1,\el),\fun{pre}(t_2,\el),s}\);

  \item (\(\equiv_2\)) est l'instance
  \(\pred{CatPre}{t_2,s}\)\index{CatPre@\predName{CatPre}} de
  l'hypothèse d'induction car nous avons \((t,s) \succ_{B \times S}
  (t_2,s)\);

  \item (\(\equiv_3\)) est l'instance \(\pred{CatPre}{t_1,
    \fun{pre}(t_2,s)}\) de l'hypothèse d'induction car \((t,s)
  \succ_{B \times S} (t_1,s')\), pour toutes les piles~\(s'\), en
  particulier si \(s' = \fun{pre}(t_2,s)\).\index{pre@\fun{pre/2}}
\end{itemize}
Nous pouvons maintenant conclure:
%We can resume conclusively in \fig~\vref{fig:pre0_pre}.\hfill\(\Box\)
%\begin{figure}
\begin{equation*}
%\boxed{%
\begin{array}{r@{\;}c@{\;}l@{\qquad}r}
\fun{pre}_0(t)
& = & \fun{pre}_0(\fun{int}(x,t_1,t_2))\\
& \xrightarrow{\smash{\delta}}
& \cons{x}{\fun{cat}(\fun{pre}_0(t_1), \fun{pre}_0(t_2))}\\
& \equiv & \cons{x}{\fun{cat}(\fun{pre}(t_1), \fun{pre}_0(t_2))}
& (\pred{Pre}{t_1})\\
& \equiv & \cons{x}{\fun{cat}(\fun{pre}(t_1), \fun{pre}(t_2))}
& (\pred{Pre}{t_2})\\
& \xrightarrow{\smash{\theta}}
& \cons{x}{\fun{cat}(\fun{pre}(t_1,\el), \fun{pre}(t_2))}\\
& \xrightarrow{\smash{\theta}}
& \cons{x}{\fun{cat}(\fun{pre}(t_1,\el), \fun{pre}(t_2,\el))}\\
& \equiv
& \cons{x}{\fun{pre}(t_1,\fun{pre}(t_2,\el))}
& (\pred{CatPre}{t_1,\fun{pre}(t_2,\el)})\\
& \xleftarrow{\smash{\kappa}}
& \fun{pre}(\fun{int}(x,t_1,t_2),\el)\\
& \xleftarrow{\smash{\theta}}
& \fun{pre}(\fun{int}(x,t_1,t_2))\\
& = & \fun{pre}(t). & \hfill\Box
\end{array}
%}
\end{equation*}
%\caption{Proof of \(\pred{Pre}{t} \colon \fun{pre}_0(t) \equiv
%  \fun{pre}(t)\) \label{fig:pre0_pre}}
%\end{figure}

\paragraph{Aplatissement revu}
\index{pile!aplatissement|(}

À la section~\vref{sec:flattening}, nous avons défini deux fonctions
pour aplatir des piles\index{pile!aplatissement} (voir
\fig~\ref{fig:flat0} à la page~\pageref{fig:flat0} et la
\fig~\ref{fig:flat} à la page~\pageref{fig:flat}). Avec le parcours
préfixe en tête, nous pourrions voir qu'aplatir une pile équivaut à
parcourir en préfixe un arbre feuillu\index{arbre binaire!$\sim$
  feuillu} (voir \fig~\vref{fig:leaf_tree} pour un exemple), en
laissant de côté les piles vides. L'essentiel est de voir une pile,
contenant peut-être d'autres piles arbitrairement imbriquées, comme
étant un arbre binaire feuillu; un exemple est montré à la
\fig~\vref{fig:stacks},
\begin{figure}[b]
\centering
\includegraphics[bb=71 643 186 721]{stacks}
\caption{Piles imbriquées vues comme un arbre feuillu}
\label{fig:stacks}
\end{figure}
où les n{\oe}uds internes (\texttt{|}) sont
des n{\oe}uds d'empilage.

La première étape consiste à définir inductivement l'ensemble des
arbres binaires feuillus comme le plus petit ensemble~\(L\) engendré
par l'application déductive (descendante) du système d'inférence
suivant:
\begin{mathpar}
\inferrule*{}{\fun{leaf}(x) \in L}
\qquad
\inferrule{t_1 \in L \and t_2 \in L}{\fun{fork}(t_1,t_2) \in L}.
\end{mathpar}
En d'autres termes, une feuille contenant la donnée~\(x\)
est notée \(\fun{leaf}(x)\)\index{leaf@\fun{leaf/1}} et les autres
n{\oe}uds internes sont des \emph{fourches}\index{arbre
  binaire!fourche} (anglais, \emph{fork}), écrites
\(\fun{fork}(t_1,t_2)\)\index{fork@\fun{fork/2}}, où
\(t_1\)~et~\(t_2\) sont des arbres binaires feuillus aussi. La seconde
étape requiert la modification de \fun{pre/1}, définie à la
\fig~\vref{fig:pre}, de telle sorte qu'elle traite des arbres binaires
feuillus. La nouvelle fonction, \fun{lpre/1}\index{lpre@\fun{lpre/1}},
est montrée à la \fig~\vref{fig:lpre}.
\begin{figure}
\begin{equation*}
\boxed{
\begin{array}{r@{\;}c@{\;}l}
\fun{lpre}(t) & \rightarrow & \fun{lpre}(t,\el).\\
\\
\fun{lpre}(\fun{leaf}(x),s) & \rightarrow & \cons{x}{s};\\
\fun{lpre}(\fun{fork}(t_1,t_2),s)
  & \rightarrow & \fun{lpre}(t_1,\fun{lpre}(t_2,s)).
\end{array}}
\end{equation*}
\caption{Parcours préfixe sur des arbres binaires feuillus}
\label{fig:lpre}
\end{figure}
L'étape finale est la traduction de \fun{lpre/1} et \fun{lpre/2} en
\fun{flat\(_2\)/1} et \fun{flat\(_2\)/2}, respectivement, à la
\fig~\vref{fig:flat2}.
\begin{figure}[b]
\begin{equation*}
\boxed{
\begin{array}{r@{\;}c@{\;}l}
\fun{flat}_2(t) & \rightarrow & \fun{flat}_2(t,\el).\\
\\
\fun{flat}_2(\el,s) & \rightarrow & s;\\
\fun{flat}_2(\cons{t_1}{t_2},s)
  & \rightarrow & \fun{flat}_2(t_1,\fun{flat}_2(t_2,s));\\
\fun{flat}_2(x,s) & \rightarrow & \cons{x}{s}.
\end{array}}
\end{equation*}
\caption{Aplatir comme \fun{lpre/1}}
\label{fig:flat2}
\end{figure}
Le point crucial est que
\(\fun{fork}(t_1,t_2)\)\index{fork@\fun{fork/2}} devient
\(\cons{t_1}{t_2}\), et \(\fun{leaf}(x)\), lorsque \(x\)~n'est pas une
pile vide, devient~\(x\) au \emph{dernier} motif. Le cas
\(\fun{leaf}(\el)\)\index{leaf@\fun{leaf/1}} devient~\(\el\) au
premier motif.

Nous avons trouvé que le coût de \fun{pre/1}\index{pre@\fun{pre/1}}
est \(\C{\fun{pre}}{n} = 2n+2\)\index{pre@$\C{\fun{pre}}{n}$} à
l'équation~\eqref{eq:pre} \vpageref{eq:pre}, où \(n\)~est le nombre de
n{\oe}uds internes. Ici, \(n\)~est la longueur de
\(\fun{flat}_2(t)\)\index{flat2@\fun{flat\(_0\)/1}}, à savoir, le
nombre de feuilles dans l'arbre feuillu qui ne sont pas des
piles. Avec cette définition, le nombre de n{\oe}uds d'empilage est
\(n + \Omega + \Gamma\), où \(\Omega\)~est le nombre de piles vides
imbriquées et \(\Gamma\)~est le nombre de piles imbriquées non-vides,
donc \(S := 1 + \Omega + \Gamma\) est le nombre total de piles. Par
conséquent, \(\C{\fun{flat}_2}{n} = 2(n +
S)\).\index{flat2@$\C{\fun{flat}_2}{n}$} Par exemple,
\begin{equation*}
\fun{flat}_2([1,[\el,[2,3]],[[4]],5]) \xrightarrow{22} [1,2,3,4,5],
\end{equation*}
car \(n=5\) et \(S=6\). (Ce dernier est le nombre de crochets
ouvrants.)  Quand \(S=1\), la pile est plate et \(\C{\fun{pre}}{n} =
\C{\fun{flat}_2}{n}\), sinon \(\C{\fun{pre}}{n} <
\C{\fun{flat}_2}{n}\).\index{pile!aplatissement|)}

\mypar{Infixe}
\index{arbre binaire!infixe|(}
\label{inorder}

Le parcours \emph{infixe} d'un arbre binaire non-vide est une pile
dans laquelle on trouve d'abord les n{\oe}uds du sous-arbre gauche en
parcours infixe, suivis par la racine et enfin les n{\oe}uds du
sous-arbre droit, en parcours infixe aussi. Par exemple, le parcours
infixe de l'arbre à la \fig~\vref{fig:bt_ex1} est
\([3,5,1,9,8,3,2]\). Il s'agit clairement d'un parcours en profondeur,
comme le parcours préfixe, parce que les enfants sont visités avant
les frères. Donc, en capitalisant sur
\fun{pre/1}\index{pre@\fun{pre/1}} à la \fig~\vref{fig:pre}, nous
comprenons que nous devrions structurer notre programme de telle sorte
qu'il suive la stratégie montrée à la \fig~\vref{fig:in_tree}, où la
seule différence avec la \fig~\ref{fig:pre_tree} est le moment où la
racine~\(x\) est empilée sur l'accumulateur\index{langage
  fonctionnel!accumulateur}: entre les parcours infixes des
sous-arbres \(t_1\)~et~\(t_2\). Les réécritures implicites dans la
figure sont \(\fun{in}(t_2,s) \twoheadrightarrow
s_1\),\index{in@\fun{in/2}} \(\fun{in}(t_1,\cons{x}{s_1})
\twoheadrightarrow s_2\) et \(\fun{in}(t,s) \twoheadrightarrow s_2\),
où \(t=\fun{int}(x,t_1,t_2)\). En éliminant les variables
intermédiaires \(s_1\)~et~\(s_2\), nous obtenons l'équivalence
\(\fun{in}(t_1,\cons{x}{\fun{in}(t_2,s)}) \equiv \fun{in}(t,s)\). Le
cas des n{\oe}uds externes est le même qu'avec un parcours préfixe. Ce
raisonnement aboutit donc à la fonction définie à la
\fig~\vref{fig:in},
\begin{figure}
\begin{equation*}
\boxed{
\begin{array}{r@{\;}c@{\;}l}
\fun{in}(t) & \xrightarrow{\xi} & \fun{in}(t,\el).\\
\\
\fun{in}(\fun{ext}(),s) & \xrightarrow{\smash{\pi}} & s;\\
\fun{in}(\fun{int}(x,t_1,t_2),s)
  & \xrightarrow{\smash{\rho}}
  & \fun{in}(t_1,\cons{x}{\fun{in}(t_2,s)}).
\end{array}}
\end{equation*}
\caption{Parcours infixe}
\label{fig:in}
\end{figure}
dont le coût est le même que celui d'un parcours préfixe:
\(\C{\fun{in}}{n} = \C{\fun{pre}}{n} = 2n +
2\).\index{in@$\C{\fun{in}}{n}$}

La \fig~\vref{fig:inorder} est un exemple d'arbre binaire qui est le
résultat d'une \emph{numérotation infixe}.\index{arbre
  binaire!numérotation infixe} Le parcours infixe de cet arbre est
\([0,1,2,3,4,5,6]\). Les numérotations infixes possèdent une propriété
intéressante: étant donné n'importe quel n{\oe}ud interne, tous les
n{\oe}uds dans son sous-arbre gauche ont des rangs inférieurs, et tous
les n{\oe}uds dans son sous-arbre droit ont des rangs supérieurs.
Soit \fun{nin/1}\index{nin@\fun{nin/1}}\index{nin@\fun{nin/2}} la
fonction qui calcule la numérotation infixe d'un arbre donnée à la
\fig~\vref{fig:nin},
\begin{figure}[b]
\centering
\framebox[0.75\columnwidth]{\vbox{%
\begin{gather*}
\inferrule
  {\fun{nin}(0,t) \twoheadrightarrow \pair{i}{t'}}
  {\fun{nin}(t) \twoheadrightarrow t'}\,.
\qquad
\fun{nin}(i,\fun{ext}()) \rightarrow \pair{i}{\fun{ext}()};
\\
\inferrule
  {\fun{nin}(i,t_1) \twoheadrightarrow \pair{j}{t'_1}
   \and
   \fun{nin}(j+1,t_2) \twoheadrightarrow \pair{k}{t'_2}}
  {\fun{nin}(i,\fun{int}(x,t_1,t_2)) \twoheadrightarrow
   \pair{k}{\fun{int}(j,t'_1,t'_2)}}\,.
\end{gather*}
}}
\caption{Numérotation infixe}
\label{fig:nin}
\end{figure}
où~\(j\), à la racine, est le plus petit rang supérieur à tout
rang dans~\(t_1\).

\paragraph{Aplatissement revu}
\label{par:rotation}

La conception de
\fun{flat/1}\index{stack!flattening}\index{flat@\fun{flat/1}} à la
\fig~\vref{fig:flat} peut suggérer une nouvelle approche du parcours
infixe. En composant des rotations à droite\index{arbre
  binaire!rotation|(}, vues aux \figs~\vrefrange{fig:lift1}{fig:lift2}
(l'inverse est, bien entendu, une \emph{rotation à gauche}), le
n{\oe}ud à visiter en premier en infixe peut être amené à devenir la
racine d'un arbre dont le sous-arbre gauche est vide. Récursivement,
le sous-arbre droit est traité, d'une façon descendante. Cet
algorithme est correct parce \emph{les parcours infixes sont
  invariants par rotation}, ce qui est symboliquement exprimé comme
suit.
\begin{equation*}
\pred{Rot}{x,y,t_1,t_2,t_3} \colon
\fun{in}(\fun{int}(y,\fun{int}(x,t_1,t_2),t_3))
\equiv
\fun{in}(\fun{int}(x,t_1,\fun{int}(y,t_2,t_3))).\index{in@\fun{in/1}}
\label{def:Rot}
\end{equation*}
À la \fig~\vref{fig:rot}, nous montrons comment un arbre binaire
devient un arbre dégénéré\index{arbre binaire!$\sim$ dégénéré}
penchant à droite, isomorphe à une pile, en répétant des rotations à
droite, de manière descendante.
\begin{figure}[b]
\centering
\includegraphics{rot}
\caption{Rotations à droite descendantes}
\label{fig:rot}
\end{figure}
Par dualité, nous pourrions composer des rotations à gauche et obtenir
un arbre dégénéré, penchant à gauche, dont le parcours infixe est
aussi égal au parcours infixe de l'arbre originel. La fonction
\fun{in\(_1\)/1}\index{in1@\fun{in\(_1\)/1}} basée sur des rotations à
droite est montrée à la \fig~\vref{fig:in1}.
\begin{figure}
\begin{equation*}
\boxed{
\begin{array}{r@{\;}c@{\;}l}
\fun{in}_1(\fun{ext}()) & \xrightarrow{\smash{\alpha}} & \el;\\
\fun{in}_1(\fun{int}(y,\fun{int}(x,t_1,t_2),t_3))
  & \xrightarrow{\smash{\beta}} & \fun{in}_1(\fun{int}(x,t_1,\fun{int}(y,t_2,t_3)));\\
\fun{in}_1(\fun{int}(y,\fun{ext}(),t_3))
  & \xrightarrow{\smash{\gamma}} & \cons{y}{\fun{in}_1(t_3)}.
\end{array}}
\end{equation*}
\caption{Parcours infixe par rotations à droite}
\label{fig:in1}
\end{figure}
Remarquons comment, à la règle~\(\gamma\), nous empilons la
racine~\(y\) dans le résultat dès que nous le pouvons, ce qui ne
serait pas possible si nous avions utilisé des rotations à gauche, et
donc nous ne construisons \emph{pas} tout l'arbre pivoté, comme à la
\fig~\vref{fig:rot}.

Le coût \(\C{\fun{in}_1}{n}\)\index{in1@$\C{\fun{in}_1}{n}$} dépend de
la topologie de l'arbre donné. Premièrement, notons que la
règle~\(\alpha\) n'est employée qu'une fois, sur le n{\oe}ud externe
le plus à droite. Deuxièmement, si l'arbre à traverser est déjà
dégénéré\index{arbre binaire!$\sim$ dégénéré} et penche à droite, la
règle~\(\beta\) est inutilisée et la règle~\(\gamma\) est appliquée
\(n\)~fois. Il est clair que cela est le meilleur des cas et
\(\B{\fun{in}_1}{n} = n + 1\).\index{in1@$\B{\fun{in}_1}{n}$}
Troisièmement, nous devrions remarquer qu'une rotation à droite
rallonge d'un n{\oe}ud exactement la \emph{branche la plus à
  droite}\index{arbre!branche}, c'est-à-dire la suite de n{\oe}uds
commençant avec la racine et atteints en suivant des arcs à droite
(par exemple, dans l'arbre initial à la \fig~\vref{fig:rot}, la
branche la plus à droite\index{arbre!branche} est \([\fun{e}, \fun{g},
\fun{h}]\)). Par conséquent, si nous voulons maximiser l'usage de la
règle~\(\beta\), nous avons besoin d'un arbre dont le sous-arbre droit
est vide, donc le sous-arbre gauche contient \(n-1\) n{\oe}uds (la
racine appartient, par définition à la branche la plus à droite): ceci
entraîne \(\W{\fun{in}_1}{n} = (n-1) + (n+1) =
2n\).\index{in1@$\W{\fun{in}_1}{n}$}\index{arbre binaire!rotation|)}

\paragraph{Exercice}

Prouvez \(\forall x,y,t_1,t_2,t_3.\pred{Rot}{x,y,t_1,t_2,t_3}\).
\index{Rot@\predName{Rot}}

\paragraph{Réflexion}

Définissons une fonction \fun{mir/1}\index{mir@\fun{mir/1}} (anglais,
\emph{mirror}) telle que \(\fun{mir}(t)\) est l'arbre symétrique à
l'arbre binaire~\(t\), par rapport à une ligne extérieure
verticale. Un exemple est montré à la \fig~\vref{fig:mirror}.
\begin{figure}[b]
\centering
\subfloat[\(t\)]{\includegraphics[bb=77 660 163 723]{bt_ex2}}
\qquad
\subfloat[\(\protect\fun{mir}(t)\)]{\includegraphics[bb=74 660 160 723]{bt_ex4}}
\caption{Réfléchissement d'un arbre binaire}
\label{fig:mirror}
\end{figure}
Nous pouvons définir cette fonction facilement:
\begin{equation*}
\fun{mir}(\fun{ext}()) \xrightarrow{\smash{\sigma}} \fun{ext}();
\quad
\fun{mir}(\fun{int}(x,t_1,t_2)) \xrightarrow{\smash{\tau}}
\fun{int}(x,\fun{mir}(t_2),\fun{mir}(t_1)).
\end{equation*}
D'après l'exemple précédent, il est assez simple de postuler la
propriété
\begin{equation*}
\pred{InMir}{t} \colon \fun{in}(\fun{mir}(t)) \equiv
\fun{rev}(\fun{in}(t)),\index{InMir@\predName{InMir}}
\end{equation*}
où \fun{rev/1}\index{rev@\fun{rev/1}}
retourne\index{pile!retournement} la pile en argument (voir
définition~\eqref{def:rev} \vpageref{def:rev}). Cette propriété est
utile parce que le membre gauche de l'équivalence est plus coûteux que
le membre droit: \(\Call{\fun{in}(\fun{mir}(t))} = \C{\fun{mir}}{n} +
\C{\fun{in}}{n} = (2n+1) + (2n+2) = 4n + 3\), à comparer avec
\(\Call{\fun{rev}(\fun{in}(t))} = \C{\fun{in}}{n} + \C{\fun{rev}}{n} =
(2n+2) + (n+2) = 3n + 4\).\index{mir@$\C{\fun{mir}}{n}$}

L'induction sur la structure des sous-arbres immédiats exige que nous
établissions
\begin{itemize}

  \item la base
  \(\pred{InMir}{\fun{ext}()}\);\index{InMir@\predName{InMir}|(}

  \item le pas \(\forall t_1.\pred{InMir}{t_1} \Rightarrow
  \forall t_2.\pred{InMir}{t_2} \Rightarrow \forall
  x.\pred{InMir}{\fun{int}(x,t_1,t_2)}.\)

\end{itemize}
La base: \(\fun{in}(\fun{mir}(\fun{ext}()))
\xrightarrow{\smash{\sigma}} \fun{in}(\fun{ext}())
\xrightarrow{\smash{\xi}} \fun{in}(\fun{ext}(),\el)
\xrightarrow{\smash{\pi}} \el \xleftarrow{\smash{\zeta}}
\fun{rcat}(\el,\el) \xleftarrow{\smash{\epsilon}} \fun{rev}(\el)
\xleftarrow{\smash{\pi}} \fun{rev}(\fun{in}(\fun{ext}(),\el))
\xleftarrow{\smash{\xi}}
\fun{rev}(\fun{in}(\fun{ext}()))\).\index{rcat@\fun{rcat/2}}\index{rev@\fun{rev/1}}
Supposons ensuite \(\pred{InMir}{t_1}\) et \(\pred{InMir}{t_2}\), puis
posons \(t := \fun{int}(x,t_1,t_2)\), pour tout~\(x\). Nous réécrivons
le membre gauche de l'équivalence à prouver jusqu'à atteindre le
membre droit ou bien être bloqué:
\begin{equation*}
\begin{array}{@{}r@{\;}l@{\;}l@{\qquad}r@{}}
\fun{in}(\fun{mir}(t)) & = &
\fun{in}(\fun{mir}(\fun{int}(x,t_1,t_2)))\\
& \xrightarrow{\smash{\tau}}
& \fun{in}(\fun{int}(x,\fun{mir}(t_2),\fun{mir}(t_1)))\\
& \xrightarrow{\smash{\xi}}
& \fun{in}(\fun{int}(x,\fun{mir}(t_2),\fun{mir}(t_1)),\el)\\
& \Rra{\rho}
& \fun{in}(\fun{mir}(t_2),\cons{x}{\fun{in}(\fun{mir}(t_1),\el)})\\
& \Lla{\xi}
& \fun{in}(\fun{mir}(t_2),\cons{x}{\fun{in}(\fun{mir}(t_1))})\\
& \equiv
& \fun{in}(\fun{mir}(t_2),\cons{x}{\fun{rev}(\fun{in}(t_1))})
& (\pred{InMir}{t_1})\\
& \xrightarrow{\smash{\xi}}
& \fun{in}(\fun{mir}(t_2),\cons{x}{\fun{rev}(\fun{in}(t_1,\el))})\\
& \Rra{\epsilon}
& \fun{in}(\fun{mir}(t_2),\cons{x}{\fun{rcat}(\fun{in}(t_1,\el),\el)}).
\end{array}
\end{equation*}
Nous ne pouvons pas utiliser l'hypothèse d'induction
\(\pred{InMir}{t_2}\) pour nous débarrasser de \(\fun{mir}(t_2)\). Un
examen attentif des termes suggère d'\emph{affaiblir} la propriété et
de surcharger \predName{InMir} avec une nouvelle définition:
\begin{equation*}
\pred{InMir}{t,s} \colon \fun{in}(\fun{mir}(t),s) \equiv
\fun{rcat}(\fun{in}(t,\el),s).
\end{equation*}
Nous avons \(\pred{InMir}{t,\el} \Leftrightarrow
\pred{InMir}{t}\). Maintenant, nous pouvons réécrire
\begin{equation*}
\begin{array}{@{}r@{\;}l@{\;}l@{\qquad}r@{}}
\fun{in}(\fun{mir}(t),s)
& =
& \fun{in}(\fun{mir}(\fun{int}(x,t_1,t_2)),s)\\
& \xrightarrow{\smash{\tau}}
& \fun{in}(\fun{int}(x,\fun{mir}(t_2),\fun{mir}(t_1)),s)\\
& \Rra{\rho}
& \fun{in}(\fun{mir}(t_2),\cons{x}{\fun{in}(\fun{mir}(t_1),s)})\\
& \equiv
& \fun{in}(\fun{mir}(t_2),\cons{x}{\fun{rcat}(\fun{in}(t_1,\el),s)})
& (\pred{InMir}{t_1,s})\\
& \equiv_0
& \fun{rcat}(\fun{in}(t_2,\el),\cons{x}{\fun{rcat}(\fun{in}(t_1,\el),s)})\\
& \Lla{\eta}
& \fun{rcat}(\cons{x}{\fun{in}(t_2,\el)},
\fun{rcat}(\fun{in}(t_1,\el),s)),
\end{array}
\end{equation*}
où (\(\equiv_0\)) est
\(\pred{InMir}{t_2,\cons{x}{\fun{rcat}(\fun{in}(t_1,\el),s)}}\).\index{rcat@\fun{rcat/2}}\index{InMir@\predName{InMir}|)}
Le membre droit maintenant:
\begin{equation*}
%\begin{array}{@{}r@{\;}c@{\;}l@{\qquad}r@{}}
\fun{rcat}(\fun{in}(t,\el),s)
\! = \!
 \fun{rcat}(\fun{in}(\fun{int}(x,t_1,t_2),\el),s)
\! \xrightarrow{\smash{\rho}}\!
 \fun{rcat}(\fun{in}(t_1,\cons{x}{\fun{in}(t_2,\el)}),s).
%\end{array}
\end{equation*}
Les deux expressions sur lesquelles nous butons partagent le
sous-terme \(\cons{x}{\fun{in}(t_2,\el)}\)\index{in@\fun{in/2}}. La
principale différence est que la première expression contient deux
appels à \fun{rcat/2}\index{rcat@\fun{rcat/2}|(}, au lieu d'un dans la
deuxième. Pouvons-nous trouver un moyen pour n'avoir qu'un seul appel
dans la première aussi? Nous cherchons une expression équivalente à
\(\fun{rcat}(s,\fun{rcat}(t,u))\), dont la forme est
\(\fun{rcat}(v,w)\), où \(v\) et~\(w\) ne contiennent aucun appel à
\fun{rcat/2}. Quelques exemple suggèrent
rapidement\index{Rcat@\predName{Rcat}}
\begin{equation*}
  \pred{Rcat}{s,t,u} \colon \fun{rcat}(s,\fun{rcat}(t,u)) \equiv \fun{rcat}(\fun{cat}(t,s),u).
\end{equation*}
Nous n'avons en fait pas besoin du principe d'induction pour prouver
ce théorème si nous nous souvenons de ce que nous avons déjà établi:
\begin{itemize}

  \item \(\pred{CatRev}{s,t} \colon \fun{cat}(\fun{rev}_0(t),
    \fun{rev}_0(s)) \equiv \fun{rev}_0(\fun{cat}(s,t))\);

  \item \(\pred{EqRev}{s} \colon \fun{rev}_0(s) \equiv \fun{rev}(s)\);

  \item \(\pred{CatAssoc}{s,t,u} \colon \fun{cat}(s,\fun{cat}(t,u))
    \equiv \fun{cat}(\fun{cat}(s,t),u)\);

  \item \(\pred{RevCat}{s,t} \colon \fun{rcat}(s,t) \equiv
    \fun{cat}(\fun{rev}(s),t)\).

\end{itemize}
Alors, nous avons les équivalences suivantes:
\begin{equation*}
\begin{array}{@{}r@{\,}c@{\,}l@{\,}r@{}}
\fun{rcat}(s,\fun{rcat}(t,u))
& \equiv & \fun{rcat}(s,\fun{cat}(\fun{rev}(t),u))
& (\pred{RevCat}{t,u}\!)\\
& \equiv & \fun{rcat}(s,\fun{cat}(\fun{rev}_0(t),u))
& (\pred{EqRev}{t}\!)\\
& \equiv & \fun{cat}(\fun{rev}(s),\fun{cat}(\fun{rev}_0(t),u))
& (\pred{RevCat}{s,\fun{cat}(\fun{rev}_0(t),u)}\!)\\
& \equiv & \fun{cat}(\fun{rev}_0(s),\fun{cat}(\fun{rev}_0(t),u)\!)
& (\pred{EqRev}{s}\!)\\
& \equiv & \fun{cat}(\fun{cat}(\fun{rev}_0(s),\fun{rev}_0(t)\!),u)\!\!
& (\!\pred{CatAssoc}{\fun{rev}(s),\fun{rev}(t),u}\!)\\
& \equiv & \fun{cat}(\fun{rev}_0(\fun{cat}(t,s)),u)
& (\pred{CatRev}{s,t}\!)\\
& \equiv & \fun{cat}(\fun{rev}(\fun{cat}(t,s)),u)
& (\pred{EqRev}{\fun{cat}(t,s)}\!)\\
& \equiv & \fun{rcat}(\fun{cat}(t,s),u)
& (\pred{RevCat}{\fun{cat}(t,s),u}\!)
\end{array}
\end{equation*}
Reprenons la réécriture de la première expression qui nous a arrêté:
\begin{equation*}
\begin{array}{@{}r@{\;}c@{\;}l@{\qquad}r@{}}
\fun{in}(\fun{mir}(t),s)
& \equiv_0
& \fun{rcat}(\cons{x}{\fun{in}(t_2,\el)},
  \fun{rcat}(\fun{in}(t_1,\el),s))\\
& \equiv_1
& \fun{rcat}(\fun{cat}(\fun{in}(t_1,\el),
  \cons{x}{\fun{in}(t_2,\el)}), s),\index{rcat@\fun{rcat/2}|)}
\end{array}
\end{equation*}
où (\(\equiv_0\)) est un raccourci pour la dérivation précédente et
(\(\equiv_1\)) est l'instance
\(\pred{Rcat}{\cons{x}{\fun{in}(t_2,\el)}, \fun{in}(t_1,\el),
  s}\). Une comparaison des expressions problématiques révèle que nous
devons prouver \(\fun{cat}(\fun{in}(t,\el),s) \equiv
\fun{in}(t,s)\). Cette équivalence est similaire à
\(\pred{CatPre}{t,s}\)\index{CatPre@\predName{CatPre}}. En supposant
qu'elle est vraie aussi, nous avons fini.\hfill\(\Box\)

\paragraph{Exercices}
\begin{enumerate}

  \item Prouvez le lemme manquant \(\pred{CatIn}{t,s} \colon
  \fun{cat}(\fun{in}(t,\el),s) \equiv
  \fun{in}(t,s)\).\index{CatIn@\predName{CatIn}}\index{cat@\fun{cat/2}}

 \item Définissez une fonction qui construit un arbre binaire à partir
  de ses parcours préfixe et infixe, en supposant que ses n{\oe}uds
  internes sont distincts. Assurez-vous que le coût est une fonction
  linéaire du nombre de n{\oe}uds internes. Comparez votre solution
  avec celle de \cite{MuBird_2003}.

\end{enumerate}
\index{arbre binaire!infixe|)}

\mypar{Postfixe}
\index{arbre binaire!postfixe|(}

Un parcours \emph{postfixe}\index{arbre binaire!postfixe} d'un arbre
binaire non-vide est une pile avec, d'abord, les n{\oe}uds du
sous-arbre droit, en parcours postfixe, suivis par les n{\oe}uds du
sous-arbre gauche, en parcours postfixe, et finalement la racine. Par
exemple, le parcours postfixe de l'arbre à la \fig~\vref{fig:bt_ex1}
est \([5,3,9,1,2,3,8]\). À l'évidence, il s'agit là d'un parcours en
profondeur\index{arbre binaire!parcours en profondeur}, comme un
parcours préfixe, mais, contrairement à ce dernier, la racine est
placée au fond de la pile. Cette approche est résumée pour les
n{\oe}uds internes à la \fig~\vref{fig:post_tree}. La différence avec
\fun{pre/1}\index{pre@\fun{pre/1}} et \fun{in/1}\index{in@\fun{in/1}}
réside dans le moment où la racine est empilée sur l'accumulateur. La
définition de la fonction est donnée à la \fig~\vref{fig:post}.
\begin{figure}
%\abovedisplayskip=0pt
%\belowdisplayskip=0pt
\begin{equation*}
\boxed{
\begin{array}{r@{\;}c@{\;}l}
\fun{post}(t) & \xrightarrow{\lambda} & \fun{post}(t,\el).\\
\\
\fun{post}(\fun{ext}(),s) & \xrightarrow{\smash{\mu}} & s;\\
\fun{post}(\fun{int}(x,t_1,t_2),s)
  & \xrightarrow{\smash{\nu}}
  & \fun{post}(t_1,\fun{post}(t_2,\cons{x}{s})).
\end{array}}
\end{equation*}
\caption{Parcours postfixe}
\label{fig:post}
\end{figure}
La signification de la pile~\(s\) dans l'appel
\(\fun{post}(t,s)\)\index{post@\fun{post/2}}\index{post@\fun{post/1}}
est le même que dans \(\fun{pre}(t,s)\), modulo l'ordre:
\(s\)~contient, en postfixe, les n{\oe}uds qui suivent, en postfixe,
les n{\oe}uds du sous-arbre~\(t\). Le coût est familier aussi:
\begin{equation*}
  \C{\fun{post}}{n} = \C{\fun{in}}{n} = \C{\fun{pre}}{n} = 2n +
  2.\index{post@$\C{\fun{post}}{n}$}
\end{equation*}

Un exemple de \emph{numérotation postfixe}\index{arbre
  binaire!numérotation postfixe} est montré à la
\fig~\vref{fig:postorder}, donc le parcours postfixe de cet arbre est
\([0,1,2,3,4,5,6]\). Remarquons que les rangs augmentent le long de
chemins ascendants. La
\fig~\vref{fig:npost}\index{npost@\fun{npost/1}}\index{npost@\fun{npost/2}}
\begin{figure}
%\abovedisplayskip=0pt
%\belowdisplayskip=0pt
\centering
\framebox[0.75\columnwidth]{\vbox{%
\begin{gather*}
\inferrule
  {\fun{npost}(0,t) \twoheadrightarrow \pair{i}{t'}}
  {\fun{npost}(t) \rightarrow t'}\,.
\qquad
\fun{npost}(i,\fun{ext}()) \rightarrow \pair{i}{\fun{ext}()};
\\
\inferrule
  {\fun{npost}(i,t_1) \twoheadrightarrow \pair{j}{t'_1}
   \and
   \fun{npost}(j,t_2) \twoheadrightarrow \pair{k}{t'_2}}
  {\fun{npost}(i,\fun{int}(x,t_1,t_2)) \rightarrow
   \pair{k+1}{\fun{int}(k,t'_1,t'_2)}}\,.
\end{gather*}
}}
\caption{Numérotation postfixe}
\label{fig:npost}
\end{figure}
montre le programme pour numéroter un arbre binaire en postfixe. La
racine est numérotée avec le nombre venant du sous-arbre droit, suivant
en cela l'organisation d'un parcours postfixe.

\paragraph{Une preuve}\label{proof_PreMir}

Soit \(\pred{PreMir}{t} \colon \fun{pre}(\fun{mir}(t)) \equiv
\fun{rev}(\fun{post}(t))\).\index{PreMir@\predName{PreMir}|(}
L'expérience acquise en prouvant \(\pred{InMir}{t}\) nous suggère
d'affaiblir (généraliser) la propriété: \(\pred{PreMir}{t,s} \colon
\fun{pre}(\fun{mir}(t),s) \equiv
\fun{rcat}(\fun{post}(t,\el),s)\).\index{rcat@\fun{rcat/2}}
Clairement, \(\pred{PreMir}{t,\el} \Leftrightarrow
\pred{PreMir}{t}\). Définissons alors
\begin{equation*}
  (t,s) \succ_{B \times S} (t',s') :\Leftrightarrow \text{\(t \succ_{B}
    t'\) ou (\(t = t'\) et \(s \succ_{S} s'\))}.
\end{equation*}
Il s'agit là essentiellement du même ordre que celui sur les appels à
\fun{pre/1}\index{pre@\fun{pre/1}}, à la
définition~\eqref{eq:BS_order} \vpageref{eq:BS_order}. L'élément
minimum pour cet ordre lexicographique est
\((\fun{ext}(),\el)\). L'induction bien fondée alors requiert la
preuve de
\begin{itemize}

  \item la base \(\pred{PreMir}{\fun{ext}(),\el}\);

  \item \(\forall t,s. (\forall t',s'.(t,s) \succ_{B \times S} (t',s')
    \Rightarrow \pred{PreMir}{t',s'}) \Rightarrow
    \pred{PreMir}{t,s}\).

\end{itemize}
Pour la base: \index{mir@\fun{mir/1}|(}
\(\fun{pre}(\fun{mir}(\fun{ext}()),\el) \xrightarrow{\smash{\sigma}}
\fun{pre}(\fun{ext}(),\el) \xrightarrow{\smash{\iota}} \el
\xleftarrow{\smash{\zeta}} \fun{rcat}(\el,\el)
\xleftarrow{\smash{\mu}}
\fun{rcat}(\fun{post}(\fun{ext}(),\el),\el)\). Posons \(t :=
\fun{int}(x,t_1,t_2)\). %À la \fig~\vref{fig:premir},
Nous avons les réécritures du membre gauche suivantes:
%\begin{figure}
\begin{equation*}
%\boxed{
\begin{array}{@{}r@{\;}l@{\;}l@{\qquad}r@{}}
\fun{pre}(\fun{mir}(t),s)
& = & \fun{pre}(\fun{mir}(\fun{int}(x,t_1,t_2)),s)\\
& \xrightarrow{\smash{\tau}}
& \fun{pre}(\fun{int}(x,\fun{mir}(t_2),\fun{mir}(t_1)),s)\\
& \Rra{\kappa}
& \cons{x}{\fun{pre}(\fun{mir}(t_2),\fun{pre}(\fun{mir}(t_1),s))}\\
& \equiv_0
& \cons{x}{\fun{pre}(\fun{mir}(t_2),
           \fun{rcat}(\fun{post}(t_1,\el),s))}\\
& \equiv_1
& \cons{x}{\fun{rcat}(\fun{post}(t_2,\el),
           \fun{rcat}(\fun{post}(t_1,\el),s))}\\
& \equiv_2
& \cons{x}{\fun{rcat}(\fun{cat}(\fun{post}(t_1,\el),
                                 \fun{post}(t_2,\el)),
                      s)}\\
& \Lla{\zeta}
& \fun{rcat}(\el,\cons{x}{\fun{rcat}(\fun{cat}(\fun{post}(t_1,\el),
                                       \fun{post}(t_2,\el)),
                            s)})\\
& \Lla{\eta}
& \fun{rcat}([x],\fun{rcat}(\fun{cat}(\fun{post}(t_1,\el),
                                       \fun{post}(t_2,\el)),
                            s))\\
& \equiv_3
& \fun{rcat}(\fun{cat}(\fun{cat}(\fun{post}(t_1,\el),
                                  \fun{post}(t_2,\el)),
                       [x]),
             s)\\
& \equiv_4
& \fun{rcat}(\fun{cat}(\fun{post}(t_1,\el),
                       \fun{cat}(\fun{post}(t_2,\el),
                                 [x])),
             s)\\
& \equiv_5
& \fun{rcat}(\fun{cat}(\fun{post}(t_1,\el),
                       \fun{post}(t_2,[x])),
             s)\\
& \equiv_6
& \fun{rcat}(\fun{post}(t_1,\fun{post}(t_2,[x])),s)\\
& \xleftarrow{\smash{\nu}}
& \fun{rcat}(\fun{post}(\fun{int}(x,t_1,t_2),\el),s)\\
& =
& \fun{rcat}(\fun{post}(t,\el),s).\hfill\Box
\end{array}
%}
\end{equation*}
%\caption{Preuve de \(\fun{pre}(\fun{mir}(t),s) \equiv %\fun{rcat}(\fun{post}(t,\el),s)\)\label{fig:premir}}
%\end{figure}
où
\begin{itemize}

  \item (\(\equiv_0\)) est \(\pred{PreMir}{t_1,s}\), une instance de
    l'hypothèse d'induction;\index{PreMir@\predName{PreMir}}

  \item (\(\equiv_1\)) est
  \(\pred{PreMir}{t_2,\fun{rcat}(\fun{post}(t_1,\el),s)}\), de
  l'hypothèse d'induction;\index{PreMir@\predName{PreMir}}

  \item (\(\equiv_2\)) est
    \(\pred{Rcat}{\fun{post}(t_2,\el),\fun{post}(t_1,\el),s}\);
    \index{Rcat@\predName{Rcat}}

  \item (\(\equiv_3\)) est
    \(\pred{Rcat}{[x],\fun{cat}(\fun{post}(t_1,\el),
      \fun{post}(t_2,\el)),s}\);
    \index{Rcat@\predName{Rcat}}

  \item (\(\equiv_4\)) est \(\pred{CatAssoc}{\fun{post}(t_1,\el),
    \fun{post}(t_2,\el),[x]}\);\index{CatAssoc@\predName{CatAssoc}}

  \item (\(\equiv_5\)) est
  \(\pred{CatPost}{t_2,[x]}\)\index{CatPost@\predName{CatPost}} si
  \(\pred{CatPost}{t,s} \colon \fun{cat}(\fun{post}(t),s) \equiv
  \fun{post}(t,s)\)\!;\index{cat@\fun{cat/2}}

  \item (\(\equiv_6\)) est \(\pred{CatPost}{t_1,\fun{post}(t_2,[x])}\).
  \index{post@\fun{post/2}}\index{CatPost@\predName{CatPost}}

\end{itemize}
Alors \index{mir@\fun{mir/1}|)} \(\pred{CatPost}{t,s} \Rightarrow
\pred{PreMir}{t,s} \Rightarrow \fun{pre}(\fun{mir}(t)) \equiv
\fun{rev}(\fun{post}(t))\).\hfill\(\Box\)

\paragraph{Dualité}
\label{thm_duality}

Le théorème dual\index{PostMir@\predName{PostMir}} \(\pred{PostMir}{t}
\colon \fun{post}(\fun{mir}(t)) \equiv
\fun{rev}(\fun{pre}(t))\)\index{mir@\fun{mir/1}}
\index{pre@\fun{pre/1}} \index{post@\fun{post/1}}
\index{rev@\fun{rev/1}} peut être démontré d'au moins deux façons:
soit nous concevons une nouvelle preuve dans l'esprit de
\(\pred{PreMir}{t}\), soit nous utilisons le fait que le théorème est
une équivalence et nous produisons alors une suite de théorèmes
équivalents, dont le dernier est simple à établir. Choisissons cette
seconde méthode et commençons par considérer
\(\pred{PreMir}{\fun{mir}(t)}\),\index{PreMir@\predName{PreMir}|)}
puis continuons en recherchant des expressions équivalentes aux deux
membres, jusqu'à atteindre \(\pred{PostMir}{t}\):
\begin{equation*}
\begin{array}{@{}r@{\;}c@{\;}l@{\qquad}r@{}}
          \fun{pre}(\fun{mir}(\fun{mir}(t)))
& \equiv
& \fun{rev}(\fun{post}(\fun{mir}(t)))
& (\pred{PreMir}{\fun{mir}(t)})\\
  \fun{pre}(t)
& \equiv
& \fun{rev}(\fun{post}(\fun{mir}(t)))
& (\pred{InvMir}{t})\\
  \fun{rev}(\fun{pre}(t))
& \equiv
& \fun{rev}(\fun{rev}(\fun{post}(\fun{mir}(t))))\\
  \fun{rev}(\fun{pre}(t))
& \equiv
& \fun{post}(\fun{mir}(t))
& (\pred{InvRev}{\fun{post}(\fun{mir}(t))}),
\end{array}
\end{equation*}
où\index{InvMir@\predName{InvMir}} \(\pred{InvMir}{t} \colon
\fun{mir}(\fun{mir}(t)) \equiv t\) et\index{InvRev@\predName{InvRev}}
\(\pred{InvRev}{s} :\Leftrightarrow \pred{Inv}{s} \wedge
\pred{EqRev}{s}\).\hfill\(\Box\)

\paragraph{Exercices}
\begin{enumerate}

  \item Prouvez le lemme\index{CatPost@\predName{CatPost}}
  \(\pred{CatPost}{t,s} \colon \fun{cat}(\fun{post}(t,\el),s) \equiv
  \fun{post}(t,s)\).

  \item Utilisez \fun{rev\(_0\)/1} au lieu de \fun{rev/1} dans
  \(\pred{InMir}{t}\) et \(\pred{PreMir}{t}\). Les preuves sont-elles
  plus aisées?

  \item Prouvez le lemme manquant\index{InvMir@\predName{InvMir}}
  \(\pred{InvMir}{t} \colon \fun{mir}(\fun{mir}(t)) \equiv
  t\).\label{ex_mir_mir}

\item Pouvez-vous reconstruire un arbre binaire d'après ses parcours
  infixe et postfixe, en supposant que ses n{\oe}uds internes sont
  tous distincts?

\end{enumerate}
\index{arbre binaire!postfixe|)}

\mypar{Parcours des niveaux}
\index{arbre binaire!parcours des niveaux|(}

Le \emph{niveau}~\(l\)\index{arbre!niveau} d'un arbre est une
sous-suite de son parcours préfixe\index{arbre binaire!préfixe} telle
que tous les n{\oe}uds ont pour longueur
interne~\(l\).\index{arbre!longueur interne} En particulier, la racine
est le seul n{\oe}ud au niveau~\(0\). Dans l'arbre de la
\fig~\vref{fig:bt_ex1}, \([3,9,2]\) est le niveau~\(2\). Pour
comprendre la contrainte du parcours préfixe, nous devons considérer
la numérotation préfixe de l'arbre, montrée à la
\fig~\vref{fig:bt_ex3}, où les rangs sont les exposants à gauche des
contenus. De cette manière, il n'y a plus d'ambiguïté lorsque nous
faisons référence à des n{\oe}uds. Par exemple, \([3,9,2]\) est en
fait ambigu parce qu'il existe deux n{\oe}uds dont l'information
associée est~\(3\). Nous voulions dire que \([{}^{2}{3}, {}^{4}{9},
{}^{6}{2}]\) est le niveau~\(2\) car tous ces n{\oe}uds ont une
longueur de chemin interne égale à~\(2\) \emph{et} possèdent des rangs
préfixes croissants (\(2,4,6\)).

Un \emph{parcours des niveaux}\index{arbre binaire!parcours des
  niveaux} consiste en une pile contenant les n{\oe}uds de tous les
niveaux par longueurs de chemins croissantes. Ainsi, le parcours par
niveaux de l'arbre à la \fig~\vref{fig:bt_ex1} est
\([8,1,3,3,9,2,5]\). Parce que cette méthode visite les frères d'un
n{\oe}uds avant ses enfants, elle est un \emph{parcours en
  largeur}.\index{arbre binaire!parcours en largeur}

À la \fig~\vref{fig:lorder} est montrée une \emph{numérotation par
  niveaux} de l'arbre à la \fig~\vref{fig:bt_ex3}, souvent appelée
\emph{numérotation en largeur}.\index{arbre binaire!numérotation en
  largeur} (Attention lors de la recherche du terme correct en
anglais, qui est \emph{breadth numbering}, car il est souvent mal
écrit, par exemple «~bread numbering~» et «~breath numbering~».) Notons
que les rangs augmentent le long de chemin descendants, comme dans une
numérotation préfixe.

Il se peut que nous comprenions maintenant que la notion de niveau
d'un arbre n'est pas intuitive. La raison en est simple: les n{\oe}uds
d'un niveau ne sont pas frères\index{arbre!n{\oe}ud!frère}, sauf au
niveau~\(1\), donc, en général, nous ne pouvons nous attendre à
construire le niveau d'un arbre \(\fun{int}(x,t_1,t_2)\) à l'aide
uniquement des niveaux de~\(t_1\) et~\(t_2\), c'est-à-dire, avec une
conception à grands pas\index{conception!grands pas}. En conséquence,
une conception à petits pas est nécessaire, en contraste avec, par
exemple, \fun{size/1}\index{size@\fun{size/1}} dans~\eqref{eq:size}
\vpageref{eq:size}.

Soit \fun{bf\(_0\)/1}\index{bf0@\fun{bf\(_0\)/1}} (anglais,
\emph{breadth-first}) la fonction telle que \(\fun{bf}_0(t)\)
calcule la pile de n{\oe}uds de~\(t\) par niveaux. Elle est
partiellement définie à la \fig~\vref{fig:lo0}.
\begin{figure}[t]
\centering
\includegraphics[bb=108 682 395 721]{lo0}
\caption{Parcours par niveaux avec \fun{bf\(_0\)/1}}
\label{fig:lo0}
\end{figure}
Si nous imaginons que nous coupons la racine d'un arbre binaire, nous
obtenons ses deux sous-arbres immédiats. Si nous coupons la racine de
ces arbres, nous obtenons encore des sous-arbres. Ceci suggère que
nous devrions travailler avec des forêts\index{arbre!forêt} au lieu
d'arbres isolément.

La fonction de coupe est \fun{def/1}\index{def@\fun{def/1}}
(anglais, \emph{deforest}), définie à la \fig~\vref{fig:def},
\begin{figure}
\abovedisplayskip=0pt
\belowdisplayskip=0pt
\centering
\includegraphics[bb=71 660 324 721]{def}
\caption{Coupe}
\label{fig:def}
\end{figure}
telle que \(\fun{def}(f)\), où \(f\)~représente une forêt, calcule une
paire \(\pair{r}{f'}\), où \(r\)~est les racines en préfixe des arbres
dans~\(f\), et \(f'\)~est la forêt immédiate de~\(f\). Remarquons qu'à
la \fig~\vref{fig:def}, la règle d'inférence augmente le niveau
partiel~\(r\) avec la racine~\(x\), et que \(t_2\)~est empilé
avant~\(t_1\) sur le reste de la forêt immédiate~\(f\), qui sera
traitée plus tard par \fun{bf\(_0\)/1}\index{bf0@\fun{bf\(_0\)/1}}. Au
lieu de construire la pile des niveaux \([[8], [1,3], [3,9,2], [5]]\),
nous aplatissons pas à pas simplement en appelant
\fun{cat/2}\index{cat@\fun{cat/2}} à la règle~\(\mu\). Si nous voulons
réellement les niveaux, nous écririons
\(\cons{r}{\fun{bf}_1(f')}\)\index{bf1@\fun{bf\(_1\)/1}} au lieu de
\(\fun{cat}(r,\fun{bf}_1(f'))\)\index{cat@\fun{cat/2}}, ce qui,
d'ailleurs, réduit le coût.

Le concept sous-jacent ici est celui de \emph{parcours d'une
  forêt}. Sauf en infixe, tous les parcours dont nous avons discuté se
généralisent naturellement aux forêts binaires: le parcours préfixe
d'une forêt est le parcours préfixe du premier arbre, suivi du
parcours préfixe du reste de la forêt. La même logique est pertinente
pour les parcours en postfixe et par niveaux. Cette uniformité
provient du fait que tous ces parcours vont vers la droite, à savoir,
un enfant à gauche est visité avant son frère. La notion de
hauteur\index{arbre!hauteur} d'un arbre se généralise à une forêt tout
aussi bien: la hauteur d'une forêt est la hauteur maximum de ses
arbres. La raison pour laquelle cela est simple est que la hauteur est
un concept qui repose sur une vue purement verticale des arbres, donc
indépendent de l'ordre des sous-arbres.

Pour évaluer le coût
\(\C{\fun{bf}_0}{n,h}\)\index{bf0@$\C{\fun{bf}_0}{n,h}$} de l'appel
\(\fun{bf}_0(t)\), où \(n\)~est la taille de l'arbre binaire~\(t\) et
\(h\)~est sa hauteur\index{arbre binaire!hauteur}, il est commode de
travailler avec les \emph{niveaux étendus}\index{arbre binaire!niveau
  étendu}. Un niveau étendu est un niveau où les n{\oe}uds externes
sont inclus (ils ne sont pas explicitement numérotés en préfixe parce
que les n{\oe}uds externes sont indistincts). Par exemple, le niveau
étendu~\(2\) de l'arbre à la \fig~\vref{fig:bt_ex3} est \([{}^{2}{3},
{}^{4}{9}, \Box, {}^{6}{2}]\). S'il est nécessaire parfois de
souligner la différence, nous pouvons appeler les niveaux habituels
\emph{niveaux taillés}, ce qui est cohérent avec la terminologie de la
\fig~\ref{fig:bt_ex} \vpageref{fig:bt_ex}. Remarquons qu'il y a
toujours un niveau étendu de plus que de niveaux taillés, constitué
entièrement de n{\oe}uds externes. (Nous évitons d'écrire que ces
n{\oe}uds sont les plus hauts, parce que le problème avec le terme
«~hauteur~»\index{arbre!hauteur} est qu'il n'a de sens que si les arbres
sont figurés avec leur racine au bas de la page. C'est peut-être pour
cela que certains auteurs préfèrent le concept, plus clair, de
\emph{profondeur}\index{arbre binaire!profondeur}, en usage en théorie
des graphes. Pour un compendium sur les différentes façons de dessiner
des arbres, voir \cite{Knuth_1997} à la section~2.3.) En d'autres
termes, \(l_h=0\), où \(l_i\)~est le nombre de n{\oe}uds internes au
niveau étendu~\(i\).
\begin{itemize}

  \item La règle~\(\iota\) est utilisée une fois;

  \item la règle~\(\kappa\) est appliquée une fois;

  \item les règles~\(\lambda\) et~\(\mu\) sont employées une fois pour
    chaque niveau étendu de l'arbre initial; ceux-ci totalisent
    \(2(h+1)\) appels;

  \item le coût de \(\fun{cat}(r,\fun{loc}_1(f'))\) est la longueur du
    niveau~\(r\), plus un, donc le coût cumulé de la concaténation est
    \(\sum_{i=0}^{h}\C{\fun{cat}}{l_i} = n + h + 1\);

  \item la règle~\(\epsilon\) est utilisée une fois par niveau étendu,
    soit \(h+1\) fois;

  \item la règle~\(\zeta\) est appliquée une fois pour chaque n{\oe}ud
    externe, donc \(n+1\) fois;

  \item les règles~\(\eta\) et~\(\theta\) sont utilisées une fois pour
    chaque n{\oe}ud interne, c'est-à-dire \(2n\) fois.

\end{itemize}
En somme, nous obtenons
\begin{equation*}
\C{\fun{bf}_0}{n,h} = 4n + 4h + 7. \index{bf0@$\C{\fun{bf}_0}{n,h}$}
\end{equation*}
Par définition, le coût minimum est \(\B{\fun{bf}_0}{n} =
\min_h\C{\fun{bf}_0}{n,h}\). La hauteur\index{arbre binaire!hauteur}
est minimum quand l'arbre binaire est \emph{parfait}\index{arbre
  binaire!$\sim$ parfait}, c'est-à-dire que tous ses niveaux sont
pleins (voir \fig~\vref{fig:comp_tree}). Dans ce cas, \(l_i=2^i\),
pour \(0 \leqslant i \leqslant h-1\), et, par extension, il y a
\(2^h\)~n{\oe}uds externes. Le théorème~\ref{thm_int_ext} donne
l'équation \(n+1 = 2^h\), d'où \(h=\lg(n+1)\), et enfin nous avons
\(\B{\fun{bf}_0}{n} = 4n + 4\lg(n+1) + 7 \sim
4n\). \index{bf0@$\B{\fun{bf}_0}{n}$}

Le coût maximum est obtenu en maximisant la hauteur\index{arbre
  binaire!hauteur}, tout en conservant la taille constante. Ceci se
produit pour les arbre dégénérés\index{arbre binaire!$\sim$ dégénéré},
comme ceux montrés à la \fig~\vref{fig:tree_stack} et à la
\fig~\vref{fig:zigzag}. Ici, \(h=n\) et le coût est
\(\W{\fun{bf}_0}{n} = 8n + 7 \sim
8n\). \index{bf0@$\W{\fun{bf}_0}{n}$}

% Wrapping figure better declared before a paragraph
%
\begin{wrapfigure}[7]{r}[0pt]{0pt}
% [7] vertical lines
% {r} mandatory right placement
% [0pt] of margin overhang
\centering
\includegraphics[bb=55 653 120 721]{zigzag}
\caption{}
\label{fig:zigzag}
\end{wrapfigure}
Ce résultat est à contraster avec les programmes que nous avons écrits
tantôt pour les parcours préfixe, infixe et postfixe, dont le coût
était \(2n+2\). Il est possible de réduire le coût du parcours des
niveaux en employant un modèle différent, basé sur
\fun{pre\(_3\)/1}\index{pre3@\fun{pre\(_3\)/1}} à la
\fig~\vref{fig:pre3}. La différence est qu'au lieu d'utiliser une pile
pour conserver les sous-arbres à traverser ultérieurement, nous
employons une file\index{file}, une structure de données linéaire que
nous avons présentée à la section~\ref{sec:queueing}. Considérons à la
\fig~\vref{fig:lo_abcde} l'algorithme à l'{\oe}uvre sur le même
exemple qu'à la \fig~\vref{fig:pre3_abcde}.
\begin{figure}[b]
\centering
\includegraphics[bb=70 655 385 723]{lo_abcde_0}
\bigskip
\includegraphics[bb=70 690 388 723]{lo_abcde_1}
\caption{Parcours des niveaux avec une file}
\label{fig:lo_abcde}
\end{figure}
Gardons à l'esprit que les arbres sont défilés du côté droit de la
forêt et enfilés à gauche. (Certains auteurs préfèrent l'autre sens.)
La racine du prochain arbre à défiler est encerclée.

En vue d'une comparaison avec
\fun{pre\(_4\)/1}\index{pre4@\fun{pre\(_4\)/1}} à la
\fig~\vref{fig:pre3}, nous écrirons
\(\enq{x}{q}\)\index{file}\index{enq@$\enq{x}{q}$} au lieu de
\(\fun{enq}(x,q)\)\index{enq@\fun{enq/2}}, et
\(\deq{x}{q}\)\index{deq@$\deq{x}{q}$} au lieu de \(\pair{q}{x}\). La
file vide est notée~\(\emq\)\index{file@$\emq$}. Crucialement, nous
permettrons à ces expressions d'apparaître dans les motifs des règles
définissant \fun{bf\(_2\)/1}\index{bf2@\fun{bf\(_2\)/1}}, qui effectue
un parcours des niveaux à la \fig~\ref{fig:bf1}.
\begin{figure}
\begin{equation*}
\boxed{
\begin{array}{r@{\;}c@{\;}l}
\fun{bf}_1(t) & \rightarrow & \fun{bf}_2(\enq{t}{\emq}).\\
\\
\fun{bf}_2(\emq) & \rightarrow & \el;\\
\fun{bf}_2(\deq{\fun{ext}()}{q}) & \rightarrow & \fun{bf}_2(q);\\
\fun{bf}_2(\deq{\fun{int}(x,t_1,t_2)}{q})
  & \rightarrow & \cons{x}{\fun{bf}_2(\enq{t_2}{\enq{t_1}{q}})}.
\end{array}}
\end{equation*}
\caption{Parcours des niveaux avec une file abstraite}
\label{fig:bf1}
\end{figure}
La différence en termes de structure de donnée (accumulateur) a déjà
été mentionnée: \fun{pre\(_4\)/1}\index{pre4@\fun{pre\(_4\)/1}}
emploie une pile et \fun{bf\(_2\)/1}\index{bf2@\fun{bf\(_2\)/1}} une
file, mais, en ce qui concerne les algorithmes, ceux-ci ne diffèrent
que dans l'ordre relatif dans lequel \(t_1\)~et~\(t_2\) sont ajoutés à
l'accumulateur.

À la section~\ref{sec:queueing}, nous avons vu comment mettre en
{\oe}uvre une file \(\fun{q}(r,f)\) avec deux piles \(r\)~et~\(f\):
\index{q@\fun{q/2}} la pile arrière~\(r\), où les éléments sont
empilés (logiquement enfilés), et la pile frontale~\(f\), de laquelle
les éléments sont dépilés (logiquement défilés). Aussi, nous avons
défini l'enfilage par \fun{enq/2}\index{enq@\fun{enq/2}}
dans~\eqref{def:enq}, \vpageref{def:enq}, et le défilage avec
\fun{deq/1}\index{deq@\fun{deq/1}} dans~\eqref{def:deq}. Ceci nous
permet de raffiner la définition
de~\fun{bf\(_2\)/1}\index{bf2@\fun{bf\(_2\)/1}} en
\fun{bf\(_3\)/1}\index{bf3@\fun{bf\(_3\)/1}} à la \fig~\vref{fig:bf3}.
\begin{figure}
\abovedisplayskip=0pt
\belowdisplayskip=0pt
\centering
\framebox[0.9\columnwidth]{\vbox{%
\begin{gather*}
\fun{bf}_3(t) \rightarrow \fun{bf}_4(\fun{enq}(t,\fun{q}(\el,\el))).
\qquad
\fun{bf}_4(\fun{q}(\el,\el)) \rightarrow \el;\\
%
\inferrule
  {\fun{deq}(q) \twoheadrightarrow \pair{q'}{\fun{ext}()}}
  {\fun{bf}_4(q) \twoheadrightarrow \fun{bf}_4(q')};
\quad
\inferrule
  {\fun{deq}(q) \twoheadrightarrow \pair{q'}{\fun{int}(x,t_1,t_2)}}
  {\fun{bf}_4(q) \twoheadrightarrow
   \cons{x}{\fun{bf}_4(\fun{enq}(t_2,\fun{enq}(t_1,q')))}}\,.
\end{gather*}}}
\caption{Raffinement de la \fig~\vref{fig:bf1}}
\label{fig:bf3}
\end{figure}

Nous pouvons spécialiser le programme davantage, de façon à économiser
de la mémoire en n'utilisant pas le constructeur
\fun{q/2}\index{q@\fun{q/2}} et en nous souvenant que son premier
argument est la pile arrière et le second la pile frontale. De plus,
au lieu d'appeler \fun{deq/1}\index{deq@\fun{deq/1}} et
\fun{enq/2}\index{enq@\fun{enq/1}}, nous pouvons insérer leur
définition et les fusionner avec la définition en cours. Le résultat
est montré à la \fig~\vref{fig:bf}.
\begin{figure}
\begin{equation*}
\boxed{
\begin{array}{r@{\;}c@{\;}l}
\fun{bf}(t) & \xrightarrow{\smash{\nu}} & \fun{bf}(\el,[t]).\\
\\
\fun{bf}(\el,\el) & \xrightarrow{\smash{\xi}} & \el;\\
\fun{bf}(\cons{t}{r},\el) & \xrightarrow{\smash{\pi}} &
  \fun{bf}(\el,\fun{rcat}(r,[t]));\\
\fun{bf}(r,\cons{\fun{ext}()}{f})
  & \xrightarrow{\smash{\rho}} & \fun{bf}(r,f);\\
\fun{bf}(r,\cons{\fun{int}(x,t_1,t_2)}{f})
  & \xrightarrow{\smash{\sigma}}
  & \cons{x}{\fun{bf}(\cons{t_2,t_1}{r},f)}.
\end{array}}
\end{equation*}
\caption{Raffinement de la \fig~\vref{fig:bf3}}
\label{fig:bf}
\end{figure}
Souvenons-nous que \fun{rcat/2}\index{rcat@\fun{rcat/2}} (anglais,
\emph{reverse and catenate}) est définie à l'équation~\eqref{def:rev}
\vpageref{def:rev}. Remarquons que
\(\fun{bf}_4(\fun{enq}(t,\fun{q}(\el,\el)))\)\index{bf4@\fun{bf\(_4\)/1}}
a été optimisé en \(\fun{bf}(\el,[t])\)\index{bf@\fun{bf/2}} de
manière à éviter un retournement de pile.

La définition de \fun{bf/1}\index{bf@\fun{bf/1}} peut être considérée
comme le plus \emph{concret} des raffinements, le programme le plus
\emph{abstrait} étant la définition originelle de
\fun{bf\(_1\)/1}\index{bf1@\fun{bf\(_1\)/1}}. La première est plus
courte que \fun{bf\(_0\)/1}\index{bf0@\fun{bf\(_0\)/1}}, mais le gain
véritable est le coût. Soit \(n\)~la taille de l'arbre binaire en
question et \(h\)~sa hauteur\index{arbre binaire!hauteur}. L'emploi
des règles est:
\begin{itemize}

  \item la règle~\(\nu\) est utilisée une fois;

  \item la règle~\(\xi\) est employée une fois;

  \item la règle~\(\pi\) est appliquée une fois par niveau, sauf le
    premier (la racine), donc, au total \(h\)~fois;

  \item tous les niveaux, sauf le premier (la racine), sont retournés
    par \fun{rev/1}: \(\sum_{i=1}^{h}\C{\fun{rev}}{e_i} =
    \sum_{i=1}^{h}(e_i+2) = (n-1)+(n+1)+ 2h = 2n +
    2h\),\index{rev@$\C{\fun{rev}}{n}$} où \(e_i\)~est le nombre de
    n{\oe}uds au niveau étendu~\(i\);

  \item la règle~\(\rho\) est utilisée une fois pour chaque n{\oe}ud
    externe, à savoir, \(n+1\) fois;

  \item la règle~\(\sigma\) est employée une fois pour chaque n{\oe}ud
    interne, donc \(n\)~fois.

\end{itemize}
La somme de toutes ces sommes partielles donne
\begin{equation*}
\C{\fun{bf}}{n,h} = 4n + 3h + 3.\index{bf@$\C{\fun{bf}}{n,h}$}
\end{equation*}
Comme avec \fun{bf\(_0\)/1}\index{bf0@\fun{bf\(_0\)/1}}, le coût
minimum se produit ici si \(h = \lg(n+1)\), donc \(\B{\fun{bf}}{n} =
4n + 3\lg(n+1) + 3 \sim 4n\).\index{bf@$\B{\fun{bf}}{n,h}$} Le coût
maximum se produit quand \(h=n\), d'où \(\W{\fun{bf}}{n} = 7n + 3 \sim
7n\). Nous pouvons maintenant comparer
\fun{bf\(_0\)/1}\index{bf0@\fun{bf\(_0\)/1}} et
\fun{bf/1}\index{bf@\fun{bf/1}}: \(\C{\fun{bf}}{n,h} <
\C{\fun{bf}_0}{n,h}\)\index{bf@$\C{\fun{bf}}{n,h}$}
\index{bf0@$\C{\fun{bf}_0}{n,h}$} et la différence de coût est la plus
notable dans leur cas le plus défavorable, qui est, pour les deux, un
arbre dégénéré. Par conséquent, \fun{bf/1} est préférable dans tous
les cas.

\paragraph{Terminaison}
\hspace*{-1pt}
\index{terminaison!parcours des niveaux|(} \index{arbre
  binaire!parcours des niveaux!terminaison|(} Dans le but de prouver
la terminaison de \fun{bf/2}\index{bf@\fun{bf/2}}, nous réutilisons
l'ordre lexicographique sur les paires de piles, basé sur la relation
de sous-pile immédiate (\(\succ_{S}\))\index{induction!ordre de la
  sous-pile immédiate} qui nous a permis de prouver la terminaison de
\fun{mrg/2}\index{mrg@\fun{mrg/2}} \vpageref{merging_termination}:
\begin{equation*}
(s,t) \succ_{S^2} (s',t') :\Leftrightarrow \text{\(s \succ_{S} s'\) ou
    (\(s = s'\) et \(t \succ_{S} t'\))}.
\end{equation*}
Malheureusement, (\(\succ_{S^2}\)) échoue dans l'ordonnancement
monotone (par rapport à la relation de réécriture) du membre gauche et
du membre droit de la règle~\(\sigma\), à cause de
\((r,\cons{\fun{int}(x,t_1,t_2)}{f}) \nsucc_{S^2}
(\cons{t_2,t_1}{r},f)\). Une autre approche consiste à définir un
ordre bien fondé sur le nombre de n{\oe}uds dans une paire de forêts:
\begin{equation*}
(r,f) \succ_{S^2} (r',f') :\Leftrightarrow \fun{dim}(r) +
  \fun{dim}(f) > \fun{dim}(r') + \fun{dim}(f'),
\end{equation*}
avec
\begin{equation*}
\fun{dim}(\el) \rightarrow \el;\qquad
\fun{dim}(\cons{t}{f}) \rightarrow \fun{size}(t) + \fun{dim}(f).
\end{equation*}
où \fun{size/1}\index{size@\fun{size/1}}\index{arbre binaire!taille}
est définie par~\eqref{eq:size} \vpageref{eq:size}. Ceci est une sorte
de mesure polynomiale\index{terminaison!mesure polynomiale} sur les
paires de dépendance\index{terminaison!paire de dépendance}, dont on
peut en voir une illustration avec
\fun{flat/1}\index{flat@\fun{flat/1}}
\vpageref{flattening_termination}. Ici, \(\measure{\fun{bf}(s,t)} :=
\fun{dim}(s) + \fun{dim}(t)\). Malheureusement, cet ordre échoue dans
l'ordonnancement monotone de la règle~\(\pi\), car \((r,\el)
\nsucc_{S^2} (\el,\fun{rev}(r))\).

L'énigme est résolue si nous visualisons l'ensemble complet des traces
d'évaluation\index{langage fonctionnel!évaluation!trace} de
\fun{bf/2}\index{bf@\fun{bf/2}} d'une manière compacte. Si nous
supposons que \fun{rev/1}\index{rev@\fun{rev/1}} est un constructeur,
les membres droits ne contiennent aucun appel ou exactement un appel
récursif. Les traces des appels à ces définitions sont joliment
représentées comme des automates finis.\index{automate fini} Un
exemple d'automate fini déterministe\index{automate fini!$\sim$
  déterministe|(} (AFD) a été vu à la \fig~\vref{fig:abacabac}. Ici,
une transition est une règle de réécriture et un état correspond à une
abstraction de la donnée. Dans le cas de \fun{bt/2}, la donnée est une
paire de piles. Décidons pour le moment que nous ne prendrons en
compte que le fait qu'une pile est vide ou non, ce qui conduit à
quatre états. Posons que «~\texttt{|}~» dénote une pile non-vide
arbitraire. En examinant les définitions de
\fun{bf/1}\index{bf@\fun{bf/1}} et \fun{bf/2}\index{bf@\fun{bf/2}} à
la \fig~\vref{fig:bf}, nous voyons que
\begin{itemize}

  \item la règle~\(\xi\) s'applique à l'état \((\el,\el)\) uniquement;

  \item la règle~\(\pi\) s'applique à l'état \((\mid,\el)\), et
    conduit à l'état \((\el,\mid)\);

  \item la règle~\(\rho\) s'applique aux états \((\el,\mid)\) et
    \((\mid,\mid)\), et mène à n'importe quel état;

  \item la règle~\(\sigma\) s'applique aux états \((\el,\mid)\) et
    \((\mid,\mid)\), et conduit aux états \((\mid,\el)\) et
    \((\mid,\mid)\).

\end{itemize}
À la \fig~\vref{fig:lo_nfa},
\begin{figure}
\centering
\subfloat[Non-déterministe (AFN)\label{fig:lo_nfa}]{
\includegraphics[bb=71 580 209 721]{lo_nfa}}
\quad
\subfloat[Déterministe (AFD)\label{fig:lo_dfa}]{
\includegraphics[bb=71 580 209 721]{lo_dfa}}
\caption{Traces de \fun{bf/2} comme automate fini}
\end{figure}
nous regroupons toute cette connectivité en un automate fini. Notons
que, par définition, l'état initial possède une transition
entrante~\(\nu\) sans source et que l'état final a une transition
sortante~\(\xi\) sans destination. Une trace est n'importe quelle
séquence de transitions de l'état initial \((\el,\mid)\) à l'état
final \((\el,\el)\), par exemple, \(\nu\rho^p\sigma^q\pi\rho\xi\),
avec \(p \geqslant 0\) et \(q \geqslant 2\). Cet automate est appelé
\emph{automate fini non-déterministe}\index{automate fini!$\sim$
  non-déterministe|(} (AFN) parce qu'un état peut avoir plus d'une
transition sortante avec le même label (considérons l'état initial et
les deux transitions de label~\(\sigma\), par exemple).

Il est toujours possible de construire un \emph{automate fini
  déterministe} (AFD) équivalent à un automate fini non-déterministe
donné \citep{VanLeeuwen_1990c,HopcroftMotwaniUllman_2003}. Les
transitions sortant de chaque état du premier ont un label
distinct. L'équivalence signifie que l'ensemble des traces de chaque
automate est le même. Si «~\texttt{?}~» dénote une pile, vide ou non,
alors la \fig~\vref{fig:lo_dfa} montre un AFD équivalent pour les
traces de \fun{bf/1}\index{bf@\fun{bf/1}} et
\fun{bf/2}\index{bf@\fun{bf/2}}.

Comme nous l'avons observé tantôt en employant l'ordre bien fondé
\((\succ_{S^2})\) basé sur la taille des piles, toutes les transitions
\(x \rightarrow y\) dans l'AFD satisfont \(x \succ_{S^2} y\),
sauf~\(\pi\), pour laquelle \(x =_{S^2} y\) est vrai (le nombre total
de n{\oe}uds est invariant). Nous pouvons néanmoins conclure que
\fun{bf/2} termine parce que la seule façon de ne pas terminer serait
l'existence d'un \emph{circuit}~\(\pi\), à savoir, une suite de
transitions successives d'un état vers lui-même, toutes portant le
label~\(\pi\), donc le long de laquelle le nombre de n{\oe}uds est
constant. En fait, toutes les traces contiennent~\(\pi\) suivie
de~\(\rho\) ou~\(\sigma\).

Encore un autre point de vue sur ce sujet serait de prouver, en
examinant toutes les règles isolément et toutes les compositions de
deux règles, que
\begin{equation*}
x \rightarrow y \Rightarrow x \succcurlyeq:{S^2} y
\quad \text{et} \quad
x \xrightarrow{\smash{2}} y \Rightarrow x \succ_{S^2} y.
\end{equation*}
Par conséquent, si \(n > 0\), alors \(x \xrightarrow{\smash{2n}} y\)
implique \(x \succ_{S^2} y\), car \((\succ_{S^2})\) est transitive. De
plus, si \(x \xrightarrow{\smash{2n}} y \rightarrow z\), alors \(x
\succ_{S^2} y \succcurlyeq:{S^2} z\), donc \(x \succ_{S^2} z\). Par
conséquent, \(x \xrightarrow{\smash{n}} y\) pour tout \(n > 1\), donc
\(x \succ_{S^2} y\).\index{automate fini!$\sim$
  non-déterministe|)}\index{automate fini!$\sim$
  déterministe|)}\index{arbre binaire!parcours des
  niveaux!terminaison|)}\index{terminaison!parcours des
  niveaux|)}\hfill\(\Box\)

\paragraph{Numérotation en largeur}
\index{arbre binaire!numérotation en largeur|(}

Comme nous l'avons mentionné plus tôt, la \fig~\vref{fig:lorder}
montre un exemple de numérotation en largeur. Ce problème a reçu une
attention particulière
\citep{JonesGibbons_1993,GibbonsJones_1998,Okasaki_2000} parce que les
programmeurs en langages fonctionnels se sentent d'habitude défiés
avec cet exercice. Une bonne démarche consiste à modifier la fonction
\fun{bf\(_1\)/2}\index{bf1@\fun{bf\(_1\)/2}} à la \fig~\vref{fig:bf1}
de telle sorte qu'elle construise un arbre au lieu d'une pile. Nous y
parvenons en enfilant ses sous-arbres immédiats, qui seront ainsi
récursivement numérotés après leur défilage, et en incrémentant un
compteur, initialisé à~\(0\), à chaque fois qu'un arbre non-vide est
défilé.

Considérons \fun{bfn\(_1\)/1}\index{bfn1@\fun{bfn\(_1\)/1}} et
\fun{bfn\(_2\)/2}\index{bfn2@\fun{bfn\(_2\)/2}} (anglais,
\emph{breadth-first numbering}) à la \fig~\vref{fig:bfn1},
\begin{figure}
\abovedisplayskip=0pt
\belowdisplayskip=0pt
\centering
\framebox[0.87\columnwidth]{\vbox{%
\begin{gather*}
\inferrule
  {\fun{bfn}_2(0,\enq{t}{\emq}) \twoheadrightarrow \deq{t'}{\emq}}
  {\fun{bfn}_1(t) \twoheadrightarrow t'}.
\\
\fun{bfn}_2(i,\emq) \rightarrow \emq;
\qquad
\fun{bfn}_2(i,\deq{\fun{ext}()}{q}) \rightarrow
  \enq{\fun{ext}()}{\fun{bfn}_2(i,q)};
\\
\inferrule
  {\fun{bfn}_2(i+1,\enq{t_2}{\enq{t_1}{q}}) \twoheadrightarrow
   \deq{t'_2}{\deq{t'_1}{q'}}}
  {\fun{bfn}_2(i,\deq{\fun{int}(x,t_1,t_2)}{q}) \twoheadrightarrow
   \enq{\fun{int}(i,t'_1,t'_2)}{q'}}.
\end{gather*}}}
\caption{Numérotation en largeur abstraite}
\label{fig:bfn1}
\end{figure}
et comparons-les avec les définitions à la \fig~\vref{fig:bf1}. En
particulier, notons que, à la différence de
\fun{bf\(_1\)/2}\index{bf1@\fun{bf\(_1\)/2}}, les n{\oe}uds externes
sont enfilés au lieu d'être écartés, parce qu'il sont ultérieurement
nécessaires pour construire l'arbre numéroté.

À la \fig~\vref{fig:bf2_ex}
\begin{figure}[t]
\centering
\includegraphics{bf2_abcd_0}
\includegraphics{bf2_abcd_1}
\includegraphics{bf2_abcd_2}
\includegraphics{bf2_abcd_3}
\includegraphics[bb=70 674 275 721]{bf2_abcd_4}
\includegraphics[bb=69 691 276 721]{bf2_abcd_5}
\includegraphics[bb=71 690 276 721]{bf2_abcd_6}
\caption{Exemple de numérotation en largeur}
\label{fig:bf2_ex}
\end{figure}
est montré un exemple où les exposants représentent les valeurs
de~\(i\) (le premier argument de
\fun{bf\(_2\)/2}\index{bf2@\fun{bf\(_2\)/2}}), les réécritures
descendantes définissent les états successifs de la file de travail
(le second argument de \fun{bf\(_2\)/2}\index{bf2@\fun{bf\(_2\)/2}}),
et les réécritures ascendantes montrent les états successifs des files
résultantes (membre droit de
\fun{bf\(_2\)/2}\index{bf2@\fun{bf\(_2\)/2}}). Souvenons-nous que les
arbres sont enfilés à gauche et défilés à droite (d'autres auteurs
utilisent la convention inverse, comme \cite{Okasaki_2000}) et faisons
attention au fait que, dans les réécritures verticales à gauche,
\(t_1\)~est enfilé d'abord, alors que, sur la droite, \(t'_2\)~est
défilé en premier, ce qui devient plus clair lorsque nous comparons
\(\enq{t_2}{\enq{t_1}{q}} =
\enq{t_2}{(\enq{t_1}{q})}\)\index{enq@$\enq{x}{q}$} avec
\(\deq{t'_2}{\deq{t'_1}{q'}} =
\deq{t'_2}{(\deq{t'_1}{q'})}\)\index{deq@$\deq{x}{q}$}.

Nous pouvons raffiner \fun{bfn\(_1\)/1}\index{bfn1@\fun{bfn\(_1\)/1}}
et \fun{bfn\(_2\)/2}\index{bfn2@\fun{bfn\(_2\)/2}} en introduisant
explicitement les appels de fonction pour enfiler et défiler, comme
cela est montré à la \fig~\vref{fig:bfn3},
\begin{figure}[!t]
\centering
\includegraphics[bb=71 599 403 721]{bfn3}
\caption{Raffinement de la \fig~\vref{fig:bfn1}}
\label{fig:bfn3}
\end{figure}
ce qui pourrait être contrasté avec la \fig~\vref{fig:bf3}.

\paragraph{Exercices}
\begin{enumerate}

  \item Comment prouveriez-vous la correction de
  \fun{bfn/1}?\index{bfn@\fun{bfn/1}}

\medskip

  \item Trouvez le coût \(\C{\fun{bfn}}{n}\) de \(\fun{bfn}(t)\), où
  \(n\)~est la taille
  de~\(t\).\index{bfn@$\C{\fun{bfn}}{n}$}\index{bfn@\fun{bfn/1}}

\end{enumerate}
\index{arbre binaire!numérotation en largeur|)}
\index{arbre binaire!parcours des niveaux|)}

\clearpage

\section{Formes classiques}

Dans cette section, nous passons brièvement en revue quelques arbres
binaires particuliers qui se révèlent utiles lors de la recherche des
extremums du coûts de nombreux algorithmes.

\paragraph{Perfection}
\label{par:perfection}

\index{arbre binaire!$\sim$ parfait|(} Nous avons déjà mentionné ce
qu'est un \emph{arbre binaire parfait} dans le contexte du tri optimal
(voir \fig~\vref{fig:comp_tree}). Une façon de définir de tels arbres
consiste à dire que tous leurs n{\oe}uds externes appartiennent au
même niveau ou, de manière équivalente, que les sous-arbres immédiats
de tous les n{\oe}uds ont la même hauteur\index{arbre!hauteur}. (La
hauteur d'un n{\oe}ud externe est~\(0\).) À la \fig~\vref{fig:per} est
montrée la définition de \fun{per/1}\index{per@\fun{per/1}}
(\emph{perfection}), où \fun{true/1} est un contructeur. Donc, si un
arbre~\(t\) est parfait, nous connaissons aussi sa hauteur~\(h\)
puisque \(\fun{per}(t) \twoheadrightarrow \fun{true}(h)\).

\bigskip

\begin{figure}[h]
\centering
\includegraphics{per}
\caption{Vérification de la perfection}
\label{fig:per}
\end{figure}

\smallskip

\noindent Notons que les règles sont ordonnées, donc la dernière ne
pourrait être sélectionnée que si les précédentes ont échoué à filtrer
l'appel en cours.

Un raffinement sans règles d'inférence est montré à la
\fig~\vref{fig:per0}, où \(\fun{per}_0(t_1)\) est évalué avant
\(\fun{per}_0(t_2)\)\index{per0@\fun{per\(_0\)/1}} puisque
\(\fun{t}(\fun{per}_0(t_1), \fun{per}_0(t_2))\) est lent si
\(\fun{per}_0(t_1) \twoheadrightarrow \fun{false}()\).

\bigskip

\begin{figure}[h]
\centering
\includegraphics[bb=71 604 251 721]{per0}
\caption{Raffinement de la \fig~\ref{fig:per}}
\label{fig:per0}
\end{figure}
\index{arbre binaire!$\sim$ parfait|)}


\paragraph{Complétude}

Un arbre binaire est \emph{complet}\index{arbre binaire!$\sim$
  complet} si les enfants de chaque n{\oe}ud interne sont deux
n{\oe}uds externes ou deux n{\oe}uds internes. Un exemple est illustré
à la \fig~\ref{fig:complete}. Récursivement, un arbre donné est
complet si, et seulement si, ses sous-arbres immédiats sont complets.
% Wrapping figure better declared before a paragraph
%
\begin{wrapfigure}[7]{r}[0pt]{0pt}
% [7] vertical lines
% {r} mandatory right placement
% [0pt] of margin overhang
\centering
\includegraphics{complete}
\caption{}
\label{fig:complete}
\end{wrapfigure}
Ceci est la même règle que nous avons utilisée pour la perfection. En
d'autres termes, la perfection et la complétude sont propagées de
façon ascendante. Par conséquent, nous devons décider que dire à
propos des n{\oe}uds externes, en particulier, l'arbre vide. Si nous
décidons que ce dernier est complet, alors
\(\fun{int}(x,\fun{ext}(),\fun{int}(y,\fun{ext}(),\fun{ext}()))\)
serait, incorrectement, considéré complet. Sinon, les
feuilles\index{arbre binaire!feuille}
\(\fun{int}(x,\fun{ext}(),\fun{ext}())\) seraient, incorrectement,
considérées incomplètes. Donc, nous pouvons opter pour l'une ou
l'autre possibilité et traiter exceptionnellement le cas problématique
correspondant; par exemple, nous pourrions décider que les n{\oe}uds
externes sont incomplets, mais que les feuilles sont des arbres
complets. Le programme se trouve à la \fig~\vref{fig:comp}. La
dernière règle s'applique si \(t=\fun{ext}()\) ou
\(t=\fun{int}(x,t_1,t_2)\), avec \(t_1\) ou~\(t_2\) incomplet.

\bigskip

\begin{figure}[h]
\centering
\includegraphics{comp}
\caption{Verification de la complétude}
\label{fig:comp}
\end{figure}

\paragraph{Équilibre}

La dernière sorte d'arbre binaire intéressante est celle des
\emph{arbres équilibrés}.\index{arbre binaire!$\sim$ équilibré} Il y a
deux espèces de critères pour définir l'équilibre:
\begin{wrapfigure}[7]{r}[0pt]{0pt}
\centering
\includegraphics[bb=71 653 153 721]{bal}
\caption{}
\label{fig:bal}
\end{wrapfigure}
la hauteur ou la taille
\citep{NievergeltReingold_1972,HiraiYamamoto_2011}. Dans ce dernier
cas, les enfants d'un même parent sont les racines d'arbres de tailles
similaires; dans le premier cas, ils ont des hauteurs similaires. Le
critère le plus courant étant la hauteur, nous l'utiliserons dans la
suite. De plus, le sens exact de «~similaire~» dépend de
l'algorithme. Par exemple, nous pourrions décider que deux arbres dont
les hauteurs diffèrent au plus de~\(1\) sont de hauteurs
similaires. Voir la \fig~\vref{fig:bal} pour un exemple. Commençons
avec une définition de la hauteur\index{arbre
  binaire!hauteur}\index{height@\fun{height/1}} d'un arbre binaire et
modifions-la pour obtenir une fonction vérifiant l'équilibre de
l'arbre:
\begin{equation*}
\fun{height}(\fun{ext}()) \rightarrow 0;
\;
\fun{height}(\fun{int}(x,t_1,t_2)) \rightarrow 1 +
\max\{\fun{height}(t_1), \fun{height}(t_2)\}.
\end{equation*}
La modification est montrée à la \fig~\vref{fig:balance},
\begin{figure}
\centering
\includegraphics{balance}
\caption{Vérification de l'équilibre}
\label{fig:balance}
\end{figure}
où la règle d'inférence de
\fun{bal\(_0\)/1}\index{bal0@\fun{bal\(_0\)/1}} est requise pour
vérifier la condition \(|h_1 - h_2| \leqslant 1\). Remarquons qu'un
arbre parfait est équilibré (\(h_1 = h_2\)).

\section{Codages d'arbres}

En général, de nombreux arbres binaires ont le même parcours préfixe,
postfixe ou infixe, donc il n'est pas possible de reconstruire l'arbre
originel à partir d'un seul parcours. Le problème de la représentation
unique d'un arbre binaire par une structure linéaire est appelé
\emph{codage d'arbre} \citep{Makinen_1991} et est relié au problème de
la génération de tous les arbre binaires d'une taille donnée
\citep[7.2.1.6]{Knuth_2011}. Une approche simple consiste à étendre un
parcours avec les n{\oe}uds externes; ainsi, nous retenons dans le
code suffisamment d'information sur les arbres, ce qui permet de
revenir sans ambiguïté vers l'arbre initial.

La fonction de codage
\fun{epost/1}\index{epost@\fun{epost/1}}\index{epost@\fun{epost/2}}
(anglais, \emph{extended postorder})\index{arbre binaire!parcours
  postfixe!codage} à la \fig~\vref{fig:epost},
\begin{figure}[b]
\begin{equation*}
\boxed{
\begin{array}{r@{\;}c@{\;}l}
\fun{epost}(t) & \rightarrow & \fun{epost}(t,\el).\\
\\
\fun{epost}(\fun{ext}(),s) & \rightarrow
  & \cons{\fun{ext}()}{s};\\
\fun{epost}(\fun{int}(x,t_1,t_2),s) & \rightarrow
  & \fun{epost}(t_1,\fun{epost}(t_2,\cons{x}{s})).
\end{array}}
\end{equation*}
\caption{Codage postfixe}
\label{fig:epost}
\end{figure}
est une simple modification de \fun{post/1}\index{post@\fun{post/1}} à
la \fig~\vref{fig:post}. Par exemple, l'arbre à la
\fig~\vref{fig:postorder} est codé par \([\Box, \Box, \Box, 0, 1,
\Box, \Box, 2, 3, \Box, \Box, \Box, 4, 5, 6]\), où \(\Box\) dénote
\(\fun{ext}()\). Puisqu'un arbre binaire avec \(n\)~n{\oe}uds internes
a \(n+1\) n{\oe}uds externes (voir théorème~\ref{thm_int_ext}
\vpageref{thm_int_ext}), le coût est très facile à déterminer:
\(\C{\fun{epost}}{n} = 2n + 2\).\index{epost@$\C{\fun{epost}}{n}$}

En ce qui concerne le décodage, nous avons déjà remarqué que la
numérotation postfixe augmente le long de chemins ascendants, ce qui
correspond à l'ordre dans lequel l'arbre est construit: des n{\oe}uds
externes jusqu'à la racine. Par conséquent, tout ce que nous avons à
faire est d'identifier les sous-arbres qui croissent en mettant les
rangs inutilisés et les sous-arbres dans une pile auxiliaire: quand le
contenu d'une racine apparaît dans le code (tout sauf
\(\fun{ext}()\)), nous pouvons créer un n{\oe}ud interne avec les deux
premiers sous-arbres dans la pile auxilaire.

La définition de \fun{post2b/1}\index{post2b@\fun{post2b/1}}
(anglais, \emph{extended postorder to binary tree}) est donnée à la
\fig~\vref{fig:post2b}.\index{arbre binaire!postfixe!décodage}
\begin{figure}
\begin{equation*}
\boxed{
\begin{array}{r@{\;}c@{\;}l}
\fun{post2b}(s) & \rightarrow & \fun{post2b}(\el,s).\\
\\
\fun{post2b}([t],\el) & \rightarrow & t;\\
\fun{post2b}(f,\cons{\fun{ext}()}{s}) & \rightarrow & \fun{post2b}(\cons{\fun{ext}()}{f},s);\\
\fun{post2b}(\cons{t_2,t_1}{f},\cons{x}{s}) & \rightarrow & \fun{post2b}(\cons{\fun{int}(x,t_1,t_2)}{f},s).
\end{array}}
\end{equation*}
\caption{Décodage postfixe}
\label{fig:post2b}
\end{figure}
La variable~\(f\) est une forêt\index{arbre!forêt}, c'est-à-dire une
pile d'arbres. Remarquons que nous aurions un problème si l'arbre
originel contenait des arbres, parce que, dans ce cas, un n{\oe}ud
externe contenu dans un n{\oe}ud interne tromperait
\fun{post2b/1}\index{post2b@\fun{post2b/1}}. Le coût est facile à
déterminer car un code postfixe doit avoir la longueur \(2n+1\), ce
qui est le nombre total de n{\oe}uds d'un arbre binaire avec
\(n\)~n{\oe}uds internes. Par conséquent, \(\C{\fun{post2b}}{n} = 2n +
3\).\index{post2b@$\C{\fun{post2b}}{n}$} Le théorème attendu est, bien
entendu,
\begin{equation}
\fun{post2b}(\fun{epost}(t)) \equiv t.
\label{thm_post2b_epost}
\end{equation}

En passant au codage préfixe, la fonction de codage
\fun{epre/1}\index{epre@\fun{epre/1}}\index{epre@\fun{epre/2}}
(anglais, \emph{extended preorder}) à la \fig~\vref{fig:epre}
\begin{figure}[b]
\begin{equation*}
\boxed{
\begin{array}{r@{\;}c@{\;}l}
\fun{epre}(t) & \rightarrow & \fun{epre}(t,\el).\\
\\
\fun{epre}(\fun{ext}(),s) & \rightarrow
  & \cons{\fun{ext}()}{s};\\
\fun{epre}(\fun{int}(x,t_1,t_2),s) & \rightarrow
  & \cons{x}{\fun{epre}(t_1,\fun{epre}(t_2,s))}.
\end{array}}
\end{equation*}
\caption{Codage préfixe}
\label{fig:epre}
\end{figure}
est une simple modification de \fun{pre/1}\index{pre@\fun{pre/1}} à la
\fig~\vref{fig:pre}. Le coût est aussi simple à déterminer que pour un
codage postfixe: \(\C{\fun{epre}}{n} =
2n+2\).\index{epre@$\C{\fun{epre}}{n}$}

Trouver la fonction inverse, du code préfixe vers l'arbre binaire, est
un peu plus délicat qu'avec les codes postfixes, parce que les rangs
préfixes augmentent le long des chemins descendants dans l'arbre, ce
qui est la direction opposée à la croissance des arbres (les
programmeurs font croître les arbres des feuilles vers la racine). Une
solution est de se souvenir de la relation
\(\pred{PreMir}{t}\)\index{PreMir@\predName{PreMir}} entre les
parcours préfixe et postfixe que nous avons prouvée
\vpageref{proof_PreMir}:
\begin{equation*}
\abovedisplayskip=7pt
\belowdisplayskip=7pt
\fun{pre}(\fun{mir}(t)) \equiv \fun{rev}(\fun{post}(t)).
\end{equation*}
Nous devrions étendre la preuve de ce théorème pour avoir
\begin{equation}
\abovedisplayskip=7pt
\belowdisplayskip=7pt
\fun{epre}(\fun{mir}(t)) \equiv \fun{rev}(\fun{epost}(t)).
\label{thm_epre_epost}
\index{rev@\fun{rev/1}}
\end{equation}
À la section~\ref{sec:reversal}, nous avons prouvé
\(\pred{Inv}{s}\)\index{Inv@\predName{Inv}} et
\(\pred{EqRev}{s}\)\index{EqRev@\predName{EqRev}}, soit
l'involution de
\fun{rev/1}\index{rev@\fun{rev/1}}\index{pile!retournement!involution}:
\begin{equation}
\abovedisplayskip=7pt
\belowdisplayskip=7pt
\fun{rev}(\fun{rev}(s)) \equiv t.
\label{thm_rev_inv}
\end{equation}
Les propriétés~\eqref{thm_rev_inv} et~\eqref{thm_epre_epost} mènent à
\begin{equation*}
\abovedisplayskip=7pt
\belowdisplayskip=7pt
\fun{rev}(\fun{epre}(\fun{mir}(t))) \equiv
\fun{rev}(\fun{rev}(\fun{epost}(t))) \equiv \fun{epost}(t).
\end{equation*}
L'application de~\eqref{thm_post2b_epost} donne
\begin{equation*}
\abovedisplayskip=7pt
\belowdisplayskip=7pt
\fun{post2b}(\fun{rev}(\fun{epre}(\fun{mir}(t))) \equiv
\fun{post2b}(\fun{epost}(t)) \equiv t.
\end{equation*}
D'après l'exercice~\ref{ex_mir_mir} \vpageref{ex_mir_mir}, nous avons
\(\fun{mir}(\fun{mir}(t)) \equiv t\), par conséquent
\begin{equation*}
\abovedisplayskip=7pt
\belowdisplayskip=7pt
\fun{post2b}(\fun{rev}(\fun{epre}(t))) \equiv \fun{mir}(t),
\quad\text{donc}\quad
\fun{mir}(\fun{post2b}(\fun{rev}(\fun{epre}(t)))) \equiv t.
\end{equation*}
Puisque nous voulons que le codage suivi du décodage soit l'identité,
\(\fun{pre2b}(\fun{epre}(t)) \equiv t\), nous avons
\(\fun{pre2b}(\fun{epre}(t)) \equiv
\fun{mir}(\fun{post2b}(\fun{rev}(\fun{epre}(t))))\), c'est-à-dire, en
posant la pile \(s := \fun{epre}(t)\),
\begin{equation*}
\abovedisplayskip=7pt
\belowdisplayskip=7pt
\fun{pre2b}(s) \equiv \fun{mir}(\fun{post2b}(\fun{rev}(s))),
\end{equation*}
Nous obtenons donc
\fun{pre2b/1}\index{pre2b@\fun{pre2b/1}}\index{arbre
  binaire!préfixe!décodage} en modifiant
\fun{post2b/1}\index{post2b@\fun{post2b/1}} à la
\fig~\vref{fig:pre2b}.
\begin{figure}[b]
\begin{equation*}
\boxed{
\begin{array}{r@{\;}c@{\;}l}
\fun{pre2b}(s) & \rightarrow & \fun{pre2b}(\el,\fun{rev}(s)).\\
\\
\fun{pre2b}([t],\el) & \rightarrow & t;\\
\fun{pre2b}(f,\cons{\fun{ext}()}{s}) & \rightarrow & \fun{pre2b}(\cons{\fun{ext}()}{f},s);\\
\fun{pre2b}(\cons{t_1,t_2}{f},\cons{x}{s}) & \rightarrow &
\fun{pre2b}(\cons{\fun{int}(x,t_1,t_2)}{f},s).
\end{array}}
\end{equation*}
\caption{Décodage préfixe}
\label{fig:pre2b}
\end{figure}
La différence entre \fun{pre2b/2}\index{pre2b@\fun{pre2b/2}} et
\fun{post2b/2}\index{post2b@\fun{post2b/2}} réside dans leur dernier
motif, à savoir \(\fun{post2b}(\cons{t_2,t_1}{f},\cons{x}{s})\) par
opposition à \(\fun{pre2b}(\cons{t_1,t_2}{f},\cons{x}{s})\), réalisant
la fusion de \fun{mir/1} et \fun{post2b/1}. Malheureusement, le coût
de \(\fun{pre2b}(t)\) \index{pre2b@\fun{pre2b/1}} est plus élevé que
le coût de \(\fun{post2b}(t)\) à cause du retournement de pile
\(\fun{rev}(s)\) au début: \(\C{\fun{pre2b}}{n} = 2n + 3 +
\C{\fun{rev}}{n} = 3n +
5\).\index{pre2b@$\C{\fun{pre2b}}{n}$}\index{rev@$\C{\fun{rev}}{n}$}

La conception de \fun{pre2b/1}\index{pre2b@\fun{pre2b/1}} est à petits
pas, avec un accumulateur. Une approche plus directe extrairait le
sous-arbre gauche et puis le sous-arbre droit du reste du code. En
d'autres termes, la nouvelle version
\(\fun{pre2b}_1(s)\)\index{pre2b1@\fun{pre2b\(_1\)/1}} calcule un
arbre construit à partir d'un préfixe du code~\(s\), accouplé avec le
suffixe. La définition est montrée à la \fig~\vref{fig:pre2b0}.
\begin{figure}
\abovedisplayskip=0pt
\belowdisplayskip=0pt
\centering
\framebox[0.87\columnwidth]{\vbox{%
\begin{gather*}
\inferrule
   {\fun{pre2b}_1(s) \twoheadrightarrow \pair{t}{\el}}
   {\fun{pre2b}_0(s) \twoheadrightarrow t}.
\qquad
\fun{pre2b}_1(\cons{\fun{ext}()}{s}) \rightarrow
\pair{\fun{ext}()}{s};
\\
\inferrule
  {\fun{pre2b}_1(s) \twoheadrightarrow \pair{t_1}{s_1}
   \and
   \fun{pre2b}_1(s_1) \twoheadrightarrow \pair{t_2}{s_2}}
  {\fun{pre2b}_1(\cons{x}{s}) \twoheadrightarrow \pair{\fun{int}(x,t_1,t_2)}{s_2}}.
\end{gather*}}}
\caption{Un autre décodage préfixe}
\label{fig:pre2b0}
\end{figure}
Notons l'absence de tout concept adventice, contrairement à
\fun{pre2b/1}\index{pre2b@\fun{pre2b/1}}, qui s'appuie sur le
retournement d'une pile\index{pile!retournement} et un théorème à
propos de la réflexion d'arbres et des parcours postfixes. Donc
\fun{pre2b\(_0\)/1}\index{pre2b0@\fun{pre2b\(_0\)/1}} est
conceptuellement plus simple, bien que son coût soit supérieur à celui
de \fun{pre2b/1}\index{pre2b@\fun{pre2b/1}} parce que nous comptons le
nombre d'appels de fonction après que les règles
d'inférence\index{système d'inférence} sont traduites dans le noyau du
langage fonctionnel (donc deux appels supplémentaires filtrant
\(\pair{t_1}{s_1}\) et \(\pair{t_2}{s_2}\) sont implicites).

Les codages d'arbres montrent qu'il est possible de représenter de
façon compacte des arbres binaires, du moment que nous ne nous
soucions pas du contenu des n{\oe}uds internes. Par exemple, nous
avons mentionné que l'arbre à la \fig~\vref{fig:postorder} a pour
parcours postfixe étendu\index{arbre binaire!parcours postfixe!codage}
\([\Box, \Box, \Box, 0, 1, \Box, \Box, 2, 3, \Box, \Box, \Box, 4, 5,
6]\). Si nous ne souhaitons retenir que la forme de l'arbre, nous
pourrions remplacer le contenu des n{\oe}uds internes par~\(0\) et les
n{\oe}uds externes par~\(1\), dont nous tirons le code
\([1,1,1,0,0,1,1,0,0,1,1,1,0,0,0]\). Un arbre binaire de taille~\(n\)
peut être représenté de manière unique par un nombre binaire de
\(2n+1\) bits. En fait, nous pouvons écarter le premier bit parce que
les deux premiers bits sont toujours~\(1\), donc \(2n\) bits sont en
fait suffisants. Pour un parcours préfixe étendu, nous choisissons de
coder les n{\oe}uds externes par~\(0\) et les n{\oe}uds internes
par~\(1\), donc l'arbre de la \fig~\vref{fig:preorder} donne \([0, 1,
2, \Box, 3, \Box, \Box, 4, \Box, \Box, 5, \Box, 6, \Box, \Box]\) et
\((111010010010100)_2\). Nous pouvons aussi écarter le bit le plus à
droite, car les deux derniers bits sont toujours~\(0\).

\section{Parcours aléatoires}

Certaines applications requièrent un parcours d'arbre qui dépende
d'une interaction avec un usager ou un autre logiciel, c'est-à-dire
qu'à l'arbre est adjointe une notion de n{\oe}ud courant, donc le
prochain n{\oe}ud a être visité peut être choisi parmi n'importe
lequel de ses enfants, le parent ou même un frère. Cette interactivité
contraste avec les parcours en préfixe, infixe et postfixe, où l'ordre
de visite est prédéterminé et ne peut être changé durant le parcours.

Normalement, la visite d'une structure de donnée fonctionnelle
commence toujours à la même place, par exemple, dans le cas d'une
pile, il s'agit de l'élément au sommet, et, dans le cas d'un arbre, la
racine. Parfois, mettre à jour une structure de donnée avec un
algorithme en-ligne (voir \vpageref{par:online_vs_offline} et
section~\ref{sec:online} \vpageref{sec:online}) nécessite de garder un
accès direct «~dans~» la structure de donnée, habituellement là où la
dernière mise à jour a eu lieu, ou près d'elle, dans l'optique d'un
coût amorti plus faible (voir page~\pageref{par:amortised_cost}) ou
d'un coû moyen inférieur (voir l'insertion bidirectionnelle à la
section~\ref{sec:2-way} \vpageref{sec:2-way}).

\begin{wrapfigure}[7]{r}[0pt]{0pt}
\centering
\includegraphics[bb=60 663 120 721]{binzip}
\caption{}
\label{fig:binzip}
\end{wrapfigure}
Appelons le n{\oe}ud courant le \emph{curseur}\index{arbre
  binaire!curseur}. Une \emph{crémaillère}\index{arbre
  binaire!crémaillère} dans un arbre binaire est faite d'un
sous-arbre, dont la racine est le curseur, et d'un \emph{chemin} de
celui-ci jusqu'à la racine. Ce chemin est la réification de la
\emph{pile des appels}\index{langage fonctionnel!pile des appels}
récursifs qui ont abouti au curseur, mais retournée, et les
\emph{sous-arbres qui n'ont pas été visités lors de la descente}.  De
cette façon, en un coup, il devient possible de visiter les enfants
dans n'importe quel ordre, le parent ou le frère. Considérons la
\fig~\ref{fig:binzip}, où le curseur est le n{\oe}ud~\fun{d}. La
crémaillère est la paire
\begin{equation*}
\pair{\fun{int}(\fun{d}(),\fun{int}(\fun{e}(),\fun{ext}(),\fun{ext}()),
\fun{int}(\fun{f}(),\fun{ext}(),\fun{ext}()))}%
{[p_1,p_2]},
\end{equation*}
où \(p_1 := \fun{right}(\fun{b}(), \fun{int}(\fun{c}(), \fun{ext}(),
\fun{ext}()))\) signifie que le n{\oe}ud~\fun{b} possède un enfant à
gauche~\fun{c} qui n'a pas été visité (ou, de manière équivalente,
nous avons bifurqué à droite lorsque nous sommes descendu
jusqu'à~\fun{d}), et \(p_2 := \fun{left}(\fun{a}(),
\fun{int}(\fun{g}(), \fun{ext}(), \fun{ext}()))\) signifie que le
n{\oe}ud~\fun{a} a un enfant à droite~\fun{g} qui n'a pas été
visité. La pile \([p_1,p_2]\) est le chemin du n{\oe}ud~\fun{d} vers
le haut, jusqu'à la racine. Bien sûr, puisque \fun{b}~est le premier
dans le chemin ascendant, il est le parent du curseur~\fun{d}.

Au début, la crémaillère ne contient que l'arbre originel~\(t\):
\(\pair{t}{\el}\). Ensuite, les opérations que nous désirons pour
parcourir un arbre binaire à la demande sont
\fun{up/1}\index{up@\fun{up/1}} (se déplacer jusqu'au parent),
\fun{left/1}\index{left@\fun{left/1}} (visiter l'enfant à gauche),
\fun{right/1}\index{right@\fun{right/1}} (voir l'enfant à droite) et
\fun{sibling/1}\index{sibling@\fun{sibling/1}} (aller au frère).
Toutes prennent une crémaillère en argument et, après n'importe lequel
de ces déplacements, une nouvelle crémaillère\index{arbre
  binaire!crémaillère} est assemblée. Voir le programme à la
\fig~\vref{fig:zip_move}
\begin{figure}
\begin{equation*}
\boxed{
\begin{array}{r@{\;}c@{\;}l}
\fun{up}(\pair{t_1}{\cons{\fun{left}(x,t_2)}{p}}) & \rightarrow &
  \pair{\fun{int}(x,t_1,t_2)}{p};\\
\fun{up}(\pair{t_2}{\cons{\fun{right}(x,t_1)}{p}}) & \rightarrow &
\pair{\fun{int}(x,t_1,t_2)}{p}.\\
\\
\fun{left}(\pair{\fun{int}(x,t_1,t_2)}{p}) & \rightarrow &
  \pair{t_1}{\cons{\fun{left}(x,t_2)}{p}}.\\
\\
\fun{right}(\pair{\fun{int}(x,t_1,t_2)}{p}) & \rightarrow &
  \pair{t_2}{\cons{\fun{right}(x,t_1)}{p}}.\\
\\
\fun{sibling}(\pair{t_1}{\cons{\fun{left}(x,t_2)}{p}}) & \rightarrow &
  \pair{t_2}{\cons{\fun{right}(x,t_1)}{p}};\\
\fun{sibling}(\pair{t_2}{\cons{\fun{right}(x,t_1)}{p}}) & \rightarrow&
  \pair{t_1}{\cons{\fun{left}(x,t_2)}{p}}.
\end{array}}
\end{equation*}
\caption{Déplacements dans un arbre binaire}
\label{fig:zip_move}
\end{figure}

Au-delà de parcours aléatoires dans un arbre binaire, cette technique,
qui est un cas particulier de la \emph{crémaillère de Huet}
\citep{Huet_1997}, permet aussi d'éditer localement la structure de
donnée.  Ceci revient en effet a simplement remplacer l'arbre courant
par un autre:
\begin{equation*}
\fun{graft}(t',\pair{t}{p}) \rightarrow \pair{t'}{p}.
\end{equation*}
Si nous ne souhaitons changer que le curseur, nous utiliserions
\begin{equation*}
\fun{slider}(x',\pair{\fun{int}(x,t_1,t_2)}{p}) \rightarrow
\pair{\fun{int}(x',t_1,t_2)}{p}.
\end{equation*}
Si nous voulons monter jusqu'à la racine et extraire le nouvel arbre:
\begin{equation*}
\fun{zip}(\pair{t}{\el}) \rightarrow t;\quad
\fun{zip}(z) \rightarrow \fun{zip}(\fun{up}(z)).
\end{equation*}

Nous n'avons pas besoin d'une crémaillère\index{arbre
  binaire!crémaillère} pour effectuer un parcours préfixe, infixe ou
postfixe, parce qu'elle est principalement conçue pour ouvrir vers le
bas et refermer vers le haut des chemins commençant à la racine d'un
arbre, à la manière d'une crémaillère dans un vêtement. Néanmoins, si
nous retenons un aspect de sa conception, à savoir l'accumulation de
n{\oe}uds et sous-arbres à visiter, nous pouvons définir les parcours
classiques en \emph{forme terminale}\index{langage fonctionnel!forme
  terminale}, c'est-à-dire que les membres droits sont soit des
valeurs ou un appel de fonction dont les arguments ne sont pas des
appels eux-mêmes. (De telles définitions sont équivalentes à des
\emph{boucles} dans les langages impératifs et peuvent être optimisées
par des compilateurs \citep{Appel_1992}.) Un parcours préfixe terminal
est montré à la \fig~\vref{fig:pre_tail},
\begin{figure}[b]
\begin{equation*}
\boxed{
\begin{array}{r@{\;}c@{\;}l}
\fun{pre}_7(t) & \rightarrow & \fun{pre}_8(\el,\el,t).\\
\\
\fun{pre}_8(s,\el,\fun{ext}()) & \rightarrow & s;\\
\fun{pre}_8(s,\cons{\fun{int}(x,t_1,\fun{ext}())}{f},\fun{ext}()) &
\rightarrow & \fun{pre}_8(s,\cons{x}{f},t_1);\\
\fun{pre}_8(s,\cons{x}{f},\fun{ext}()) & \rightarrow &
\fun{pre}_8(\cons{x}{s},f,\fun{ext}());\\
\fun{pre}_8(s,f,\fun{int}(x,t_1,t_2)) & \rightarrow &
\fun{pre}_8(s,\cons{\fun{int}(x,t_1,\fun{ext}())}{f},t_2).
\end{array}}
\end{equation*}
\caption{Parcours préfixe en forme terminale}
\label{fig:pre_tail}
\end{figure}
où, dans \(\fun{pre}_8(s,f,t)\), la pile~\(s\) collecte les n{\oe}uds
visités en ordre préfixe, la pile~\(f\) (forêt) est un accumulateur de
sous-arbres à visiter et \(t\)~est l'arbre restant à traverser. Le
coût est: \(\C{\fun{pre}_7}{n} = 3n + 2\).

\section{Dénombrement}

De nombreuses publications \cite[\S~2.3.4.4]{Knuth_1997}
\cite[chap.~5]{SedgewickFlajolet_1996} montrent comment trouver le
nombre d'arbres binaires de taille~\(n\) en utilisant des outils
mathématiques puissants, appelés \emph{fonctions génératrices}
\index{dénombrement combinatoire!fonction génératrice}
\cite[chap.~7]{GrahamKnuthPatashnik_1994}. À leur place, pour des
raisons didactiques, nous choisissons une technique plus intuitive
issue du dénombrement combinatoire\index{dénombrement combinatoire}
qui consiste à construire une bijection entre deux ensembles finis, le
cardinal de l'un étant donc le cardinal de l'autre. Pour une taille
donnée, nous allons associer chaque arbre binaire à un objet de
manière exclusive, et aucun objet ne sera laissé de côté. De plus, ces
objets doivent être aisément dénombrables.

Nous connaissons déjà les objets adéquats: les \emph{chemins de
  Dyck}\index{chemin de Dyck}, que nous avons rencontrés à la
section~\ref{sec:queueing} sur les files d'attente\index{file}. Un
chemin de Dyck est une ligne brisée dans un repère orthonormé, allant
du point \((0,0)\) à \((2n,0)\) et constituée de deux sortes de
segments montrés à la \fig~\vref{fig:rise_fall},
\begin{figure}[b]
\centering
\subfloat[Montée\label{fig:rise}]{%
  \includegraphics[bb=69 662 132 721,scale=0.75]{enqueue}
}
\qquad
\subfloat[Descente\label{fig:fall}]{%
  \includegraphics[bb=69 662 132 721,scale=0.75]{dequeue}
}
\caption{Pas de base dans un repère}
\label{fig:rise_fall}
\end{figure}
de telle sorte qu'ils demeurent au-dessus de l'axe des abscisses, mais
peuvent l'atteindre. Considérons à nouveau l'exemple que nous avons
donné à la \fig~\vref{fig:dyck_path}, sans prendre en compte les coûts
de chaque pas.

Si nous suivons la même convention qu'au
chapitre~\ref{chap:factoring}, nous dirions ici qu'un \emph{mot de
  Dyck}\index{mot de Dyck|see{chemin de Dyck}} est un mot fini sur
l'alphabet contenant les lettres~\fun{r} (montée; anglais,
\emph{rise}\index{chemin de Dyck!montée}) et~\fun{f} (descente;
anglais, \emph{fall}\index{chemin de Dyck!descente}), de telle manière
que tous ses préfixes\index{recherche de motif!préfixe} contiennent
plus de lettres~\fun{r} que~\fun{f}, ou un nombre égal. Cette
condition équivaut à la caractérisation géométrique «~au-dessus de
l'axe des abscisses ou l'atteignant~». Par exemple, \fun{rff} n'est pas
un mot de Dyck parce que le préfixe \fun{rff} (en fait, le mot
entier), contient plus de descentes que de montées, donc le chemin
associé termine sous l'axe des abscisses. Le mot de Dyck correspondant
au chemin de Dyck à la \fig~\vref{fig:dyck_path} est
\fun{rrrfrfrrfffrff}. À ladite page, nous avons utilisé une parenthèse
ouvrante et fermante au lieu de~\word{r}
et~\word{f}. Conceptuellement, il n'y a pas de différence entre un
chemin de Dyck et un mot de Dyck, nous parlons de chemin quand un
cadre géométrique est plus intuitif et de mot pour un raisonnement
symbolique et de la programmation.

Tout d'abord, associons injectivement des arbres binaires à des mots
de Dyck; en d'autres termes, nous voulons parcourir un arbre donné et
produire un mot de Dyck qui n'est le parcours d'aucun autre. Puisque,
par définition, les arbres binaires non-vides sont faits de n{\oe}uds
internes connectés à deux autres arbres binaires, nous pourrions nous
demander comment découper un mot de Dyck en trois parties: une
correspondant à la racine et deux correspondant aux sous-arbres
immédiats. Étant donné que tout mot de Dyck commence par une montée et
se termine par une descente, nous pourrions nous interroger à propos
du mot entre les deux extrémités. En général, ce n'est pas un mot de
Dyck; par exemple, couper les bouts de \fun{rfrrff} donne
\fun{frrf}. Au lieu de cela, nous recherchons une décomposition des
mots de Dyck en mots de Dyck. Si un mot de Dyck a exactement un
\emph{retour}, c'est-à-dire une descente qui mène à l'axe des
abscisses, alors découper la première montée et cet unique retour (qui
doit aussi être la dernière descente) produit un autre mot de
Dyck. Par exemple, \(\fun{rrfrrfff} = \fun{r} \cdot \fun{rfrrff} \cdot
\fun{f}\). Ces mots sont appelés \emph{premiers}, parce que tout mot
de Dyck peut être décomposé de façon unique comme la concaténation de
tels mots (d'où la référence à la décomposition en facteurs premiers
en théorie élémentaire des nombres): pour tout mot de Dyck
non-vide~\(d\), il existe \({n > 0}\) mots de Dyck~\(p_i\) premiers et
uniques tels que \(d = p_1 \cdot p_2 \cdots p_n\). Ceci conduit
naturellement à la \emph{décomposition en arches}\index{chemin de
  Dyck!décomposition!$\sim$ en arches}, dont le nom provient d'une
analogie architecturale: pour tout mot de Dyck~\(d\), il y a \({n >
  0}\) mots de Dyck~\(d_i\) et retours~\(\fun{f}_i\) tels que
\begin{equation*}
d = (\fun{r} \cdot d_1 \cdot \fun{f}_1) \cdots (\fun{r} \cdot d_n
\cdot \fun{f}_n).
\end{equation*}
Voir \cite{PanayotopoulosSapounakis_1995,
  Lothaire_2005,FlajoletSedgewick_2009}. Mais cette analyse n'est pas
adaptée telle quelle, parce que \(n\)~peut être supérieur à~\(2\), ce
qui empêche toute analogie avec les arbres binaires. La solution est
assez simple: conservons le premier facteur premier \(\fun{r} \cdot
d_1 \cdot \fun{f}_1\) et ne factorisons \emph{pas} le suffixe, qui est
un mot de Dyck. Autrement dit, pour tout mot de Dyck non-vide~\(d\),
il existe un seul retour~\(\fun{f}_1\) et deux sous-mots de Dyck
\(d_1\) et~\(d_2\) (vides ou non) tels que
\begin{equation*}
d = (\fun{r} \cdot d_1 \cdot \fun{f}_1) \cdot d_2.
\end{equation*}
Ceci est la \emph{décomposition par premier retour}\index{chemin de
  Dyck!décomposition!$\sim$ par premier retour}, connue aussi comme la
\emph{décomposition quadratique}\index{chemin de
  Dyck!décomposition!$\sim$ quadratique} ---~il est aussi possible
d'écrire \(d = d_1 \cdot (\fun{r} \cdot d_2 \cdot \fun{f}_1)\). Par
exemple, le mot de Dyck \(\fun{rrfrffrrrffrff}\), montré à la
\fig~\ref{fig:dyck_quad}
\begin{figure}
\centering
\includegraphics[bb=83 616 508 727,scale=0.8]{dyck_quad}
\caption{Décomposition quadratique d'un chemin de Dyck}
\label{fig:dyck_quad}
\end{figure}
admet la décomposition quadratique \(\fun{r} \cdot \fun{rfrf} \cdot
\fun{f} \cdot \fun{rrrffrff}\). Cette décomposition est unique parce
que la factorisation en facteurs premiers est unique.

Étant donné un arbre \(\fun{int}(x,t_1,t_2)\), la montée et la
descente explicitement distinguées dans la décomposition quadratique
doivent être comprises comme formant une paire qui est en
correspondance avec~\(x\), alors que \(d_1\)~est associé à~\(t_1\)
et~\(d_2\) à~\(t_2\). Plus précisément, la valeur de~\(x\) n'est pas
pertinente ici, seulement l'existence d'un n{\oe}ud interne, et une
fourche\index{arbre binaire!fourche} dans un arbre feuillu\index{arbre
  binaire!$\sim$ feuillu} serait codée tout aussi
bien. Symboliquement, si~\(\delta(t)\) est le mot de Dyck en relation
avec l'arbre binaire~\(t\), alors nous nous attendons aux équations
suivantes:
\begin{equation}
\delta(\fun{ext}()) = \varepsilon; \quad \delta(\fun{int}(x,t_1,t_2))
= \fun{r}_{x} \cdot \delta(t_1) \cdot \fun{f} \cdot \delta(t_2).
\label{eq:delta}
\end{equation}
Nous avons attaché le contenu du n{\oe}ud~\(x\) à la montée
correspondante, donc nous ne perdons pas d'information. Par exemple,
l'arbre à la \fig~\vref{fig:dyck2bt}
\begin{figure}[b]
\centering
\subfloat[Arbre binaire\label{fig:dyck2bt}]{%
\includegraphics[bb=71 655 142 721]{dyck2bt}}
\qquad\quad
\subfloat[Chemin de Dyck correspondant au codage préfixe $[0,1,\Box,2,\Box,\Box,3,\Box,\Box\char93$\label{fig:bt2dyck}]{%
\includegraphics[scale=0.9,bb=93 650 330 724]{bt2dyck}}
\caption{Bijection entre un arbre binaire et un chemin de Dyck}
\end{figure}
est \(t=\fun{int}(0, \fun{int}(1, \fun{ext}(), \fun{int}(2,
\fun{ext}(), \fun{ext}())), \fun{int}(3, \fun{ext}(), \fun{ext}())\)
et mis en correspondance avec le chemin de Dyck à la
\fig~\vref{fig:bt2dyck}\index{arbre binaire!préordre!codage} de la
façon suivante:
\begin{align*}
\delta(t)
&= \fun{r}_0 \cdot \delta(\fun{int}(1, \fun{ext}(),
\fun{int}(2, \fun{ext}(), \fun{ext}()))) \cdot \fun{f} \cdot
\delta(\fun{int}(3, \fun{ext}(), \fun{ext}()))\\ &= \fun{r}_0
\cdot (\fun{r}_1 \cdot \delta(\fun{ext}()) \cdot \fun{f} \cdot
\delta(\fun{int}(2, \fun{ext}(), \fun{ext}()))) \cdot
\fun{f} \cdot \delta(\fun{int}(3, \fun{ext}(), \fun{ext}()))\\
&= \fun{r}_0 \fun{r}_1 \varepsilon \cdot \fun{f} \cdot (\fun{r}_2
\cdot \delta(\fun{ext}()) \cdot \fun{f} \cdot \delta(\fun{ext}()))
\cdot \fun{f} \cdot (\fun{r}_3 \cdot \delta(\fun{ext}()) \cdot \fun{f}
\cdot \delta(\fun{ext}()))\\
&= \fun{r}_0 \fun{r}_1 \fun{f} \cdot (\fun{r}_2 \cdot
\varepsilon \cdot \fun{f} \cdot \varepsilon) \cdot
\fun{f} \cdot (\fun{r}_3 \cdot \varepsilon \cdot \fun{f} \cdot
\varepsilon)
= \fun{r}_0\fun{r}_1\fun{f}\fun{r}_2\fun{f}\fun{f}\fun{r}_3\fun{f}.
\end{align*}
Remarquons que si nous remplaçons les montées par leur contenu associé
(en indice) et les descentes par~\(\Box\), nous obtenons \([0, 1,
\Box, 2, \Box, \Box, 3, \Box]\), ce qui est le codage
préfixe\index{arbre binaire!parcours préfixe!codage} de l'arbre sans
son dernier \(\Box\). Nous pourrions alors modifier
\fun{epre/2}\index{epre@\fun{epre/2}} à la \fig~\vref{fig:epre} pour
qu'elle code un arbre binaire en un chemin de Dyck\index{chemin de
  Dyck}, mais nous devrions enlever le dernier élément de la pile
résultante, donc il est plus efficace de mettre en {\oe}uvre
directement \(\delta\) comme la fonction
\fun{dpre/1}\index{dpre@\fun{dpre/1}} (anglais, \emph{Dyck path as
  preorder}) à la \fig~\vref{fig:dpre}.
\begin{figure}
\begin{equation*}
\boxed{
\begin{array}{r@{\;}c@{\;}l}
\fun{dpre}(t) & \rightarrow & \fun{dpre}(t,\el).\\
\\
\fun{dpre}(\fun{ext}(),s) & \rightarrow & s;\\
\fun{dpre}(\fun{int}(x,t_1,t_2),s)
  & \rightarrow
  & \cons{\fun{r}(x)}{\fun{dpre}(t_1,\cons{\fun{f}()}{\fun{dpre}(t_2,s)})}.
\end{array}}
\end{equation*}
\caption{Codage préfixe d'un arbre en un chemin de Dyck}
\label{fig:dpre}
\end{figure}
Si la taille de l'arbre binaire est~\(n\), alors \(\C{\fun{dpre}}{n} =
2n+2\)\index{dpre@$\C{\fun{dpre}}{n}$} et la longueur du chemin de
Dyck est~\(2n\).

Cette correspondance est clairement inversible, car nous avons déjà
résolu le décodage préfixe aux
\figs~\vrefrange{fig:pre2b}{fig:pre2b0},\index{pre2b@\fun{pre2b/1}}
\index{pre2b0@\fun{pre2b\(_0\)/1}} et nous comprenons maintenant que
l'inverse est fondé sur la décomposition quadratique du
chemin.\index{chemin de Dyck!décomposition!$\sim$ quadratique}

Ceci dit, si nous sommes préoccupés par l'efficacité, nous pouvons
nous souvenir qu'un codage postfixe\index{arbre binaire!parcours
  préfixe!codage} conduit à un décodage plus rapide, comme nous
l'avons vu aux
\figs~\vrefrange{fig:epost}{fig:post2b}. \index{epost@\fun{epost/1}}\index{post2b@\fun{post2b/1}}
Pour créer un chemin de Dyck\index{chemin de Dyck} basé sur un
parcours postfixe, nous associons les n{\oe}uds externes à des montées
et les n{\oe}uds internes à des descentes (avec les contenus), enfin,
nous ôtons la première montée. Voir la \fig~\vref{fig:dyck_post}
\begin{figure}[b]
\centering
\subfloat[Arbre binaire]{\includegraphics[bb=71 655 142 721]{dyck2bt}}
\qquad\quad
\subfloat[Chemin de Dyck correspondant au codage postfixe
$[\Box,\Box,\Box,2,1,\Box,\Box,3,0\char93$\label{fig:dyck_post}]{%
\includegraphics[scale=0.9,bb=93 650 330 724]{dyck_post}}
\caption{Bijection entre un arbre binaire et un chemin de Dyck}
\end{figure}
pour le chemin de Dyck obtenu à partir du parcours postfixe du même
arbre que précédemment. Bien sûr, tout comme nous l'avons fait avec la
correspondance préfixe, nous n'allons pas construire le codage
postfixe\index{arbre binaire!parcours postfixe!codage}, mais plutôt
aller directement de l'arbre binaire au chemin de Dyck\index{chemin de
  Dyck}, comme à la
\fig~\vref{fig:dpost}.\index{dpost@\fun{dpost/1}}\index{dpost@\fun{dpost/2}}
\begin{figure}[t]
\begin{equation*}
\boxed{
\begin{array}{r@{\;}c@{\;}l}
\fun{dpost}(t) & \rightarrow & \fun{dpost}(t,\el).\\
\\
\fun{dpost}(\fun{ext}(),s) & \rightarrow & s;\\
\fun{dpost}(\fun{int}(x,t_1,t_2),s)
  & \rightarrow
  & \fun{dpost}(t_1,\cons{\fun{r}()}{\fun{dpost}(t_2,\cons{\fun{f}(x)}{s})}).
\end{array}}
\end{equation*}
\caption{Codage postfixe d'un arbre en un chemin de Dyck}
\label{fig:dpost}
\end{figure}
Remarquons que nous empilons \(\fun{r}()\) et \(\fun{f}(x)\) dans la
même règle, donc nous n'avons pas besoin d'ôter la première montée à
la fin. (Nous employons une optimisation semblable avec
\fun{dpre/2}\index{dpre@\fun{dpre/2}} à la \fig~\vref{fig:dpre}.) En
termes structurels, l'inverse de cette correspondance postfixe est une
décomposition \(d = d_1 \cdot (\fun{r} \cdot d_2 \cdot \fun{f}_1)\),
que nous avons mentionnée précédemment en passant comme étant une
option à la décomposition quadratique.\index{chemin de
  Dyck!décomposition!$\sim$ quadratique}

La fonction qui décode les chemins de Dyck\index{chemin de Dyck}
postfixes en arbres binaires est une simple variation de
\fun{post2b/1}\index{post2b@\fun{post2b/1}} et
\fun{post2b/2}\index{post2b@\fun{post2b/2}} à la
\fig~\vref{fig:post2b}: nous initialisons la pile auxiliaire avec
\([\fun{ext}()]\). La fonction est nommée
\fun{d2b/1}\index{d2b@\fun{d2b/1}} et définie à la
\fig~\vref{fig:d2b}. Le coût est \(\C{\fun{d2b}}{n} = 2n +
2\).\index{d2b@$\C{\fun{d2b}}{n}$}

\bigskip

\begin{figure}[h]
\begin{equation*}
\boxed{
\begin{array}{r@{\;}c@{\;}l}
\fun{d2b}(s) & \rightarrow & \fun{d2b}([\fun{ext}()],s).\\
\\
\fun{d2b}([t],\el) & \rightarrow & t;\\
\fun{d2b}(f,\cons{\fun{r}()}{s}) & \rightarrow & \fun{d2b}(\cons{\fun{ext}()}{f},s);\\
\fun{d2b}(\cons{t_2,t_1}{f},\cons{\fun{f}(x)}{s}) & \rightarrow & \fun{d2b}(\cons{\fun{int}(x,t_1,t_2)}{f},s).
\end{array}}
\end{equation*}
\caption{Décodage postfixe d'un chemin de Dyck en un arbre}
\label{fig:d2b}
\end{figure}

\bigskip

Que nous choisissions un codage préfixe ou postfixe, une conséquence
des bijections est qu'il y a autant d'arbres binaires de taille~\(n\)
que de chemins de Dyck de longueur~\(2n\). Nous savons déjà, voir le chapitre~\ref{chap:Catalan}, et l'équatio~\eqref{eq:Cn}
\vpageref{eq:Cn}, qu'il y a \(C_n = \frac{1}{n+1}\binom{2n}{n}
\sim \frac{4^n}{n\sqrt{\pi n}}\) de ces chemins.

D'autres encodages des arbres binaires sont proposés dans
\cite[2.3.3]{Knuth_1997} et \cite[5.11]{SedgewickFlajolet_1996}.


\mypar{Longueur moyenne des chemins}

La plupart des paramètres moyens usuels des arbres binaires, comme la
longueur moyenne des chemins internes, la hauteur et largeur moyenne,
sont obtenus très difficilement et requièrent des outils mathématiques
qui vont au-delà de la portée de ce livre.

La longueur interne~\(I(t)\)\index{arbre binaire!longueur interne}
d'un arbre binaire~\(t\) est la somme des longueurs des chemins de la
racine à chaque n{\oe}ud interne. Nous avons déjà rencontré le concept
de longueur externe~\(E(t)\)\index{arbre binaire!longueur externe}, à
savoir la somme des longueurs des chemins de la racine à chaque
n{\oe}ud externe, à la section~\vref{sec:opt_sort} à propos du tri
optimal, où nous avons montré que l'arbre binaire qui minimise la
longueur interne moyenne possède tous ses n{\oe}uds externes sur deux
niveaux successifs. La relation entre ces deux longueurs est assez
simple parce que la structure binaire donne lieu à une équation qui ne
dépend que de la taille~\(n\):
\begin{equation}
E_n = I_n + 2n.\label{eq:EI}
\end{equation}
En effet, soit \(\fun{int}(x,t_1,t_2)\) un arbre avec \(n\)~n{\oe}uds
internes. Alors, nous avons
\begin{equation}
  I(\fun{ext}()) = 0,\quad
  I(\fun{int}(x,t_1,t_2)) = I(t_1) + I(t_2) + n - 1,
\label{eq:I}
\end{equation}
parce que chaque chemin dans~\(t_1\) et~\(t_2\) est rallongé avec un
arc de plus jusqu'à la racine~\(x\), et il y a \(n-1\) chemins ainsi,
par définition. D'un autre côté,
\begin{equation}
  E(\fun{ext}()) = 0,\quad
  E(\fun{int}(x,t_1,t_2)) = E(t_1) + E(t_2) + n + 1,
\label{eq:E}
\end{equation}
parce que chaque chemin dans~\(t_1\) et~\(t_2\) est rallongé avec un
arc de plus jusqu'à la racine~\(x\) et il y a \(n+1\) chemins en
question, d'après le théorème~\vref{thm_int_ext}. En soustrayant
l'équation~\eqref{eq:I} de~\eqref{eq:E} donne
\begin{align*}
  E(\fun{ext}()) - I(\fun{ext}()) &= 0,\\
  E(\fun{int}(x,t_1,t_2)) - I(\fun{int}(x,t_1,t_2))
  &= (E(t_1) - I(t_1)) + (E(t_2) - I(t_2)) + 2.\!\!
\end{align*}
En d'autres termes, tout n{\oe}ud interne ajoute~\(2\) à la différence
des longueurs externe et interne à partir de lui. Puisque cette
différence est nulle aux n{\oe}uds externes, nous obtenons
l'équation~\eqref{eq:EI} pour l'arbre de taille~\(n\). Malheureusement,
tout autre résultat est franchement difficile à obtenir. Par exemple,
la \emph{longueur interne moyenne}~\(\Expected{I_n}\)\index{arbre
  binaire!longueur interne moyenne}, qui vaut
\begin{equation*}
\Expected{I_n}  = \frac{4^n}{b_n} - 3n - 1 \sim n \sqrt{\pi n},
\end{equation*}
est expliquée par \cite{Knuth_1997} à l'exercice~5 de la
section~2.3.4.5, et \cite{SedgewickFlajolet_1996}, au théorème~5.3 de
la section~5.6.  En utilisant l'équation~\eqref{eq:EI}, nous déduisons
\(\Expected{E_n} = \Expected{I_n} + 2n\), ce qui implique que le coût
du parcours d'un arbre binaire aléatoire de taille~\(n\) de la racine
à un n{\oe}ud externe est \(\Expected{E_n}/(n+1) \sim \sqrt{\pi
  n}\). De plus, la valeur \(\Expected{I_n}/n\) peut être comprise
comme le niveau moyen d'un n{\oe}ud interne choisi au hasard.

\mypar{Hauteur moyenne}

La hauteur moyenne~\(h_n\)\index{arbre binaire!hauteur moyenne} d'un
arbre binaire de taille~\(n\) est encore plus difficile à obtenir et a
été étudiée par
\cite{FlajoletOdlyzko_1981,BrownShubert_1984,FlajoletOdlyzko_1984,Odlyzko_1984}:
\begin{equation*}
h_n \sim 2 \sqrt{\pi n}.
\end{equation*}
Dans le cas des arbres de Catalan, à savoir, les arbres dont les
n{\oe}uds internes peuvent avoir un nombre quelconque d'enfants,
l'analyse de la hauteur moyenne a été menée par
\cite{DasarathyYang_1980,DershowitzZaks_1981}, \cite{Kemp_1984} à la
section~5.1.1, \cite{DershowitzZaks_1990,KnuthdeBruijnRice_2000b} et
\cite{SedgewickFlajolet_1996}, à la
section~5.9.\index{arbre!hauteur!$\sim$ moyenne}

\mypar{Largeur moyenne}

La \emph{largeur}\index{arbre binaire!largeur} d'un arbre binaire est
la longueur de son niveau étendu le plus grand\index{arbre
  binaire!niveau étendu}. On peut montrer que la \emph{largeur
  moyenne}~\(w_n\)\index{arbre binaire} d'un arbre binaire de
taille~\(n\) satisfait
\begin{equation*}
w_n \sim \sqrt{\pi n} \sim \tfrac{1}{2} h_n.
\end{equation*}
Ce résultat implique, en particulier, que la taille moyenne de la pile
nécessaire à l'exécution d'un parcours préfixe avec
\fun{pre\(_4\)/2}\index{pre4@\fun{pre\(_4\)/2}} à la
\fig~\vref{fig:pre3} est deux fois la taille moyenne de la file
nécessaire au parcours en largeur avec \fun{bf/1}\index{bf@\fun{bf/1}}
à la \fig~\vref{fig:bf}. Ceci n'est pas du tout évident, car les deux
piles qui simulent la file ne contiennent pas toujours un niveau
complet.

Pour les arbres généraux, \cite{DasarathyYang_1980} ont joliment
présenté une correspondance bijective avec les arbres binaires et le
transfert de certains paramètres moyens.

\section*{Exercices}

\begin{enumerate}

  \item Prouvez \(\fun{post2b}(\fun{epost}(t)) \equiv t\).
    \index{post2b@\fun{post2b/1}}

  \item Prouvez \(\fun{pre2b}(\fun{epre}(t)) \equiv t\).
    \index{pre2b@\fun{pre2b/1}}

  \item Prouvez \(\fun{epre}(\fun{mir}(t)) \equiv
    \fun{rev}(\fun{epost}(t))\).\index{epost@\fun{epost/1}}
    \index{rev@\fun{rev/1}}\index{epre@\fun{epre/1}}
    \index{mir@\fun{mir/1}}

  \item Définissez le codage d'un arbre binaire fondé sur son parcours
    infixe.\index{arbre binaire!parcours infixe!codage}

\end{enumerate}
