\chapter{Tri par interclassement}
\label{chap:merge_sort}
\index{tri par interclassement|(}

\cite{Knuth_1996} rapporte que le premier programme informatique,
écrit en~\oldstylenums{1945} par le mathématicien John von~Neumann,
était un algorithme de tri que l'on appelle de nos jours \emph{tri par
interclassement}, ou \emph{tri par fusion}. Il figure aujourd'hui
parmi les plus méthodes de tri les plus enseignées parce qu'il
illustre une stratégie de résolution de problèmes connue sous le nom
de «~diviser pour régner~»\index{diviser pour régner}: la donnée est
divisée, les parties non-triviales sont récursivement traitées et
leurs solutions sont finalement combinées pour former une solution au
problème initial. On peut reconnaître là la double méthode
d'\emph{analyse et synthèse}, prônée en mathématiques: l'analyse
divise le problème en sous-problèmes et la synthèse combine chacune
des solutions. Bien sûr, il faut parvenir à des sous-problèmes
suffisamment simples pour que leur solution soit évidente ou obtenue
par une autre méthode. Bien que le tri par interclassement ne soit pas
difficile à programmer, la détermination de son coût nécessite des
connaissances mathématiques approfondies. La plupart des manuels
\citep{GrahamKnuthPatashnik_1994,CLRS_2009} montrent comment
déterminer l'ordre de grandeur d'un majorant du coût (exprimée à
l'aide de la notation de Bachmann \(\mathcal{O}\)) à partir
d'équations de récurrences qu'il satisfait, mais le cas général n'est
souvent présenté qu'en annexe ou pas du tout, parce qu'une solution
asymptotique requiert une certaine habileté en \emph{combinatoire
analytique} \citep{FlajoletSedgewick_2001, FlajoletSedgewick_2009,
  FlajoletGolin_1994, Hwang_1998, ChenHwangChen_1999}.

De plus, il y a plusieurs variantes du tri par interclassement
\citep{Knuth_1998,GolinSedgewick_1993} et, souvent, la variante
\emph{descendante} est la seule présentée et son fonctionnement
illustré sur des tableaux. Ici, nous montrons que les piles, en tant
que structures de données purement fonctionnelles
\citep{Okasaki_1998b}, sont adaptées aussi bien à la version
descendante qu'à la version \emph{ascendante}
\citep{PannyProdinger_1995}.


\section{Interclassement}
\label{sec:merging}
\index{tri par interclassement!interclassement|(}

John von~Neumann n'a pas en fait vraiment décrit le tri par
interclassement, mais l'opération sur laquelle il s'appuie,
l'\emph{interclassement}, qu'il nomma \emph{meshing}
(maillage). L'interclassement consiste à combiner deux piles de clés
ordonnées en une seule pile ordonnée. Sans perte de généralité, nous
ne nous intéresserons qu'au tri en ordre croissant. Par exemple,
l'interclassement de \([10,12,17]\) et \([13,14,16]\) donne
\([10,12,13,14,16,17]\). Une façon de parvenir à ce résultat consiste
à comparer les deux clés les plus petites, retenir la plus petite et
répéter le procédé jusqu'à ce qu'une des piles soit épuisée, auquel
cas l'autre est ajoutée en entier aux clés précédemment retenues. Nous
avons ainsi (les clés comparées sont soulignées):
\begin{equation*}
\left\{
\begin{aligned}
&\underline{10}~12~17\\
&\underline{13}~14~16
\end{aligned}
\right.
\rightarrow 10
\left\{
\begin{aligned}
&\underline{12}~17\\
&\underline{13}~14~16
\end{aligned}
\right.
\rightarrow 10~12
\left\{
\begin{aligned}
&\underline{17}\\
&\underline{13}~14~16
\end{aligned}
\right.
\rightarrow 10~12~13
\left\{
\begin{aligned}
&\underline{17}\\
&\underline{14}~16
\end{aligned}
\right.
%\quad\text{etc.}
\end{equation*}
La fonction \fun{mrg/2}\index{mrg@\fun{mrg/2}} (anglais: \emph{merge})
à la \fig~\vref{fig:mrg}
\begin{figure}[b]
\begin{equation*}
\boxed{%
\begin{array}{r@{\;}l@{\;}l}
\fun{mrg}(\el,t)         & \xrightarrow{\smash{\theta}} & t;\\
\fun{mrg}(s,\el)         & \xrightarrow{\smash{\iota}} & s;\\
\fun{mrg}(\cons{x}{s},\cons{y}{t}) & \xrightarrow{\smash{\kappa}}
& \cons{y}{\fun{mrg}(\cons{x}{s},t)},\;\text{si \(x \succ y\)};\\
\fun{mrg}(\cons{x}{s},t) & \xrightarrow{\smash{\lambda}}
                         & \cons{x}{\fun{mrg}(s,t)}.
\end{array}}
\end{equation*}
\caption{Interclasser deux piles triées}
\label{fig:mrg}
\index{tri par interclassement!interclassement!programme}
\end{figure}
met en œuvre cette technique. La règle~\(\iota\) n'est pas
nécessaire mais nous la conservons tout de même car elle rend le coût
symétrique, tout comme \fun{mrg/2}\index{mrg@\fun{mrg/2}} l'est:
\(\C{\fun{mrg}}{m,n} =
\C{\fun{mrg}}{n,m}\)\index{mrg@$\C{\fun{mrg}}{m,n}$} et
\(\fun{mrg}(s,t) \equiv \fun{mrg}(t,s)\), où \(m\)~et~\(n\) sont les
longueurs de~\(s\) et~\(t\). Cette propriété permet d'exprimer le coût
plus facilement et d'accélérer l'évaluation. Remarquons que dans la
définition de \fun{cat/2}\index{cat@\fun{cat/2}}
(équation~\eqref{def:cat} à la page~\pageref{def:cat}), nous
n'incluons pas une règle similaire, \(\fun{cat}(s,\el) \rightarrow
s\), parce que, malgré le gain en termes de coût, la fonction est
asymétrique et les calculs de coûts sont simplifiés lorsque nous
employons \(\C{\fun{cat}}{n}\)\index{cat@$\C{\fun{cat}}{n}$} plutôt
que \(\C{\fun{cat}}{m,n}\). La \fig~\vref{fig:mrg_247}
\begin{figure}
\begin{equation*}
\boxed{%
\begin{array}{r@{\;}l@{\;}l}
  \fun{mrg}([3,4,7],[1,2,5,6])
& \xrightarrow{\smash{\kappa}}
& \cons{1}{\fun{mrg}([3,4,7],[2,5,6])}\\
& \xrightarrow{\smash{\kappa}}
& \cons{1,2}{\fun{mrg}([3,4,7],[5,6])}\\
& \xrightarrow{\smash{\lambda}}
& \cons{1,2,3}{\fun{mrg}([4,7],[5,6])}\\
& \xrightarrow{\smash{\lambda}}
& \cons{1,2,3,4}{\fun{mrg}([7],[5,6])}\\
& \xrightarrow{\smash{\kappa}}
& \cons{1,2,3,4,5}{\fun{mrg}([7],[6])}\\
& \xrightarrow{\smash{\kappa}}
& \cons{1,2,3,4,5,6}{\fun{mrg}([7],\el)}\\
& \xrightarrow{\smash{\iota}}
& [1,2,3,4,5,6,7].
\end{array}}
\end{equation*}
\caption{\(\fun{mrg}([3,4,7],[1,2,5,6]) \twoheadrightarrow
  [1,2,3,4,5,6,7]\)}
\label{fig:mrg_247}
\index{tri par interclassement!interclassement!exemple}
\index{langage fonctionnel!évaluation!trace}
\end{figure}
montre une trace d'évaluation de \fun{mrg/2}. Les règles
\(\kappa\)~et~\(\lambda\) impliquent une comparaison, contrairement à
\(\theta\)~et~\(\iota\) qui terminent les évaluations; par conséquent,
si \(\OC{\fun{mrg}}{m,n}\)\index{mrg@$\OC{\fun{mrg}}{m,n}$} est le
nombre de comparaisons pour interclasser avec
\fun{mrg/2}\index{mrg@\fun{mrg/2}} deux piles de longueurs
\(m\)~et~\(n\), nous avons\index{mrg@$\C{\fun{mrg}}{m,n}$}
\begin{equation}
\C{\fun{mrg}}{m,n} = \OC{\fun{mrg}}{m,n} + 1.\label{def:cost_mrg}
\end{equation}
Pour gagner en généralité, nous étudierons
\(\OC{\fun{mrg}}{m,n}\). Graphiquement, nous représentons les clés
d'une pile par un disque blanc \((\circ)\) et les clés de l'autre par
un disque noir \((\bullet)\). Nous les appellerons \emph{nœuds} et
nous les dessinerons sur une ligne horizontale, celui la plus à gauche
étant le plus petit. Les comparaisons sont toujours effectuées entre
un nœud blanc et un nœud noir, et sont représentées comme des
\emph{arcs} à la \fig~\vref{fig:merged}.  Un arc entrant (côté pointu)
signifie que la clé du nœud est inférieure à celle de l'autre
extrémité, donc tous les arcs pointent vers la gauche et le nombre de
comparaisons est le nombre de nœuds avec un arc entrant.
%\par\vskip\baselineskip
%\begin{figure}[h]
%\centering
\begin{equation}
\includegraphics[bb=73 708 292 718]{merged}%[bb=73 702 292 724]
\label{fig:merged}
\end{equation}
%\caption{Interclassement de deux piles\label{fig:merged}}
%\end{figure}

%\addcontentsline{toc}{subsection}{Cost}
\paragraph{Coût minimal}
\label{merge_best_case}
\index{tri par interclassement!interclassement!coût minimal|(}

Il y a deux nœuds blancs consécutifs sans arcs à l'extrême droite
de la \fig~\vref{fig:merged}, ce qui suggère que plus il y a de clés
provenant d'une même pile au fond du résultat, moins il y a eu de
comparaisons pour l'interclassement: le nombre minimal est atteint
quand \emph{la pile la plus courte apparaît la première dans le
  résultat} (donc à gauche sur la figure). Considérons l'exemple
suivant, où le nombre de comparaisons est le nombre de nœuds
noirs:
\begin{center}
\includegraphics[bb=73 705 292 720]{min_mrg}
\end{center}
Le nombre minimal de comparaisons
\(\OB{\fun{mrg}}{m,n}\)\index{mrg@$\OB{\fun{mrg}}{m,n}$} pour
l'interclassement de piles de taille \(m\)~et~\(n\) est donc
\begin{equation}
\OB{\fun{mrg}}{m,n} = \min\{m,n\}.\label{eq:best_merge}
\end{equation}
\index{tri par interclassement!interclassement!coût minimal|)}

\paragraph{Coût maximal}
\index{tri par interclassement!interclassement!coût maximal|(}

Nous pouvons augmenter le nombre de comparaisons par rapport à \(m+n\)
en ôtant, à la \fig~\vref{fig:merged}, les nœuds à droite
\emph{qui ne sont pas comparés}, comme on peut le voir ici:
\begin{center}
\includegraphics[bb=73 702 258 718]{max_mrg2}
\end{center}
Ceci maximise les comparaisons parce tous les nœuds, sauf le
dernier (le plus à droite), sont pointés par un arc. Le nombre maximal
de comparaisons
\(\OW{\fun{mrg}}{m,n}\)\index{mrg@$\OW{\fun{mrg}}{m,n}$} est donc
\begin{equation}
\OW{\fun{mrg}}{m,n} = m + n - 1.\label{eq:worst_merge}
\end{equation}
Échanger les deux nœuds les plus à droite dans l'exemple précédent
laisse \(m+n-1\) invariant:
\begin{center}
\includegraphics[bb=71 700 260 726]{max_mrg1}
\end{center}
donc le nombre maximal de comparaisons se produit quand \emph{les deux
  dernières clés dans le résultat proviennent de deux piles.}
\index{tri par interclassement!interclassement!coût maximal|)}


\paragraph{Coût moyen}
\index{tri par interclassement!interclassement!coût moyen|(}

Cherchons le nombre moyen de comparaisons dans tous les
interclassements de deux piles de longueurs
\(m\)~et~\(n\). Considérons la \fig~\vref{fig:mean_mrg1},
\begin{figure}[b]
\centering
\includegraphics[bb=66 648 350 725]{mean_mrg1}
\caption{Tous les interclassements avec \(m=3\) (\(\circ\)) et \(n=2\)
  (\(\bullet\))}
\label{fig:mean_mrg1}
\end{figure}
avec \(m=3\) nœuds blancs et \(n=2\) nœuds noirs qui sont
interclassés de toutes les manières possibles. Examinons la structure
en jeu. La première colonne liste toutes les configurations où le
nœud noir le plus à droite est aussi le plus à droite dans le
résultat (le dernier). La deuxième colonne liste tous les cas où le
nœud noir le plus à droite est avant-dernier dans le résultat. La
troisième colonne est divisée en deux groupes, le premier dénombrant
les cas où le nœud noir le plus à droite est l'antépénultième dans
le résultat. Le nombre total de comparaisons est~\(35\) et le nombre
de configurations est~\(10\), donc le nombre moyen de comparaisons est
\(35/10 = 7/2\).\label{seven_two} Cherchons une méthode qui fournit ce
ratio pour tout \(m\)~et~\(n\).

Tout d'abord, le nombre de configurations: combien y a-t-il de
manières de combiner \(m\)~nœuds blancs et \(n\)~nœuds noirs?
Cela revient à se demander combien il y a de façons de peindre en noir
\(n\)~nœuds choisis parmi \(m+n\) nœuds blancs. Plus
abstraitement, cela équivaut à trouver combien il y a de manières de
choisir \(n\)~objets parmi \(m+n\). Ce nombre est appelé
\emph{coefficient binomial}\index{coefficient binomial} et est noté
\(\binom{m+n}{n}\). Par exemple, considérons l'ensemble
\(\{a,b,c,d,e\}\) et les \emph{combinaisons}\index{combinaison} de
\(3\)~objets pris parmi eux sont
\begin{gather*}
\{a,b,c\},\{a,b,d\},\{a,b,e\},\{a,c,d\},\{a,c,e\},\{a,d,e\},\\
\{b,c,d\},\{b,c,e\},\{b,d,e\},\\
\{c,d,e\}.
\end{gather*}
Ce dénombrement établit que \(\binom{5}{3} = 10\). Remarquons que nous
utilisons des ensembles mathématiques, donc l'ordre des éléments ou
leur répétition ne sont pas significatifs. Il est aisé de compter les
combinaisons si nous nous souvenons comment nous avons compté les
permutations, \vpageref{par:permutations}. Déterminons donc
\(\binom{r}{k}\). Nous pouvons choisir le premier objet parmi~\(r\),
le deuxième parmi \(r-1\) etc. jusqu'à ce que nous choisissions le
\(r^\text{e}\)~objet parmi \(r-k+1\), donc nous avons réalisé
\(r(r-1)\dots(r-k+1)\) choix. Mais ces arrangements contiennent des
doublons, par exemple, nous devons identifier \(\{a,b,c\}\) à
\(\{b,a,c\}\), puisque l'ordre est insignifiant. Par conséquent, nous
devons diviser le nombre d'arrangements par le nombre de doublons, qui
n'est autre que le nombre de permutations de \(k\)~objets,
c'est-à-dire \(k!\). Au bout du compte:
\begin{equation*}
\binom{r}{k} := \frac{r(r-1)\ldots(r-k+1)}{k!} =
\frac{r!}{k!(r-k)!}.
\end{equation*}
Nous pouvons vérifier maintenant qu'à la \fig~\vref{fig:mean_mrg1},
nous devons bien avoir \(10\)~cas car \(\binom{5}{2} = 5!/(2!3!) =
10\). La symétrie du problème signifie qu'interclasser une pile de
\(m\)~clés avec une pile de \(n\)~clés aboutit au même résultat
qu'interclasser une pile de \(n\)~clés avec une pile de \(m\)~clés:
\begin{equation*}
\binom{m+n}{n} = \binom{m+n}{m}.
%\label{eq:comb_m_n}
\end{equation*}
Ceci peut aussi être facilement prouvé via la définition:
\begin{equation*}
\binom{m+n}{n} := \frac{(m+n)!}{n!(m+n-n)!} =
\frac{(m+n)!}{m!n!} =: \binom{m+n}{m}.
\end{equation*}
Comme nous l'avons remarqué précédemment, le nombre total \(K(m,n)\)
de comparaisons nécessaires pour interclasser \(m\)~et~\(n\) clés de
toutes les façons possibles avec notre méthode est le nombre de
nœuds avec un arc entrant. Soit \(\overline{K}(m,n)\) le nombre
total de nœuds \emph{sans} arc entrant, entourés à la
\fig~\ref{fig:mean_mrg2}.
\begin{figure}
\centering
\includegraphics[bb=66 625 252 730]{mean_mrg2}
\caption{Dénombrement vertical}
\label{fig:mean_mrg2}
\end{figure}
Cette figure a été obtenue en déplaçant la troisième colonne de la
\fig~\vref{fig:mean_mrg1} sous la deuxième colonne et en ôtant les
arcs. Le nombre total de nœuds est donc \(K(m,n) +
\overline{K}(m,n)\). Puisqu'à chaque interclassement, on a \(m+n\)
nœuds et qu'il y a \(\binom{m+n}{n}\) interclassements, ce même
nombre est \((m + n) \binom{m+n}{n}\), donc
\begin{equation}
K(m,n) + \overline{K}(m,n) = (m + n) \binom{m+n}{n}.
\label{eq:KoverK}
\end{equation}
Nous pouvons facilement caractériser les nœuds cerclés: ils
constituent la plus longue série monochromatique et continue à droite
dans le résultat. Puisqu'il n'y a que deux couleurs, la détermination
du nombre total \(W(m,n)\) de nœuds blancs cerclés est duale à la
détermination du nombre total \(B(m,n)\) de nœuds noirs,
précisément:
\begin{equation*}
B(m,n) = W(n,m).
\end{equation*}
Par conséquent,
\begin{equation}
\overline{K}(m,n) = W(m,n) + B(m,n) = W(m,n) + W(n,m).
\label{eq:K}
\end{equation}
D'après les équations~\eqref{eq:KoverK} et~\eqref{eq:K}, nous tirons
\begin{equation}
K(m,n) = (m + n) \binom{m+n}{n} - W(m,n) - W(n,m).
\label{eq:K_temp}
\end{equation}
Nous pouvons décomposer \(W(m,n)\) en dénombrant \emph{verticalement}
les nœuds blancs cerclés. À la \fig~\ref{fig:mean_mrg2},
\(W(3,2)\) est la somme des nombres d'interclassements avec au moins
trois, deux et un nœuds blancs cerclés terminaux: \(W(3,2) = 1+ 3
+ 6 = 10\). La première colonne donne \(B(3,2) = 1 + 4 = 5\). En
général, le nombre d'interclassements avec un nœud blanc cerclé à
droite est le nombre de manières de combiner \(n\)~nœuds noirs
avec \(m-1\) nœuds blancs: \(\binom{n+m-1}{n}\). Le nombre
d'interclassements avec au moins deux nœuds blancs à droite est
\(\binom{n+m-2}{n}\), et ainsi de suite. Par conséquent
\begin{equation*}
W(m,n)
  = \binom{n+m-1}{n} + \binom{n+m-2}{n} + \dots + \binom{n+0}{n}
  = \!\sum_{j=0}^{m-1}{\!\binom{n+j}{n}}.
\end{equation*}
Cette somme peut en fait être simplifiée car elle possède une forme
close, mais pour bien la comprendre, nous devons d'abord aiguiser
notre intuition à propos des combinaisons. En calculant des
combinaisons \(\binom{r}{k}\) pour de petites valeurs de~\(r\)
et~\(k\) à l'aide de la définition, nous pouvons remplir une table
nommée traditionnellement \emph{le triangle de Pascal}\index{triangle
  de Pascal} et montrée à la \fig~\vref{fig:pascal_triangle}.
\begin{figure}
\centering
\includegraphics{pascal}
\caption{Le coin du triangle de Pascal (en gras)}
\label{fig:pascal_triangle}
\end{figure}
Remarquons comment nous avons posé, par convention, \(\binom{r}{k} =
0\) si \(k > r\). Le triangle de Pascal vérifie de nombreuses
propriétés intéressantes, en particulier à propos de sommes de
certaines de ses valeurs. Par exemple, si nous choisissons un nombre
dans le triangle et son voisin de droite, leur somme se trouve sous le
voisin. Nous pouvons voir cela en extrayant de la
\fig~\vref{fig:pascal_triangle} les lignes \(r=7\) et \(r=8\):
\begin{center}
\begin{tabular}{||c|c||rrrrrrrrrr||}
\cline{5-6}\cline{8-9}
      & 7 & \textbf{1} & \textbf{7} & \multicolumn{1}{|c}{\textbf{21}} & \multicolumn{1}{c|}{\textbf{35}} &  \textbf{35} &  \multicolumn{1}{|c}{\textbf{21}} & \multicolumn{1}{c|}{\textbf{7}} &  \textbf{1} & 0 & 0\\
\cline{5-5}\cline{8-8}
      & 8 & \textbf{1} & \textbf{8} & \textbf{28} & \multicolumn{1}{|c|}{\textbf{56}} &  \textbf{70} &  \textbf{56} & \multicolumn{1}{|c|}{\textbf{28}} &  \textbf{8} & \textbf{1} & 0\\
\cline{6-6}\cline{9-9}
\end{tabular}
\end{center}
Nous avons encadré deux exemples de cette propriété additive des
combinaisons: \(21 + 35 = 56\) et \(21 + 7 = 28\). Nous pouvons parier
alors
\begin{equation*}
\binom{r-1}{k-1} + \binom{r-1}{k} = \binom{r}{k}.
\end{equation*}
Ceci n'est en fait pas difficile à prouver si nous revenons à la
définition:
\begin{align*}
\binom{r}{k} &:= \frac{r!}{k!(r-k)!}
              = \frac{r}{k} \cdot \frac{(r-1)!}{(k-1)!((r-1)-(k-1))!}
              = \frac{r}{k} \binom{r-1}{k-1}.\\
\binom{r}{k} &:= \frac{r!}{k!(r-k)!}
              = \frac{r}{r-k} \cdot \frac{(r-1)!}{k!((r - 1) - k)!}
              = \frac{r}{r-k} \binom{r-1}{k}.
\end{align*}
La première égalité est valable si \(k > 0\) et la seconde si \(r \neq
k\). Nous pouvons alors exprimer \(\binom{r-1}{k-1}\) et
\(\binom{r-1}{k}\) en termes de \(\binom{r}{k}\) dans la somme
\begin{equation*}
\binom{r-1}{k-1} + \binom{r-1}{k} = \frac{k}{r}\binom{r}{k}
+ \frac{r-k}{r}\binom{r}{k} = \binom{r}{k}.
\end{equation*}
La somme est valide si \(r > 0\). Nous pouvons prouver directement
cette formule en ayant recours au dénombrement combinatoire, sans
passer par l'algèbre. Supposons que nous ayons \emph{déjà} tous les
sous-ensembles de \(k\)~clés choisies parmi~\(r\). Par définition, il
y en a \(\binom{r}{k}\). Nous distinguons alors une clé arbitrairement
parmi les~\(r\) et nous voulons séparer la collection d'ensembles en
deux: d'un côté, toutes les combinaisons qui contiennent cette clé
particulière, de l'autre, toutes celles qui ne la contiennent pas. Le
premier ensemble a pour cardinal \(\binom{r-1}{k-1}\) car ses
combinaisons sont construites en prenant la clé en question et en
choisissant \(k-1\) clés restantes parmi \(r-1\). Le second ensemble
contient \(\binom{r-1}{k}\) combinaisons qui sont faites à partir de
\(r-1\) clés (la clé spéciale est mise de côté), dont \(k\)~ont alors
été sélectionnées de toutes les manières possibles. Ce simple
raisonnement par partition donne la même formule additive que
ci-dessus. Retournons maintenant à notre somme:
\begin{equation*}
W(m,n) = \sum_{j=0}^{m-1}{\binom{n+j}{n}}.
\end{equation*}

% Wrapping figure better declared before a paragraph
%
\begin{wrapfigure}[6]{l}[0pt]{0pt}
% 6 vertical lines
% left placement
% 0pt of margin overhang
\centering
\includegraphics[bb=73 650 110 708]{triangle1}%... 721
\end{wrapfigure}
En regardant le triangle de Pascal, nous comprenons que cette somme
porte sur des nombres appartenant à une même colonne. Plus
précisément, elle débute à la diagonale avec le nombre \(\binom{n}{n}
= 1\) et descend jusqu'à ce que \(m\)~nombres aient été
ajoutés. Choisissons donc un petit exemple sous la forme de deux
colonnes adjacentes, où la somme est petite. À gauche est montré un
extrait pour \(n=4\) (la colonne de gauche est la cinquième dans le
triangle de Pascal) et \(m=4\) (hauteur de la colonne de gauche).  La
somme de la colonne gauche, qui est la somme recherchée, égale le
nombre en bas de la colonne droite: \(1 + 5 + 15 + 35 = 56\). En
vérifiant que cela vaut aussi pour d'autres colonnes, nous gagnons le
sentiment que nous tenons-là un motif général. Avant de tenter une
preuve dans le cas général, voyons comme faire avec notre
exemple. Commençons avec le bas de la colonne droite, \(56\), et
utilisons la formule d'addition à l'envers, c'est-à-dire, exprimons
\(56\) comme la somme de deux nombres dans la ligne au-dessus: \(56 =
35 + 21\). Nous voulons garder le nombre \(35\) parce qu'il fait
partie de la somme recherchée. Appliquons encore la formule additive à
propos de~\(21\) et tirons \(21 = 15 + 6\). Gardons~\(15\) et
recommençons avec~\(6\), d'où \(6 = 5 + 1\). Finalement, \(1 = 1 +
0\). Nous venons donc de vérifier \(56 = 35 + (15 + (5 + (1 + 0)))\),
qui est exactement ce que nous recherchions. Puisque nous voulons que
le nombre correspondant à~\(35\) soit, en général,
\(\binom{n+m-1}{n}\), nous avons la dérivation
\begin{align}
\binom{n+m}{n+1}
  &= \binom{n+m-1}{n} + \binom{n+m-1}{n+1}\notag\\
  &= \binom{n+m-1}{n} + \left[\binom{n+m-2}{n} +
     \binom{n+m-2}{n+1}\right]\notag\\
  &= \binom{n+m-1}{n} + \binom{n+m-2}{n} + \dots +
     \left[\binom{n}{n} + \binom{n}{n+1}\right],\notag\\
\binom{n+m}{n+1}
  &= \sum_{j=0}^{m-1}{\binom{n+j}{n}} = W(m,n).
\label{eq:binom_sum}
\end{align}
Maintenant, nous pouvons remplacer cette forme close dans
l'équation~\eqref{eq:K_temp} \vpageref{eq:K_temp}\index{K@$K(m,n)$}:
\begin{equation*}
K(m,n) = (m + n)
\binom{m+n}{n} - \binom{m+n}{n+1} - \binom{m+n}{m+1}.
\end{equation*}
Par définition, le nombre moyen de comparaisons
\(\OM{\fun{mrg}}{m,n}\)\index{mrg@$\OM{\fun{mrg}}{m,n}$} est le ratio
de \(K(m,n)\) par \(\binom{m+n}{n}\), donc
\begin{equation}
\OM{\fun{mrg}}{m,n} = m + n - \frac{m}{n+1} - \frac{n}{m+1}
  = \frac{mn}{m+1} + \frac{mn}{n+1}.
\label{eq:Amrg}
\end{equation}
Nous avons nécessairement \(\OB{\fun{mrg}}{m,n} \leqslant
\OM{\fun{mrg}}{m,n} \leqslant \OW{\fun{mrg}}{m,n}\) et nous pourrions
nous demander si et quand le coût moyen atteint les bornes. Le
majorant est atteint si, et seulement si, les entiers naturels
\((m,n)\) satisfont l'équation \(m^2 + n^2 - mn = 1\), dont les
uniques solutions sont \((0,1)\), \((1,0)\) et \((1,1)\). Le minorant,
quant à lui, est atteint si, et seulement si, \(mn/(m+1) + mn/(n+1) =
\min\{m,n\}\), dont les uniques solutions en entiers naturels sont
\((0,n)\), \((m,0)\) et \((1,1)\). De plus, les cas \((m,1)\) et
\((1,n)\) suggèrent que l'interclassement d'une pile contenant une
seule clé (un singleton) avec une autre pile est équivalent à
l'insertion de ladite clé parmi les autres, comme nous l'avons fait
avec l'insertion simple à la section~\ref{sec:straight_ins}
\vpageref{sec:straight_ins}. En d'autres termes, nous devrions avoir
le théorème
\begin{equation}
\fun{ins}(x,s) \equiv \fun{mrg}([x],s).
\label{eq:ins_mrg}
\index{ins@\fun{ins/2}}\index{mrg@\fun{mrg/2}}
\end{equation}
Par conséquent, \emph{l'insertion simple est un cas spécial
  d'interclassement.} Néanmoins, les coûts moyens ne sont pas
exactement les mêmes. D'abord, nous avons \(\M{\fun{mrg}}{m,n} =
\OM{\fun{mrg}}{m,n} + 1\), car nous devons tenir compte de l'emploi de
la règle~\(\theta\) ou~\(\iota\) à la \fig~\vref{fig:mrg}, comme nous
l'avons reconnu à l'équation~\eqref{def:cost_mrg}. Alors, les
équations~\eqref{eq:ins} \vpageref{eq:ins} et~\eqref{eq:Amrg} donnent
\begin{equation*}
\M{\fun{mrg}}{1,n} = \frac{1}{2}{n} + 2 - \frac{1}{n+1}\quad\text{et}\quad
\M{\fun{ins}}{n} = \frac{1}{2}{n} + 1.
\index{ins@$\M{\fun{ins}}{n}$}\index{mrg@$\M{\fun{mrg}}{1,n}$}
\end{equation*}
Asymptotiquement, les coûts sont équivalents:
\begin{equation*}
\M{\fun{mrg}}{1,n} \sim \M{\fun{ins}}{n}.
\end{equation*}
Mais \fun{mrg/2}\index{mrg@\fun{mrg/2}} est légèrement plus lente en
moyenne que \fun{ins/2}\index{ins@\fun{ins/2}} dans ce cas:
\begin{equation*}
\M{\fun{mrg}}{1,n} - \M{\fun{ins}}{n} = 1 - \frac{1}{n+1} < 1
\quad\text{et}\quad
\M{\fun{mrg}}{1,n} - \M{\fun{ins}}{n} \sim 1.
\end{equation*}
Par ailleurs, il est intéressant d'observer ce qui advient quand \(m =
n\), c'est-à-dire quand les deux piles interclassées ont la même
longueur:
\begin{equation}
  \M{\fun{mrg}}{n,n} = 2n - 1 + \frac{2}{n+1} = \W{\fun{mrg}}{n,n} - 1 +
  \frac{2}{n+1} \sim 2n.\index{mrg@$\M{\fun{mrg}}{n,n}$}
\label{eq:Amrg_n_n}
\end{equation}
En d'autres termes, le coût moyen de l'interclassement de deux piles
de même taille est asymptotiquement le nombre total de clés, ce qui
est le pire des cas.
\index{tri par interclassement!interclassement!coût moyen|)}

\paragraph{Terminaison}
\label{merging_termination}
\index{tri par interclassement!interclassement!terminaison|(}

La terminaison de \fun{mrg/2}\index{mrg@\fun{mrg/2}} à la
\fig~\vref{fig:mrg} est facile à établir en prenant un ordre
lexicographique \index{induction!ordre lexicographique}
(page~\pageref{par:ackermann}) sur les paires de piles qui sont, à
leur tour, partiellement ordonnées par la relation de sous-terme
propre\index{induction!ordre des sous-termes propres}
(page~\pageref{par:well-founded}), ou, plus restrictive, la
\emph{relation de sous-pile immediate}\index{induction!ordre de la
  sous-pile immédiate}, c'est-à-dire, \(\cons{x}{s} \succ s\). Les
paires de dépendance\index{terminaison!paire de dépendance} des règles
\(\kappa\)~et~\(\lambda\) sont ordonnées par
\((\cons{x}{s},\cons{y}{t}) \succ (\cons{x}{s},t)\) et
\((\cons{x}{s},t) \succ (s,t)\).\index{tri par
  interclassement!interclassement!terminaison|)}\index{tri par
  interclassement!interclassement|)}\hfill\(\Box\)

\section{Trier $2^n$ clés}
\label{sec:power_of_two}
\index{tri par interclassement!puissance@$2^n$ clés|(}

L'interclassement peut être utilisé pour trier \emph{une} pile de clés
comme suit. La pile initiale est coupée en deux, puis les deux parties
sont coupées à leur tour etc. jusqu'à ce qu'il ne reste que des
singletons. Ceux-ci sont alors interclassés deux à deux etc. jusqu'à
ce qu'une seule pile demeure, une pile qui est inductivement triée car
un singleton est trié par définition et
\(\fun{mrg}(s,t)\)\index{mrg@\fun{mrg/2}} est triée si~\(s\) et~\(t\)
le sont. Cette technique ne spécifie pas la stratégie de coupure et,
peut-être, la plus intuitive est celle qui consiste à couper en deux
moitiés égales ou presque, l'égalité étant atteinte pour toutes les
sections seulement s'il y a originellement \(2^p\)~clés. Nous verrons
plus loin comment s'accommoder au cas général et à une stratégie de
partition différente.

Pour l'instant, considérons à la \fig~\vref{fig:bot_up1}
\begin{figure}
\centering
\includegraphics{bot_up1}
\caption{Tri de \([7,3,5,1,6,8,4,2]\)}
\label{fig:bot_up1}
\end{figure}
tous les interclassements et leur ordre relatif dans le tri de la pile
\([7, 3, 5, 1, 6, 8, 4, 2]\). Nous nommons cette structure un
\emph{arbre d'interclassement}\index{tri par interclassement!arbre
  d'interclassement}\index{arbre!$\sim$ d'interclassement}, parce que
chaque nœud de l'arbre est une pile ordonnée, soit un singleton,
soit le résultat de l'interclassement de ses deux enfants. La racine
contient logiquement le résultat. L'arbre d'interclassement est mieux
compris lors d'un examen du bas vers le haut, niveau par
niveau.

Dénotons par \(\C{\Join}{p}\) le nombre de comparaisons pour trier
\(2^p\)~clés et considérons un arbre d'interclassement avec
\(2^{p+1}\)~feuilles. Il est constitué de deux sous-arbres immédiats
de \(2^p\)~feuilles chacun, et la racine contient
\(2^{p+1}\)~clés. Par conséquent
\begin{equation*}
%\abovedisplayskip=4pt
%\belowdisplayskip=4pt
\C{\Join}{0} = 0,
\quad
\C{\Join}{p+1} = 2 \cdot \C{\Join}{p} + \OC{\fun{mrg}}{2^p,2^p}.
\end{equation*}
En déroulant la récurrence, nous aboutissons à
\begin{equation}
%\abovedisplayskip=2pt
%\belowdisplayskip=0pt
\C{\Join}{p+1}
  = 2^p\sum_{k=0}^p{\frac{1}{2^k}\OC{\fun{mrg}}{2^k,2^k}}.
\index{mrg@$\C{\Join}{n}$}
\label{eq:cost_power_2}
\end{equation}

%\addcontentsline{toc}{subsection}{Cost}
\paragraph{Coût minimal}
\index{tri par interclassement!puissance@$2^n$ clés!coût minimal|(}

Quand la pile donnée est déjà triée en ordre croissant ou décroissant,
le nombre de comparaisons est minimal. En fait, étant donné un arbre
d'interclassement au nombre minimal de comparaisons,
\index{arbre!$\sim$ d'interclassement} l'échange de n'importe quels
sous-arbres dont les racines sont interclassées laisse le nombre de
comparaisons invariant. La raison en est que l'arbre d'interclassement
est construit de façon ascendante et le nombre de comparaisons est une
fonction symétrique.  Notons \(\B{\Join}{p}\) le nombre minimal de
comparaisons pour trier \(2^p\)~clés. D'après les équations
\eqref{eq:cost_power_2} et~\eqref{eq:best_merge}, nous avons alors
\begin{equation}
%\abovedisplayskip=0pt
%\belowdisplayskip=0pt
\B{\Join}{p}
  = 2^{p-1}\!\sum_{k=0}^{p-1}{\frac{1}{2^k}\OB{\fun{mrg}}{2^k,2^k}}
  = p2^{p-1}.\label{eq:best_power}\index{mrg@$\B{\Join}{n}$}
\end{equation}
\index{tri par interclassement!puissance@$2^n$ clés!coût minimal|)}

\paragraph{Coût maximal}
\index{tri par interclassement!puissance@$2^n$ clés!coût maximal|(}

Tout comme avec le meilleur des cas, la construction d'un arbre
d'interclassement avec un nombre maximal de comparaisons exige que
tous les sous-arbres soient aussi dans le pire des cas, par exemple,
\([7,3,5,1,4,8,6,2]\). Soit~\(\W{\Join}{p}\) le nombre maximal de
comparaisons pour trier \(2^p\)~clés. D'après les équations
\eqref{eq:cost_power_2}~et~\eqref{eq:worst_merge},
\begin{equation}
%\abovedisplayskip=2pt
%\belowdisplayskip=0pt
\W{\Join}{p}
  = 2^{p-1}\!\sum_{k=0}^{p-1}{\frac{1}{2^k}\OW{\fun{mrg}}{2^k,2^k}}
  = (p-1)2^p + 1.
\label{eq:worst_power}\index{mrg@$\W{\Join}{n}$}
\end{equation}
\index{tri par interclassement!puissance@$2^n$ clés!coût maximal|)}

\paragraph{Coût moyen}
\index{tri par interclassement!puissance@$2^n$ clés!coût moyen|(}
\label{par:Atms_2p}

Pour une pile donnée, dont toutes les permutations sont équiprobables,
le coût moyen de son tri par interclassement est obtenu en considérant
les coûts moyens de tous les sous-arbres de l'arbre d'interclassement:
toutes les permutations des clés sont considérées pour une longueur
donnée. Donc l'équation~\eqref{eq:cost_power_2} est satisfaite par
\(\OM{\fun{mrg}}{2^k,2^k}\) et le coût moyen
\(\M{\Join}{p}\)\index{mrg@$\M{\Join}{n}$}, soit le nombre moyen de
comparaisons pour trier \(2^p\)~clés. Les
équations~\eqref{eq:Amrg_n_n} et~\eqref{def:cost_mrg} donnent
\begin{equation*}
%\abovedisplayskip=2pt
%\belowdisplayskip=2pt
\OM{\fun{mrg}}{n,n} = 2n - 2 + \frac{2}{n+1}.
\index{mrg@$\OM{\fun{mrg}}{n,n}$}
\end{equation*}
Avec l'équation \eqref{eq:cost_power_2}, nous obtenons de plus, pour
\(p > 0\),
\begin{align}
%\abovedisplayskip=0pt
%\belowdisplayskip=0pt
\M{\Join}{p}
  &= 2^{p-1}\sum_{k=0}^{p-1}{\frac{1}{2^k}\OM{\fun{mrg}}{2^k,2^k}}
  = 2^p\sum_{k=0}^{p-1}{\frac{1}{2^k}\left(2^k - 1 + \frac{1}{2^k +
      1}\right)}\notag\\
  &= 2^{p}\left(p - \sum_{k=0}^{p-1}{\frac{1}{2^k}}
     + \sum_{k=0}^{p-1}{\frac{1}{2^k(2^k+1)}}\right)\notag\\
  &= 2^{p}\left(p - \sum_{k=0}^{p-1}{\frac{1}{2^k}}
     + \sum_{k=0}^{p-1}\left(\frac{1}{2^k}
     - \frac{1}{2^k+1}\right)\!\!\right)
   = p2^p - 2^p \sum_{k=0}^{p-1}\frac{1}{2^k+1}\notag\\
  &= p2^p - 2^p \sum_{k \geqslant 0}\frac{1}{2^k+1}
     + 2^p \sum_{k \geqslant p}\frac{1}{2^k+1}
= p2^p - \alpha 2^p + \sum_{k \geqslant 0}\frac{1}{2^{k}+2^{-p}},
\label{eq:Mjoin}
\end{align}
où \(\alpha := \sum_{k \geqslant 0}\frac{1}{2^k+1} \simeq 1.264499\)
est irrationnel \citep{Borwein_1992}. Puisque \(0 < 2^{-p} < 1\), nous
avons \(1/(2^k + 1) < 1/(2^k+2^{-p}) < 1/2^k\) et
\begin{equation}
(p - \alpha)2^p + \alpha < \M{\Join}{p} < (p-\alpha)2^p + 2.
\label{ineq:M_join}
\end{equation}
La convergence uniforme de la série \(\sum_{k \geqslant
  0}\frac{1}{2^{k}+2^{-p}}\) nous permet d'échanger les limites
sur~\(k\) et~\(p\), puis de déduire \(\M{\Join}{p} - (p-\alpha)2^p - 2
\rightarrow 0^{-}\), si \(p \rightarrow \infty\). En d'autres termes,
\(\M{\Join}{p}\) est mieux approché par sa borne supérieure, pour des
valeurs croissantes de la variable~\(p\).\index{tri par interclassement!puissance@$2^n$ clés!coût
  moyen|)}\index{tri par interclassement!puissance@$2^n$ clés|)}


\section{Tri descendant}
\index{tri par interclassement!$\sim$ descendant|(}

Lorsque nous appliquons la stratégie de coupure au milieu à un nombre
arbitraire de clés, nous obtenons deux piles de longueurs
\(\floor{n/2}\) et \(\ceiling{n/2}\), et l'algorithme prend le nom
général de tri par interclassement \emph{ascendant}. Le programme
réalisant ce tri est montré à la \fig~\vref{fig:tms}\index{tri par
  interclassement!$\sim$ descendant!programme}.
\begin{figure}[!b]
\begin{equation*}
\boxed{%
\begin{array}{r@{\;}l@{\;}l}
\fun{tms}(\cons{x,y}{t}) & \rightarrow
                         & \fun{cutr}([x],\cons{y}{t},t);\\
\fun{tms}(t)             & \rightarrow & t.\\
\\
\fun{cutr}(s,\cons{y}{t},\cons{a,b}{u})
                       & \rightarrow & \fun{cutr}(\cons{y}{s},t,u);\\
\fun{cutr}(s,t,u)        & \rightarrow
                         & \fun{mrg}(\fun{tms}(s),\fun{tms}(t)).\\
\\
\fun{mrg}(\el,t)         & \rightarrow & t;\\
\fun{mrg}(s,\el)         & \rightarrow & s;\\
\fun{mrg}(\cons{x}{s},\cons{y}{t}) & \rightarrow
                         & \cons{y}{\fun{mrg}(\cons{x}{s},t)},\;
                           \text{si \(x \succ y\)};\\
\fun{mrg}(\cons{x}{s},t) & \rightarrow & \cons{x}{\fun{mrg}(s,t)}.
\end{array}}
\end{equation*}
\caption{Tri par interclassement descendant avec \fun{tms/1}}
\label{fig:tms}
\end{figure}

Remarquons que l'appel \(\fun{cutr}(s,t,u)\)\index{cutr@\fun{cutr/3}}
retourne la première moitié de~\(t\) sur~\(s\) si~\(s=t\). La
technique consiste à commencer avec \(s=\el\) et à continuer en
projetant les clés de~\(t\) une par une et celles de~\(u\) deux par
deux, donc lorsque \(\fun{cutr}(s,t,\el)\) ou \(\fun{cutr}(s,t,[y])\)
sont atteints, nous savons que \(t\)~est la seconde moitié de la pile
initiale et \(s\)~est la première moitié retournée (de longueur
\(\floor{n/2}\) si \(n\)~est la longueur de la pile originelle).

Dans la première règle de~\fun{tms/1} (anglais: \emph{top-down merge
  sort}), nous économisons un appel récursif à \fun{cutr/3} et un peu
de mémoire en appelant \(\fun{cutr}([x],\cons{y}{t}, t)\) au lieu de
\(\fun{cutr}(\el,\cons{x,y}{t},\cons{x,y}{t})\). De plus, de cette
manière, la seconde règle prend en charge les deux cas de base, à
savoir \(\fun{tms}(\el)\) et \(\fun{tms}([y])\).

Par ailleurs, notons que, dans la seconde règle de \fun{cutr/2}, si
\(u = \el\), alors la longueur de la pile initiale est paire, et, si
\(u = [a]\), elle est impaire. Selon les contextes d'utilisation, un
inconvénient de \fun{tms/1}\index{tms@\fun{tms/1}} peut être que le
tri est \emph{instable}, c'est-à-dire que l'ordre relatif de clés
égales n'est pas toujours invariant.

Puisque toutes les comparaisons sont faites par
\fun{mrg/2}\index{mrg@\fun{mrg/2}}, la définition de
\fun{tms/1}\index{tms@\fun{tms/1}} implique que le nombre de
comparaisons satisfait
\begin{equation}
\OC{\fun{tms}}{0} = \OC{\fun{tms}}{1} = 0,
\qquad
\OC{\fun{tms}}{n} = \OC{\fun{tms}}{\floor{n/2}}
+ \OC{\fun{tms}}{\ceiling{n/2}}
+ \OC{\fun{mrg}}{\floor{n/2},\ceiling{n/2}}.
\index{tms@$\OC{\fun{tms}}{n}$}
\label{eq:cost_tms}
\end{equation}

%\addcontentsline{toc}{subsection}{Cost}
\mypar{Coût minimal}
\index{tri par interclassement!$\sim$ descendant!coût minimal|(}

Le nombre minimal de comparaisons vérifie
\begin{equation*}
%\abovedisplayskip=0pt
\OB{\fun{tms}}{0} = \OB{\fun{tms}}{1} = 0,
\;
\OB{\fun{tms}}{n} = \OB{\fun{tms}}{\floor{n/2}}
+ \OB{\fun{tms}}{\ceiling{n/2}}
+ \OB{\fun{mrg}}{\floor{n/2},\ceiling{n/2}}.
\index{mrg@\fun{mrg/2}}
\index{tms@$\OB{\fun{tms}}{n}$}
\end{equation*}
% Wrapping figure better declared before a paragraph
%
{\setlength{\intextsep}{0pt} % No space before and after a figure
\begin{wrapfigure}[23]{r}[0pt]{0pt}
% 23 vertical lines (3 for each display)
% mandatory right placement (better because of a list)
% 0pt of margin overhang
\centering
\includegraphics[bb=71 435 149 714]{bits}
\caption{}
\label{fig:bits}
%\caption{Binary numbers from \(1\) to \(n\)\label{fig:bits}}
\end{wrapfigure}
Nous avons \(\OB{\fun{tms}}{n} = \OB{\fun{tms}}{\floor{n/2}} +
\OB{\fun{tms}}{\ceiling{n/2}} +
\floor{n/2}\),\index{tms@$\OB{\fun{tms}}{n}$} via
l'équation~\eqref{eq:best_merge} \vpageref{eq:best_merge}.
En particulier,
\begin{equation*}
\OB{\fun{tms}}{2p} = 2 \cdot \OB{\fun{tms}}{p} + p,\quad
\OB{\fun{tms}}{2p+1} = \OB{\fun{tms}}{p} + \OB{\fun{tms}}{p+1} + p.
\end{equation*}
Introduisons la différence de deux termes successifs, \(\Delta_n :=
\OB{\fun{tms}}{n+1} - \OB{\fun{tms}}{n}\), donc \(\Delta_0 = 0\), et
cherchons une contrainte. Puisque nous prenons les parties entières
par défaut et par excès de~\(n/2\), nous considérerons deux cas, selon
la parité de~\(n\):
\begin{itemize}

 \item \(\Delta_{2p} = \OB{\fun{tms}}{2p+1} - \OB{\fun{tms}}{2p} =
  \OB{\fun{tms}}{p+1} - \OB{\fun{tms}}{p} = \Delta_{p}\).

  \item \(\Delta_{2p+1} = \OB{\fun{tms}}{p+1} - \OB{\fun{tms}}{p} + 1 =
  \Delta_{p} + 1\).

\end{itemize}
Nous avons déjà rencontré \(\Delta_n\) sous le nom~\(\nu_n\) à
l'équation~\eqref{eq:ruler_nu} \vpageref{eq:ruler_nu}. Définissons-là
récursivement:
\begin{equation}
\nu_{0} := 0,\quad \nu_{2n} := \nu_{n},\quad
\nu_{2n+1} := \nu_{n} + 1.\label{def:nu}\index{somme des bits}
\end{equation}
Cette définition devient évidente quand nous considérons les
représentations binaires de~\(2n\) et~\(2n+1\). Remarquons que
\(\nu\)~est une fonction à la simplicité trompeuse, par exemple, elle
est périodique car \(\nu_{2^p} = 1\), mais
\(\nu_{2^p-1}=p\). Revenons à notre propos: nous avons
\(\OB{\fun{tms}}{n+1} = \OB{\fun{tms}}{n} + \nu_n\) et, en sommant
membre à membre, nous obtenons
\begin{equation}
\abovedisplayskip=6pt
\abovedisplayshortskip=6pt
\belowdisplayskip=2pt
\OB{\fun{tms}}{n} = \sum_{k=0}^{n-1}{\nu_k}.\label{eq:OB_tms}
\index{tms@$\OB{\fun{tms}}{n}$}
\end{equation}}

\noindent \cite{Trollope_1968} a été le premier à trouver une forme
close pour \(\sum_{k=0}^{n-1}\nu_k\), avec une démonstration qui a été
ensuite simplifiée par \cite{Delange_1975}, qui en a étendu l'analyse
avec des séries de Fourier. \cite{Stolarsky_1977} a collecté de
nombreuses références sur ce sujet. À
l'équation~\eqref{eq:best_power}, nous avons \(\OB{\fun{tms}}{2^p} =
\frac{1}{2}p2^p\), soit \(\OB{\fun{tms}}{n} = \frac{1}{2}n\lg n\) si
\(n=2^p\). Ceci devrait nous inciter à rechercher, avec
\cite{McIlroy_1974}, un terme linéaire supplémentaire dans le cas
général, soit les plus grandes constantes réelles \(a\)~et~\(b\)
telles que, si \(n \geqslant 2\),
\begin{equation}
\pred{Low}{n} \colon \frac{1}{2}n\lg n + an + b \leqslant \OB{\fun{tms}}{n}.
\label{ineq:McIlroy}
\end{equation}
Le cas de base est \(\pred{Low}{2} \colon 2a + b \leqslant 0\). Le
moyen le plus simple de structurer l'argument inductif est de suivre
la définition de \(\OB{\fun{tms}}{n}\) quand \(n=2p\) et \(n=2p+1\),
mais un minorant de \(\OB{\fun{tms}}{2p+1}\) dépendrait des minorants
de \(\OB{\fun{tms}}{p}\) et \(\OB{\fun{tms}}{p+1}\), ce qui cumulerait
les imprécisions. Au lieu de cela, si nous pouvions nous appuyer sur
au moins une valeur exacte à partir de laquelle construire
inductivement la borne, nous gagnerions en précision. Par conséquent,
nous devrions obtenir de meilleurs résultats si nous pouvions
décomposer \(\OB{\fun{tms}}{2^p+i}\), où \(0 < i \leqslant 2^p\), en
termes de \(\OB{\fun{tms}}{2^p}\) (exactement) et
\(\OB{\fun{tms}}{i}\). Ceci devient facile si nous comptons les bits à
la \fig~\vref{fig:Btms_table},
\begin{figure}
\centering
\includegraphics[bb=71 595 210 721]{Btms_table}
\caption{$\protect\OB{\fun{tms}}{2^p+i} = \protect\OB{\fun{tms}}{2^p}
  + \protect\OB{\fun{tms}}{i} + i$}
\label{fig:Btms_table}
\end{figure}
qui est la même table qu'à la \fig~\ref{fig:bits}, où
\(n=2^p+i\). (Gardons en tête que \(\OB{\fun{tms}}{n}\) est la somme
des bits jusqu'à \(n-1\), comme nous pouvons le constater à
l'équation~\eqref{eq:OB_tms}.) Nous constatons que
\begin{equation}
\OB{\fun{tms}}{2^p+i} = \OB{\fun{tms}}{2^p} + \OB{\fun{tms}}{i} + i.
\label{eq:OBtms_2m_i}
\end{equation}
(Le terme~\(i\) est la somme des bits les plus à gauche.) Donc,
supposons \(\pred{Low}{n}\), pour tout \(1 \leqslant n \leqslant
2^p\), et prouvons \(\pred{Low}{2^p+i}\) si \(0 < i \leqslant
2^p\). Le principe d'induction entraîne alors que \(\pred{Low}{n}\)
vaut pour tout~\(n \geqslant 2\). Le pas inductif
\(\pred{Low}{2^p+i}\) devrait nous offrir l'opportunité de maximiser
les constantes \(a\)~et~\(b\).\index{Low@\predName{Low}} Soit
\(m=2^p\). En employant \(\OB{\fun{tms}}{2^p} = \tfrac{1}{2}p2^p\) et
l'hypothèse d'induction \(\pred{Low}{i}\), nous avons:
\begin{equation}
\frac{1}{2}m\lg m + \left(\frac{1}{2}i\lg i + ai + b\right) + i
\leqslant
\OB{\fun{tms}}{m} + \OB{\fun{tms}}{i} + i = \OB{\fun{tms}}{m+i}.
\label{ineq:Btms_n_i}
\end{equation}
Pour le pas inductif, il nous faut \(\tfrac{1}{2}(m+i)\lg(m+i) +
a(m+i) + b \leqslant
\OB{\fun{tms}}{m+i}\). Avec~\eqref{ineq:Btms_n_i}, il est impliqué
par
\begin{equation*}
\frac{1}{2}(m+i)\lg(m+i) + a(m+i) + b
\leqslant
\frac{1}{2}m\lg m + \left(\frac{1}{2}i\lg i + ai + b\right) + i.
\end{equation*}
Nous pouvons d'ores et déjà remarquer que cette inégalité est
équivalente à la suivante:
\begin{equation}
\frac{1}{2}m\lg(m+i) + \frac{1}{2}i\lg(m+i) + am
\leqslant \frac{1}{2}m\lg m + \frac{1}{2}i\lg i + i.
\label{ineq:Btms_n_i_details}
\end{equation}
Mais \(\tfrac{1}{2}m\lg m < \tfrac{1}{2}m\lg(m+i)\) et
\(\tfrac{1}{2}i\lg i < \tfrac{1}{2}i\lg(m+i)\), par conséquent, la
constante~\(a\) que nous cherchons doit satisfaire \(am \leqslant
i\), pour tout \(0 < i < m\), donc~\(a\) est strictement négatif.

Nous plongeons~\(i\) dans les nombres réels en posant \(i=x2^p=xm\),
où \(x\)~est un nombre réel tel que \(0 < x \leqslant 1\). En
remplaçant \(i\)~par~\(xm\) dans
l'inégalité~\eqref{ineq:Btms_n_i_details}, nous tirons
\begin{equation*}
\frac{1}{2}(1+x)\lg(1+x) + a \leqslant \frac{1}{2}x\lg x + x.
\end{equation*}
Soit \(\Phi(x) := \tfrac{1}{2}x\lg x - \tfrac{1}{2}(1+x)\lg(1+x) +
x\). L'inégalité précedente est alors équivalente à \(a \leqslant
\Phi(x)\). La fonction~\(\Phi\) est continûment prolongée en~\(0\) car
\(\lim_{x \to 0} x\lg x = 0\), et dérivable sur l'intervalle
fermé \([0,1]\):
\begin{equation}
\frac{d\Phi}{dx} = \frac{1}{2}\lg\frac{4x}{x+1}.
\label{eq:der_Phi}
\end{equation}
La racine de \(d\Phi/dx = 0\) est \myfrac{1}/{3}, et la dérivée est
négative avant et positive après. Par conséquent, \(a_{\max} :=
\min_{0 \leqslant x \leqslant 1}\Phi(x) = \Phi(\tfrac{1}{3}) =
-\tfrac{1}{2}\lg\tfrac{4}{3}\). Le cas de base était \(b \leqslant
-2a\), donc \(b_{\max} := -2a_{\max} = \lg\tfrac{4}{3}\). Finalement,
\begin{equation}
  \frac{1}{2}n\lg n - \left(\frac{1}{2}\lg\frac{4}{3}\right)n + \lg\frac{4}{3}
  \leqslant \OB{\fun{tms}}{n},
\label{ineq:lower_Btms}
\end{equation}
où \(\tfrac{1}{2}\lg\tfrac{4}{3} \simeq 0.2075\). Le plus important
est que le minorant est atteint quand \(x=\myfrac{1}/{3}\),
c'est-à-dire si \(2^p+i=2^p+x2^p=(1+1/3)2^p=2^{p+2}\!/3\), ou, en
général, \(2^k\!/3\). Les entiers les plus proches sont
\(\floor{2^k\!/3}\) et \(\ceiling{2^k\!/3}\), donc nous devons trouver
lequel minimise \(\OB{\fun{tms}}{n} -
\tfrac{1}{2}n\lg(\tfrac{3}{4}n)\), étant donné que \(\tfrac{1}{2}n\lg
n - \left(\tfrac{1}{2}\lg\tfrac{4}{3}\right)n =
\tfrac{1}{2}n\lg(\tfrac{3}{4}n)\). Nous débutons avec:
\begin{lemma}
\label{lem_div3}
\textsl{Les entiers de la forme \(4^p-1\) sont divisibles par~\(3\).}
\end{lemma}
\begin{proof}
  Soit \(\pred{Div}{p}\)\index{Div@\predName{Div}} la proposition à
  prouver. Clairement, \(\pred{Div}{1}\) est vraie. Supposons
  \(\pred{Div}{p}\) et prouvons \(\pred{Div}{p+1}\). L'hypothèse
  d'induction veut dire qu'il existe un entier~\(q\) tel que \(4^p - 1
  = 3q\). Par conséquent, \(4^{p+1} - 1 = 3(4q+1)\), ce qui signifie
  que \(\pred{Div}{p+1}\) est vraie. Le principe d'induction implique
  alors que le lemme est vrai pour tout entier~\(p\).
\end{proof}
\begin{thm}
\label{thm_OB_lambda}
\textsl{Nous avons \(\OB{\mathsf{tms}}{1+\phi_k} -
  \OB{\mathsf{tms}}{\phi_k} = \floor{k/2}\), où \(\phi_k :=
  \floor{2^k\!/3}\).}
\end{thm}
\noindent \emph{Démonstration.} Soit \(\phi_k := \floor{2^k\!/3}\). Soit
\(k\)~est pair ou impair.
\begin{itemize}

  \item Si \(k=2m\), alors \(2^k\!/3 = (4^m-1)/3 + 1/3\). Puisque
    \(1/3<1\) et que, par le lemme~\ref{lem_div3}, \((4^m-1)/3\) est un
    entier: \(2^k\!/3 = \floor{2^k\!/3} + 1/3\) et
    \begin{align*}
      \floor{2^k\!/3} &= (4^m-1)/3 = 4^{m-1} + 4^{m-2} + \dots + 1\\
                      &= 2^{2m-2} + 2^{2m-4} + \dots + 1\\
                      &= (1010\dots01)_2.
    \end{align*}
    Donc \(\nu_{\phi_{2m}} = m\). On sait que \(\OB{\fun{tms}}{m+1} =
    \OB{\fun{tms}}{m} + \nu_m\), par conséquent:
    \(\OB{\fun{tms}}{1+\phi_{2m}} - \OB{\fun{tms}}{\phi_{2m}} = m\),
    soit \(\OB{\fun{tms}}{1+\phi_k} - \OB{\fun{tms}}{\phi_k} =
    \floor{k/2}\).

  \item Si \(k=2m+1\), alors \(2^k\!/3 = 2(4^m\!-\!1)/3 + 2/3\). Puisque
    \(2/3\!<\!1\) et que, par le lemme~\ref{lem_div3},
    \((4^m-1)/3\) est un entier, nous avons donc
    \begin{equation*}
      2^k\!/3 = \floor{2^k\!/3} + 2/3 = \ceiling{2^k\!/3} - 1/3
    \end{equation*}
    et
    \begin{align*}
      \floor{2^k\!/3} &= 2(4^m-1)/3\\
                      &= 2^{2m-1} + 2^{2m-3} + \dots + 2\\
                      &= (1010\dots10)_2;
    \end{align*}
    d'où \(\nu_{\phi_{2m+1}} = m\). D'après \(\OB{\fun{tms}}{m+1} =
    \OB{\fun{tms}}{m} + \nu_m\), nous pouvons alors déduire
    \(\OB{\fun{tms}}{1+\phi_{2m+1}} - \OB{\fun{tms}}{\phi_{2m+1}} =
    m\), soit:\index{tms@$\OB{\fun{tms}}{n}$|(}
    \(\OB{\fun{tms}}{1+\phi_k} - \OB{\fun{tms}}{\phi_k} =
    \floor{k/2}\).\hfill\(\Box\)

\end{itemize}

\bigskip

\noindent Soit \(Q(x) := \tfrac{1}{2}x\lg(\tfrac{3}{4}x)\).  Comparons
\(\OB{\fun{tms}}{\phi_k} - Q(\phi_k)\) avec \(\OB{\fun{tms}}{1+\phi_k}
- Q(1+\phi_k)\) selon la parité de~\(k\). Si la première différence
est plus petite, alors l'entier \(p=\phi_k\) minimise
\(\OB{\fun{tms}}{p} - \tfrac{1}{2}p\lg(\tfrac{3}{4}p)\); sinon c'est
\(p=1+\phi_k\):
\begin{itemize}

  \item Si \(k=2m+2\), alors \(\phi_{2m+2} = 2^{2m} + \phi_{2m}\)
    (théorème~\ref{thm_OB_lambda}). D'après
    l'équation~\eqref{eq:OBtms_2m_i},
    \begin{equation*}
      \OB{\fun{tms}}{\phi_{2m+2}} = \OB{\fun{tms}}{2^{2m}} +
    \OB{\fun{tms}}{\phi_{2m}} + \phi_{2m} = \OB{\fun{tms}}{\phi_{2m}}
    - m4^m + \phi_{2m}.
    \end{equation*}
    En sommant les deux membres, de \(m=0\) à \(m=n-1\), mène à
    \begin{equation*}
      \OB{\fun{tms}}{\phi_{2n}} = \OB{\fun{tms}}{\phi_0} + S_n
    + \sum_{m=0}^{n-1}{\phi_{2m}},\; \text{where}\; S_n :=
    \sum_{m=0}^{n-1}{m4^m}.
    \end{equation*}
    Nous devons maintenant trouver une forme close de~\(S_n\):
    \begin{equation*}
      S_n + n4^n = \sum_{m=1}^{n}{m4^m}
                 = \sum_{m=0}^{n-1}(m+1)4^{m+1}
                 = 4 \cdot S_n + 4\sum_{m=0}^{n-1}4^m.
    \end{equation*}
    De \(\sum_{m=0}^{n-1}4^m = (4^n-1)/3\), nous tirons \(9 \cdot S_n
    = (3n -4)4^n + 4\). D'un autre côté, \(9
    \sum_{m=0}^{n-1}{\phi_{2m}} = 4^n - 3n - 1\). Finalement,
    remarquons que \(\phi_0 = 0\) et \(\OB{\fun{tms}}{0} = 0\), nous
    déduisons
    \begin{equation}
      \OB{\fun{tms}}{\phi_{2n}} = (n-1)\phi_{2n}.
      \label{eq:OBtms_phi_2n}
    \end{equation}
    Cherchons maintenant
    \begin{equation*}
      Q(\phi_{2n}) = \frac{1}{2}\phi_{2n}(2(n-1)
      + \lg(1-1/4^n)) = \OB{\fun{tms}}{\phi_{2n}}
      + \frac{1}{2}\phi_{2n}\lg(1-1/4^n),
    \end{equation*}
    en usant de~\eqref{eq:OBtms_phi_2n}. Si nous définissons \(f(x) :=
    (1-x) \ln(1-1/x)\), alors \(\OB{\fun{tms}}{\phi_{2n}} -
    Q(\phi_{2n}) = f(4^n)/(6\ln 2)\). Un peu d'analyse réelle
    élémentaire révèle que \(3\ln(3/4) \leqslant f(x) < 1\),
    si \(x \geqslant 4\), c'est-à-dire
    \begin{equation*}
      1 - \frac{1}{2}\lg 3 \leqslant
    \OB{\fun{tms}}{\phi_{2n}} - Q(\phi_{2n}) < \frac{1}{6\ln 2},\; \text{si \(n
      \geqslant 1\)}.
    \end{equation*}
    Cela donne l'encadrement
    \begin{equation}
      0.2075 < \OB{\fun{tms}}{\phi_{2n}} - Q(\phi_{2n}) < 0.2405.
      \label{ineq:OBtms_phi_2n}
    \end{equation}
    D'après le théorème~\ref{thm_OB_lambda} et
    l'équation~\eqref{eq:OBtms_phi_2n}, nous avons
    \begin{equation}
      \OB{\fun{tms}}{1+\phi_{2n}} = (1+\phi_{2n})(n-1) + 1.
      \label{eq:OBtms_succ_phi_2n}
    \end{equation}
    De plus,
    \begin{align*}
      Q(1+\phi_{2n}) &= \frac{1}{2}(1+\phi_{2n}) (2(n-1) + \lg(1+1/2^{2n-1}))\\
                    &= \OB{\fun{tms}}{1+\phi_{2n}} - 1 +
      \frac{1}{6}(4^n + 2)\lg(1+2/4^n),
    \end{align*}
    d'après~\eqref{eq:OBtms_succ_phi_2n}. Si l'on pose \(g(x) := 1 -
    \tfrac{1}{6}(x+2)\lg(1+2/4^n)\), alors
    \(\OB{\fun{tms}}{1+\phi_{2n}} - Q(1+\phi_{2n}) = g(4^n)\). Une
    analyse élémentaire montre que \(2 - \lg 3 \leqslant g(x) < 1 -
    1/(3\ln 2)\), quand \(x \geqslant 4\), c'est-à-dire \(2 - \lg 3
    \leqslant g(4^n) < 1 - 1/(3\ln 2)\), où \(n \geqslant 1\). D'où
    l'encadrement
    \begin{equation}
      0.4150 < \smash{\OB{\fun{tms}}{1+\phi_{2n}}} - Q(1+\phi_{2n}) <
      0.5192.
      \label{ineq:OBtms_1_phi_2n}
    \end{equation}

    \bigskip
    \noindent \textsl{En comparant~\eqref{ineq:OBtms_phi_2n}
      et~\eqref{ineq:OBtms_1_phi_2n}, on a \(p=\phi_{2n} =
      (1010\dots01)_2\) qui minimise \(\OB{\fun{tms}}{p} -
      \tfrac{1}{2}p\lg(\tfrac{3}{4}p)\).}
  \bigskip

  \item Si \(k=2m+1\), alors \(\phi_{2m+1} = 2^{2m-1} + \phi_{2m-1}\)
    (théorème~\ref{thm_OB_lambda}). D'après
    l'équation~\eqref{eq:OBtms_2m_i}, nous tirons
    \begin{align*}
      \OB{\fun{tms}}{\phi_{2m+1}} &= \OB{\fun{tms}}{2^{2m-1}} +
      \OB{\fun{tms}}{\phi_{2m-1}} + \phi_{2m-1}\\
      &= \OB{\fun{tms}}{\phi_{2m-1}} - (2m-1)4^{m-1} + \phi_{2m-1}.
    \end{align*}
    En sommant membre à membre pour \(m=1\) à~\(m=n-1\) entraîne
    \(9\OB{\fun{tms}}{\phi_{2n+1}} = S_{n+1}/2 - 3(4^n-1) +
    \sum_{m=0}^{n-1}\phi_{2m+1}\), qui se simplifie en
    \begin{equation}
      \OB{\fun{tms}}{\phi_{2n+1}} = \frac{1}{2}\phi_{2n+1}(2n - 1).
      \label{eq:OBtms_phi_2n_1}
    \end{equation}
    Nous avons alors
    \begin{align*}
      Q(\phi_{2n+1}) &= \frac{1}{2}\phi_{2n+1}(2n-1 + \lg(1-1/4^n))\\
      &= \OB{\fun{tms}}{\phi_{2n+1}} + \frac{1}{2}\phi_{2n+1}\lg(1-1/4^n),
    \end{align*}
    où la dernière égalité résulte de~\eqref{eq:OBtms_phi_2n_1}. Si,
    comme nous l'avons fait pour \(Q(\phi_{2n})\), nous posons \(f(x)
    := (1-x)\ln(1-1/x)\), nous avons \(\OB{\fun{tms}}{\phi_{2n+1}} -
    Q(\phi_{2n+1}) = f(4^n)/(3\ln 2)\). Nous savons déjà que
    \(3\ln(3/4) \leqslant f(x) < 1\), si \(x \geqslant 4\),
    soit
    \begin{equation*}
      \lg 3 - 2 \leqslant \OB{\fun{tms}}{\phi_{2n+1}} -
      Q(\phi_{2n+1}) < \frac{1}{3\ln 2},\; \text{si \(n \geqslant 1\)}.
    \end{equation*}
    Nous avons donc établi:
    \begin{equation}
      0.4150 < \smash{\OB{\fun{tms}}{\phi_{2n+1}}} -
      Q(\phi_{2n+1}) < 0.4809.
      \label{ineq:OBtms_phi_2n_1}
    \end{equation}
    Par ailleurs, d'après le théorème~\ref{thm_OB_lambda} et
    l'équation~\eqref{eq:OBtms_phi_2n_1}, nous tirons
    \begin{equation}
      \OB{\fun{tms}}{1+\phi_{2n+1}} = \frac{1}{2}(1+\phi_{2n+1})(2n - 1)
      + \frac{1}{2}.
      \label{eq:OBtms_succ_phi_2n_1}
    \end{equation}
    De plus,
    \begin{align*}
      Q(1+\phi_{2n+1}) &= \frac{1}{2}(1+\phi_{2n+1})(2n-1 + \lg(1+1/2^{2n+1}))\\
                      &= \OB{\fun{tms}}{1+\phi_{2n+1}} - \frac{1}{2}
                         + \frac{1}{6}(1+2^{2n+1})\lg(1+1/2^{2n+1}),
    \end{align*}
    d'après~\eqref{eq:OBtms_succ_phi_2n_1}. Alors
    \(\OB{\fun{tms}}{1+\phi_{2n+1}} - Q(1+\phi_{2n+1}) =
    h(2^{2n+1})\), où \(h(x) := \tfrac{1}{2} -
    \tfrac{1}{6}(1+x)\lg(1+1/x)\). Une analyse élémentaire montre que
    \(5 - 3\lg 3 \leqslant h(x) < 1/2 - 1/(6\ln 2)\), si \(x \geqslant
    8\), c'est-à-dire que \(5 - 3\lg 3 \leqslant
    \OB{\fun{tms}}{1+\phi_{2n+1}} - Q(1+\phi_{2n+1}) < 1/2 - 1/(6\ln
    2)\), si \(n \geqslant 1\). D'où
    \begin{equation}
      0.2450 < \smash{\OB{\fun{tms}}{1+\phi_{2n+1}}} -
      Q(1+\phi_{2n+1}) < 0.2596.
      \label{ineq:OBtms_1_phi_2n_1}
    \end{equation}

    \bigskip
    \noindent
    \textsl{De~\eqref{ineq:OBtms_phi_2n_1}
      et~\eqref{ineq:OBtms_1_phi_2n_1}, on a \(p = 1+\phi_{2m+1} =
      (1010\dots1011)_2\) qui minimise \(\OB{\fun{tms}}{p} -
      \tfrac{1}{2}p\lg p\).}

\end{itemize}

\bigskip

Finalement, nous déduisons de l'analyse précédente que le minorant
de~\eqref{ineq:lower_Btms} est atteint si \(n=2\) (le cas de base) et
est sinon approché au plus près quand \(n=(1010\dots01)_2\) ou
\(n=(1010\dots1011)_2\). Prises comme un tout, ces valeurs constituent
la \emph{suite de Jacobsthal},\index{Jacobsthal!nombre de $\sim$}
définie par
\begin{equation}
J_0 = 0; \; J_1=1; \; J_{n+2} = J_{n+1} + 2J_{n},\; \text{for \(n
  \geqslant 0\).}
\label{eq:Jacobsthal}
\end{equation}

Employons maintenant la même approche inductive pour trouver un bon
majorant à \(\OB{\fun{tms}}{n}\). En d'autres termes, nous voulons
minimiser les constantes réelles \(a'\)~et~\(b'\) telles que, si \(n
\geqslant 2\),
\begin{equation*}
\OB{\fun{tms}}{n} \leqslant \frac{1}{2}n\lg n + a'n + b'.
\end{equation*}
La seule différence avec la recherche du minorant est que les
inégalités sont inversées, donc nous voulons
\begin{equation*}
\Phi(x) \leqslant a', \;\text{où \(\Phi(x) := \frac{1}{2}x\lg x - \frac{1}{2}(1+x)\lg(1+x) + x\)}.
\end{equation*}
Ici, il nous faut trouver le maximum de~\(\Phi\) sur l'intervalle
fermé \([0,1]\). Les deux racines positives de~\(\Phi\) sont
\(0\)~et~\(1\), et \(\Phi\)~est négative entre elles
(voir~\eqref{eq:der_Phi}). D'où: \(a'_{\min} := \max_{0 \leqslant x
  \leqslant 1}\Phi(x) = \Phi(0) = \Phi(1) = 0\). Par ailleurs, d'après
le cas de base: \(b'_{\min} = -2a_{\min} = 0\). Finalement, nous
avons dérivé l'encadrement suivant:
\begin{equation}
  \tfrac{1}{2}n\lg n - \left(\frac{1}{2}\lg\frac{4}{3}\right)n +
  \lg\frac{4}{3} \leqslant \OB{\fun{tms}}{n} \leqslant
  \frac{1}{2}n\lg n.
\label{ineq:bounds_Btms}
\index{tms@$\OB{\fun{tms}}{n}$}
\end{equation}
Le majorant est clairement atteint si \(n=2^p\) à cause de
l'équation~\eqref{eq:best_power}. Il est aussi évident maintenant que
\(\OB{\fun{tms}}{n} \sim \frac{1}{2}n\lg n\), mais si nous n'étions
seulement intéressés que par ce résultat asymptotique,
\cite{Bush_1940} en donne une dérivation très simple grâce à un
dénombrement de bits de la \fig~\vref{fig:bits}. Rappelons que
\cite{Delange_1975} a étudié \(\OB{\fun{tms}}{n}\) dans le cadre de
l'analyse réelle et a montré que \(\OB{\fun{tms}}{n} =
\tfrac{1}{2}n\lg n + F_0(\lg n) \cdot n\), où \(F_0\)~est une fonction
périodique de période~\(1\), continue et nulle part dérivable, et dont
la série de Fourier montre que sa valeur moyenne est environ
\(-0.145599\).  \index{tms@$\OB{\fun{tms}}{n}$|)} \index{tri par
  interclassement!$\sim$ descendant!coût minimal|)}

\mypar{Coût maximal}
\index{tri par interclassement!$\sim$ descendant!coût maximal|(}

Le nombre maximal de comparaisons satisfait
\begin{equation*}
\OW{\fun{tms}}{0} = \OW{\fun{tms}}{1} = 0,
\qquad
\OW{\fun{tms}}{n} = \OW{\fun{tms}}{\floor{n/2}}
+ \OW{\fun{tms}}{\ceiling{n/2}}
+ \OW{\fun{mrg}}{\floor{n/2},\ceiling{n/2}}.
\index{tms@$\OW{\fun{tms}}{n}$|(}
\index{mrg@\fun{mrg/2}}
\end{equation*}
L'équation~\eqref{eq:worst_merge} \vpageref{eq:worst_merge} done
\(\OW{\fun{tms}}{n} = \OW{\fun{tms}}{\floor{n/2}} +
\OW{\fun{tms}}{\ceiling{n/2}} + n - 1\) et
\begin{equation*}
\OW{\fun{tms}}{0} = \OW{\fun{tms}}{1} = 0;\;
\OW{\fun{tms}}{2p} = 2\OW{\fun{tms}}{p} + 2p - 1,\;
\OW{\fun{tms}}{2p+1} = \OW{\fun{tms}}{p} + \OW{\fun{tms}}{p+1} + 2p.
\end{equation*}
Posons la différence entre deux termes successifs: \(\Delta_n :=
\smash[t]{\OW{\fun{tms}}{n+1}} - \OW{\fun{tms}}{n}\).
\begin{itemize}

  \item Si \(n=2p\), alors \(\Delta_{2p} = \Delta_{p} + 1\).

  \item Si \(n=2p+1\), alors \(\OW{\fun{tms}}{2p+2} = 2 \cdot
  \OW{\fun{tms}}{p+1} + 2p + 1\), d'où \(\Delta_{2p+1} = \Delta_{p} +
  1\).

\end{itemize}
En somme, \(\Delta_0 = 0\) et \(\Delta_n = \Delta_{\floor{n/2}} +
1\). Si nous déroulons la récurrence, nous trouvons que \(\Delta_n =
\Delta_{\floor{\floor{n/2}/2}}+ 2\), donc nous devons simplifier
un terme de la forme \(\floor{\floor{\floor{\dots}/2}/2}\).
\begin{thm}[Parties entières par défaut et fractions]
\label{thm_floors}
\textsl{Soit \(x\)~un nombre réel et \(q\)~un entier naturel. Alors
  \(\floor{\floor{x}/q} = \floor{x/q}\).}
\end{thm}
\begin{proof}
  L'égalité est équivalente à la conjonction des deux inégalités
  complémentaires \(\floor{\floor{x}/q} \leqslant \floor{x/q}\) et
  \(\floor{x/q} \leqslant \floor{\floor{x}/q}\). La première est une
  simple conséquence de \(\floor{x} \leqslant x\). Puisque les deux
  membres de la seconde sont des entiers, \(\floor{x/q} \leqslant
  \floor{\floor{x}/q}\) est équivalente à \(p \leqslant \floor{x/q}
  \Rightarrow p \leqslant \floor{\floor{x}/q}\), pour tout
  entier~\(p\). Un lemme évident est que si \(i\)~est un entier et
  \(y\)~un réel, \(i \leqslant \floor{y} \Leftrightarrow i
  \leqslant y\), donc l'inégalité originelle est équivalente à
  \(p \leqslant x/q \Rightarrow p \leqslant \floor{x}/q\), qui est
  trivialement équivalente à \(pq \leqslant x \Rightarrow pq \leqslant
  \floor{x}\). Puisque \(pq\)~est un entier, cette implication est
  vraie par le même petit lemme précédent.
\end{proof}
En usant du théorème~\vref{thm_floors}, nous déduisons \(\Delta_n =
m\), où \(m\)~est le plus petit entier naturel tel que \(\floor{n/2^m}
= 0\). En d'autres termes, \(m\)~est le nombre de bits dans la
notation binaire de~\(n\), que nous retrouvons à
l'équation~\eqref{eq:e_r}, donc \(\Delta_n = \floor{\lg n} +
1\). En revenant à la définition de~\(\Delta_n\), formons
\(\sum_{k=1}^{n-1}\Delta_k\) et donc
\begin{equation}
\OW{\fun{tms}}{n} = \sum_{k=1}^{n-1}(\floor{\lg k}+1).
\label{eq:tms_n_tmp}
\end{equation}
Alors que le coût minimal est le nombre de bits à~\(1\) jusqu'à
\(n-1\), nous voyons maintenant que le coût maximal est le nombre
total de bits jusqu'à \(n-1\). Cela nous incite à parier que
\(\OW{\fun{tms}}{n} \sim 2 \cdot \OB{\fun{tms}}{n} \sim n\lg n\), car
nous nous attendons à ce que le nombre de bits à~\(0\) et à~\(1\)
soient les mêmes en moyenne. Considérons à nouveau la table des bits à
la figure \fig~\vref{fig:bits}. La plus grande puissance de~\(2\) qui
est inférieure à~\(n\) est~\(2^{\floor{\lg n}}\) parce que c'est le
nombre binaire \((10\dots0)_2\) qui possède le même nombre de bits
que~\(n\); il figure donc dans la même section de la table
que~\(n\). La technique consiste à compter les bits en colonnes, du
haut vers le bas, et vers la gauche. Dans la colonne la plus à droite,
nous trouvons \(n\)~bits. Dans la deuxième colonne à partir de la
droite, nous lisons \(n-2^1+1\) bits. La troisième contient
\(n-2^2+1\) bits etc. jusqu'à la colonne la plus à gauche qui regroupe
\(n-2^{\floor{\lg n}}+1\) bits. Le nombre total de bits dans la table
est donc
\begin{equation*}
\sum_{k=1}^{n}{\!(\floor{\lg k}+1)}
   = \sum_{k=0}^{\floor{\lg n}}{\!(n-2^k+1)}
   = (n + 1)(\floor{\lg n} + 1) - 2^{\floor{\lg n}+1} + 1.
\end{equation*}
Exprimons la représentation binaire de~\(n\): \(n := (b_{m-1}\dots
b_0)_2\), alors \(2^{m-1} \leqslant n \leqslant 2^m - 1\) et \(2^{m-1}
< 2^{m-1} + 1 \leqslant n + 1 \leqslant 2^m\), et donc \(m-1 < \lg(n+1)
\leqslant m\), d'où \(m = \ceiling{\lg(n+1)}\), ce qui, avec l'aide de
l'équation~\eqref{eq:e_r} \vpageref{eq:e_r}, prouve \(1 + \floor{\lg
  n} = \ceiling{\lg(n+1)}\). En conséquence,
l'équation~\eqref{eq:tms_n_tmp} peut être réécrite comme
\begin{equation}
\OW{\fun{tms}}{0} = \OW{\fun{tms}}{1} = 0,
\qquad
\OW{\fun{tms}}{n} = n\ceiling{\lg n} - 2^{\ceiling{\lg n}} + 1.
\label{eq:top}
\end{equation}
Cette équation est plus subtile qu'elle ne paraît, à cause de la
périodicité cachée dans \(2^{\ceiling{\lg n}}\). Nous analysons selon
que \(n = 2^p\) ou non:
\begin{itemize}

  \item si~\(n=2^p\), alors \(\OW{\fun{tms}}{n} = n\lg n - n + 1\);

  \item sinon, nous avons \(\ceiling{\lg n} = \floor{\lg n} + 1 = \lg
    n - \{\lg n\} + 1\) et puis \(\OW{\fun{tms}}{n} = n\lg n +
    \theta(1 - \{\lg n\}) \cdot n + 1\), où \(\theta(x) := x - 2^x\)
    et la \emph{partie fractionnaire}\index{partie
      fractionnaire@$\{x\}$|see{partie fractionnaire}}\index{partie
      fractionnaire} du réel~\(x\) est \(\{x\} := x - \floor{x}\). En
    particulier, \(0 \leqslant \{x\} < 1\). La dérivée est
    \(\theta'(x) = 1 - 2^x\ln 2\); elle a une racine \(\theta'(x_0) =
    0 \Leftrightarrow x_0 = -\lg\ln 2\); elle est positive
    avant~\(x_0\), et négative après.  Donc \(\theta(x)\) atteint son
    maximal en~\(x_0\):
    \begin{equation*}
      \max_{0<x\leqslant 1}\theta(x) = \theta(x_0)
      = -(1+\ln\ln{2})/\!\ln{2} \simeq -0.9139,
    \end{equation*}
    et \(\min_{0<x\leqslant 1}\theta(x) = \theta(1) = -1\). Par
    injectivité, \(\theta(1) = \theta(1-\{\lg n\})\) implique \(\{\lg
    n\} = 0\), soit \(n=2^p\) (premier cas).
\end{itemize}
En conclusion: \(\OW{\fun{tms}}{n} = n\lg n + A(\lg n) \cdot n + 1\),
où \(A(x) := 1 - \{x\} - 2^{1 - \{x\}}\) est une fonction périodique,
car \(A(x) = A(\{x\})\), encadrée comme suit: \(-1 \leqslant A(x) <
-1/\!\ln 2 - \lg\ln 2 < -0.91\). Une analyse plus poussée de~\(A(x)\)
fait appel à des séries de Fourier ou à l'analyse complexe; sa valeur
moyenne est environ \(-0.942695\). Les auteurs de référence sur ce
sujet sont \cite{FlajoletGolin_1994}, et aussi
\cite{PannyProdinger_1995}. Nous pouvons maintenant conclure:
\begin{equation}
n\lg n - n + 1 \leqslant \OW{\fun{tms}}{n} <
n\lg n - 0.91 n + 1.\label{ineq:OWtms}
\end{equation}
Le minorant est atteint si \(n=2^p\). Le majorant est approché au
mieux quand \(\{\lg n\} = 1 + \lg\ln 2\), soit quand \(n\)~est
l'entier le plus proche de \(2^p\ln 2\) (prendre la notation binaire
de \(\ln 2\), décaler le point décimal (ou virgule) \(p\)~fois vers la
droite et arrondir). Très clairement, \(\OW{\fun{tms}}{n} \sim n\lg
n\).\index{tms@$\OW{\fun{tms}}{n}$|)} \index{tri par
  interclassement!$\sim$ descendant!coût maximal|)}

\mypar{Coût moyen}
\index{tri par interclassement!$\sim$ descendant!coût moyen|(}

Soit \(\OM{\fun{tms}}{n}\)\index{tms@$\OM{\fun{tms}}{n}$|(} le nombre
moyen de comparaisons pour trier \(n\)~clés de façon
descendante. Toutes les permutations de la pile donnée étant également
probables, l'équation~\eqref{eq:cost_tms} devient
\begin{equation*}
\OM{\fun{tms}}{0} = \OM{\fun{tms}}{1} = 0,\qquad
\OM{\fun{tms}}{n} = \OM{\fun{tms}}{\floor{n/2}} +
\OM{\fun{tms}}{\ceiling{n/2}} +
\OM{\fun{mrg}}{\floor{n/2},\ceiling{n/2}}.\index{mrg@\fun{mrg/2}}
\end{equation*}
L'équation~\eqref{eq:Amrg} implique, à son tour,
\begin{equation*}
\OM{\fun{tms}}{n} = \OM{\fun{tms}}{\floor{n/2}} +
\OM{\fun{tms}}{\ceiling{n/2}} + n -
\frac{\floor{n/2}}{\ceiling{n/2}+1}
- \frac{\ceiling{n/2}}{\floor{n/2}+1}.
\end{equation*}
Si nous procédons comme avec les extremums du coût, nous obtenons
\begin{equation*}
\OM{\fun{tms}}{2p} = 2\cdot\OM{\fun{tms}}{p} + 2p - 2 +
\frac{2}{p+1},\; \OM{\fun{tms}}{2p+1} = \OM{\fun{tms}}{p} +
\OM{\fun{tms}}{p+1} + 2p - 1 + \frac{2}{p+2}.
\end{equation*}
Ces récurrences sont assez délicates. Poser \(\Delta_n :=
\OM{\fun{tms}}{n+1} - \OM{\fun{tms}}{n}\) mène à
\begin{equation*}
\Delta_{2p} = \Delta_p + 1 + \frac{2}{p+2} - \frac{2}{p+1},
\quad
\Delta_{2p+1} = \Delta_{p} + 1.
\end{equation*}
Contrairement aux équations aux différences que nous avons dérivées
pour les extremums du coût, celles-ci ne sont pas d'une grande aide,
par conséquent nous devrions opter pour une approche inductive, comme
nous l'avons fait pour trouver l'encadrement serré de
\(\OB{\fun{tms}}{n}\). Les inéquations~\eqref{ineq:M_join}
\vpageref{ineq:M_join} sont équivalentes à
\begin{equation*}
  n\lg n - \alpha n +
\alpha < \OM{\fun{tms}}{n} < n\lg n - \alpha n + 2,\; \text{si \(n =
  2^p\),}
\end{equation*}
et cela nous suggère de rechercher des bornes de la forme \(n\lg n +
an + b\) aussi quand \(n \neq 2^p\). Commençons par la borne
inférieure et entreprenons de maximiser les constantes réelles
\(a\)~et~\(b\) dans
\begin{equation*}
\abovedisplayskip=0pt
\belowdisplayskip=2pt
\pred{H}{n} \colon n\lg n + an + b \leqslant \OM{\fun{tms}}{n},
\; \text{avec \(n \geqslant 2\).}
\end{equation*}
Puisque \(\pred{H}{2p}\) dépend de \(\pred{H}{p}\), et que
\(\pred{H}{2p\!+\!1}\) dépend de \(\pred{H}{p}\) et
\(\pred{H}{p\!+\!1}\), la propriété \(\pred{H}{n}\), pour tout
\(n>1\), dépend donc transitivement de \(\pred{H}{2}\) seul, parce que
nous itérons des divisions par~\(2\). Si nous écrivons \(\pred{H}{n}
\leadsto \pred{H}{m}\) pour dire: «~\(\pred{H}{n}\) dépend de
\(\pred{H}{m}\)~», nous avons alors, par exemple, \(\pred{H}{2^3}
\leadsto \pred{H}{2^2} \leadsto \pred{H}{2^1}\); \(\pred{H}{7}
\leadsto \pred{H}{3} \leadsto \pred{H}{2}\) et \(\pred{H}{7} \leadsto
\pred{H}{4} \leadsto \pred{H}{2}\). \(\pred{H}{2}\) équivaut à
\begin{equation}
2a + b + 1 \leqslant 0.
\label{ineq:base_lower_Atms}
\end{equation}
Parce que la définition de \(\OM{\fun{tms}}{n}\) dépend de la parité
de~\(n\), le pas inductif sera double. Supposons
\(\pred{H}{n}\) pour \(n < 2p\), en particulier, nous supposerons
\(\pred{H}{p}\), qui, avec l'expression de \(\OM{\fun{tms}}{2p}\)
ci-dessus, équivaut à
\begin{equation*}
\abovedisplayskip=2pt
\belowdisplayskip=2pt
 (2p\lg p + 2ap + 2b) + 2p - 2 + \frac{2}{p+1} \leqslant \OM{\fun{tms}}{2p}.
\end{equation*}
Nous voulons \(\pred{H}{2p} \colon 2p\lg(2p) + 2ap + b
= 2p\lg p + 2ap + 2p + b \leqslant \OM{\fun{tms}}{2p}\), qui tient si
la condition suffisante suivante est satisfaite:
\begin{equation*}
\abovedisplayskip=0pt
\belowdisplayskip=2pt
2p\lg p + 2ap + 2p + b \leqslant 2p\lg p + 2ap + 2b + 2p - 2 + \frac{2}{p+1},
\end{equation*}
ce qui équivaut à
\begin{equation*}
\abovedisplayskip=0pt
\belowdisplayskip=2pt
2 - \frac{2}{p+1} = \frac{2p}{p+1} \leqslant b.
\end{equation*}
Soit \(\Phi(p) := 2p/(p+1)\). Cette fonction est strictement
croissante pour les valeurs \(p > 0\) et \(\Phi(p) \rightarrow
2^{-}\), lorsque \(p \rightarrow +\infty\).

L'autre pas inductif concerne les valeurs impaires de~\(n\). Nous
supposerons \(\pred{H}{n}\) pour tout \(n < 2p+1\), en particulier,
nous supposons \(\pred{H}{p}\) et \(\pred{H}{p+1}\), qui, avec
l'expression de \(\OM{\fun{tms}}{2p+1}\) ci-dessus, implique
\begin{equation*}
%\abovedisplayskip=0pt
%\belowdisplayskip=2pt
(p\lg p + ap + b) + ((p+1)\lg(p+1) + a(p+1) + b) + 2p - 1 +
\frac{2}{p+2} \leqslant \OM{\fun{tms}}{2p+1},
\end{equation*}
qui peut être simplifiée légèrement comme suit:
\begin{equation*}
%\abovedisplayskip=0pt
%\belowdisplayskip=2pt
p\lg p + (p+1)\lg(p+1) + a(2p+1) + 2b + 2p - 1 + \frac{2}{p+2}
\leqslant \OM{\fun{tms}}{2p+1}.
\end{equation*}
Nous désirons établir \(\pred{H}{2p+1} \colon (2p+1)\lg(2p+1) +
a(2p+1) + b \leqslant \OM{\fun{tms}}{2p+1}\), qui est donc impliquée
par
\begin{equation}
%\abovedisplayskip=0pt
%\belowdisplayskip=2pt
  (2p+1)\lg(2p+1) \leqslant
  p\lg p + (p+1)\lg(p+1) + b + 2p - 1 + \frac{2}{p+2}.
  \label{ineq:Psi_temp}
\end{equation}
Soit \(\Psi(p) := (2p+1)\lg(2p+1) - (p+1)\lg(p+1) - p\lg p - 2p + 1 -
2/(p+2)\). Alors~\eqref{ineq:Psi_temp} équivaut à \(\Psi(p) \leqslant
b\). De plus,
\begin{equation*}
\frac{d\Psi}{dp}(p) = \frac{2}{(p+2)^2} + \lg\left(1+\frac{1}{4p(p+1)}\right).
\end{equation*}
Clairement, \(d\Psi/dp > 0\), si \(p > 0\), donc \(\Psi(p)\)~croît
strictement sur \(p > 0\). Cherchons \(\lim_{p \to +\infty}\Psi(p)\)
en réécrivant \(\Psi(p)\) comme suit:
\begin{align*}
\Psi(p)
  &= 2 - \frac{2}{p+2} + (2p+1)\lg(p+1/2) - (p+1)\lg(p+1)
     - p\lg p\\
  &= 2 - \frac{2}{p+2} + p\left(\lg(p+1/2)^2 - \lg(p+1)
   - \lg p\right) + \lg(p+1/2)\\
  &\phantom{=} \quad - \lg(p+1)\\
  &= 2 - \frac{2}{p+2} + p\lg\left(1 + \frac{1}{4p(p+1)}\right) +
  \lg\frac{p + 1/2}{p+1}.
\end{align*}
La limite de \(x\ln(1+1/x^2)\) quand \(x \to +\infty\) est obtenue en
changeant \(x\)~en \(1/y\) et en considérant la limite quand \(y \to
0^{+}\), qui est~\(0\) par la règle de l'Hôpital. Ce résultat peut
être étendu et appliqué au grand terme dans \(\Psi(p)\), et, puisque
tous les autres termes variables convergent vers~\(0\), nous pouvons
conclure que \(\Psi(p) \to 2^{-}\), lorsque \(p \to +\infty\).

Étant donné que nous devons satisfaire les conditions \(\Psi(p)
\leqslant b\) et \(\Phi(p) \leqslant b\) pour que le double pas
inductif tienne, nous devons comparer \(\Psi(p)\)~et~\(\Phi(p)\)
lorsque \(p\)~est un entier naturel: nous avons \(\Phi(1) < \Psi(1)\)
et \(\Phi(2) < \Psi(2)\), mais \(\Psi(p) < \Phi(p)\) si \(p \geqslant
3\). Par conséquent, pour que la constante~\(b\) ne dépende pas
de~\(p\), elle doit être supérieure à~\(2\), qui est le plus petit
majorant de~\(f\) et~\(g\).  Mais
l'inéquation~\eqref{ineq:base_lower_Atms} nous dit que nous devons
minimiser~\(b\) pour maximiser~\(a\) (ce qui constitue la priorité),
donc devons choisir la limite: \(b_{\min} = 2\). La même inégalité
entraîne \(a \leqslant -3/2\), donc \(a_{\max} = -3/2\) et le principe
d'induction complète établit
\begin{equation}
  n\lg n - \frac{3}{2} n + 2 < \OM{\fun{tms}}{n},\; \text{pour \(n
    \geqslant 2\)}.
\label{ineq:lower_Atms}
\end{equation}
Ce minorant n'est pas très bon, mais il a été obtenu sans grande
peine. Nous pouvons nous souvenir du minorant au cas où \(n=2^p\),
dans~\eqref{ineq:M_join} \vpageref{ineq:M_join}: \(n\lg n - \alpha n +
\alpha < \OM{\fun{tms}}{n}\), où \(\alpha \simeq 1.264499\). En fait,
\cite{FlajoletGolin_1994} ont prouvé
\begin{equation}
n\lg n - \alpha n < \OM{\fun{tms}}{n}.
\label{ineq:best_lower_Atms}
\end{equation}
À l'asymptote, ce minorant est identique à celui pour le cas
\(n=2^p\). Notre méthode inductive ne peut mener à ce beau résultat
car elle produit des conditions suffisantes qui sont trop fortes, en
particulier, nous n'avons pas trouvé de décomposition de la forme
\(\OM{\fun{tms}}{2^p+i} = \OM{\fun{tms}}{2^p} + \OM{\fun{tms}}{i} +
\dots\)

Cherchons maintenant les plus petites constantes réelles
\(a'\)~et~\(b'\) telles que pour \(n \geqslant 2\),
\(\OM{\fun{tms}}{n} \leqslant n\lg n + a'n + b'\). Le cas de base de
\(\pred{H}{n}\) dans~\eqref{ineq:base_lower_Atms} est ici inversé:
\(2a' + b' + 1 \geqslant 0\). Par conséquent, pour minimiser~\(a'\),
nous devons maximiser~\(b'\). Par ailleurs, les conditions sur~\(b'\)
déduites des pas inductifs sont aussi inversées par rapport à~\(b\):
\(b' \leqslant \Phi(p)\) et \(b' \leqslant \Psi(p)\). Le cas de base
est \(\pred{H}{2}\), soit \(p=1\), et nous avons vu tantôt que
\(\Phi(1) \leqslant \Psi(1)\), donc nous devons avoir \(b'< \Phi(1) =
1\). Par conséquent, la valeur maximale est \(b'_{\max} =
1\). Finalement, ceci implique que \(a'\geqslant -1\), d'où
\(a'_{\min} = -1\).

En regroupant nos bornes, nous avons l'encadrement
\begin{equation*}
n\lg n - \frac{3}{2}n + 2 < \OM{\fun{tms}}{n} < n\lg n - n + 1.
\end{equation*}
Trivialement, nous avons \(\OM{\fun{tms}}{n} \sim n\lg n \sim
\OW{\fun{tms}}{n} \sim 2 \cdot \OB{\fun{tms}}{n}\).
\cite{FlajoletGolin_1994} ont prouvé, au moyen d'une analyse bien
au-delà de la portée de ce livre, le résultat très fort suivant:
\begin{equation*}
\OM{\fun{tms}}{n} = n\lg n + B(\lg n) \cdot n + \mathcal{O}(1),
\end{equation*}
où \(B\)~est une fonction périodique de période~\(1\), continue et
nulle part dérivable, dont la valeur moyenne est environ
\(-1.2481520\). La notation \(\mathcal{O}(1)\) est un cas particulier
de la notation dite de Bachmann pour désigner une constante positive
inconnue. La valeur maximale atteinte par \(B(x)\) est
approximativement \(-1.24075\), donc
\begin{equation*}
\belowdisplayskip=0pt
\OM{\fun{tms}}{n} = n\lg n - (1.25 \pm 0.01) \cdot n + \mathcal{O}(1).
\end{equation*}
\index{tms@$\OM{\fun{tms}}{n}$|)} \index{tri par
  interclassement!$\sim$ descendant!coût moyen|)} \index{tri par
  interclassement!$\sim$ descendant|)}


\section{Tri ascendant}
\label{sec:general_case}
\index{tri par interclassement!$\sim$ ascendant|(}

Au lieu de couper une pile de \(n\)~clés en deux moitiés, nous
pourrions la diviser en deux piles de longueur \(2^{\ceiling{\lg
    n}-1}\) et \(n-2^{\ceiling{\lg n}-1}\), où le premier nombre
représente la plus grande puissance de~\(2\) strictement inférieure
à~\(n\). Par exemple, si \(n=11=2^3+2^1+2^0\), nous partagerions en
\(2^3=8\) et \(2^1+2^0=3\). Bien sûr, si \(n=2^p\), cette stratégie,
nommée \emph{ascendante}, coïncide avec celle du tri par
interclassement descendant, ce qui, en termes de coût, s'exprime
ainsi: \(\OC{\fun{bms}}{2^p} = \OC{\fun{tms}}{2^p} = \C{\Join}{p}\),
où \fun{bms/1}\index{bms@\fun{bms/1}} réalise le \emph{tri par
  interclassement ascendant} (anglais: \emph{bottom-up merge
sort}). La différence entre les tris descendant et ascendant se voit
aisément à la \fig~\ref{fig:top_vs_bot}.
\begin{figure}[!b]
\centering
\subfloat[Ascendant\label{fig:bot_up}]{%
\includegraphics{bot_up}}
\qquad
\subfloat[Descendant\label{fig:top_down}]{%
\includegraphics{top_down}}
\caption{Comparaison des tris de \([6,3,2,4,1,5]\)}
\label{fig:top_vs_bot}
\end{figure}
En général, nous avons les équations suivantes:
\begin{equation}
\OC{\fun{bms}}{0} = \OC{\fun{bms}}{1} = 0,
\quad
\OC{\fun{bms}}{n} = \OC{\fun{bms}}{2^{\ceiling{\lg n}-1}}
+ \OC{\fun{bms}}{n - 2^{\ceiling{\lg n}-1}}
+ \OC{\fun{mrg}}{2^{\ceiling{\lg n}-1},n - 2^{\ceiling{\lg n}-1}}.
\index{bms@$\OC{\fun{bms}}{n}$}
\label{eq:cost_bms}
\end{equation}

La \fig~\vref{fig:bot_up2}
\begin{figure}[!b]
\centering
\subfloat[Arbre pour \({[}7,3,5,1,6,2,4{]}\)\label{fig:bot_up2}]{%
\includegraphics[bb=58 632 192 721]{bot_up2}}
\qquad
\subfloat[Longueurs seules\label{fig:msort_abs}]{%
\includegraphics[bb=71 632 161 721]{msort_abs}}
\caption{Arbre d'interclassement pour sept clés}
\end{figure}
montre l'arbre d'interclassement\index{arbre!$\sim$ d'interclassement}
de sept clés triées de cette manière. Remarquons que le singleton
\([4]\) est interclassé avec \([2,6]\), une pile deux fois plus
longue. Le déséquilibre des longueurs est propagé vers le haut. Le cas
général est mieux suggéré en ne conservant de chaque nœud que la
longueur de la pile associée, comme on peut en voir un exemple à la
\fig~\vref{fig:msort_abs}.

%\addcontentsline{toc}{subsection}{Coût}
\mypar{Coût minimal}
\index{tri par interclassement!$\sim$ ascendant!coût minimal|(}

Soit \(\OB{\fun{bms}}{n}\)\index{bms@$\OB{\fun{bms}}{n}$|(} le coût
minimal pour trier \(n\)~clés de manière ascendante. Soit \(n=2^p+i\),
avec \(0 < i < 2^p\). Alors, de l'équation~\eqref{eq:cost_bms}
\vpageref{eq:cost_bms}, et~\eqref{eq:best_merge}
\vpageref{eq:best_merge}, nous déduisons:
\begin{equation*}
\OB{\fun{bms}}{2^p+i} = \OB{\fun{bms}}{2^p} + \OB{\fun{bms}}{i} + i,
\end{equation*}
que nous identifions comme une instance des équations fonctionnelles
suivantes: \(f(0)=f(1)=0\), \(f(2)=1\) et \(f(2^p+i) = f(2^p) + f(i) +
i\), avec \(f=\OB{\fun{tms}}{}\), comme on le voit à
l'équation~\eqref{eq:OBtms_2m_i} \vpageref{eq:OBtms_2m_i}. Par
conséquent,
\begin{equation}
\OB{\fun{bms}}{n} = \OB{\fun{tms}}{n} = \sum_{k=0}^{n-1}{\nu_k}.
\label{eq:OBbms}
\end{equation}
Nous pouvons donc réutiliser l'encadrement de \(\OB{\fun{tms}}{n}\):
\begin{equation}
\frac{1}{2}n\lg n - \left(\frac{1}{2}\lg\frac{4}{3}\right)n + \lg\frac{4}{3}
\leqslant \OB{\fun{bms}}{n} \leqslant
\frac{1}{2}n\lg n.
\index{bms@$\OB{\fun{bms}}{n}$}
\end{equation}
Le minorant est atteint si \(n=2\) et est approché au plus près
lorsque \(n\)~est un nombre de Jacobsthal\index{Jacobsthal!nombre de
  $\sim$} (voir équations~\eqref{eq:Jacobsthal}
\vpageref{eq:Jacobsthal}). Le majorant est atteint quand \(n=2^p\).
\index{bms@$\OB{\fun{bms}}{n}$|)}
\index{tri par interclassement!$\sim$ ascendant!coût minimal|)}


\mypar{Coût maximal}
\index{tri par interclassement!$\sim$ ascendant!coût maximal|(}

Soit \(\OW{\fun{bms}}{n}\)\index{bms@$\OW{\fun{bms}}{n}$|(} le coût
maximal pour trier \(n\)~clés de manière ascendante. Soit \(n=2^p+i\),
avec \(0 < i < 2^p\). Alors, de l'équation~\eqref{eq:cost_bms}
\vpageref{eq:cost_bms}, et~\eqref{eq:worst_merge}
\vpageref{eq:worst_merge}, nous déduisons
\begin{equation}
\OW{\fun{bms}}{2^p+i} = \OW{\fun{bms}}{2^p} + \OW{\fun{bms}}{i} + 2^p
+ i - 1.
\label{eq:Wbms_2p_i}
\end{equation}
Cherchons un minorant de \(\OW{\fun{bms}}{n}\) en fondant une
induction sur cette équation. Précisément, trouvons les plus grandes
constantes réelles \(a\)~et~\(b\) telles que, pour \(n \geqslant 2\),
\begin{equation*}
n\lg n + an + b \leqslant \OW{\fun{bms}}{n}.
\end{equation*}
Le cas de base est \(n=2\), soit, \(b \leqslant -2a - 1\). Supposons
que le minorant tienne pour \(n=i\) et souvenons-nous de
l'équation~\eqref{eq:worst_power} \vpageref{eq:worst_power}, qui prend
ici la forme \(\OW{\fun{bms}}{2^p} = p2^p - 2^p +
1\). Alors~\eqref{eq:Wbms_2p_i} entraîne:
\begin{equation*}
(p2^p - 2^p + 1) + (i\lg i + ai + b) + 2^p + i - 1 \leqslant
\OW{\fun{bms}}{2^p+i},
\end{equation*}
qui est équivalente à \(p2^p + i\lg i + i + ai + b \leqslant
\OW{\fun{bms}}{2^p+i}\). Nous voulons que le minorant tienne si
\(n=2^p+i\), soit \((2^p+i)\lg(2^p+i) + a(2^p+i) + b \leqslant
\OW{\fun{bms}}{2^p+i}\). Clairement, cela est vrai si la contrainte
plus forte qui suit est vraie:
\begin{equation*}
(2^p+i)\lg(2^p+i) + a(2^p+i) + b \leqslant p2^p + i\lg i + i + ai + b.
\end{equation*}
Cette inégalité est équivalente à \(a2^p \leqslant p2^p -
(2^p+i)\lg(2^p+i) + i\lg i + i\). Plongeons~\(i\) dans les nombres
réels en posant \(i=x2^p\), où \(x\)~est un nombre réel tel que \(0 <
x \leqslant 1\). Alors, l'inégalité en question équivaut à
\begin{equation*}
a \leqslant \Phi(x),\; \text{où \(\Phi(x) := x\lg x -
  (1+x)\lg(1+x) + x\).}
\end{equation*}
La fonction~\(\Phi\) peut être étendue par continuité en~\(0\), car
\(\lim_{x \to 0} x\lg x = 0\), et elle est dérivable sur l'intervalle
fermé \([0,1]\):
\begin{equation*}
\frac{d\Phi}{dx} = \lg\frac{2x}{x+1}.
\end{equation*}
La racine de \(d\Phi/dx = 0\) est~\(1\), la dérivée est négative
avant, puis positive; donc \(\Phi\)~décroît jusqu'à \(x=1\):
\(a_{\max} := \min_{0 < x \leqslant 1}\Phi(x) = \Phi(1) = -1\). Du cas
de base, nous tirons: \(b_{\max} := -2a_{\max} - 1 = 1\). Par
conséquent,
\begin{equation*}
n\lg n - n + 1 \leqslant \OW{\fun{bms}}{n}.
\end{equation*}
Le minorant est atteint quand \(x=1\), c'est-à-dire \(i=2^p\), soit
\(n=2^{p+1}\).

Cherchons maintenant les plus petites constantes réelles
\(a'\)~et~\(b'\) telles que, pour \(n \geqslant 2\),
\begin{equation*}
\OW{\fun{bms}}{n} \leqslant n\lg n + a'n + b'.
\end{equation*}
La différence avec le minorant est que les inégalités sont inversées
et nous minimisons les inconnues, au lieu de les maximiser. Par
conséquent, le cas de base est ici \(b' \geqslant -2a - 1\) et la
condition pour l'induction est \(a' \geqslant \Phi(x)\). D'après les
variations de~\(\Phi\): \(a'_{\min} := \max_{0 \leqslant x \leqslant
  1}\Phi(x) = \Phi(0) = 0\), et \(b'_{\min} := -2a'_{\min} - 1 =
-1\). En conclusion,
\begin{equation}
n\lg n - n + 1 \leqslant \OW{\fun{bms}}{n} < n\lg n - 1.
\label{ineq:OWbms}
\end{equation}
Puisque \(\Phi(x)\) a été étendue en~\(x=0\), le majorant est approché
au plus près lorsque \(i=1\), la plus petite valeur possible, soit
quand \(n=2^p+1\) (l'interclassement le plus déséquilibré: des piles
de taille~\(2^p\) et~\(1\)). Une étude plus approfondie par
\cite{PannyProdinger_1995}, fondée sur une analyse de Fourier,
confirme que les termes linéaires de cet encadrement ne peuvent être
améliorés et elle montre aussi que la valeur moyenne du coefficient
linéaire est environ \(-0.70057\).


\paragraph{Une autre approche}

Bien que nous ayons déjà encadré \(\OW{\fun{bms}}{n}\) au plus près,
nous pouvons apprendre quelque chose de plus en exprimant le coût
d'une façon différente de celle de sa définition, d'une manière plus
adaptée à des calculs élémentaires aussi. En toute généralité, soient
\(r \geqslant 0\) et \(e_r > \dots > e_1 > e_0 \geqslant 0\) tels que
\(n := 2^{e_r} + \dots + 2^{e_1} + 2^{e_0} > 0\). Nous avons employé
cette décomposition à l'équation~\eqref{eq:e_r} à la
page~\pageref{eq:e_r}.

\begin{wrapfigure}[9]{r}[0pt]{0pt}
\centering
\includegraphics[bb=63 631 202 721]{msort_gen}% [... 721]
\caption{\(\sum_{j=0}^{r}2^{e_j}\) clés}
\label{fig:msort_gen}
\end{wrapfigure}
Considérons alors à la \fig~\ref{fig:msort_gen} l'arbre de tous les
interclassements\index{arbre!$\sim$ d'interclassement|(} possibles
dont on ne retient que les longueurs. Les triangles sont des
sous-arbres constitués d'\emph{interclassements équilibrés},
c'est-à-dire d'interclassements entre piles de même longueurs, pour
lesquels nous connaissons déjà le nombre de comparaisons. Les
longueurs des \emph{interclassements déséquilibrés} se trouvent dans
la branche descendant à droite, c'est-à-dire les nœuds à partir de
la racine, \(2^{e_r}+ \dots + 2^{e_0}\), jusqu'à \(2^{e_1} +
2^{e_0}\).\index{arbre!$\sim$ d'interclassement|)} À la
\fig~\vref{fig:Wn_even}
\begin{figure}[b]
\centering
\subfloat[La somme des nœuds est \(\protect\OW{\protect\fun{bms}}{n}\)
\label{fig:w2p}]%
{\includegraphics[bb=55 602 199 721]{w2p}} % 62 ... ... ...
\qquad
\subfloat[La somme des nœuds est \(\protect\OW{\protect\fun{bms}}{n+1}\)
\label{fig:w2p_succ}]%
{\includegraphics[bb=71 604 214 721]{w2p_succ}}
\caption{Arbres de coût maximal pour \(n\)~pair et \(n+1\)}
\label{fig:Wn_even}
\end{figure}
sont montrés les arbres de coût maximal pour \(n\)~pair et
\(n+1\). Les expressions encadrées ne se trouvent pas dans l'arbre en
opposition, donc, la somme dans chaque arbre des termes non-encadrés
est identique.
\begin{itemize}

  \item \emph{Si \(n\)~est pair}, à la \fig~\vref{fig:w2p}, cette somme
  est \(\OW{\fun{bms}}{n} - r\). Elle
  égale \(\OW{\fun{bms}}{n+1} - 2^{e_0} - \OW{\fun{bms}}{2^0}\) à la
  \fig~\vref{fig:w2p_succ}. En égalant les deux:
  \begin{equation}
    \OW{\fun{bms}}{n} - r = \OW{\fun{bms}}{n+1} - 2^{e_0} -
    \OW{\fun{bms}}{2^0}.
      \label{eq:OW1}
  \end{equation}
  Explicitons que \(e_0\)~est une fonction de~\(n\) (il s'agit de la
  plus grande puissance de~\(2\) qui divise~\(n\)): \(e_0 :=
  \rho_n\). De plus, nous savons déjà que \(\nu_n = r+1\) et
  \(\OW{\fun{bms}}{1} = 0\). Prendre \(n=2k\) dans
  l'équation~\eqref{eq:OW1} implique alors la récurrence
    \begin{equation}
      \OW{\fun{bms}}{2k+1} = \OW{\fun{bms}}{2k} + 2^{\rho_{2k}} +
      \nu_{2k} - 1.
      \label{eq:OWbms_2k_1_tmp}
    \end{equation}
    La fonction \(\rho_n\) (anglais: \emph{ruler function}) est
    étudiée par \cite{GrahamKnuthPatashnik_1994,Knuth_2011}
    \index{rho@$\rho_n$} et satisfait les récurrences suivantes
    \begin{equation}
      \rho_{1} = 0,\qquad \rho_{2n} = \rho_{n} + 1,\qquad
      \rho_{2n+1} = 0,\label{eq:ruler}
    \end{equation}
    pour \(n>0\). On les devine facilement à partir de la notation
    binaire de~\(n\) car \(\rho_n\)~compte le nombre de zéros à
    droite. Ceci nous permet de simplifier un petit peu
    l'équation~\eqref{eq:OWbms_2k_1_tmp} en
    \begin{equation}
     \OW{\fun{bms}}{2k+1} = \OW{\fun{bms}}{2k} + 2 \cdot 2^{\rho_{k}}
     + \nu_{k} - 1.
     \label{eq:OWbms_2k_1}
    \end{equation}

  \item \emph{Si \(n\)~est impair}, nous construisons les arbres à la
    \fig~\vref{fig:Wn_odd}.
\begin{figure}[t]
\centering
\subfloat[La somme des nœuds est \(\protect\OW{\protect\fun{bms}}{n}\)
\label{fig:w2p1}]{%
\includegraphics{w2p1}}
\qquad
\subfloat[La somme des nœuds est \(\protect\OW{\protect\fun{bms}}{n+1}\)
\label{fig:w2p1_succ}]{%
\includegraphics[bb=48 602 194 721]{w2p1_succ}} % 56 ... ... ...
\caption{Arbres de coût maximal pour \(n\)~impair et \(n+1\)}
\label{fig:Wn_odd}
\end{figure}
    La somme des expressions non-encadrées à la \fig~\ref{fig:w2p1} est
    \begin{multline*}
      \OW{\fun{bms}}{n} - \sum_{k=0}^{q-1}\OW{\fun{bms}}{2^k} -
      \sum_{k=2}^{q}2^k + 2((q-1)+(r-q+1))\\
      \begin{aligned}
        & = \OW{\fun{bms}}{n} - \sum_{k=0}^{q-1}((k-1)2^{k}+1)
        - \sum_{k=2}^{q}2^k + 2r\\
        & = \OW{\fun{bms}}{n} - (q-1)2^q - q + 2r + 1,
      \end{aligned}
    \end{multline*}
    en utilisant l'équation~\eqref{eq:worst_power} et
    \(\sum_{k=0}^{q-1}k2^k = (q-2)2^{q}+2\). En effet, soit \(S_{q} :=
    \sum_{k=0}^{q-1}{k2^{k-1}}\). Alors
    \begin{align*}
      S_{q} + q2^{q-1} & = \smash[t]{\sum_{k=1}^{q}{k2^{k-1}}}
      = \smash[t]{\sum_{k=0}^{q-1}{(k+1)2^{k}}} \\
      & = \smash[t]{\sum_{k=0}^{q-1}{k2^{k}}} + \smash[t]{\sum_{k=0}^{q-1}{2^{k}}}
      = 2 \cdot S_{q-1} + 2^{q} - 1,
    \end{align*}
    d'où
    \begin{equation}
      \abovedisplayskip=4pt
      \belowdisplayskip=4pt
      S_{q} = \smash[t]{\textstyle\sum_{k=1}^{q-1}{k2^{k-1}}} =
      (q-2)2^{q-1} + 1.
      \label{eq:Sj}
    \end{equation}
    La même somme à la \fig~\ref{fig:w2p1_succ}
    vaut
    \begin{equation*}
      \OW{\fun{bms}}{n+1} - \OW{\fun{bms}}{2^q} + (r-q+1) =
      \OW{\fun{bms}}{n+1} - (q-1)2^{q} - q + r.
    \end{equation*}
    Nous égalons ces deux sommes et les simplifions en
    \begin{equation*}
      \OW{\fun{bms}}{n+1} = \OW{\fun{bms}}{n} + r + 1=
      \OW{\fun{bms}}{n} + \nu_n.
    \end{equation*}
    En nous souvenant des récurrences~\eqref{def:nu} \vpageref{def:nu} et
    en posant \(n=2k-1\), cette équation est simplifiée en
    \begin{equation}
      \OW{\fun{bms}}{2k} = \OW{\fun{bms}}{2k-1} + \nu_{k-1} + 1.
      \label{eq:OWbms_2k}
    \end{equation}

\end{itemize}
D'après les équations~\eqref{eq:OWbms_2k} et~\eqref{eq:OWbms_2k_1},
nous déduisons
\begin{equation*}
\OW{\fun{bms}}{2k+1} = \OW{\fun{bms}}{2k-1} + 2 \cdot 2^{\rho_k} +
\nu_{k-1} + \nu_k,\quad \OW{\fun{bms}}{2k+2} = \OW{\fun{bms}}{2k} + 2
\cdot 2^{\rho_k} + 2\nu_k.
\end{equation*}
Ces équations sont un bon moyen pour calculer les valeurs de
\(\OW{\fun{bms}}{n}\). En sommant les côtés pour des valeurs
croissantes de la variable~\(k\) donne
\begin{align}
\OW{\fun{bms}}{2p+1}
 &= \OW{\fun{bms}}{1} + 2\!\sum_{k=1}^{p} 2^{\rho_k} +
    \!\sum_{k=1}^{p}\nu_{k-1} + \!\sum_{k=1}^p\nu_k
  = 2 \!\sum_{k=1}^{p} 2^{\rho_k}\! + 2\!\sum_{k=1}^{p-1}\nu_k +
    \!\nu_p.\label{eq:OWbms_2p_1}\\
\OW{\fun{bms}}{2p}
  &= \OW{\fun{bms}}{2} + 2\sum_{k=1}^{p-1}2^{\rho_k} +
2\sum_{k=1}^{p-1}\nu_k
  = 1 +  2\sum_{k=1}^{p-1}2^{\rho_k} + 2\sum_{k=1}^{p-1}\nu_k.
\label{eq:OWbms_2p}
\end{align}
Ces expressions contiennent deux fonctions intéressantes, qui jouent
un rôle en théorie élémentaire des nombres:
\(\sum_{k=1}^{p-1}2^{\rho_k}\) et \(\sum_{k=1}^{p-1}\nu_k\), cette
dernière étant \(\OB{\fun{bms}}{p}\), comme nous l'avons vu à
l'équation~\eqref{eq:OBbms}.  \index{bms@$\OW{\fun{bms}}{n}$|)}
\index{tri par interclassement!$\sim$ ascendant!coût maximal|)}


\mypar{Coût moyen}
\index{tri par interclassement!$\sim$ ascendant!coût moyen|(}

Soit \(\OM{\fun{bms}}{n}\) le nombre moyen de comparaisons pour trier
\(n\)~clés de manière ascendante. Toutes les permutations de la pile
étant également probables, l'équation~\eqref{eq:cost_bms}
\vpageref{eq:cost_bms} donne \(\OM{\fun{bms}}{0} = \OM{\fun{bms}}{1}
= 0\),\index{bms@$\OM{\fun{bms}}{n}$}
\begin{equation*}
\OM{\fun{bms}}{n} = \OM{\fun{bms}}{2^{\ceiling{\lg n}-1}}
+ \OM{\fun{bms}}{n - 2^{\ceiling{\lg n}-1}}
+ \OM{\fun{mrg}}{2^{\ceiling{\lg n}-1},n - 2^{\ceiling{\lg n}-1}},
\end{equation*}
qui, avec l'équation~\eqref{eq:Amrg}, à son tour implique
\(\OM{\fun{bms}}{0} = \OM{\fun{bms}}{1} = 0\) et
\begin{equation*}
\OM{\fun{bms}}{n} = \OM{\fun{bms}}{2^{\ceiling{\lg n}-1}}
+ \OM{\fun{bms}}{n - 2^{\ceiling{\lg n}-1}}
+ n - \frac{2^{\ceiling{\lg n}-1}}{n - 2^{\ceiling{\lg n}-1} + 1}
- \frac{n - 2^{\ceiling{\lg n}-1}}{2^{\ceiling{\lg n}-1}+1}.
\end{equation*}
Cette définition est plutôt intimidante, donc tournons-nous vers
l'induction pour trouver un encadrement, comme nous l'avons fait pour
\(\OB{\fun{tms}}{n}\) dans l'inégalité~\eqref{ineq:McIlroy}
\vpageref{ineq:McIlroy}. Commençons avec le minorant et cherchons à
maximiser les constantes réelles \(a\)~et~\(b\) telles que, si \(n \geqslant 2\),
\begin{equation*}
\pred{H}{n} \colon n\lg n + an + b \leqslant \OM{\fun{bms}}{n}.
\end{equation*}
Le cas de base de l'induction est \(\pred{H}{2}\):
\begin{equation}
2a + b + 1 \leqslant 0.
\label{ineq:base_lower_Btms}
\end{equation}
Supposons alors \(\pred{H}{n}\) pour tout \(2 \leqslant n \leqslant
2^p\), et montrons \(\pred{H}{2^p+i}\), pour tout \(0 < i \leqslant
2^p\). Le principe d'induction ensuite entraîne que \(\pred{H}{n}\)
est vrai pour tout \(n \geqslant 2\). Si \(n=2^p+i\), alors
\(\ceiling{\lg n} - 1 = p\), donc
\begin{equation}
\OM{\fun{bms}}{2^p+i} = \OM{\fun{bms}}{2^p} + \OM{\fun{bms}}{i}
+ 2^p + i - \frac{2^p}{i+1} - \frac{i}{2^p+1}.
\label{eq:Abms_2p_i}
\end{equation}
Par hypothèse, \(\pred{H}{i}\) tient, soit \(i\lg i + ai + b \leqslant
\OM{\fun{bms}}{i}\), mais, au lieu d'user de \(\pred{H}{2^p}\), nous
emploierons la valeur exacte à l'équation~\eqref{eq:Mjoin}
\vpageref{eq:Mjoin}, où \(\alpha := \sum_{k \geqslant
  0}1/(2^k+1)\). D'après l'équation~\eqref{eq:Abms_2p_i}, nous
dérivons
\begin{equation*}
(p-\alpha)2^p + \sum_{k \geqslant
    0}\frac{1}{2^{k}+2^{-p}}
+ (i\lg i + ai + b) + 2^p + i -
\frac{2^p}{i+1} - \frac{i}{2^p+1} < \OM{\fun{bms}}{2^p+i}.
\end{equation*}
Nous voulons prouver \(\pred{H}{2^p+i} \colon (2^p+i)\lg(2^p+i) +
a(2^p+i) + b \leqslant \OM{\fun{bms}}{2^p+i}\), qui est donc impliquée
par
\begin{equation*}
(2^p+i)\lg(2^p+i) + a2^p \leqslant (p - \alpha + 1)2^p -
\frac{2^p}{i+1} + i\lg i + i - \frac{i}{2^p+1} + c_p,
\end{equation*}
où \(c_p := \sum_{k \geqslant 0}1/(2^{k}+2^{-p})\). Posons
\begin{equation*}
  \Psi(p,i) := p - \alpha + 1 - \frac{1}{i+1} + \frac{i}{2^p+1} -
  \frac{1}{2^p}((2^p+i)\lg(2^p+i) - i\lg i - c_p).
\end{equation*}
Alors la condition suffisante ci-dessus équivaut à \(a \leqslant
\Psi(p,i)\). Pour l'étude de \(\Psi(p,i)\), fixons~\(p\) et
laissons~\(i\) varier sur l'intervalle réel \(]0,2^p]\). La dérivée
partielle de~\(\Psi\) par rapport à~\(i\) est
\begin{equation*}
\frac{\partial\Psi}{\partial i}(p,i) = \frac{1}{2^p+1}
+ \frac{1}{(i+1)^2} - \frac{1}{2^p}\lg\left(\frac{2^p}{i}+1\right).
\end{equation*}
Déterminons aussi la dérivée deuxième par rapport à~\(i\):
\begin{equation*}
\frac{\partial^2\Psi}{\partial i^2}(p,i) = \frac{1}{(2^p+i)i\ln 2} - \frac{2}{(i+1)^3},
\end{equation*}
où \(\ln x\) est le logarithme naturel de~\(x\). Soit le polynôme
cubique suivant:
\begin{equation*}
K_p(i) := i^3 + (3 - 2\ln 2)i^2 + (3 - 2^{p+1}\ln 2)i + 1.
\end{equation*}
Alors \(\partial^2\Psi/\partial i^2 = 0 \Leftrightarrow K_p(i) = 0\)
et le signe de \(\partial^2\Psi/\partial i^2\) est le signe de
\(K_p(i)\). En général, une équation cubique prend la forme
\begin{equation*}
ax^3 + bx^2 + cx + d = 0, \; \text{avec \(a \neq 0\)}.
\end{equation*}
Un résultat classique à propos de la nature des racines est le
suivant. Soit \(\Delta := 18abcd - 4b^3d + b^2c^2 - 4ac^3 - 27a^2d^2\)
le \emph{discriminant}\index{discriminant|(} de la cubique.
\begin{itemize}

  \item Si \(\Delta > 0\), l'équation a trois racine réelles
    distinctes;

  \item si \(\Delta = 0\), l'équation possède une racine multiple et
    toutes ses racines sont réelles;

  \item si \(\Delta < 0\), l'équation possède une racine réelle et
    deux racines complexes, non-réelles et conjuguées.

\end{itemize}
Reprenons le fil de la discussion. Soit le polynôme cubique suivant:
\begin{equation*}
\Delta(x) \!:= (4\ln 2)x^3 - (9 - 2\ln 2)(3 + 2\ln 2)x^2 + 12(9 - 2\ln
29)x - 4(27 - 8\ln 2).
\end{equation*}
Alors le discriminant de \(K_p(i) = 0\) est \(\Delta(2^{p+1}) \cdot
\ln^2 2\). Le discriminant de \(\Delta(x) = 0 \) est négatif, donc
\(\Delta(x)\) a une racine réelle~\(x_0 \simeq 8.64872\). Puisque le
coefficient de~\(x^3\) est positif, \(\Delta(x)\) est négatif si \(x <
x_0\) et positif si \(x > x_0\).
\begin{enumerate}

  \item Étant donné que \(p \geqslant 3\) implique \(2^{p+1} > x_0\),
    le discriminant\index{discriminant|)} de \(K_p(i) = 0\) est
    positif, ce qui veut dire que \(K_p(i)\) possède trois racines
    réelles distinctes si \(p \geqslant 3\), de même que
    \(\partial^2\Psi/\partial i^2\).

  \item Sinon, \(K_p(i)\) a une racine
    réelle si \(0 \leqslant p \leqslant 2\).

\end{enumerate}
Avant d'étudier ces deux cas en détail, rafraîchissons-nous la mémoire
à propos des polynômes cubiques. Soient \(\rho_0\), \(\rho_1\)
et~\(\rho_2\) les racines du polynôme \(P(x) = ax^3 + bx^2 + cx +
d\). Alors, sa forme factorisée, puis développée donne \(P(x) =
a(x-\rho_0)(x-\rho_1)(x-\rho_2) = ax^3 - a(\rho_0+\rho_1+\rho_2)x^2 +
a(\rho_0\rho_1 + \rho_0\rho_2 + \rho_1 \rho_2)x -
a(\rho_0\rho_1\rho_2)\), donc \(\rho_0\rho_1\rho_2 = -d/a\).
\begin{itemize}

\item Soit \(p \in \{0,1,2\}\). Nous venons de trouver que \(K_p(i)\)
  possède une racine réelle, disons~\(\rho_0\), et deux non-réelles et
  conjuguées, disons~\(\rho_1\) et \(\rho_2=\overline{\rho_1}\). Alors
  \(\rho_0\rho_1\rho_2 = \rho_0 \len{\rho_1}^2 = -1\), d'où \(\rho_0 <
  0\). Puisque le coefficient de~\(x^3\) est positif, ceci entraîne
  \(K_p(i) > 0\) si \(i > 0\), ce qui est vrai de
  \(\partial^2\Psi/\partial i^2\) aussi: \(i > 0 \) implique
  \(\partial^2\Psi/\partial i^2 > 0\), par conséquent
  \(\partial\Psi/\partial i\) croît. Puisque
  \begin{equation*}
    \frac{\partial\Psi}{\partial i}(p,i) \xrightarrow[i\to 0^{+}]{}
    -\infty < 0, \;\text{et}\; \left.\frac{\partial\Psi}{\partial
        i}(p,i)\right|_{i=2^p} = -\frac{1}{2^p(2^p+1)^2} < 0,
  \end{equation*}
  nous déduisons que \(\partial\Psi/\partial i < 0\) si \(i > 0\), ce
  qui signifie que \(\Psi(p,i)\) décroît lorsque \(i \in\;
  ]0,2^p]\). Puisque nous voulons minimiser \(\Psi(p,i)\), nous avons
  \(\min_{0 < i \leqslant 2^p}\Psi(p,i) = \Psi(p,2^p)\).

\item Si \(p \geqslant 3\), alors \(K_p(i)\) possède trois racines
  réelles. Ici, le produit des racines de \(K_p(i)\) est~\(-1\), donc
  au plus deux d'entre elles sont positives. Puisque nous avons
  \(K_p(0) = 1 > 0\), \(K_p(1) < 0\) et \(\lim_{i\to+\infty}K_p(i) >
  0\), nous voyons que \(K_p(i)\) a une racine dans \(]0,1[\) et une
  autre dans \(]1,+\infty[\), de même que \(\partial^2\Psi/\partial
  i^2\). De plus, \(\left.\partial\Psi/\partial i\right|_{i=1} > 0\)
  et \(\left.\partial\Psi/\partial i\right|_{i=2^p} < 0\), donc le
  théorème de la valeur intermédiaire entraîne l'existence d'un
  réel~\(i_p \in\; ]1,2^p[\) tel que \(\left.\partial\Psi/\partial
    i\right|_{i=i_p} = 0\), et nous le savons unique parce que
  \(\partial\Psi^2/\partial i^2\) ne change de signe qu'une fois dans
  l'intervalle \(]1,+\infty[\). Ceci signifie aussi que \(\Psi(p,i)\)
  croît si \(i\)~croît sur \([1,i_p[\), atteint son maximum quand
  \(i=i_p\), et décroît sur \(]i_p,2^p]\). Or \(\lim_{i \to
    0^{+}}\Psi(p,i) = -\infty\) et nous sommes à la recherche d'un
  minorant de \(\Psi(p,i)\), donc nous devons déterminer si \(i=1\) ou
  \(i=2^p\) minimise \(\Psi(p,i)\): en fait, nous avons \(\Psi(p,1)
  \geqslant \Psi(p,2^p)\), donc nous en concluons que \(\min_{0 < i
    \leqslant 2^p}\Psi(p,i) = \Psi(p,2^p)\).
\end{itemize}
Dans tous les cas, nous devons minimiser \(\Psi(p,2^p)\). Nous avons:
\begin{equation*}
  \Psi(p,2^p) = - \frac{1}{2^p+1} - \sum_{k=0}^{p}\frac{1}{2^k+1}.
\end{equation*}
Nous vérifions que \(\Psi(p,2^p) > \Psi(p+1,2^{p+1})\), donc la
fonction décroît aux points entiers et \(a_{\max} = \min_{p >
  0}\Psi(p,2^p) = \lim_{p \to \infty}\Psi(p,2^p) = -\alpha^{+}\). De
l'inéquation~\eqref{ineq:base_lower_Btms}, nous tirons \(b_{\max} =
-2a_{\max} - 1 = 2\alpha - 1 \simeq 1.52899\). Au total, par le
principe d'induction, nous avons établi, pour \(n \geqslant 2\),
\begin{equation*}
n\lg n - \alpha n + 2\alpha -1 < \OM{\fun{bms}}{n}.
\end{equation*}
Ce minorant est meilleur que celui pour le coût moyen du tri par
interclassement descendant, à l'inégalité~\eqref{ineq:lower_Atms}
\vpageref{ineq:lower_Atms}, parce que là-bas, nous avions
décomposé~\(n\) en valeurs paires et impaires, pas \(n=2^p+i\) qui
nous permet ici d'utiliser la valeur exacte de
\(\OM{\fun{bms}}{2^p}\). Il est même un peu meilleur
que~\eqref{ineq:M_join} \vpageref{ineq:M_join}, ce qui est une bonne
surprise.

Nous devons tâcher à présent d'obtenir un majorant à l'aide de la même
technique. En d'autres termes, nous souhaitons ici minimiser les
constantes réelles \(a'\)~et~\(b'\) telles que \(\OM{\fun{bms}}{n}
\leqslant n\lg n + a'n + b'\), pour \(n \geqslant 2\). La différence
avec la quête du minorant est que les inégalités sont inversées:
\(a'\geqslant \Psi(p,i)\) et \(b' \geqslant -2a' - 1\). Nous
revisitons les deux cas ci-dessus:
\begin{itemize}

\item Si \(0 \leqslant p \leqslant 2\), alors \(\max_{0 < i \leqslant
    2^p}\Psi(p,i) = \Psi(p,1)\). Nous vérifions aisément que \(\max_{0
    \leqslant p \leqslant 2}\Psi(p,1) = \Psi(0,1) = 1 - \alpha\).

\item Si \(p \geqslant 3\), nous devons exprimer \(i_p\)~en fonction
  de~\(p\), mais il est difficile de résoudre l'équation
  \(\left.\partial\Psi/\partial i\right|_{i=i_p} = 0\), même de
  manière approchée.

\end{itemize}
Avant d'abandonner la partie, nous pourrions tenter de
dériver~\(\Psi\) par rapport à~\(p\), au lieu de~\(i\). En effet,
\((p,i,\Psi(p,i))\) définit une surface dans l'espace, et en
privilégiant~\(p\) par rapport à~\(i\), nous découpons la surface par
des plans perpendiculaires à l'axe des~\(i\). Parfois, découper dans
une direction plutôt qu'une autre facilite l'analyse. Le problème ici
est de dériver~\(c_p\). Nous pouvons contourner la difficulté avec le
majorant \(c_p < 2\) des inégalités~\eqref{ineq:M_join}
\vpageref{ineq:M_join} et en définissant
\begin{equation*}
  \Phi(p,i) := p - \alpha + 1 - \frac{1}{i+1} + \frac{i}{2^p+1} -
  \frac{1}{2^p}((2^p+i)\lg(2^p+i) - i\lg i - 2).
\end{equation*}
Nous avons ainsi \(\Psi(p,i) < \Phi(p,i)\) et, au lieu de \(\Psi(p,i)
\leqslant a'\), nous imposons la contrainte plus forte \(\Phi(p,i)
\leqslant a'\) et croisons nos doigts. Dans la \fig~\vref{fig:phi},
\begin{figure}
\centering
\includegraphics[bb=71 565 400 725]{phi}
\caption{\(\Phi(p,1)\), \(\Phi(p,2)\) et \(\Phi(p,3)\)}
\label{fig:phi}
\end{figure}
sont esquissées \(\Phi(p,1)\), \(\Phi(p,2)\) et \(\Phi(p,3)\). (Le
point de départ pour chaque courbe est marqué d'un disque blanc.)
La dérivation par rapport à~\(p\) produit
\begin{equation*}
\frac{\partial\Phi}{\partial p}(p,i) =
\frac{i}{2^p}\ln\left(\frac{2^p}{i}+1\right)
- \frac{\ln 2}{2^{p-1}} - \frac{i2^p\ln 2}{(2^p+1)^2}.
\end{equation*}
Pour étudier le signe de \(\partial\Phi(p,i)/\partial p\) quand~\(p\)
varie, définissons
\begin{equation*}
  \varphi(x,i) :=
    \frac{x}{i\ln 2} \cdot
    \left.\frac{\partial\Phi}{\partial p}(p,i)\right|_{p=\lg x}.
\end{equation*}
Puisque \(x \geqslant 1\) implique \(x/i\ln 2 > 0\) et \(\lg x
\geqslant 0\), le signe de \(\varphi(x,i)\) quand~\(x \geqslant 1\)
varie est le même que le signe de \(\partial\Phi(p,i)/\partial p\)
quand~\(p \geqslant 0\) varie, en gardant à l'esprit
que \(x=2^p\). Nous avons
\begin{align*}
\varphi(x,i) &= \lg\left(\frac{x}{i}+1\right) -
\left(\!\frac{x}{x+1}\!\right)^2 - \frac{2}{i},\\
\frac{\partial\varphi}{\partial x}(x,i) &=
\frac{1}{(x+i)\ln 2} - \frac{2x}{(x+1)^3}.
\end{align*}
Ceci devrait nous rappeler quelque chose de familier:
\begin{equation*}
\frac{\partial\varphi}{\partial x}(x,i) =
  x \cdot \left.\frac{\partial^2\Psi}{\partial x^2}(p,x)\right|_{p=\lg
  i}.
\end{equation*}
Quand \(x \geqslant 1\) varie, le signe de
\(\partial\varphi(x,i)/\partial x\) est le même que le signe de
\(\left.\partial^2\Psi(p,x)/\partial x^2\right|_{p=\lg i}\), donc nous
pouvons réactiver la discussion préalable sur les racines de
\(K_p(i)\), tout en prenant soin de remplacer~\(i\) par~\(x\),
et~\(2^p\) par~\(i\):
\begin{itemize}

\item Si \(i \in \{1,2,3,4\}\), alors \(\partial\varphi(x,i)/\partial x
  > 0\) quand \(x > 0\).

\item Si \(i \geqslant 5\), alors \(\partial\varphi(x,i)/\partial x >
  0\) quand \(x \geqslant 1\).

\end{itemize}
Dans les deux cas, \(\varphi(x,i)\) croît quand \(x \geqslant 1\), ce
qui, sachant par ailleurs que \(\lim_{x\to 0^{+}}\varphi(x,i) =
-\infty < 0\) et \(\lim_{x\to\infty}\varphi(x,i) = +\infty > 0\),
entraîne l'existence d'une racine unique~\(\rho > 0\) telle que
\(\varphi(x,i) < 0\) si \(x < \rho\), et \(\varphi(x,i) > 0\) si \(x >
\rho\), et de même pour \(\partial\Phi/\partial p\) (avec une racine
différente). Ainsi, \(\Phi(p,i)\) décroît jusqu'à son minimum, avant
de croître. (Voir à nouveau la \fig~\vref{fig:phi}.)

De plus, nous avons la limite
\begin{equation*}
  \overline\lim_{p \to \infty}\Phi(p,i) = \frac{i}{i+1} - \alpha <
  1 - \alpha = \Phi(0,1),
\end{equation*}
donc les courbes ont des asymptotes. Puisque nous recherchons le
maximum, nous déduisons: \(a'_{\min} = \max_{0 < i \leqslant
  2^p}\Phi(p,i) = 1 - \alpha \simeq -0.2645\), et la constante est
\(b'_{\min} = -2a'_{\min} - 1 = 2\alpha - 3 \simeq -0.471\). En somme,
nous avons trouvé, pour \(n \geqslant 2\),
\begin{equation}
n\lg n - \alpha n + (2\alpha - 1) < \OM{\fun{bms}}{n}
< n\lg n - (\alpha - 1)n - (3 - 2\alpha).
\label{ineq:bounds_Mbms}
\end{equation}
Le minorant est approché au plus près lorsque \(n=2^p\). Pour
interpréter les valeurs de~\(n\) pour lesquelles le majorant est
approché au plus près, nous devons jeter un autre coup d'œil à la
\fig~\vref{fig:phi}. Nous avons \(i/(i+1) - \alpha \to 1 - \alpha\),
si \(p \to \infty\), mais ceci ne nous dit rien
sur~\(p\).

Malheureusement, comme nous l'avions remarqué plus tôt, pour un~\(p\)
donné, nous ne savons caractériser explicitement~\(i_p\), qui est la
valeur de~\(i\) qui maximise \(\Phi(p,i)\) (dans les plans
perpendiculaires à cette page). Malgré tout, les termes linéaires de
cet encadrement ne peuvent être améliorés. Cela veut dire que le
nombre supplémentaire de comparaisons dû au tri de \(n=2^p+i\) clés au
lieu de~\(2^p\) est au plus~\(n\).

Comme c'était déjà le cas avec le tri par interclassement descendant,
des mathématiques approfondies par \cite{PannyProdinger_1995} montrent
que \(\OM{\fun{bms}}{n} = n\lg n + B^*(\lg n) \cdot
n\)\index{bms@$\OM{\fun{bms}}{n}$}, où \(B^*\)~est une fonction
continue, non-dérivable et périodique, dont la valeur moyenne est
environ \(-0.965\). Trivialement, nous avons
\begin{equation*}
\OM{\fun{bms}}{n} \sim n\lg n \sim \OW{\fun{bms}}{n} \sim 2 \cdot
\OB{\fun{bms}}{n}.
\end{equation*}
\index{tri par interclassement!$\sim$ ascendant!coût moyen|)}

\begin{figure}[H]
\begin{equation*}
\boxed{%
\begin{array}{r@{\;}l@{\;}lr@{\;}l@{\;}l}
  \fun{bms}(\el) & \xrightarrow{\smash{\mu}} & \el;
& \fun{solo}(\el) & \xrightarrow{\smash{\xi}} & \el;\\
  \fun{bms}(s) & \xrightarrow{\smash{\nu}}
               & \fun{all}(\fun{solo}(s)).
& \fun{solo}(\cons{x}{s}) & \xrightarrow{\smash{\pi}}
                          & \cons{[x]}{\fun{solo}(s)}.\\
\\
  \fun{all}([s]) & \xrightarrow{\smash{\rho}} & s;
& \fun{nxt}(\cons{s,t}{u}) & \xrightarrow{\smash{\tau}}
                         & \cons{\fun{mrg}(s,t)}{\fun{nxt}(u)};\\
  \fun{all}(s) & \xrightarrow{\smash{\sigma}}
               & \fun{all}(\fun{nxt}(s)).
& \fun{nxt}(u) & \xrightarrow{\smash{\upsilon}} & u.\\
\\
\fun{mrg}(\el,t)         & \xrightarrow{\smash{\theta}} & t;\\
\fun{mrg}(s,\el)         & \xrightarrow{\smash{\iota}} & s;\\
\fun{mrg}(\cons{x}{s},\cons{y}{t}) & \xrightarrow{\smash{\kappa}}
& \multicolumn{4}{@{}l}{\cons{y}{\fun{mrg}(\cons{x}{s},t)},
\;\text{si \(x \succ y\);}}\\
\fun{mrg}(\cons{x}{s},t) & \xrightarrow{\smash{\lambda}}
                         & \cons{x}{\fun{mrg}(s,t)}.
\end{array}}
\end{equation*}
\caption{Tri par interclassement ascendant avec \fun{bms/1}}
\label{fig:bms}
\end{figure}

\mypar{Programme}
\index{tri par interclassement!$\sim$ ascendant!programme}

Nous avons réussi à analyser le nombre de comparaisons pour trier par
interclassement parce que le processus, dans sa globalité, peut être
aisément visualisé par un arbre\index{arbre!$\sim$
  d'interclassement|(}. Il est grand tant d'écrire un programme dont
les traces d'évaluation sont conformes à ces arbres
d'interclassement. Nous montrons à la \fig~\vref{fig:bms} la
définition de la fonction de tri \fun{bms/1}\index{bms@\fun{bms/1}} et
de plusieurs auxiliaires.

\begin{itemize}

  \item L'appel \(\fun{solo}(s)\)\index{solo@\fun{solo/1}} s'évalue en
    une pile contenant des singletons avec toutes les clés de~\(s\)
    dans le même ordre. En d'autres termes, il s'agit des feuilles de
    l'arbre d'interclassement.

  \item L'appel \(\fun{all}(u)\)\index{all@\fun{all/1}} s'évalue en
    une pile contenant les interclassements des piles adjacentes
    dans~\(u\). Dit encore autrement, il s'agit du niveau juste
    au-dessus de~\(u\) dans l'arbre d'interclassement.

  \item L'appel \(\fun{all}(\fun{solo}(s))\), quant à lui, calcule la
    pile ordonnée correspondant à la pile de
    singletons~\(\fun{solo}(s)\), c'est-à-dire, en commençant par les
    feuilles, il construit les niveaux de plus en plus haut en faisant
    appel à \fun{mrg/2}\index{mrg@\fun{mrg/2}}, jusqu'à atteindre la
    racine.

\end{itemize}
La beauté de ce programme est qu'il n'y a pas besoin de deux phases
distinctes, d'abord construire les arbres des interclassements
équilibrés et puis effectuer les interclassement déséquilibrés à
partir des racines de ceux-ci: il est possible de parvenir au même
résultat en procédant vers la droite et vers le haut d'une manière
uniforme.\index{arbre!$\sim$ d'interclassement|)}

\paragraph{Coût supplémentaire}

Pour déterminer le coût
\(\C{\fun{bms}}{n}\)\index{bms@$\C{\fun{bms}}{n}$} nous avons besoin
d'ajouter au nombre de comparaisons le nombre de réécritures qui
n'impliquent pas de comparaisons, c'est-à-dire, autres que par les
règles~\(\kappa\) et~\(\lambda\).
\begin{itemize}

\item Les règles \(\theta\)~et~\(\iota\) sont utilisées une fois pour
  conclure chaque interclassement. Soit
  \(\OC{\ltimes}{n}\)\index{mrg@$\OC{\ltimes}{n}$} le nombre de
  comparaisons effectuées lors des interclassements déséquilibrés
  quand il y a \(n\)~clés à trier. Un coup d'œil à la
  \fig~\vref{fig:msort_gen} révèle
    \begin{equation}
      \OC{\ltimes}{n} :=
      \sum_{i=1}^{r}{\OC{\fun{mrg}}{2^{e_i},2^{e_{i-1}}+\dots+2^{e_0}}}.
      \label{eq:C_unbal}
    \end{equation}
    Le nombre total de comparaisons est la somme des nombres de
    comparaisons des interclassements équilibrés et déséquilibrés:
    \begin{equation}
      \OC{\fun{bms}}{n}
      = \sum_{i=0}^{r}{\C{\Join}{e_i}}
      +
      \OC{\ltimes}{n}
      = \sum_{i=0}^{r}{\OC{\fun{bms}}{2^{e_i}}}
      +
      \sum_{i=1}^{r}{\OC{\fun{mrg}}{2^{e_i},2^{e_{i-1}}+\dots+2^{e_0}}}.
      \label{eq:msort_gen}
    \end{equation}
    Pour trouver le nombre d'interclassements, posons
    \(\OC{\fun{mrg}}{m,n} = 1\) à l'équation~\eqref{eq:cost_power_2}
    \vpageref{eq:cost_power_2}, donnant \(\OC{\fun{bms}}{2^p} = 2^p -
    1\). En substituant ce résultat dans
    l'équation~\eqref{eq:msort_gen}, nous tirons \(\OC{\fun{bms}}{n} =
    n - 1\). En d'autres termes, les règles \(\theta\)~et~\(\iota\)
    sont employées \(n-1\)\label{eq:bms_merges} fois au
    total.\index{bms@$\OC{\fun{bms}}{n}$}

  \item À la règle~\(\tau\), un appel
    \(\fun{nxt}(\cons{s,t}{u})\)\index{nxt@\fun{nxt/1}} correspond à
    un appel \(\fun{mrg}(s,t)\)\index{mrg@\fun{mrg/2}}, un par
    interclassement. Donc \(\tau\)~est utilisée \(n-1\)~fois.

  \item La règle~\(\upsilon\) est employée une fois par niveau dans
    l'arbre d'interclassement\index{arbre!$\sim$ d'interclassement},
    sauf pour la racine, avec \(u\)~étant soit vide, soit un
    singleton. Soit~\(\Lambda(j)\) le nombre de nœuds au
    niveau~\(j\), où \(j=0\) représente le niveau des feuilles. Alors,
    le nombre~\(z\) que nous recherchons est le plus grand entier
    natural satisfaisant l'équation \(\Lambda(z) = 1\), à la racine.
    La fonction~\fun{nxt/1}\index{nxt@\fun{nxt/1}} implique
    \begin{equation*}
      \Lambda(j+1) = \ceiling{\Lambda(j)/2},\; \text{avec \(\Lambda(0)
        = n\)}.
    \end{equation*}
    Cette récurrence équivaut à la forme close \(\Lambda(j) =
    \ceiling{n/2^j}\), en conséquence du théorème suivant.

    \medskip

\begin{thm}[Parties entières par excès et fractions]
\label{thm_ceilings}
\textsl{Soit \(x\)~un nom\-bre réel et \(q\)~un entier naturel. Alors
  \(\ceiling{\ceiling{x}/q} = \ceiling{x/q}\).}
\end{thm}
\begin{proof}
  L'égalité est équivalente à la conjonction des deux inégalités
  complémentaires \(\ceiling{\ceiling{x}/q} \geqslant \ceiling{x/q}\)
  et \(\ceiling{\ceiling{x}/q} \leqslant \ceiling{x/q}\). La première
  est simple car \(\ceiling{x} \geqslant x \Rightarrow \ceiling{x}/q
  \geqslant x/q \Rightarrow \ceiling{\ceiling{x}/q} \geqslant
  \ceiling{x/q}\). Ensuite, puisque les deux membres de l'inégalité
  sont des entiers, \(\ceiling{\ceiling{x}/q} \leqslant
  \ceiling{x/q}\) équivaut à \(p \leqslant \ceiling{\ceiling{x}/q}
  \Rightarrow p \leqslant \ceiling{x/q}\), pour tout entier~\(p\). Un
  lemme évident est que si \(i\)~est un entier et \(y\)~un réel, \(i
  \leqslant \ceiling{y} \Leftrightarrow i \leqslant y\), donc
  l'inéquation initiale est équivalente à \(p \leqslant \ceiling{x}/q
  \Rightarrow p \leqslant x/q\), pour tout entier~\(p\), c'est-à-dire
  \(pq \leqslant \ceiling{x} \Rightarrow pq \leqslant x\). Le lemme
  valide cette implication et achève la preuve.
 \end{proof}
    \noindent Pour trouver~\(z\), nous devons exprimer~\(n\) comme une
    suite de bits \(n :=\! \sum_{k=0}^{m-1}{b_k2^k} = (b_{m-1}\ldots
    b_0)_2\), où \(b_k \in \{0,1\}\) et \(b_{m-1} = 1\). Il est facile
    de déduire une formule pour~\(b_i\). Nous avons
    \begin{equation}
      \frac{n}{2^{i+1}}
      = \frac{1}{2^{i+1}}\sum_{k=0}^{m-1}{b_k2^{k}}
      = \frac{1}{2^{i+1}}\sum_{k=0}^{i}{b_k2^k} + (b_{m-1}\dots b_{i+1})_2.
      \label{eq:n_on_power_2}
    \end{equation}
    Nous prouvons alors \(\floor{n/2^{i+1}} = (b_{m-1}\dots
    b_{i+1})_2\) comme suit:
    \begin{equation*}
      \sum_{k=0}^{i}{2^k} < 2^{i+1}
      \Rightarrow
      0 \leqslant \sum_{k=0}^{i}{b_k2^k} < 2^{i+1}
      \Leftrightarrow
      0 \leqslant \frac{1}{2^{i+1}}\sum_{k=0}^{i}{b_k2^k} < 1.
    \end{equation*}
    Ceci et l'équation~\eqref{eq:n_on_power_2} impliquent que
    \begin{equation*}
      \left\lceil\frac{n}{2^i}\right\rceil =
      (b_{m-1}\dots b_i)_2
      + \begin{cases}
          0, & \text{si \((b_{i-1}\dots b_0)_2=0\)};\\
          1, & \text{sinon}.
        \end{cases}
    \end{equation*}
    Donc, \(\ceiling{n/2^z} = 1\) équivaut à \(z=m-1\) si
    \(n=2^{m-1}\), et \(z=m\) sinon. L'équation~\eqref{eq:e_r}
    \vpageref{eq:e_r} stipule \(m = \floor{\lg n} + 1\), et par
    conséquent \(z=\floor{\lg n}\) si \(n\)~est une puissance
    de~\(2\), et \(z=\floor{\lg n} + 1\) sinon. Plus simplement, ceci
    signifie que \(z=\ceiling{\lg n}\).

  \item La règle~\(\rho\) est utilisée une fois, à la racine. La
    règle~\(\sigma\) est employée \(z\)~fois.

  \item La trace de \(\fun{solo}(s)\)\index{solo@\fun{solo/1}}
    est~\(\pi^n\xi\) si~\(s\) contient \(n\)~clés, donc
    \(\C{\fun{solo}}{n} = n + 1\).

  \item La contribution au coût total des règles
    \(\mu\)~et~\(\nu\) est simplement~\(1\).

\end{itemize}
Au total, \(\C{\fun{bms}}{n} = \OC{\fun{bms}}{n} + 3n + 2\ceiling{\lg
  n} + 1\) et \(\C{\fun{bms}}{n} \sim
\OC{\fun{bms}}{n}\).\index{bms@$\C{\fun{bms}}{n}$}

\paragraph{Amélioration}

Il est aisé d'améliorer \fun{bms/1} en construisant directement le
deuxième niveau de l'arbre d'interclassement\index{arbre!$\sim$
  d'interclassement} \emph{sans utiliser \fun{mrg/2}}.
\index{mrg@\fun{mrg/2}} Considérons le programme à la
\fig~\vref{fig:bms0}, où \fun{solo/1}\index{solo@\fun{solo/1}} a
été remplacée par~\fun{duo/1}\index{duo@\fun{duo/1}}.
\begin{figure}
\begin{equation*}
\boxed{%
\begin{array}{r@{\;}l@{\;}l@{\;}l}
\fun{bms}_0(s)   & \rightarrow & \fun{all}(\fun{duo}(s)).\\
\\
\fun{duo}(\cons{x,y}{s}) & \rightarrow & \cons{[y,x]}{\fun{duo}(s)},
                                       & \text{si \(x \succ y\)};\\
\fun{duo}(\cons{x,y}{s}) & \rightarrow & \cons{[x,y]}{\fun{duo}(s)};\\
\fun{duo}(s)             & \rightarrow & [s].\\
\\
\fun{all}([s]) & \rightarrow & s;\\
\fun{all}(s)   & \rightarrow & \fun{all}(\fun{nxt}(s)).\\
\\
\fun{nxt}(\cons{s,t}{u}) & \rightarrow
                         & \cons{\fun{mrg}(s,t)}{\fun{nxt}(u)};\\
\fun{nxt}(u)             & \rightarrow & u.\\
\\
\fun{mrg}(\el,t)         & \rightarrow & t;\\
\fun{mrg}(s,\el)         & \rightarrow & s;\\
\fun{mrg}(\cons{x}{s},\cons{y}{t}) & \rightarrow
                         & \cons{y}{\fun{mrg}(\cons{x}{s},t)},
                         & \text{si \(x \succ y\)};\\
\fun{mrg}(\cons{x}{s},t) & \rightarrow & \cons{x}{\fun{mrg}(s,t)}.
\end{array}}
\end{equation*}
\caption{Tri ascendant accéléré avec \fun{bms\(_0\)/1}}
\label{fig:bms0}
\end{figure}
Le nombre de comparaisons est inchangé, mais le coût, quantifié par le
nombre de réécritures, est légèrement inférieur. Le coût additionnel
de~\(\fun{duo}(s)\)\index{duo@\fun{duo/1}} est \(\floor{n/2}+1\), où
\(n\)~est la longueur de~\(s\). D'un autre côté, nous économisons le
coût de~\(\fun{solo}(s)\)\index{solo@\fun{solo/1}}. La première
réécriture par la règle~\(\sigma\) n'est pas effectuée, de même que
l'appel subséquent \(\fun{nxt}(s)\)\index{nxt@\fun{nxt/1}},
c'est-à-dire, \(\floor{n/2}\) appels à
\fun{mrg/2}\index{mrg@\fun{mrg/2}} sur des paires de singletons
par~\(\kappa\) ou~\(\lambda\), plus une réécriture par~\(\theta\)
ou~\(\iota\) pour le dernier singleton ou la pile vide, faisant un
total de \(\floor{n/2}\C{\fun{mrg}}{1,1} + 1 = 2\floor{n/2} + 1\). Au
bout du compte, le coût total est diminué de
\begin{equation*}
  ((n+1) + 1 + (2\floor{n/2}+1)) - (\floor{n/2}+1) = n + \floor{n/2} +
  2.
\end{equation*}
Donc, \(\C{\fun{bms}_0}{n} = \OC{\fun{bms}}{n} + \ceiling{3n/2} +
2\ceiling{\lg n} - 1\), pour \(n>0\), et \(\C{\fun{bms}_0}{0} =
3\). Asymptotiquement, nous avons \(\C{\fun{bms}_0}{n} \sim
\OC{\fun{bms}}{n}\).\index{bms@$\OC{\fun{bms}}{n}$}

\section{Comparaison}

Dans cette section, nous réunissons nos résultats à propos des tris
par interclassement descendant et ascendant pour rendre plus aisée
leur comparaison, et nous présentons aussi quelques propriétés qui
mettent en relation le coût de ces deux algorithmes.

\mypar{Coût minimal}

Le coût minimal des deux variantes du tri par interclassement est le
même: \(\OB{\fun{tms}}{n} = \OB{\fun{bms}}{n}\) et
\begin{equation*}
\frac{1}{2}n\lg n - \left(\frac{1}{2}\lg\frac{4}{3}\right)n +
\lg\frac{4}{3} \leqslant \OB{\fun{tms}}{n} \leqslant \frac{1}{2}n\lg
n.
\end{equation*}
Le minorant est atteint si \(n=2\) et est approché au plus près quand
\(n\)~est un nombre de Jacobsthal\index{Jacobsthal!nombre de $\sim$}
(voir~\eqref{eq:Jacobsthal} \vpageref{eq:Jacobsthal}). Le majorant est
atteint quand \(n=2^p\). Ces résultats ne sont peut-être pas
intuitifs \emph{a priori}.

\mypar{Coût maximal}

Dans les sections précédentes, nous avons obtenu les encadrements
suivants:
\begin{align*}
n\lg n - n + 1 &\leqslant \OW{\fun{tms}}{n} <
n\lg n - 0.91 n + 1;\\
n\lg n - n + 1 &\leqslant \OW{\fun{bms}}{n} < n\lg n - 1.
\end{align*}
Dans les deux cas, le minorant est atteint si, et seulement si
\(n=2^p\). Le majorant du tri descendant est approché au plus près
quand \(n\)~est l'entier le plus proche de \(2^p\ln 2\). Le majorant
du tri ascendant est approché au plus près si \(n=2^p+1\).

Il est intéressant d'encadrer \(\OW{\fun{bms}}{n}\) en fonction de
\(\OW{\fun{tms}}{n}\) pour éclairer davantage la relation entre ces
deux variantes du tri par interclassement.

Nous avons déjà remarqué \(\OC{\fun{bms}}{2^p} =
\OC{\fun{tms}}{2^p}\), d'où \(\OW{\fun{bms}}{2^p} =
\OW{\fun{tms}}{2^p}\). Par ailleurs, \(\OW{\fun{bms}}{2^p} =
\OW{\fun{bms}}{2^p-1} + p\), donc \(\OW{\fun{bms}}{2^p-1} =
\OW{\fun{tms}}{2^p-1}\). Un autre cas intéressant est
\(\OW{\fun{tms}}{2^p+1} = (p-1)2^p + p + 2\), donc
\(\OW{\fun{bms}}{2^p+1} - \OW{\fun{tms}}{2^p+1} = 2^p - p - 1\). Ceci
nous incite à conjecturer l'encadrement suivant, aux bornes
atteignables, qui relie le tri descendant et ascendant:
\begin{equation*}
\OW{\fun{tms}}{n} \leqslant \OW{\fun{bms}}{n} \leqslant
\OW{\fun{tms}}{n} + n - \ceiling{\lg n} - 1.
\index{tms@$\OW{\fun{tms}}{n}$}
\index{bms@$\OW{\fun{bms}}{n}$}
\end{equation*}
Nous établirons ces inégalités grâce à l'induction complète sur~\(n\)
et, ce faisant, nous découvrirons quand elles deviennent des
égalités. D'abord, déduisons de la récurrence générale satisfaite par
le coût du tri ascendant la récurrence pour le coût maximal:
\begin{equation}
\OW{\fun{bms}}{0} = \OW{\fun{bms}}{1} = 0;\quad \OW{\fun{bms}}{n} =
\OW{\fun{bms}}{2^{\ceiling{\lg n}-1}} +
\OW{\fun{bms}}{n-2^{\ceiling{\lg n}-1}} + n - 1.
\label{eq:bot}
\end{equation}
Par ailleurs, nous pouvons vérifier aisément, pour tout \(p \geqslant
0\), que nous avons l'égalité intéressante
\begin{equation}
\OW{\fun{tms}}{2^p} = \OW{\fun{bms}}{2^p}.
\label{eq:2p}
\end{equation}

\paragraph{Minorant}

Prouvons, pour tout \(n \geqslant 0\),
\begin{equation}
  \abovedisplayskip=2pt
  \belowdisplayskip=2pt
  \pred{W\(_L\)}{n} \colon \OW{\fun{tms}}{n} \leqslant
  \OW{\fun{bms}}{n}.
  \index{WL@$\predName{W}_L$}
\label{ineq:topbot}
\end{equation}
Par~\eqref{eq:2p}, il est clair que \(\pred{W\(_L\)}{2^0}\) est
vraie. Soit l'hypothèse d'induction \(\forall m \leqslant
2^p.\pred{W\(_L\)}{m}\). Le principe d'induction exige que nous
prouvions alors \(\pred{W\(_L\)}{2^p+i}\), pour tout \(0 < i <
2^p\).

Des équations \eqref{eq:bot} et \eqref{eq:2p}, il vient
\index{bms@$\OW{\fun{bms}}{n}$}
\begin{align*}
\OW{\fun{bms}}{2^p+i} &=
\OW{\fun{bms}}{2^p} + \OW{\fun{bms}}{i} + 2^p + i - 1\\
& = \OW{\fun{tms}}{2^p} + \OW{\fun{bms}}{i} + 2^p + i - 1 \geqslant
\OW{\fun{tms}}{2^p} + \OW{\fun{tms}}{i} + 2^p + i - 1,
\end{align*}
l'inégalité étant l'instance \(\pred{W\(_L\)}{i}\) de l'hypothèse
d'induction.  Par conséquent, si l'inégalité
\begin{equation}
\OW{\fun{tms}}{2^p} + \OW{\fun{tms}}{i} + 2^p + i - 1 \geqslant
\OW{\fun{tms}}{2^p+i}
\label{ineq:suff_cond}
\end{equation}
est vraie, alors le résultat \(\pred{W\(_L\)}{2^p+i}\)
s'ensuit. Essayons donc de prouver cette condition suffisante.

Soit \(n = 2^p + i\). Alors on a \(p = \floor{\lg n}\) et
\(\ceiling{\lg n} = \floor{\lg n} + 1\). L'équation~\eqref{eq:top}
\vpageref{eq:top} implique
\begin{align*}
\OW{\fun{tms}}{2^p+i} &=
(2^p+i)(p+1)-2^{p+1}+1 = ((p-1)2^p+1)+(p+1)i\\
&= \OW{\fun{tms}}{2^p} + (p+1)i.
\end{align*}
Par conséquent, l'inégalité~\eqref{ineq:suff_cond} est équivalente à
\(pi \leqslant \OW{\fun{tms}}{i} + 2^p - 1\). En employant
l'équation~\eqref{eq:top}, cette inégalité équivaut à
\begin{equation}
%  \abovedisplayskip=2pt
%  \belowdisplayskip=2pt
  (p - \ceiling{\lg i})i \leqslant 2^p - 2^{\ceiling{\lg i}}.
\label{conj}
\end{equation}
Pour la prouver, nous devons analyser deux cas complémentaires:
\begin{itemize}

  \item \(i=2^q\), avec \(0 \leqslant q < p\). Alors \(\lg i = q\) et
    l'équation~\eqref{conj} est équivalente à \((p-q)2^q \leqslant 2^p
    - 2^q\), c'est-à-dire,
    \begin{equation}
      p-q \leqslant 2^{p-q} - 1.\label{conj0}
    \end{equation}
    Posons \(f(x) := 2^x - x - 1\), pour \(x > 0\). Nous avons \(f(0)
    = f(1) = 0\) et \(f(x) > 0\) pour \(x>1\), donc
    l'inégalité~\eqref{conj0} est vraie et la borne est atteinte si,
    et seulement si, \(x=1\), soit \(q=p-1\).

  \item \(i = 2^q + j\), avec \(0 \leqslant q < p\) et \(0 < j <
    2^q\). Alors \(\floor{\lg i} = q = \ceiling{\lg i} - 1\) et
    l'inégalité~\eqref{conj} équivaut à \((p-q-1)i \leqslant 2^p -
    2^{q+1}\), c'est-à-dire
  \begin{equation}
%    \abovedisplayskip=2pt
%    \belowdisplayskip=2pt
    (p-q+1)2^q + (p-q-1)j \leqslant 2^p.\label{conj1}
  \end{equation}
  Puisque \(p-q-1 \geqslant 0\) et \(j < 2^q\), on a \((p-q-1)j
  \leqslant (p-q-1)2^q\) (égalité si \(q=p-1\)), donc
  \begin{equation*}
    (p-q+1)2^q+(p-q-1)j \leqslant (p-q)2^{q+1}.
  \end{equation*}
  L'inégalité~\eqref{conj1} est impliquée par l'inégalité \(2(p-q)
  \leqslant 2^{p-q}\). Posons \(g(x) := 2^x - 2x\), pour \(x>0\). Nous
  avons \(g(1) = g(2) = 0\) et \(f(x) > 0\) si \(x > 2\). Donc,
  l'inégalité~\eqref{conj1} est vraie et la borne atteinte si, et
  seulement si, \(x=1\), soit \(q = p -1\) (le cas \(x = 2\) implique
  \(i \leqslant 2^{p-1}\), donc une borne stricte). \hfill\(\Box\)

\end{itemize}

Cherchons maintenant la forme de~\(n\) quand \(\pred{W\(_L\)}{n}\) est
stricte. Nous avons établi ci-dessus que, si \(q=p-1\), c'est-à-dire
que la notation binaire de~\(n\) commence par deux bits à~\(1\),
formellement \((11(0+1)^*)_2\), alors l'inégalité suivante est vraie:
\begin{equation*}
\OW{\fun{tms}}{2^p+i} =
\OW{\fun{tms}}{2^p} + \OW{\fun{tms}}{i} + 2^p + i - 1 \leqslant
\OW{\fun{tms}}{2^p} + \OW{\fun{bms}}{i} + 2^p + i - 1 =
\OW{\fun{bms}}{2^p+i}.
\end{equation*}
Cette inégalité n'est pas stricte: \(\OW{\fun{tms}}{2^p+i} =
\OW{\fun{bms}}{2^p+i} \Leftrightarrow \OW{\fun{tms}}{i} =
\OW{\fun{bms}}{i}\). \index{bms@$\OW{\fun{bms}}{n}$}
\index{tms@$\OW{\fun{tms}}{n}$}

En reprenant l'analyse par cas plus haut, si \(i = 2^q + j\), nous
avons \(\OW{\fun{tms}}{2^p+i} = \OW{\fun{bms}}{2^p+i}\), si, et
seulement si, \(\OW{\fun{tms}}{2^{p-1}+j} =
\OW{\fun{bms}}{2^{p-1}+j}\). Ces équivalences peuvent être répétées,
engendrant deux séries strictement décroissantes d'entiers positifs,
\(2^p+i > 2^{p-1}+j > 2^{p-2}+k > \dots\) et \(i > j > k > \dots\) La
fin de la dernière série est simplement~\(0\), ce qui veut dire que la
première termine à une puissance de~\(2\), à laquelle s'applique
l'équation~\eqref{eq:2p}. En d'autres termes, la représentation
binaire de~\(n\) est constituée d'une série d'au moins un bit à~\(1\)
(d'après \(2^p\), \(2^{p-1}\), \(2^{p-2}\), \ldots), possiblement
suivie par des bits à~\(0\), ce que nous pouvons formellement écrire
\(n=(1^+0^*)_2\), c'est-à-dire, que \(n\)~est la différence de deux
puissances de~\(2\). Nous souvenant par ailleurs que
\(\OW{\fun{bms}}{2^q} = \OW{\fun{tms}}{2^q}\), nous avons établi
\begin{equation*}
\boxed{\OW{\fun{bms}}{n} = \OW{\fun{tms}}{n} \Leftrightarrow n=2^q \;
  \text{ou} \; n=2^p - 2^q, \text{avec \(p > q \geqslant 0\)}.}
\end{equation*}
Remarquons que si \(n=2^p-1\), le nombre d'interclassements ascendants
déséquilibrés est maximal, et les coûts maximaux sont les mêmes dans
les deux variantes.

\paragraph{Majorant}

Si \(n=2^p+1\), alors \(p=\floor{\lg n}=\ceiling{\lg
  n}-1\). Par ailleurs, la définition~\eqref{eq:bot} implique
\(\OW{\fun{bms}}{2^p+1} = p2^p+1\) et la définition~\eqref{eq:top}
\vpageref{eq:top} \(\OW{\fun{tms}}{2^p+1} = (p-1)2^p+p+2\), donc
\(\OW{\fun{bms}}{2^p+i} - \OW{\fun{tms}}{2^p+i} = 2^p - p - 1\). En
termes de~\(n\), ceci veut dire: \(\OW{\fun{bms}}{n} -
\OW{\fun{tms}}{n} = n - \ceiling{\lg n} - 1\), si \(n=2^p+1\). Nous
souhaitons montrer que cette différence est maximale:
\begin{equation}
\pred{W\(_U\)}{n} \colon \OW{\fun{bms}}{n} \leqslant \OW{\fun{tms}}{n} + n -
\ceiling{\lg n} - 1.
\label{ineq:upper_Wbms}
\index{WU@$\predName{W}_U$}
\end{equation}
Notons comment l'équation~\eqref{eq:2p} implique
\(\pred{W\(_U\)}{2^p}\). Par conséquent, soit l'hypothèse d'induction
\(\forall m \leqslant 2^p.\pred{W\(_U\)}{m}\) et montrons que
\(\pred{W\(_U\)}{2^p+i}\), pour tout \(0 < i < 2^p\). Soit
\(n=2^p+i\). De \eqref{eq:bot} \& \eqref{eq:2p} vient
\begin{align*}
\OW{\fun{bms}}{2^p+i} &= \OW{\fun{bms}}{2^p} + \OW{\fun{bms}}{i} +
2^p + i - 1
= \OW{\fun{tms}}{2^p} + \OW{\fun{bms}}{i} + 2^p + i - 1\\
&\leqslant \OW{\fun{tms}}{2^p} + \OW{\fun{tms}}{i} + 2^p + 2i -
\ceiling{\lg i} - 2,
\end{align*}
où l'inégalité est l'instance \(\pred{W\(_U\)}{i}\) de l'hypothèse
d'induction. De plus, \(n - \ceiling{\lg n} - 1 = 2^p + i - p -
2\). Par conséquent, si nous avions l'inégalité
\begin{equation*}
\OW{\fun{tms}}{2^p} + \OW{\fun{tms}}{i} +
2^p + 2i - \ceiling{\lg i} - 2 \leqslant \OW{\fun{tms}}{2^p+i} + 2^p +
i - p - 2,
\end{equation*}
alors \(\pred{W\(_U\)}{2^p+i}\) s'ensuivrait. En employant
l'équation~\eqref{eq:top}, nous
avons\index{bms@$\OW{\fun{bms}}{n}$}\index{tms@$\OW{\fun{tms}}{n}$}
\begin{align*}
\OW{\fun{tms}}{i} &= i\ceiling{\lg i} - 2^{\ceiling{\lg i}} + 1,\\
\OW{\fun{tms}}{2^p} &= (p-1)2^p + 1,\\
\OW{\fun{tms}}{2^p+i} &= \OW{\fun{tms}}{2^p} + (p+1)i.
\end{align*}
L'inégalité à prouver devient \(\OW{\fun{tms}}{i} + i - \ceiling{\lg i} \leqslant (p+1)i - p\), soit
\begin{equation}
1 \leqslant (i-1)(p-\ceiling{\lg i}) + 2^{\ceiling{\lg i}}.
\label{eq:conj2}
\end{equation}
Nous avons deux cas complémentaires à considérer:
\begin{itemize}

\item \(i = 2^q\), avec \(0 \leqslant q < p\). Alors \(\lg i = q\) et
  l'inéquation~\eqref{eq:conj2} équivaut à \((p-q+1)(2^q-1) \geqslant
  0\). Puisque \(0 \leqslant q < p\) implique \(p-q+1>1\) et \(2^q
  \geqslant 1\), l'inégalité est prouvée et devient une égalité si, et
  seulement si, \(q=0\).

  \item \(i = 2^q + j\), avec \(0 \leqslant q < p\) et \(0 < j <
    2^q\). Nous avons alors \(\floor{\lg i} = q = \ceiling{\lg i} -
    1\) et l'inéquation~\eqref{eq:conj2} est donc équivalente à \(1
    \leqslant (2^q + j - 1) (p-q) + 2^q\), ou
    \begin{equation}
     1 \leqslant (p-q+1)2^q + (p-q-1)(j-1).\label{ineq:2q_j}
    \end{equation}
    De \(q < p\) nous déduisons \(p-q+1 \geqslant 2\) et \(p-q-1
    \geqslant 0\); nous avons aussi \(2^q \geqslant 1\) et \(j
    \geqslant 1\). Par conséquent, \((p-q+1)2^q \geqslant 2\) et
    \((p-q-1)(j-1) \geqslant 0\), donc l'inéquation~\eqref{ineq:2q_j}
    est vraie au sens strict (pas d'égalité).\hfill\(\Box\)

\end{itemize}
En passant, nous avons aussi prouvé que si \(i=1\), soit \(n=2^p +
1\), alors
\begin{equation*}
\OW{\fun{bms}}{2^p+1} = \OW{\fun{tms}}{2^p} +
\OW{\fun{bms}}{1} + 2^p \leqslant \OW{\fun{tms}}{2^p} +
\OW{\fun{tms}}{1} + 2^p = \OW{\fun{tms}}{2^p+1} + 2^p - p - 1.
\end{equation*}
Mais, puisque \(\OW{\fun{tms}}{1} = \OW{\fun{bms}}{1} = 0\),
l'inégalité restante est en fait une égalité. En conclusion,
\begin{equation*}
\boxed{\OW{\fun{bms}}{n} = \OW{\fun{tms}}{n} + n - \ceiling{\lg n} - 1
  \Leftrightarrow n=1 \;\text{ou}\; n=2^p+1, \text{avec \(p \geqslant
    0\)}.}
\end{equation*}

\paragraph{Programme}

Nous présenterons le langage de programmation \Erlang dans la
partie~\ref{part:implementation}, mais voici déjà comment calculer
efficacement les coûts maximaux: \ErlangInUnchecked{max} Remarquons
que l'exponentiation binaire~\(2^n\) est évaluée efficacement au moyen
des équations récurrentes suivantes:
\begin{equation*}
2^0 = 1,\quad 2^{2m} = (2^m)^2,\quad 2^{2m+1} =  2(2^m)^2.
\end{equation*}
Le coût \(\C{\fun{exp2}}{n}\) satisfait donc \(\C{\fun{exp2}}{0} = 1\)
et \(\C{\fun{exp2}}{n} = 1 + \C{\fun{exp2}}{\floor{n/2}}\), si \(n >
0\). Par conséquent, si \(n > 0\), il vaut \(1\)~plus le nombre de
bits de~\(n\), soit \(\C{\fun{exp2}}{n} = \floor{\lg n} + 2\), sinon
\(\C{\fun{exp2}}{0} = 1\).

\mypar{Coût moyen}

En somme, nous avons établi, pour \(n \geqslant 2\),
\begin{align*}
n\lg n - \frac{3}{2}n + 2 &< \OM{\fun{tms}}{n} < n\lg n - n + 1,\\
n\lg n - \alpha n + (2\alpha - 1) &< \OM{\fun{bms}}{n}
< n\lg n - (\alpha - 1)n - (3 - 2\alpha),
\end{align*}
où \(\alpha \simeq 1.2645\), \(2\alpha - 1 \simeq 1.52899\) et \(3 -
2\alpha \simeq 0.471\). Pour le tri descendant, la nature de~\(n\)
quand l'encadrement est le plus serré n'a pas été élucidée par notre
méthode inductive. Pour la variante ascendante, le minorant est
approché au plus près quand \(n=2^p\), mais nous n'avons pu déterminer
les valeurs de~\(n\) qui permettent la meilleure approximation du
majorant.

L'encadrement de \(\OM{\fun{bms}}{n}\) ne nous permet pas de comparer
les coûts moyens des deux variantes du tri par interclassement que
nous avons étudiées. Nous allons donc prouver que le tri descendant
requiert moins de comparaisons, en moyenne, que le tri
ascendant. Puisque nous avons déjà établi que ceci est vrai aussi pour
le cas le plus défavorable (voir~\eqref{ineq:topbot}
\vpageref{ineq:topbot}), et que leur coût minimal sont égaux
(voir~\eqref{eq:OBbms} \vpageref{eq:OBbms}), ceci scelle le sort de la
variante ascendante, qui renaîtra sous une forme plus utile à la
section~\vref{sec:online}. Nous voulons prouver par induction:
\begin{equation*}
\OM{\fun{tms}}{n} \leqslant \OM{\fun{bms}}{n}.
\end{equation*}
Nous savons déjà que l'inégalité est une égalité quand \(n=2^p\), donc
vérifions-la pour \(n=2\), puis supposons qu'elle soit valable
jusqu'à~\(2^p\) et tâchons ensuite de la prouver pour \(2^p+i\), où
\(0 < i \leqslant 2^p\), dont notre objectif découlera. Souvenons-nous
de l'équation~\eqref{eq:Abms_2p_i} \vpageref{eq:Abms_2p_i}:
\begin{equation*}
\OM{\fun{bms}}{2^p+i} = \OM{\fun{bms}}{2^p} + \OM{\fun{bms}}{i}
+ 2^p + i - \frac{2^p}{i+1} - \frac{i}{2^p+1}.
\end{equation*}
Puisque \(\OM{\fun{bms}}{2^p} = \OM{\fun{tms}}{2^p}\) et que, par
hypothèse, \(\OM{\fun{tms}}{i} \leqslant \OM{\fun{bms}}{i}\), nous
avons
\begin{equation}
\OM{\fun{bms}}{2^p+i} \geqslant \OM{\fun{tms}}{2^p} + \OM{\fun{tms}}{i}
+ 2^p + i - \frac{2^p}{i+1} - \frac{i}{2^p+1}.
\label{ineq:Atms_Abms}
\end{equation}
Si nous pouvions montrer que le membre droit est supérieur ou égal à
\(\OM{\fun{tms}}{2^p+i}\), nous aurions gagné. Généralisons cette
condition suffisante et exprimons-la comme le lemme suivant:
\begin{equation*}
  \pred{T}{m,n} \colon
  \OM{\fun{tms}}{m+n} \leqslant \OM{\fun{tms}}{m} + \OM{\fun{tms}}{n} +
  m + n - \frac{m}{n+1} - \frac{n}{m+1}.
\end{equation*}
Ordonnons les paires \((m,n)\) d'entiers naturels \(m\)~et~\(n\) selon
un ordre lexicographique (voir définition~\eqref{def:lexico}
\vpageref{def:lexico}). Le cas de base, \((0,0)\), est aisément
vérifiable. Nous observons que la proposition à établir est
symétrique, \(\pred{T}{m,n} \Leftrightarrow \pred{T}{n,m}\), donc nous
n'avons besoin que de trois cas: \((2p,2q)\), \((2p,2q+1)\) et
\((2p+1,2q+1)\).
\begin{enumerate}

  \item \((m,n) = (2p,2q)\). Dans ce cas,
    \begin{itemize}

      \item \(\OM{\fun{tms}}{m+n} = \OM{\fun{tms}}{2(p+q)} =
        2\OM{\fun{tms}}{p+q} + 2(p+q) - 2 + 2/(p+q+1)\);

      \item \(\OM{\fun{tms}}{m} = \OM{\fun{tms}}{2p} =
        2\OM{\fun{tms}}{p} + 2p - 2 + 2/(p+1)\);

      \item \(\OM{\fun{tms}}{n} = \OM{\fun{tms}}{2q} =
        2\OM{\fun{tms}}{q} + 2q - 2 + 2/(q+1)\).

    \end{itemize}
    Alors, le membre droit de \(\pred{T}{m,n}\) est
    \begin{equation*}
      r := 2\left(\OM{\fun{tms}}{p} + \OM{\fun{tms}}{q} + 2(p+q) - 2 +
        \tfrac{1}{p+1} + \tfrac{1}{q+1} - \tfrac{p}{2q+1} -
        \tfrac{q}{2p+1}\right).
    \end{equation*}
    L'hypothèse d'induction \(\pred{T}{p,q}\) est
    \begin{equation*}
      \OM{\fun{tms}}{p+q} \leqslant \OM{\fun{tms}}{p} +
      \OM{\fun{tms}}{q} + p + q - \frac{p}{q+1} - \frac{q}{p+1}.
    \end{equation*}
    Donc \(r/2 \geqslant \OM{\fun{tms}}{p+q} + p + q - 2 +
    \tfrac{q+1}{p+1} + \tfrac{p+1}{q+1} - \tfrac{p}{2q+1} -
    \tfrac{q}{2p+1}\). Si le membre droit est supérieur ou égal à
    \(\OM{\fun{tms}}{m+n}/2\), alors \(\pred{T}{m,n}\) est
    prouvé. En d'autres termes, nous devons établir
    \begin{equation*}
      \frac{p+1}{q+1} + \frac{q+1}{p+1} \geqslant 1 +
      \frac{p}{2q+1} + \frac{q}{2p+1} + \frac{1}{p+q+1}.
    \end{equation*}
    Nous développons tout pour nous défaire des fractions; nous
    observons alors que nous pouvons factoriser~\(pq\) et le polynôme
    bivarié restant est nul si \(p=q\) (l'inégalité est une égalité),
    ce qui signifie que nous pouvons factoriser \(p-q\) (en fait, deux
    fois). Finalement, cette inéquation équivaut à
    \(pq(p-q)^2(2p+2q+3) \geqslant 0\), où \(p,q \geqslant 0\), ce qui
    veut dire que \(\pred{T}{m,n}\) est vrai.

  \item \((m,n) = (2p,2q+1)\). Dans ce cas,
    \begin{itemize}

      \item \(\OM{\fun{tms}}{m+n} = \OM{\fun{tms}}{2(p+q)+1} =
        \OM{\fun{tms}}{p+q} + \OM{\fun{tms}}{p+q+1} + 2(p+q) - 1 +
        \tfrac{2}{(p+q+2)}\);

      \item \(\OM{\fun{tms}}{m} = \OM{\fun{tms}}{2p} =
        2\OM{\fun{tms}}{p} + 2p - 2 + 2/(p+1)\);

      \item \(\OM{\fun{tms}}{n} = \OM{\fun{tms}}{2q+1} =
        \OM{\fun{tms}}{q} + \OM{\fun{tms}}{q+1} + 2q - 1 + 2/(q+2)\).

    \end{itemize}
    Alors, le membre droit de \(\pred{T}{m,n}\) est
    \begin{equation*}
      r := 2\OM{\fun{tms}}{p} + \OM{\fun{tms}}{q} +
      \OM{\fun{tms}}{q+1} + 4(p+q) - 2 + \tfrac{2}{p+1} +
      \tfrac{2}{q+2} - \tfrac{p}{q+1} - \tfrac{2q+1}{2p+1}.
    \end{equation*}
    Les hypothèses d'induction \(\pred{T}{p,q}\) et
    \(\pred{T}{p,q+1}\) sont
    \begin{itemize}

      \item \(\OM{\fun{tms}}{p+q} \leqslant \OM{\fun{tms}}{p} +
      \OM{\fun{tms}}{q} + p + q - \frac{p}{q+1} - \frac{q}{p+1}\),

      \item \(\OM{\fun{tms}}{p+(q+1)} \leqslant \OM{\fun{tms}}{p} +
      \OM{\fun{tms}}{q+1} + p + (q + 1) - \frac{p}{q+2} -
      \frac{q+1}{p+1}\).

    \end{itemize}
    Donc \(r \geqslant \OM{\fun{tms}}{p+q} + \OM{\fun{tms}}{p+q+1} +
    2(p+q) - 3 + \tfrac{2q+3}{p+1} + \tfrac{p+2}{q+2} -
    \tfrac{2q+1}{2p+1}\). Si le membre droit est supérieur ou égal à
    \(\OM{\fun{tms}}{m+n}\), alors \(\pred{T}{m,n}\) est prouvé.  En
    d'autres termes, nous voulons
    \begin{equation*}
      \frac{2q+3}{p+1} + \frac{p+2}{q+2} \geqslant 2 +
      \frac{2q+1}{2p+1} + \frac{2}{p+q+2}.
    \end{equation*}
    En développant tout pour nous débarrasser des fractions, nous
    obtenons un polynôme bivarié avec les facteurs triviaux \(p\)~et
    \(p-q\) (car, si \(p=q\), l'inégalité devient une égalité). Après,
    un système de calcul formel peut finir la factorisation, montrant
    que l'inégalité équivaut à \(p(p-q)(p-q-1)(2p+2q+5) \geqslant 0\),
    donc \(\pred{T}{m,n}\) tient.

  \item \((m,n) = (2p+1,2q+1)\). Dans ce cas,
    \begin{itemize}

      \item \(\OM{\fun{tms}}{m+n} = \OM{\fun{tms}}{2(p+q+1)} =
        2\OM{\fun{tms}}{p+q+1} + 2(p+q) + 2/(p+q+2)\);

      \item \(\OM{\fun{tms}}{n} = \OM{\fun{tms}}{2p+1} =
        \OM{\fun{tms}}{p} + \OM{\fun{tms}}{p+1} + 2p - 1 + 2/(p+2)\).

      \item \(\OM{\fun{tms}}{n} = \OM{\fun{tms}}{2q+1} =
        \OM{\fun{tms}}{q} + \OM{\fun{tms}}{q+1} + 2q - 1 + 2/(q+2)\).

    \end{itemize}
    Alors, le membre droit de \(\pred{T}{m,n}\) est
    \begin{equation*}
      r := \OM{\fun{tms}}{p} + \OM{\fun{tms}}{q} + \OM{\fun{tms}}{p+1}
      + \OM{\fun{tms}}{q+1} + 4(p+q) + \tfrac{2}{p+2} + \tfrac{2}{q+2}
      - \tfrac{2p+1}{2q+2} - \tfrac{2q+1}{2p+2}.
    \end{equation*}
    Les hypothèses d'induction symétriques \(\pred{T}{p,q+1}\) et
    \(\pred{T}{p+1,q}\):
    \begin{itemize}

      \item \(\OM{\fun{tms}}{p+(q+1)} \leqslant \OM{\fun{tms}}{p} +
        \OM{\fun{tms}}{q+1} + p + q + 1 - \tfrac{p}{q+2}
        - \tfrac{q+1}{p+1}\);

      \item \(\OM{\fun{tms}}{(p+1)+q} \leqslant \OM{\fun{tms}}{p+1} +
        \OM{\fun{tms}}{q} + p + q + 1 - \tfrac{p+1}{q+1} -
        \tfrac{q}{p+2}\).

    \end{itemize}
    Donc, \(r \geqslant 2\OM{\fun{tms}}{p+q+1} + 2(p+q) - 2 +
    \tfrac{q+1}{p+1} + \tfrac{q}{p+2} + \tfrac{p+1}{q+1} +
    \tfrac{p}{q+2} + \tfrac{2}{p+2} + \tfrac{2}{q+2} -
    \tfrac{2p+1}{2q+2} - \tfrac{2q+1}{2p+2}\). Si le membre droit est
    supérieur ou égal à \(\OM{\fun{tms}}{m+n}\), alors
    \(\pred{T}{m,n}\) est prouvé. En d'autres termes, nous voulons
    \begin{equation*}
      \frac{q+1}{p+1} + \frac{q+2}{p+2} + \frac{p+2}{q+2} +
      \frac{p+1}{q+1} \geqslant 2 + \frac{2p+1}{2q+2} +
      \frac{2q+1}{2p+2} + \frac{2}{p+q+2}.
    \end{equation*}
    Après développement pour former un polynôme positif, nous
    remarquons que l'inégalité est une égalité si \(p=q\), donc
    \(p-q\) est un facteur. Après division, un autre facteur \(p-q\)
    apparaît. L'inégalité équivaut donc à \((p-q)^2(2p^2(q+1) + p(2q^2
    + 9q + 8) + 2(q+2)^2) \geqslant 0\), donc \(\pred{T}{m,n}\) est
    vrai dans ce cas aussi.

\end{enumerate}
Au total, \(\pred{T}{m,n}\) est vrai dans chaque cas, donc le lemme
est vrai pour tout \(m\)~et~\(n\). En appliquant le lemme
à~\eqref{ineq:Atms_Abms}, nous prouvons le théorème
\(\OM{\fun{tms}}{n} \leqslant \OM{\fun{bms}}{n}\), pour tout~\(n\). En
regroupant tous les cas où l'inégalité est une égalité, nous vérifions
ce que nous attendions: \(m=n\), \(m=n+1\) ou
\(n=m+1\). Pour~\eqref{ineq:Atms_Abms}, ceci signifie \(i=2^p\) ou
\(i=2^p-1\), soit
\begin{equation*}
  \boxed{\OM{\fun{tms}}{n} = \OM{\fun{bms}}{n} \Leftrightarrow n=2^p
    \;\text{ou}\; n=2^p-1, \text{où \(p \geqslant 0\)}.}
\end{equation*}

\paragraph{Programme}

Voici une réalisation en \Erlang du calcul des coûts moyens du tri par
interclassement descendant et ascendant:
\ErlangInUnchecked{mean}

\mypar{Interclassement ou insertion}
\index{tri par interclassement!comparaison avec le tri par insertion|(}

Comparons les tris par insertion et interclassement dans leurs
variantes les plus rapides. Nous avons trouvé à
l'équation~\eqref{eq:ave_i2wb} \vpageref{eq:ave_i2wb} le coût moyen du
tri par insertion bidirectionnelle équilibrée:
\begin{equation*}
\M{\fun{i2wb}}{n}\index{12wb@$\M{\fun{i2wb}}{n}$}
  = \frac{1}{8}(n^2 + 13n - \ln n + 10 - \ln 2) + \epsilon_n,\;
\text{avec \(0 < \epsilon_n < \frac{7}{8}\)}.
\end{equation*}
Par ailleurs, nous venons tout juste de trouver que le coût en plus
des comparaisons est \(\ceiling{3n/2} + 2\ceiling{\lg n} - 1\) pour
\fun{bms\(_0\)/1}\index{bms0@\fun{bms\(_0\)/1}}, et
\(\OC{\fun{bms\(_0\)}}{n} = \OC{\fun{bms}}{n}\). En outre, nous avons
obtenu l'encadrement~\eqref{ineq:bounds_Mbms}
\vpageref{ineq:bounds_Mbms} de \(\OM{\fun{bms}}{n}\), le majorant
étant excellent. Par conséquent,
\begin{align*}
\M{\fun{bms\(_0\)}}{n} &< (n\lg n - (\alpha - 1)n - (3-2\alpha))
+ (\ceiling{3n/2} + 2\ceiling{\lg n} - 1)\\
& < (n+2)\lg n + 1.236n + 1.529;\\
\M{\fun{bms\(_0\)}}{n} &> (n\lg n - 1.35n + 1.69) + (\ceiling{3n/2} +
2\ceiling{\lg n} - 1)\\
&> (n+2)\lg n + 0.152n + 0.69;
\end{align*}
\begin{equation*}
(n^2 + 13n - \ln 2n + 10)/8 < \M{\fun{i2wb}}{n} < (n^2 + 13n - \ln 2n + 17)/8.
\end{equation*}
où \(\alpha \simeq 1.2645\) et \(\ceiling{x} < x + 1\). Donc,
\begin{equation*}
  (n+2)\lg n + 1.236n + 1.529 < (n^2 + 13n - \ln 2n + 10)/8
\end{equation*}
implique \(\M{\fun{bms\(_0\)}}{n} < \M{\fun{i2wb}}{n}\), et aussi
l'inégalité
\begin{equation*}
  (n^2 + 13n - \ln 2n + 17)/8 < (n+2)\lg n + 0.152n + 0.69
\end{equation*}
entraîne la relation \(\M{\fun{i2wb}}{n} <
\M{\fun{bms\(_0\)}}{n}\). Avec l'aide d'un système de calcul formel,
nous trouvons que \(\M{\fun{bms\(_0\)}}{n} < \M{\fun{i2wb}}{n}\) si
\(n \geqslant 43\), et \(\M{\fun{i2wb}}{n} < \M{\fun{bms\(_0\)}}{n}\)
si \(3 \leqslant n \leqslant 29\). Pour le cas \(n=2\), nous avons:
\(\M{\fun{i2wb}}{2} = 11/2 > 5 = \M{\fun{bms\(_0\)}}{2}\). Si nous
laissons de côté ce cas particulier, nous pouvons conclure que le tri
par insertion est plus rapide, en moyenne, pour des piles de moins de
\(30\)~clés, et le contraire est vrai pour des piles d'au moins
\(43\)~clés. Entre-deux, nous ne savons rien, mais nous pouvons
calculer efficacement les coûts moyens et procéder par dichotomie sur
l'intervalle de~\(30\) à~\(43\).

En utilisant le programme \Erlang ci-dessus, nous trouvons rapidement
que le tri par insertion est battu le plus tôt par le tri par
interclassement à la valeur \(n=36\). Ceci suggère de laisser tomber
\fun{duo/1} en faveur d'une fonction qui construit des segments de
\(35\)~clés à partir de la pile originelle, puis les trie avec des
insertions équilibrées bidirectionnelles et, finalement, s'il y a plus
de \(35\)~clés, commence à interclasser ces piles ordonnées. Cette
amélioration revient à ne pas construire les premiers \(35\)~niveaux
dans l'arbre d'interclassement\index{arbre!$\sim$ d'interclassement}
mais, plutôt, à construire le \(35^{\text{e}}\) niveau par insertions.

Malgré l'analyse qui précède, nous devrions garder à l'esprit qu'elle
s'appuie sur la mesure qu'est le nombre d'appels de fonction, qui
suppose que chaque appel est vraiment effectué par l'environnement
d'exécution (pas d'expansion en ligne), que tous les changements de
contextes ont la même durée, que les autres opérations prennent un
temps négligeable en comparaison, que l'antémémoire, les prévisions de
saut et le pipeline des instructions sont sans effet etc. Même
l'utilisation du même compilateur sur la même machine ne nous dédouane
pas de procéder à un prudent banc de test.
\index{tri par interclassement!comparaison avec le tri par insertion|)}
\index{tri par interclassement!$\sim$ ascendant|)}


\section{Tri en ligne}
\label{sec:online}
\index{tri par interclassement!$\sim$ en ligne|(}

Les algorithmes de tri peuvent être distingués selon qu'ils opèrent
sur la totalité de la pile de clés, ou bien clé par clé. Les premiers
sont dits \emph{hors ligne}, car les clés ne sont pas ordonnées
pendant qu'elles arrivent, et les derniers sont appelés \emph{en
  ligne}, car le tri est temporellement entrelacé avec l'acquisition
des données. Le tri par interclassement ascendant est un algorithme
hors ligne, mais il peut être facilement modifié pour devenir en ligne
en remarquant que les interclassements équilibrés peuvent être répétés
à chaque fois qu'une nouvelle clé arrive et les interclassements
déséquilibrés ne sont effectués que lorsque la pile courante ordonnée
est demandée.

Plus précisément, considérons à nouveau la \fig~\ref{fig:msort_gen}
\vpageref{fig:msort_gen} sans les interclassements
déséquilibrés. L'ajout d'une nouvelle clé (par la droite) mène à deux
circonstances: si~\(n\) est pair, c'est-à-dire \(e_0 > 0\), alors rien
n'est fait et la clé devient une pile singleton ordonnée de longueur
\(2^0\); sinon, une cascade d'interclassements entre piles de
longueurs égales à~\(2^{e_i}\), avec \(e_i=i\), est déclenchée jusqu'à
ce que \(e_j > j\). Ce procédé est exactement l'addition en binaire
de~\(1\) à~\(n\), sauf que des interclassements, au lieu d'additions
de bits, sont effectués tant qu'une retenue est produite et propagée.

À notre connaissance, seul \cite{Okasaki_1998a} mentionne cette
variante; il montre qu'elle peut être efficacement programmée avec des
structures de données purement fonctionnelles, tout comme la version
hors ligne. (Remarquons que le contexte dans lequel il écrit est
néanmoins différent du nôtre car il suppose une évaluation paresseuse
et procède à une analyse du coût amorti.) Le tri hors ligne est
utilisé dans la bibliothèque de référence de l'assistant de preuve
\index{Coq@\textsf{Coq}} \textsf{Coq} \citep{BertotCasteran_2004}.

Notre programme est montré à la \fig~\ref{fig:oms}\index{tri par
  interclassement!$\sim$ en ligne!programme}.
\begin{figure}
\begin{equation*}
\boxed{%
\begin{array}{r@{\;}l@{\;}l}
\fun{oms}(s)   & \xrightarrow{\smash{\phi}}
               & \fun{unb}(\fun{sum}(s,\el),\el).\\
\\
\fun{sum}(\el,t)         & \xrightarrow{\smash{\chi}} & t;\\
\fun{sum}(\cons{x}{s},t) & \xrightarrow{\smash{\psi}}
                         & \fun{sum}(s,\fun{add}([x],t)).\\
\\
\fun{add}(s,\el) & \xrightarrow{\smash{\omega}} & [\fun{one}(s)];\\
\fun{add}(s,\cons{\fun{zero}()}{t})
                    & \xrightarrow{\smash{\gamma}}
                    & \cons{\fun{one}(s)}{t};\\
\fun{add}(s,\cons{\fun{one}(u)}{t}) & \xrightarrow{\smash{\delta}}
                    & [\fun{zero}()|\fun{add}(\fun{mrg}(s,u),t)].\\
\\
\fun{unb}(\el,u) & \xrightarrow{\smash{\mu}} & u;\\
\fun{unb}(\cons{\fun{zero}()}{s},u)
                 & \xrightarrow{\smash{\nu}} & \fun{unb}(s,u);\\
\fun{unb}(\cons{\fun{one}(t)}{s},u)
                 & \xrightarrow{\smash{\xi}}
                 & \fun{unb}(s,\fun{mrg}(t,u)).\\
\\
\fun{mrg}(\el,t)         & \xrightarrow{\smash{\theta}} & t;\\
\fun{mrg}(s,\el)         & \xrightarrow{\smash{\iota}} & s;\\
\fun{mrg}(\cons{x}{s},\cons{y}{t}) & \xrightarrow{\smash{\kappa}}
                         & \cons{y}{\fun{mrg}(\cons{x}{s},t)},\;
                           \text{si \(x \succ y\)};\\
\fun{mrg}(\cons{x}{s},t) & \xrightarrow{\smash{\lambda}}
                         & \cons{x}{\fun{mrg}(s,t)}.
\end{array}}
\end{equation*}
\caption{Tri par interclassement en ligne avec \fun{oms/1}}
\label{fig:oms}
\end{figure}
Nous usons de \(\fun{zero}()\)\index{zero@\fun{zero/0}} pour
représenter un bit à~\(0\) dans la notation binaire du nombre de clés
ordonnées présentement. Par dualité, l'appel
\(\fun{one}(s)\)\index{one@\fun{one/1}} dénote un bit à~\(1\), où la
pile~\(s\) contient un nombre de clés triées égal à la puissance
de~\(2\) associée dans la notation binaire. Chaque appel à \fun{one/1}
correspond à un sous-arbre\index{arbre!$\sim$ d'interclassement} à la
\fig~\vref{fig:msort_gen}. Par exemple, la configuration de la pile
\([\fun{one}([4]), \fun{zero}(), \fun{one}([3,6,7,9])]\) correspond au
nombre binaire \((101)_2\), donc elle contient \(1 \cdot 2^2 + 0 \cdot
2^1 + 1 \cdot 2^0 = 5\) clés en tout. Gardons présent à l'esprit que
les bits sont à l'envers dans la pile, ainsi l'arrivée de la clé~\(5\)
donnerait \([\fun{zero}(),\fun{one}([4,5]),\fun{one}([3,6,7,9])]\).

Notons que le programme à la \fig~\vref{fig:oms} ne capture pas
l'usage normal du tri en ligne, car, en pratique, l'argument~\(s\) de
l'appel \(\fun{oms}(s)\)\index{oms@\fun{oms/1}} (anglais:
\emph{on-line merge sort}) ne serait pas connu dans sa totalité, donc
\fun{add/2}\index{add@\fun{add/2}} ne serait appelé que quand une clé
est disponible. Toutefois, dans l'analyse suivante, nous nous
intéressons au nombre de comparaisons d'une série de mises à jour
par~\fun{sum/2}\index{sum@\fun{sum/2}} (un cadre de travail que nous
avons déjà rencontré à la section~\ref{sec:queueing}), suivie d'une
suite d'interclassements déséquilibrés
par~\fun{unb/2}\index{unb@\fun{unb/2}} (anglais: \emph{unbalanced}),
avec pour résultat une pile ordonnée; par conséquent, notre programme
convient à notre objectif qui est d'évaluer
\(\OC{\fun{oms}}{n}\)\index{oms@$\OC{\fun{oms}}{n}$}.

Soit \(\OC{\fun{add}}{n}\)\index{add@$\OC{\fun{add}}{n}$} le nombre de
comparaisons pour ajouter une nouvelle clé à la pile courante de
longueur~\(n\) et souvenons-nous que
\(\OC{\fun{mrg}}{m,n}\)\index{mrg@$\OC{\fun{mrg}}{m,n}$} est le nombre
de comparaisons pour interclasser deux piles de longueurs
\(m\)~et~\(n\) en appelant \fun{mrg/2}\index{mrg@\fun{mrg/2}}. Si
\(n\)~est pair, alors il n'y a pas de comparaisons, puisque ceci est
similaire à l'ajout de~\(1\) à une séquence binaire \((\Xi{0})_2\), où
\(\Xi\)~est une chaîne de bits arbitraire. Sinon, une suite
d'interclassements équilibrés de taille~\(2^i\) sont effectués, car
cela est similaire à l'ajout de~\(1\) à \((\Xi{011}\ldots 1)_2\), où
\(\Xi\)~est arbitraire. Par conséquent,
\begin{equation*}
\abovedisplayskip=0pt
\belowdisplayskip=0pt
\OC{\fun{add}}{2j} = 0,\qquad
\OC{\fun{add}}{2j-1} = \sum_{i=0}^{\rho_{2j}}{\OC{\fun{mrg}}{2^i,2^i}},
\end{equation*}
où \(\rho_n\)~est la plus grande puissance de~\(2\) qui divise~\(n\).
Soit \(\OC{\fun{sum}}{n}\)\index{sum@$\OC{\fun{sum}}{n}$} le nombre de
comparaisons pour ajouter \(n\)~clés à \(\el\). Nous avons
\begin{gather}
\abovedisplayskip=0pt
\belowdisplayskip=0pt
\OC{\fun{sum}}{n} = \sum_{k=0}^{n-1}{\OC{\fun{add}}{k}}.\notag\\
\OC{\fun{sum}}{2p} = \OC{\fun{sum}}{2p+1}
= \sum_{k=1}^{2p-1}{\OC{\fun{add}}{k}}
= \sum_{j=1}^{p}{\OC{\fun{add}}{2j-1}}
= \sum_{j=1}^{p}\sum_{i=0}^{1+\rho_{j}}{\OC{\fun{mrg}}{2^i,2^i}}.
\label{eq:sum_2p}
\end{gather}
Soit \(\OC{\fun{unb}}{n}\)\index{unb@$\OC{\fun{unb}}{n}$} le nombre de
comparaisons de tous les interclassements déséquilibrés pour trier
\(n\)~clés. Un coup d'œil en arrière, à la
définition~\eqref{eq:C_unbal} page~\pageref{eq:C_unbal} nous rappelle
que
\begin{equation}
\OC{\fun{unb}}{n}
= \OC{\ltimes}{n}
= \sum_{i=1}^{r}{\OC{\fun{mrg}}{2^{e_i},2^{e_{i-1}}+\dots+2^{e_0}}}.
\index{unb@$\OC{\fun{unb}}{n}$}
\label{eq:cost_unb}
\end{equation}
Soit \(\OC{\fun{oms}}{n}\)\index{oms@$\OC{\fun{oms}}{n}$} le nombre de
comparaisons pour trier \(n\)~clés en ligne. On a
\begin{equation}
\OC{\fun{oms}}{n} = \OC{\fun{sum}}{n} + \OC{\fun{unb}}{n}.
\index{sum@$\OC{\fun{sum}}{n}$}
\label{eq:ocost_online}
\end{equation}

%\addcontentsline{toc}{subsection}{Cost}
\paragraph{Coût minimal}
\index{tri par interclassement!$\sim$ en ligne!coût minimal|(}

En remplaçant~\(\Cost\) par~\(\Best\) dans
l'équation~\eqref{eq:sum_2p}, nous obtenons les équations pour le
nombre minimal de comparaisons, ce qui nous permet de simplifier
\(\OB{\fun{sum}}{n}\)\index{sum@$\OB{\fun{sum}}{n}$} avec l'aide de
l'équation~\eqref{eq:best_merge} \vpageref{eq:best_merge}:
\begin{equation}
\OB{\fun{sum}}{2p}
  = \OB{\fun{sum}}{2p+1}
  = \sum_{j=1}^{p}\sum_{i=0}^{1+\rho_{j}}{\OB{\fun{mrg}}{2^i,2^i}}
  = \sum_{j=1}^{p}\sum_{i=0}^{1+\rho_{j}}{2^{i}}
  = 4\sum_{j=1}^{p}{2^{\rho_{j}}} - p.
\label{eq:B_oplus}
\end{equation}
Soit \(T_p := \sum_{j=1}^{p}{2^{\rho_{j}}}\). Les
récurrences~\eqref{eq:ruler} \vpageref{eq:ruler} sur la
fonction~\(\rho\) nous aident à trouver une récurrence pour~\(T_p\)
comme suit:
\begin{align*}
T_{2q} &= \sum_{k=0}^{\smash[t]{q-1}}{2^{\rho_{2k+1}}} +
\sum_{k=1}^{q}{2^{\rho_{2k}}} = q + 2 \cdot T_{q},\\[2mm]
T_{2q+1}
&= \sum_{j=1}^{\smash[t]{2q+1}}{2^{\rho_{j}}} = 1 + T_{2q} = (q + 1) +
2 \cdot T_{q}.
\end{align*}
De manière équivalente, \(T_{p} = 2 \cdot T_{\floor{p/2}} +
\ceiling{p/2} = 2 \cdot T_{\floor{p/2}} + p - \floor{p/2}\). Donc, en
déroulant quelques pas de la récurrence révèle rapidement l'équation:
\begin{equation*}
2 \cdot T_p = 2p + \sum_{j=1}^{\floor{\lg p}}
            {\left\lfloor{\frac{p}{2^j}}\right\rfloor 2^j},
\end{equation*}
en usant du Théorème~\vref{thm_floors}. Par définition, \(\{x\} := x -
\floor{x}\), d'où
\begin{equation*}
2 \cdot T_p = p\floor{\lg p} + 2p - \sum_{j=1}^{\floor{\lg
    p}}\left\lbrace\frac{p}{2^j}\right\rbrace 2^j.
\end{equation*}
Puisque \(0 \leqslant \{x\} < 1\), nous obtenons l'encadrement
\begin{equation*}
p\floor{\lg p} + 2p - 2^{\floor{\lg p}+1} + 2 < 2 \cdot T_p \leqslant
p\floor{\lg p} + 2p.
\end{equation*}
De même, avec \(x - 1 < \floor{x} \leqslant x\) et \(\floor{x} = x -
\{x\}\), nous dérivons:
\begin{align*}
p(\lg p - \{\lg p\}) + 2p - 2^{\lg p - \{\lg p\} +
  1} + 2 < 2 \cdot T_p &\leqslant p\lg p + 2p,\\
p\lg p + 2p + 2 - p \cdot \theta_L(\{\lg p\}) < 2 \cdot T_p
&\leqslant p\lg p + 2p,
\end{align*}
où \(\theta_L(x) := x + 2^{1 - x}\). Puisque \(\max_{0 \leqslant x <
  1}\theta_L(x) = \theta_L(0) = 2\), nous concluons:
\begin{equation*}
p\lg p + 2 < 2 \cdot T_p \leqslant p\lg p + 2p.
\end{equation*}
Le majorant est atteint si \(p=2^q\). L'application de cet encadrement
à la définition de~\(\OB{\fun{sum}}{2p}\) dans~\eqref{eq:B_oplus}
produit
\begin{equation}
2p\lg p - p + 4 < \OB{\fun{sum}}{2p} \leqslant 2p\lg p + 3p.
\label{ineq:B_oplus_2p}
\end{equation}
Par conséquent, \(\OB{\fun{sum}}{2p} = \OB{\fun{sum}}{2p+1} \sim 2p\lg
p\), donc \(\OB{\fun{sum}}{n} \sim n\lg
n\).\index{sum@$\OB{\fun{sum}}{n}$}

\bigskip

\noindent Grâce à~\eqref{eq:best_merge} \&~\eqref{eq:cost_unb} on a
\(\OB{\fun{unb}}{n} = \sum_{i=1}^{r}{\min\{2^{e_i},2^{e_{i-1}} + \dots
  + 2^{e_0}\}}\)\index{unb@$\OB{\fun{unb}}{n}$|(}. Commençons par
noter que
\begin{equation*}
  \sum_{j=0}^{i}{2^{e_j}} \leqslant \sum_{j=0}^{e_i}{2^j} = 2 \cdot
  2^{e_i} - 1.
\end{equation*}
Ceci équivaut au fait qu'un nombre binaire donné est toujours
inférieur ou égal au nombre avec le même nombre de bits tous à~\(1\),
par exemple, \((10110111)_2 \leqslant (11111111)_2\). Par définition
de~\(e_i\), on a \(e_{i-1} + 1 \leqslant e_i\), donc
\begin{equation*}
  \sum_{j=0}^{i-1}{2^{e_j}} \leqslant 2^{e_{i-1}+1} - 1 \leqslant 2^{e_i} - 1 < 2^{e_i}
\end{equation*}
et \(\min\{2^{e_i},2^{e_{i-1}} + \dots + 2^{e_0}\} = 2^{e_{i-1}} +
\dots + 2^{e_0}\). Nous avons alors
\begin{equation}
\OB{\fun{unb}}{n} = \sum_{i=1}^{r}\sum_{j=0}^{i-1}{2^{e_j}} < n.
\label{ineq:OBunb}
\end{equation}
Trivialement, \(0 < \OB{\fun{unb}}{n}\), donc~\eqref{eq:ocost_online}
implique \(\OB{\fun{oms}}{n} \sim n\lg n \sim 2
\cdot \OB{\fun{bms}}{n}\).  \index{tri par interclassement!$\sim$ en
  ligne!coût minimal|)} \index{unb@$\OB{\fun{unb}}{n}$|)}

\paragraph{Coût maximal}
\index{tri par interclassement!$\sim$ en ligne!coût maximal|(}

En remplaçant~\(\Cost\) par~\(\Worst\) dans
l'équation~\eqref{eq:sum_2p}, nous obtenons des équations pour le
nombre maximal de comparaisons, que nous pouvons simplifier avec
l'aide de l'équation~\eqref{eq:worst_merge} \vpageref{eq:worst_merge}
en
\begin{equation}
\OW{\fun{sum}}{2p}
  = \OW{\fun{sum}}{2p+1}
  = \sum_{j=1}^{p}\sum_{i=0}^{1+\rho_{j}}{\OW{\fun{mrg}}{2^i,2^i}}
%  = \sum_{j=1}^{p}\sum_{i=0}^{1+\rho_{j}}{(2^{i+1}-1)}\\
  = 8\sum_{j=1}^{p}{2^{\rho_j}} - \sum_{j=1}^{p}\rho_j - 4p.
\label{eq:OW_sum_2p}
\index{sum@$\OW{\fun{sum}}{n}$}
\end{equation}
\index{somme des bits|(}Nous pouvons obtenir une forme close de
\(\sum_{j=1}^{p}\rho_j\) si nous pensons à la propagation de la
retenue et au nombre de bits à~\(1\) lorsque nous ajoutons~\(1\) à un
nombre binaire (car \(j\)~décrit les entiers successifs). Ceci revient
à trouver une relation entre \(\rho_j\), \(\rho_{j+1}\), \(\nu_j\) et
\(\nu_{j+1}\).
\begin{itemize}

  \item Supposons d'abord que \(2n+1 = (\Xi 01^a)_2\), où \(\Xi\)~est
    une chaîne de bits arbitraire et \((1^a)_2\)~est une chaîne
    de~\(1\) répétés \(a\)~fois. Alors \(\nu_{2n+1} = \nu_{\Xi} + a\)
    et \(\rho_{2n+1} = 0\).

  \item Supposons \(2n+2 = (\Xi 10^a)_2\), donc \(\nu_{2n+2} =
    \nu_{\Xi} + 1\) et \(\rho_{2n+2} = a\). Nous pouvons maintenant
    mettre en relation \(\rho\)~et~\(\nu\) grâce à~\(a\):
    \begin{equation*}
      \rho_{2n+2} = \nu_{2n+1} - \nu_{\Xi} = \nu_{2n+1} - (\nu_{2n+2} - 1)
      = 1 + \nu_{2n+1} - \nu_{2n+2}.
    \end{equation*}
    Nous pouvons vérifier la répétition du même motif
    avec~\(\rho_{2n+1}\), avec simplement les définitions de
    \(\rho\)~et~\(\nu\): \(\rho_{2n+1} = 1 + \nu_{2n} - \nu_{2n+1}\).

\end{itemize}
Ceci termine la preuve, pour tout entier \(n>0\), que \(\rho_n = 1 +
\nu_{n-1} - \nu_{n}\). En sommant membre à membre, nous obtenons
\begin{equation*}
\sum_{j=1}^{p}{\rho_j} = p - \nu_p.
\end{equation*}
Il est intéressant de noter que nous avons déjà rencontré \(p -
\nu_p\) \index{somme des bits|)} à l'équation~\eqref{eq:ruler_nu},
\vpageref{eq:ruler_nu}. Nous pouvons maintenant
simplifier~\eqref{eq:OW_sum_2p} ainsi:
\begin{equation*}
\OW{\fun{sum}}{2p}
 = \OW{\fun{sum}}{2p+1}
 = 8\sum_{j=1}^{p}{2^{\rho_j}} - 5p - \nu_p
 = 2 \cdot \OB{\fun{sum}}{2p} - 3p - \nu_p.
\index{sum@$\OW{\fun{sum}}{n}$}
\end{equation*}
En réutilisant l'encadrement de~\(\OB{\fun{sum}}{2p}\) dans
l'inégalité~\eqref{ineq:B_oplus_2p} nous amène à \(\OW{\fun{sum}}{2p}
= \OW{\fun{sum}}{2p+1} \sim 4p\lg p\).

Les équations~\eqref{eq:worst_merge} et~\eqref{eq:cost_unb}, ainsi que
l'inéquation~\eqref{ineq:OBunb} impliquent
\begin{equation*}
\OW{\fun{unb}}{n} = \sum_{i=1}^{r}\sum_{j=0}^{i}{2^{e_j}} - \nu_n + 1
                  = \OB{\fun{unb}}{n} + n - \rho_n - \nu_n + 1
                  < 2n + 1.
\end{equation*}
Par conséquent, \(\OW{\fun{oms}}{n} \sim 2n\lg n \sim 2
\cdot \OW{\fun{bms}}{n}\).
\index{tri par interclassement!$\sim$ en ligne!coût maximal|)}

\paragraph{Coût supplémentaire}

Prenons maintenant en compte toutes les réécritures dans l'évaluation
d'un appel \(\fun{oms}(s)\). Soit
\(\C{\fun{oms}}{n}\)\index{oms@$\C{\fun{oms}}{n}$} ce nombre. Nous
connaissons déjà la contribution due aux comparaisons,
\(\OC{\fun{oms}}{n}\)\index{oms@$\OC{\fun{oms}}{n}$}, soit dans la
règle~\(\kappa\) ou~\(\lambda\), donc cherchons \(\C{\fun{oms}}{n} -
\OC{\fun{oms}}{n}\):
\begin{itemize}

  \item La règle~\(\phi\) est employée une fois.

  \item Les règles \(\chi\)~et~\(\psi\) constituent la sous-trace
    \(\psi^n\chi\), donc sont utilisées \(n+1\)~fois.

  \item Les règles \(\omega\), \(\gamma\) et~\(\delta\) sont employées
    \(F(n) = 2n - \nu_n\) fois, comme on peut le constater à
    l'équation~\eqref{eq:ruler_nu}. Nous devons aussi tenir compte des
    règles \(\theta\)~et~\(\iota\) requises pour les appels
    \(\fun{mrg}(s,u)\) dans la règle~\(\delta\). Chaque bit à~\(1\)
    dans la notation binaire des nombres de~\(1\) à~\(n-1\) déclenche
    un tel appel, c'est-à-dire \(\sum_{k=1}^{n-1}\nu_k\).

  \item Les règles \(\nu\)~et~\(\xi\) sont utilisées pour chaque bit
    dans la notation binaire de~\(n\) et la règle~\(\mu\) est employée
    une fois, pour un total de \(\floor{\lg n} + 2\) appels. Nous
    avons aussi besoin d'ajouter le nombre d'appels
    \(\fun{mrg}(t,u)\)\index{mrg@\fun{mrg/2}} à la règle~\(\xi\), qui
    témoignent de l'application des règles
    \(\theta\)~et~\(\iota\). Ceci est le nombre de bits à~\(1\)
    dans~\(n\), faisant donc~\(\nu_n\).

\end{itemize}
Au total, nous avons \(\C{\fun{oms}}{n} - \OC{\fun{oms}}{n} = 3n +
\floor{\lg n} + \sum_{k=1}^{n-1}\nu_k +
2\). L'équation~\eqref{eq:OBbms} \vpageref{eq:OBbms} implique
\(\C{\fun{oms}}{n} = \OC{\fun{oms}}{n} + 3n + \floor{\lg n} +
\OB{\fun{bms}}{n} + 2\). L'encadrement~\eqref{ineq:bounds_Btms}
\vpageref{ineq:bounds_Btms} implique \(\OB{\fun{bms}}{n} \sim
\tfrac{1}{2}n\lg n\), d'où \(\C{\fun{oms}}{n} \sim
\OC{\fun{oms}}{n}\).  \index{tri par interclassement|)} \index{tri par
  interclassement!$\sim$ en ligne|)}

\section*{Exercices}

\begin{enumerate}

  \item Prouvez \(\fun{mrg}(s,t) \equiv \fun{mrg}(t,s)\).

  \item Prouvez que \(\fun{mrg}(s,t)\)\index{mrg@\fun{mrg/2}} est une
    pile ordonnée si \(s\)~et~\(t\) sont ordonnés.

  \item Prouvez que toutes les clés de~\(s\) et~\(t\) sont dans
  \(\fun{mrg}(s,t)\).

  \item Prouvez la terminaison de \(\fun{bms/1}\), \(\fun{oms/1}\) et
    \(\fun{tms/1}\).\index{bms@\fun{bms/1}}\index{oms@\fun{oms/1}}
    \index{tms@\fun{tms/1}}

  \item Est-ce que \fun{bms/1} est stable? Qu'en est-il de \fun{tms/1}?

  \item Trouvez \(\C{\fun{tms}}{n} - \OC{\fun{tms}}{n}\). \emph{Aide:}
    pensez à l'équation~\eqref{eq:ruler_nu} \vpageref{eq:ruler_nu}.

  \item À la page~\pageref{eq:bms_merges}, nous avons trouvé que le
    nombre d'interclassements effectués par \(\fun{bms}(s)\)
    est~\(n-1\) si~\(n\) est le nombre de clés dans~\(s\). Montrez que
    \(\fun{tms}(s)\) effectue le même nombre
    d'interclassements. (\emph{Aide}: Considérez
    l'équation~\eqref{eq:cost_tms} \vpageref{eq:cost_tms}.)

  \item Trouvez un dénombrement d'éléments de la table à la
    \fig~\vref{fig:bits} montrant que
    \begin{equation*}
      \sum_{k=1}^{p-1}{2^{\rho_k}}
      = \sum_{i=0}^{\ceiling{\lg p}-1}%
      {\left\lceil\frac{p-2^i}{2^{i+1}}\right\rceil 2^i}.
    \end{equation*}

  \item Comparez le nombre de nœuds d'empilage\index{nœud
      d'empilage} créés par \fun{bms/1} et \fun{tms/1}.

\end{enumerate}
