\section{Découpage}
\label{sec:cutting}
\index{pile!découpage}

\index{cut@\fun{cut/2}|(} Étudions le problème du découpage d'une
pile~\(s\) à la position~\(k\). Évidemment, le résultat est une paire
de piles. Plus précisément, soit \(\pair{t}{u}\) la valeur de
\(\fun{cut}(s,k)\), telle que \(\fun{cat}(t,u) \twoheadrightarrow
s\)\index{cat@\fun{cat/2}} et \(t\)~contient \(k\)~éléments,
c'est-à-dire \(\fun{len}(t) \twoheadrightarrow k\). En particulier,
si~\(k = 0\), alors \(t = \el\); les données invalides mènent à des
résultats non-spécifiés. Par exemple, \(\fun{cut}([4,2],0)
\twoheadrightarrow \pair{\el}{[4,2]}\) et \(\fun{cut}([5,3,6,0,2],3)
\twoheadrightarrow \pair{[5,3,6]}{[0,2]}\), mais, pour gagner en
simplicité, on ne dira rien à propos de \(\fun{cut}([0],7)\) et
\(\fun{cut}([0],-1)\). Nous déduisons deux cas: \(k = 0\) ou bien la
pile n'est pas vide. Le premier est facile à deviner:
\begin{equation*}
\fun{cut}(s,0)           \rightarrow \pair{\el}{s};\qquad
\fun{cut}(\cons{x}{s},k) \rightarrow \fbcode{CCCCCC}\,.
\end{equation*}
Une conception à grand pas\index{conception!grands pas} emploie des
appels récursifs sur des sous-structures pour établir la structure de
la valeur dans le membre droit. Étant donné que \fun{cut/2} prend deux
arguments, nous prévoyons l'usage d'un ordre lexicographique\index{induction!ordre lexicographique}
(définition~\eqref{def:lexico}, page~\pageref{def:lexico}):
\begin{equation*}
\fun{cut}(s_0,k_0) \succ \fun{cut}(s_1,k_1)
:\Leftrightarrow
\text{\(s_0 \succ s_1\) ou (\(s_0 = s_1\) et \(k_0 > k_1\))}.
\end{equation*}
En définissant (\(\succ\)) comme étant l'ordre des sous-termes
propres\index{induction!ordre des sous-termes propres} sur les piles
(section~\ref{par:well-founded}, page~\pageref{par:well-founded}),
nous obtenons
\begin{equation*}
\fun{cut}(\cons{x}{s},k) \succ \fun{cut}(s,j);\quad
\fun{cut}(\cons{x}{s},k) \succ \fun{cut}(\cons{x}{s},j),\;
\text{si \(k > j\)}.
\end{equation*}
Dans le dernier cas, nous voulons poser \(j=k-1\), mais la valeur de
\(\fun{cut}(s,j)\) doit être projetée en \(\pair{t}{u}\) pour
que~\(x\) puisse être injecté et donne \(\pair{\cons{x}{t}}{u}\). Cela
peut être réalisé à l'aide d'une fonction
auxiliaire~\fun{push/2}\index{push@\fun{push/2}}:
\begin{equation*}
\begin{array}{@{}r@{\;}l@{\;}lr@{\;}l@{\;}l@{}}
\fun{cut}(s,0) & \rightarrow & \pair{\el}{s};
& \fun{push}(x,\pair{t}{u}) & \rightarrow & \pair{\cons{x}{t}}{u}.\\
\fun{cut}(\cons{x}{s},k) & \rightarrow
& \fun{push}(x,\fun{cut}(s,k-1)).
\end{array}
\end{equation*}


\paragraph{Systèmes d'inférence}
\label{par:infsys}
\index{système d'inférence}

Quand la valeur d'un appel récursif a besoin d'être déstructurée, il
est commode d'utiliser une extension de notre langage pour éviter de
créer des fonctions auxiliaires comme \fun{push/2}:
\begin{mathpar}
\inferrule*{}{\fun{cut}(s,0) \rightarrow \pair{\el}{s}}
\;\TirName{Nil}
\qquad
\inferrule
  {\fun{cut}(s,k-1)         \twoheadrightarrow \pair{t}{u}}
  {\fun{cut}(\cons{x}{s},k) \twoheadrightarrow \pair{\cons{x}{t}}{u}}
\,\TirName{Pref}
\end{mathpar}
La nouvelle construction s'appelle une \emph{règle
  d'inférence}\index{système d'inférence!règle} parce qu'elle
signifie: «~Pour que la valeur de \(\fun{cut}(\cons{x}{s},k)\) soit
\(\pair{\cons{x}{t}}{u}\), nous inférons que la valeur de
\(\fun{cut}(s,k-1)\) doit être \(\pair{t}{u}\).~» Cette interprétation
correspond à une lecture ascendante de la règle \TirName{Pref}
(\emph{préfixe}). Tout comme nous pouvons composer horizontalement les
règles de réécriture, nous composons les règles d'inférence
verticalement, en les empilant:
\begin{mathpar}
\inferrule
  {\inferrule{
     \inferrule{\fun{cut}([0,2],0) \rightarrow \pair{\el}{[0,2]}}
               {\fun{cut}([6,0,2],1) \twoheadrightarrow \pair{[6]}{[0,2]}}}
      {\fun{cut}([3,6,0,2],2) \twoheadrightarrow \pair{[3,6]}{[0,2]}}}
  {\fun{cut}([5,3,6,0,2],3) \twoheadrightarrow \pair{[5,3,6]}{[0,2]}}
\end{mathpar}
Pour déterminer le coût de \(\fun{cut}(s,k)\), nous comptons~\(1\)
pour chaque occurrence de \((\twoheadrightarrow)\) et nous prenons en
compte la fonction auxiliaire cachée \fun{push/2}, donc
\(\Call{\fun{cut}([5,3,6,0,2],3)} = 7\). En général,
\index{cut@$\C{\fun{cut}}{n}$} \(\C{\fun{cut}}{k} = 2k +
1\). Remarquons que les systèmes d'inférence définissent
\((\twoheadrightarrow)\) au lieu de \((\rightarrow)\).

Au-delà de la simplification des programmes, ce qui rend ce formalisme
intéressant est qu'il rend possible deux sortes d'interprétations:
l'une, logique, et, l'autre, algorithmique. Le lecture algorithmique,
appelée \emph{inductive}\index{système d'inférence!lecture inductive}
dans certains contextes, a été illustrée tantôt. La lecture logique
voit les règles d'inférences comme des implications logiques de la
forme \(P_1 \wedge P_2 \wedge \ldots \wedge P_n \Rightarrow C\)
écrites
\begin{equation*}
\inferrule
  {P_1 \\ P_2 \\ \ldots \\ P_n}
  {C}
\end{equation*}
Les propositions~\(P_i\) sont les \emph{prémisses}\index{système
  d'inférence!règle!prémisse} et \(C\)~est la \emph{conclusion}. Dans
le cas de \TirName{Pref}, il n'y a qu'une seule prémisse. S'il n'y a
aucune prémisse, comme dans le cas de \TirName{Nil}, alors \(C\)~est
un \emph{axiome} et aucune ligne horizontale n'est tirée. La
composition de règles d'inférence est une \emph{dérivation}. Dans le
cas de \fun{cut/2}, toutes les dérivations sont isomorphes à une pile
dont le sommet est la conclusion.

La lecture logique de la règle \TirName{Pref} est:
\begin{center}
  «~Si
\(\fun{cut}(s,k-1) \twoheadrightarrow \pair{t}{u}\), alors
\(\fun{cut}(\cons{x}{s},k) \twoheadrightarrow
\pair{\cons{x}{t}}{u}\).~»
\end{center}
Une telle lecture descendante est dite \emph{déductive}\index{système
  d'inférence!lecture déductive}. On peut alors comprendre la
dérivation précédente comme étant la preuve du théorème
\(\fun{cut}([5,3,6,0,2],3) \twoheadrightarrow \pair{[5,3,6]}{[0,2]}\).

\paragraph{Induction sur les dérivations}
\index{induction!preuve par $\sim$}

La double herméneutique des règles d'inférence rend possible à la fois
la spécification de programmes et la preuve de théorèmes à leurs
propos par \emph{induction sur la structure des dérivations}. Comme
nous l'avons vu plus haut, l'induction structurelle peut être
appliquée avec profit aux piles, considérées comme des types de
données (objets). Puisque dans le cas de \fun{cut/2} les dérivations
sont elles-mêmes des piles, nous pouvons aussi appliquer à leur
structure (en tant que méta-objets) le même principe
d'induction. Illustrons cette élégante technique inductive avec la
preuve de la \emph{correction}\index{correction} de
\fun{cut/2}\index{cut@\fun{cut/2}}.

\mypar{Correction}
\label{par:cut_sound}

Le concept de correction~\citep{McCarthy_1962, Floyd_1967, Hoare_1971,
  Dijkstra_1976} est une relation, donc nous devons toujours parler de
la correction d'un programme par rapport à sa
\emph{spécification}\index{spécification}. Une spécification est une
description logique des propriétés attendues du résultat de
l'exécution d'un programme, étant données des propriétés de son
entrée. Dans le cas de \(\fun{cut}(s,k)\), nous avons déjà mentionné
ce que nous attendions: la valeur doit être une paire \(\pair{t}{u}\)
telle que la concaténation de \(t\)~et~\(u\) est~\(s\) et la longueur
de~\(t\) est~\(k\).

Formellement, soit
\(\pred{CorCut}{s,k}\)\index{CorCut@\predName{CorCut}|(} la
proposition
\begin{quote}
  \textsl{Si \(\fun{cut}(s,k) \twoheadrightarrow \pair{t}{u}\), alors
    \(\fun{cat}(t,u) \twoheadrightarrow s\) et \(\fun{len}(t)
    \twoheadrightarrow k\),}
\end{quote}
où la fonction \fun{len/1}\index{len@\fun{len/1}|(} est définie à
l'équation~\eqref{eq:len} \vpageref{eq:len}.

Supposons la véracité de l'antécédent de l'implication, sinon le
théorème est trivialement vrai (vacuité), donc il existe une
dérivation~\(\Delta\) dont la conclusion est \(\fun{cut}(s,k)
\twoheadrightarrow \pair{t}{u}\). Cette dérivation est une (méta) pile
dont le sommet est la conclusion en question, ce qui rend possible le
raisonnement par induction sur sa structure, c'est-à-dire que nous
supposons que \predName{CorCut} est vraie pour la sous-dérivation
immédiate de~\(\Delta\) (l'hypothèse d'induction) et nous procédons
ensuite en montrant que \predName{CorCut} est vraie pour la dérivation
entière. Ceci n'est autre que l'induction sur les sous-termes
immédiats que nous avons utilisé sur des piles considérées comme des
objets du discours: nous supposons que le théorème est vrai pour~\(s\)
et nous procédons ensuite en prouvant qu'il est vrai
pour~\(\cons{x}{s}\).

La preuve est guidée par une analyse par cas qui discrimine sur la
sorte de règle pouvant terminer~\(\Delta\). Pour éviter des collisions
entre des variables du théorème et du système d'inférence, nous
surlignons ces dernières, ainsi \(\overline{s}\), \(\overline{t}\)
etc. qui sont alors différentes \(s\)~et~\(t\) dans \predName{CorCut}.
\begin{itemize}

\item \emph{Cas où \(\Delta\) se termine par \TirName{Nil}.} Il n'y a
  pas de prémisses, car~\TirName{Nil} est un axiome. Dans ce cas, nous
  devons établir \predName{CorCut} sans induction. Le filtrage de
  \(\fun{cut}(s,k) \twoheadrightarrow \pair{t}{u}\) par
  \(\fun{cut}(\overline{s},0) \rightarrow \pair{\el}{\overline{s}}\)
  résulte en \(\overline{s} = s\), \(0=k\), \(\el=t\) et
  \(\overline{s}=u\). D'où, \index{cat@\fun{cat/2}} \(\fun{cat}(t,u) =
  \fun{cat}(\el,s) \xrightarrow{\smash{\alpha}} s\), ce qui prouve la
  moitié de la conjonction. Aussi, \(\fun{len}(t) = \fun{len}(\el)
  \xrightarrow{\smash{a}} 0\). Ceci est cohérent avec \(k=0\), donc
  \(\pred{CorCut}{s,0}\) est établie.\index{CorCut@\predName{CorCut}|)}

  \item \emph{Cas où \(\Delta\) se termine avec par \TirName{Pref}.}
    La forme de~\(\Delta\) est donc
    \begin{mathpar}
      \inferrule*[right=Pref]
        {\inferrule*
           {\inferrule*[vdots=1.5em]{}{ }}
           {\fun{cut}(\overline{s},\overline{k}-1)
            \twoheadrightarrow \pair{\overline{t}}{\overline{u}}}
        }
        {\fun{cut}(\cons{\overline{x}}{\overline{s}},\overline{k})
         \twoheadrightarrow
         \pair{\cons{\overline{x}}{\overline{t}}}{\overline{u}}}
    \end{mathpar}
    Le filtrage de \(\fun{cut}(s,k) \twoheadrightarrow \pair{t}{u}\)
    par la conclusion mène à \(\cons{\overline{x}}{\overline{s}} =
    s\), \(\overline{k} = k\), \(\cons{\overline{x}}{\overline{t}} =
    t\) et \(\overline{u} = u\). L'hypothèse d'induction est alors que
    le théorème est vrai pour la sous-dérivation:
    \index{cat@\fun{cat/2}|(} \(\fun{cat}(\overline{t}, \overline{u})
    \twoheadrightarrow \overline{s}\) et aussi
    \(\fun{len}(\overline{t}) \twoheadrightarrow \overline{k} -
    1\).\index{len@\fun{len/1}|)} Le principe d'induction requiert que
    nous établissions alors
    \(\fun{cat}(\cons{\overline{x}}{\overline{t}}, \overline{u})
    \twoheadrightarrow \cons{\overline{x}}{\overline{s}}\) et
    \(\fun{len}(\cons{\overline{x}}{\overline{t}}) \twoheadrightarrow
    \overline{k}\). D'après la définition de \fun{cat/2} et une partie
    de l'hypothèse, nous déduisons aisément
    \(\fun{cat}(\cons{\overline{x}}{\overline{t}}, \overline{u})
    \xrightarrow{\smash{\beta}}
    \cons{\overline{x}}{\fun{cat}(\overline{t}, \overline{u})}
    \twoheadrightarrow
    \cons{\overline{x}}{\overline{s}}\). Maintenant, l'autre partie:
    \(\fun{len}(\cons{\overline{x}}{\overline{t}})
    \xrightarrow{\smash{b}} 1 + \fun{len}(\overline{t})
    \twoheadrightarrow 1 + (\overline{k} - 1) =
    \overline{k}\).\index{cat@\fun{cat/2}|)}\hfill\(\Box\)

\end{itemize}

\paragraph{Exercice}

Écrivez une définition équivalente à \fun{cut/2} qui soit en forme
terminale.\index{langage fonctionnel!forme terminale}
\index{cut@\fun{cut/2}|)}
