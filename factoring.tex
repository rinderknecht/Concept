\chapter{Recherche de motifs}
\label{chap:factoring}
\index{recherche de motifs|(}

Nous appellerons \emph{alphabet}\index{recherche de motifs!alphabet}
un ensemble fini non-vide de symboles, appelés
\emph{lettres}\index{recherche de motifs!lettre}, que nous composerons
avec une police linéale, par exemple \word{a}, \word{b} etc. Un
\emph{mot}\index{recherche de motifs!mot} est une suite finie de
lettres, comme \word{mot}; en particulier, une lettre est un mot,
comme en français. Nous dénotons la répétition d'une lettre ou d'un
mot avec un exposant, par exemple, \(\word{a}^3 = \word{aaa}\). Les
mots, tout comme les lettres, peuvent être joints pour former des
mots: le mot \(u \cdot v\) est constitué des lettres du mot~\(u\)
suivies des lettres du mot~\(v\); par exemple, si \(u=\word{or}\) et
\(v=\word{ange}\), alors \(u \cdot v = \word{orange}\). Cette
opération est associative: \((u \cdot v) \cdot w = u \cdot (v \cdot
w)\). Pour abréger, l'opérateur peut être omis: \((uv)w = u(vw)\). La
concaténation de mots se comporte comme un produit non-commutatif,
donc elle a un élément neutre \(\varepsilon\), appelé le \emph{mot
  vide}: \(u \cdot \varepsilon = \varepsilon \cdot u = u\).

Un mot~\(x\) est un \emph{facteur}\index{recherche de motifs!facteur}
d'un mot~\(y\) s'il existe deux mots \(u\)~et~\(v\) tels que \(y =
uxv\). Le mot~\(x\) est un \emph{préfixe}\index{recherche de
  motifs!préfixe} de~\(y\), noté \(x \prefeq y\),
si~\(u=\varepsilon\), c'est-à-dire si \(y = xv\). De plus, si \(v \neq
\varepsilon\), c'est un \emph{préfixe propre}, noté \(x \pref
y\). Étant donné \(y = uxv\), le mot~\(x\) est un
\emph{suffixe}\index{recherche de motifs!suffixe} de~\(y\) si \(v =
\varepsilon\). Si, en plus, \(u \neq \varepsilon\), il est un
\emph{suffixe propre}. Soit~\(a\) une lettre quelconque et \(x\),
\(y\)~des mots quelconques; alors la relation de préfixe est facile à
définir par un système d'inférence comme suit:
\begin{mathpar}
\inferrule*{}{\varepsilon \prefeq y}
\qquad
\inferrule{x \prefeq y}{a \cdot x \prefeq a \cdot y}
\end{mathpar}

Le but étant d'écrire un programme pour chercher un facteur dans un
texte, nous avons besoin de traduire les mots et les opérations en
termes du langage fonctionnel. Une lettre est traduite en un
constructeur constant; par exemple, \word{a} devient \(\fun{a}()\). Un
mot de plus d'une lettre est traduit en une pile de lettres traduites,
comme \word{je} dans \([\fun{j}(),\fun{e}()]\). La concaténation d'une
lettre et d'un mot est traduite par un empilage, ainsi \(\word{a}
\cdot \word{mie}\) devient \([\fun{a}(), \fun{m}(), \fun{i}(),
\fun{e}()]\). La concaténation de deux mots est traduite par la
concaténation de piles, donc \(\word{ab} \cdot \word{cd}\) mène
à\index{cat@\fun{cat/2}}
\(\fun{cat}([\fun{a}(),\fun{b}()],[\fun{c}(),\fun{d}()])\).

% The transposition of a word is the reversal of the stack implementing
% the word.

Comme toujours, la traduction du système d'inférence définissant
\((\prefeqName)\) en une fonction \fun{pre/2}\index{pre@\fun{pre/2}}
requiert que les cas correspondant aux axiomes s'évaluent en
\(\fun{true}()\) et que les cas qui ne sont pas spécifiés
\((\nprefeqName)\) s'évaluent en \(\fun{false}()\):
\begin{equation*}
\fun{pre}(\el,y) \rightarrow \fun{true}();\quad
\fun{pre}(\cons{a}{x},\cons{a}{y}) \rightarrow \fun{pre}(x,y);\quad
\fun{pre}(x,y) \rightarrow \fun{false}().
\end{equation*}
Le système d'inférence est alors une spécification formelle du
programme.

Une lettre d'un mot peut être caractérisée de façon unique par un
entier naturel appelé \emph{index}\index{recherche de motifs!index},
en supposant que la première lettre a pour index~\(0\)
\citep{Dijkstra_1982}. Si \(x=\word{pot}\), alors la lettre à
l'index~\(0\) est \(\ind{x}{0}=\word{p}\) et celle à l'index~\(2\) est
\(\ind{x}{2}=\word{t}\). Un facteur~\(x\) de~\(y\) peut être identifié
par l'index de~\(\ind{x}{0}\) dans~\(y\). La fin du facteur peut aussi
être isolée; par exemple, \(x=\word{fin}\) est un facteur de
\(y=\word{affiner}\) à l'index~\(2\), noté \(\ind{y}{2,4} = x\) et
signifiant \(\ind{y}{2} = \ind{x}{0}\), \(\ind{y}{3} = \ind{x}{1}\) et
\(\ind{y}{4} = \ind{x}{2}\).

La recherche de motifs est courante dans l'édition de textes, bien
qu'elle soit mieux connue en tant que \emph{recherche d'occurrences},
un sujet d'études en \emph{algorithmique du texte}, appelé parfois en
anglais \emph{stringology}
\citep{CharrasLecroq_2004,CrochemoreHancartLecroq_2007}
\citep[\S{}32]{CLRS_2009}. Dû à la nature asymétrique de la recherche,
le mot~\(p\) est appelé le \emph{motif}\index{recherche de
  motifs!motif} (à ne pas confondre avec le motif d'une règle de
réécriture) et le mot~\(t\) est le \emph{texte}\index{recherche de
  motifs!texte}.

\section{Recherche na\"{\i}ve}
\label{sec:naive_factoring}
\index{recherche de motifs!$\sim$ na\"{\i}ve|(}

À la section~\ref{def:linear_search}, \vpageref{def:linear_search},
nous avons présenté la recherche linéaire\index{recherche linéaire},
c'est-à-dire la recherche pas à pas de l'occurrence d'un élément dans
une pile. Nous pouvons la généraliser à la recherche d'une suite
d'éléments contigus dans une pile, c'est-à-dire qu'elle résout alors
le problème de la recherche d'un mot. Cette approche est qualifiée de
na\"{\i}ve parce qu'elle est une extension simple d'une idée simple,
et il est sous-entendu qu'elle n'est pas la plus efficace. Tout
commence avec la comparaison de~\(\ind{p}{0}\) et~\(\ind{t}{0}\),
puis, en supposant que \(\ind{p}{0} = \ind{t}{0}\), les lettres
\(\ind{p}{1}\) et~\(\ind{t}{1}\) sont à leur tour comparées etc.
jusqu'à ce qu'un des mots soit vide ou une inégalité se produise. En
supposant que \(p\)~est plus court que~\(t\), le premier cas signifie
que \(p\)~est un préfixe de~\(t\). Dans le dernier cas, \(p\)~est
décalé de telle sorte que \(\ind{p}{0}\) soit aligné avec
\(\ind{t}{1}\) et les comparaisons reprennent à partir de ce
point. Si~\(p\) ne peut être décalé davantage parce que son extrémité
dépasserait la fin de~\(t\), alors il n'est pas un facteur. L'essence
de cette procédure est résumée à la \fig~\vref{fig:naive},
\begin{figure}[t]
\centering
\includegraphics[bb=75 621 352 715]{naive}
\caption{Recherche na\"{\i}ve du motif~\(p\) dans le texte~\(t\)
(échec grisé)\label{fig:naive}}
\end{figure}
où \(\ind{p}{i} \neq \ind{t}{j}\) (les lettres \word{a}~et~\word{b} ne
sont pas significatives en elles-mêmes).

La \fig~\vref{fig:loc0}
\begin{figure}[t]
\centering
\includegraphics[bb=71 548 253 721]{loc0}
\caption{Recherche na\"{\i}ve avec \fun{loc\(_0\)/2}
\label{fig:loc0}}
\end{figure}
montre un programme fonctionnel réalisant ce plan. L'appel
\(\fun{loc}_0(p,t)\)\index{loc0@\fun{loc\(_0\)/2}} est évalué en
\(\fun{absent}()\) si le motif~\(p\) n'est pas un facteur du
texte~\(t\), sinon en \(\fun{factor}(k)\), où \(k\)~est l'index
dans~\(t\) où~\(p\) apparaît en premier. Du point de vue conceptuel,
ce dessein consiste à combiner une recherche linéaire de la première
lettre du motif et un test de préfixe pour le reste du motif et du
texte. Il est important de vérifier si les invariants implicites en
général ne sont pas brisés en présence de cas aux limites. Par
exemple, dans le traitement de piles, utilisons des piles vides
partout où cela est possible et interprétons chaque réécriture et
l'évaluation complète. Nous avons ainsi \(\fun{pre}(\el,t)
\twoheadrightarrow \fun{true}()\), parce que \(t = \varepsilon \cdot
t\). Mais aussi \(\fun{loc}_0(\el,t) \twoheadrightarrow \fun{factor}(0)\).

\paragraph{Raffinements}

Bien que la composition de ce programme soit intuitive, elle est trop
longue. Nous pourrions remarquer qu'après un appel à
\fun{pre/2}\index{pre@\fun{pre/2}} s'évalue en \(\fun{true}()\),
l'évaluation s'achève avec \(\fun{factor}(j)\). De même, une valeur
\(\fun{false}()\) est suivie par l'appel
\(\fun{loc}_0(p,t,j+1)\)\index{loc0@\fun{loc\(_0\)/3}}. Par
conséquent, au lieu d'appeler \fun{pre/2} et puis inspecter la valeur
résultante pour déterminer la suite, nous pourrions donner
l'initiative à \fun{pre/2}. Ceci implique qu'elle a besoin de recevoir
des arguments supplémentaires pour pouvoir terminer avec
\(\fun{factor}(j)\) ou recommencer avec \(\fun{loc}_0(p,t,j+1)\),
comme on s'y attend. Le programme correspondant est montré à la
\fig~\ref{fig:loc1}.
\begin{figure}[t]
\begin{equation*}
\boxed{%
\begin{array}{r@{\;}l@{\;}l}
\fun{loc}_1(p,t)   & \rightarrow & \fun{loc}_1(p,t,0).\\
\\
\fun{loc}_1(\cons{a}{p},\el,j) & \rightarrow & \fun{absent}();\\
\fun{loc}_1(p,t,j) & \rightarrow & \fun{pre}_1(p,t,p,t,j).\\
\\
\fun{pre}_1(\el,t,p',t',j) & \rightarrow & \fun{factor}(j);\\
\fun{pre}_1(\cons{a}{p},\cons{a}{t},p',t',j)
                   & \rightarrow & \fun{pre}_1(p,t,p',t',j);\\
\fun{pre}_1(p,t,p',\cons{a}{t'},j) & \rightarrow & \fun{loc}_1(p',t',j+1).
\end{array}}
\end{equation*}
\caption{Raffinement de la \fig~\vref{fig:loc0}
\label{fig:loc1}}
\end{figure}

Un examen plus approfondi révèle que nous pouvons mêler
\fun{loc\(_1\)/3}\index{loc1@\fun{loc\(_1\)/3}} et
\fun{pre\(_1\)/5}\index{pre1@\fun{pre\(_1\)/5}} en
\fun{pre/5}\index{pre@\fun{pre/5}} à\index{recherche de motifs!$\sim$
  na\"{\i}ve!programme} la \fig~\vref{fig:loc}. Cette sorte de
conception progressive, où un programme est transformé en une série de
programmes équivalents est un \emph{raffinement}. Ici, chaque
raffinement est plus efficace que le précédent, mais moins lisible que
l'original, donc chaque étape doit être vérifiée attentivement.
\begin{figure}[h]
\begin{equation*}
\boxed{%
\begin{array}{r@{\;}l@{\;}l}
\fun{loc}(p,t)   & \xrightarrow{\smash{\pi}} & \fun{pre}(p,t,p,t,0).\\
\\
\fun{pre}(\el,t,p',t',j) & \xrightarrow{\smash{\rho}}
                         & \fun{factor}(j);\\
\fun{pre}(p,\el,p',t',j) & \xrightarrow{\smash{\sigma}}
                         & \fun{absent}();\\
\fun{pre}(\cons{a}{p},\cons{a}{t},p',t',j)
                         & \xrightarrow{\smash{\tau}}
                         & \fun{pre}(p,t,p',t',j);\\
\fun{pre}(p,t,p',\cons{b}{t'},j) & \xrightarrow{\smash{\upsilon}}
                         & \fun{pre}(p',t',p',t',j+1).
\end{array}}
\end{equation*}
\caption{Raffinement de la \fig~\vref{fig:loc1}
\label{fig:loc}}
\end{figure}

\paragraph{Terminaison}
\index{recherche de motifs!$\sim$ na\"{\i}ve!terminaison|(}

On veut montrer que l'index dans le texte augmente toujours, qu'une
comparaison échoue ou non, donc nous définissons un ordre
lexicographique sur les paires de dépendance de \fun{pre/5}
\index{terminaison!paire de dépendance} constituées par le quatrième
et le deuxième argument (définition~\eqref{def:lexico}
\vpageref{def:lexico}), avec \(s \succ t\) si~\(t\) est la sous-pile
immédiate de~\(s\). La troisième règle satisfait \((t',\cons{a}{t})
\succ (t',t)\). La quatrième est ordonnée aussi: \((\cons{b}{t'},t)
\succ (t',t')\).\index{recherche de motifs!$\sim$
  na\"{\i}ve!terminaison|)}\hfill\(\Box\)

\paragraph{Complétude}
\index{recherche de motifs!$\sim$ na\"{\i}ve!complétude|(}
\index{induction!exemple|(}

Remarquons comment, à la règle~\(\sigma\), le motif~\(p\) ne peut pas
être vide car les règles sont ordonnées et ce cas serait alors filtré
par la règle~\(\rho\). La complétude de la définition de \fun{pre/5}
requiert un examen attentif et nous devons justifier pourquoi l'appel
\(\fun{pre}(\cons{a}{p},\cons{b}{t},p',\el,j)\), où \(a \neq b\), ne
peut se produire.

Il est peut-être surprenant qu'une assertion plus générale soit plus
facile à établir:
\begin{center}
\(\fun{loc}(p,t) \twoheadrightarrow \fun{pre}(p_0,t_0,p_0',t_0',j)\)
\textsl{implique} \(t_0' \succcurlyeq t_0\),
\end{center}
où (\(\succcurlyeq\)) est la relation de sous-pile réflexive.
Prouvons cette propriété par \emph{induction sur la longueur de la
  dérivation}.\index{induction!$\sim$ sur la longueur de la
  dérivation} Plus précisément, nous voulons établir la proposition
\begin{equation*}
\pred{Comp}{n} \colon \fun{loc}(p,t) \xrightarrow{\smash{n}}
\fun{pre}(p_0,t_0,p_0',t_0',j) \Rightarrow t_0' \succcurlyeq t_0.
\index{Comp@\predName{Comp}}
\end{equation*}
\begin{itemize}

  \item La base \(\pred{Comp}{0}\) est aisément prouvée sans induction
  grâce à la règle~\(\pi\): \(\fun{loc}(p,t) \xrightarrow{\smash{\pi}}
  \fun{pre}(p,t,p,t,0)\) et \(t \succcurlyeq t\) est triviale.

  \item L'hypothèse d'induction est \(\pred{Comp}{n}\) et nous voulons
  montrer que, avec cette hypothèse, \(\pred{Comp}{n+1}\) est vraie
  aussi. Autrement dit, supposons que \(\fun{loc}(p,t)
  \xrightarrow{\smash{n}} \fun{pre}(p_0,t_0,p_0',t_0',j)\) implique
  \(t_0' \succcurlyeq t_0\) et nous voulons prouver que
  \(\fun{pre}(p_0,t_0,p_0',t_0',j) \rightarrow
  \fun{pre}(p_1,t_1,p_1',t_1',k)\) implique \(t_1' \succcurlyeq
  t_1\). Cette réécriture ne peut être que via~\(\tau\)
  ou~\(\upsilon\).
  \begin{itemize}

    \item Si~\(\tau\), l'hypothèse d'induction appliquée au membre
    gauche entraîne \(t' \succcurlyeq \cons{a}{t}\), donc \(t'
    \succcurlyeq t\) dans le membre droit;

    \item sinon, le membre droit de~\(\upsilon\) satisfait clairement
    \(t' \succcurlyeq t'\).

      \end{itemize}
\end{itemize}
En somme, \(\pred{Comp}{0}\) est vraie et \(\pred{Comp}{n} \Rightarrow
\pred{Comp}{n+1}\). Par conséquent, le principe d'induction implique
\(\forall n.\pred{Comp}{n}\), qui, à son tour, entraîne notre
formulation avec~(\(\twoheadrightarrow\)). Notons comment, dans ce
cas, cette technique de preuve se réduit à une induction sur les
entiers naturels.\hfill\(\Box\) \index{induction!exemple|)}
\index{recherche de motifs!$\sim$ na\"{\i}ve!complétude|)}

\addcontentsline{toc}{subsection}{Coût}
\paragraph{Coût}

Dans l'analyse de coût qui suit, soit~\(m\) la taille du motif~\(p\)
et~\(n\) la longueur du texte~\(t\). De plus, comme c'est souvent le
cas avec les algorithmes de recherche, nous discriminerons selon que
\(p\)~est un facteur de~\(t\) ou non.

\paragraph{Coût minimum}
\index{recherche de motifs!$\sim$ na\"{\i}ve!coût minimum}

Si \(m \leqslant n\), le meilleur cas se produit quand le motif est un
préfixe du texte, donc la trace d'évaluation est \(\pi\tau^m\rho\) et
sa longueur \(\B{\fun{loc}}{m,n} = m + 2\). Si \(m > n \), le coût
minimum est \(\B{\fun{loc}}{m,n} = \len{\pi\tau^n\sigma} = n +
2\). Nous pouvons regrouper ces deux cas en une seule formule:
\begin{equation*}
\B{\fun{loc}}{m,n} = \min\{m,n\} + 2.\index{loc@$\B{\fun{loc}}{m,n}$}
\end{equation*}

\paragraph{Coût maximum}
\index{recherche de motifs!$\sim$ na\"{\i}ve!coût maximum|(}

Pour trouver le coût maximum, étudions les cas où le motif est un
facteur du texte et quand il ne l'est pas.
\begin{itemize}

\item \emph{Le texte contient le motif.} La découverte du motif doit
  être retardée le plus possible, donc le pire des cas est quand
  \(w\)~est un suffixe de~\(t\) et chaque inégalité de lettre porte
  sur la dernière lettre du motif. Par exemple, prenons \(p =
  \word{a}^{m-1}\word{b}\) et \(t = \word{a}^{n-1}\word{b}\). La trace
  d'évaluation est \(\pi (\tau^{m-1}\upsilon)^{n-m} \tau^m\rho\), de
  longueur \(mn-m^2+m+2\).

  \item \emph{Le texte ne contient pas le motif.} Le motif n'est le
  préfixe d'aucun suffixe du texte. La comparaison la plus tardive qui
  échoue se produit sur la dernière lettre du motif, comme avec \(p =
  \word{a}^{m-1}\word{b}\) et \(t = \word{a}^{n}\). Le coût est
  \(\len{\pi(\tau^{m-1}\upsilon)^{n-m+1}\tau^{m-1}\sigma} = mn - m^2 +
  2m + 1\).

\end{itemize}
Par conséquent, le coût maximum est \(\W{\fun{loc}}{m,n} = mn - m^2 +
2m + 1\) \index{loc@$\W{\fun{loc}}{m,n}$}, quand le motif n'est pas un
facteur du texte et \(m \geqslant 1\). L'analyse précédente suggère
une amélioration dans ce cas, mais qui empirerait le cas où le texte
contient le motif: juste après la règle~\(\tau\), ajoutons
\index{recherche de motifs!$\sim$ na\"{\i}ve!coût maximum|)}
\begin{equation*}
\belowdisplayskip=0pt
\fun{pre}([a],[b],p',t',j) \rightarrow \fun{absent}();\index{pre@\fun{pre/5}}
\end{equation*}

\paragraph{Coût moyen}
\index{recherche de motifs!$\sim$ na\"{\i}ve!coût moyen|(}

Supposons que \(0 < m \leqslant n\) et que les lettres de
\(p\)~et~\(t\) sont puisées dans un même alphabet de cardinal
\(\breve{a} > 1\). La recherche de motifs na\"{\i}ve consiste à
confronter un motif et les préfixes des suffixes du texte, par
longueurs décroissantes. Soit~\(\OM{\breve{a}}{m}\)
\index{loc@$\OM{\breve{a}}{n}$} le nombre moyen de comparaisons de
lettres pour comparer deux mots de longueur~\(m\), de
l'alphabet~\(\breve{a}\). Le nombre moyen \(\OM{\fun{loc}}{m,n}\) de
comparaisons pour la recherche na\"{\i}ve est
\begin{equation}
\OM{\fun{loc}}{m,n} = (n-m+1)\OM{\breve{a}}{m} + \OM{\breve{a}}{m-1},
\label{eq:loc}
\end{equation}
car il y a \(n-m+1\) suffixes de longueur au moins~\(m\) et
\(1\)~suffixe de longueur~\(m-1\), avec lesquels le motif est comparé.
La détermination de \(\OM{\breve{a}}{m}\) est obtenue en figeant le
motif~\(p\) et en laissant le texte~\(t\) prendre toutes les formes
possibles. Il y a \(\breve{a}^{m}\)~comparaisons entre \(\ind{p}{0}\)
et~\(\ind{t}{0}\), autant qu'il y a de textes différents; si
\(\ind{p}{0}=\ind{t}{0}\), il y a \(\breve{a}^{m-1}\)~comparaisons
entre \(\ind{p}{1}\) et~\(\ind{t}{1}\), autant qu'il a de différents
\(\ind{t}{1,m-1}\) etc. Au total, il y a \(\breve{a}^{m} +
\breve{a}^{m-1} + \dots + \breve{a} = \breve{a}(\breve{a}^{m} -
1)/(\breve{a}-1)\) comparaisons. Il y a \(\breve{a}^m\)~textes
possibles, donc la moyenne est
\begin{equation*}
\OM{\breve{a}}{m}
 = \frac{\breve{a}(\breve{a}^{m}-1)}{\breve{a}^m(\breve{a}-1)}
 = \frac{\breve{a}}{\breve{a}-1}\left(1
                          - \frac{1}{\breve{a}^{m}}\right)
 < \frac{\breve{a}}{\breve{a}-1}
 \leqslant 2.
\end{equation*}
Puisque \(\OM{\breve{a}}{1} = 1\), il vient l'encadrement suivant
grâce à l'équation~\eqref{eq:loc}:
\begin{equation*}
n - m + 2 \leqslant \OM{\fun{loc}}{m,n} < 2(n-m+2) \leqslant 2n + 4.
\end{equation*}
La recherche na\"{\i}ve est donc efficace en moyenne, mais son
hypothèse n'est pas applicable à un corpus en français. Le coût moyen
est d'autant moindre que l'alphabet est grand car \(\lim_{\breve{a}
  \rightarrow \infty}\OM{\breve{a}}{m} = 1\).\index{recherche de
  motifs!$\sim$ na\"{\i}ve!coût moyen|)} \index{recherche de
  motifs!$\sim$ na\"{\i}ve|)}


\section{Algorithme de Morris et Pratt}
\index{recherche de motifs!algorithme de Morris et Pratt|(}

En cas d'échec de comparaison, la recherche na\"{\i}ve recommence à
comparer les premières lettres de~\(p\) sans utiliser l'information du
succès partiel, à savoir: \(\ind{p}{0,i-1} = \ind{t}{j-i,j-1}\) et
\(\ind{p}{i} \neq \ind{t}{j}\) (voir \fig~\vref{fig:naive}). La
comparaison de~\(p\) avec \(\ind{t}{j-i+1,j-1}\) pourrait réutiliser
\(\ind{t}{j-i+1,j-1} = \ind{p}{1,i-1}\), en d'autres termes,
\(\ind{p}{0,i-2}\)~est comparé à~\(\ind{p}{1,i-1}\): le motif~\(p\)
est comparé à une partie de lui-même. Si nous connaissons un
index~\(k\) tel que \(\ind{p}{0,k-1} = \ind{p}{i-k,i-1}\),
c'est-à-dire si \(\ind{p}{0,k-1}\)~est un \emph{bord}
de~\(\ind{p}{0,i-1}\), alors nous pouvons recommencer à comparer
\(\ind{t}{j}\) avec~\(\ind{p}{k}\). Clairement, plus~\(k\) est grand,
plus on évite de comparaisons, donc nous voulons trouver le \emph{bord
  maximum} de chaque préfixe de~\(p\).

\paragraph{Bord}
\index{recherche de motifs!bord|(}

Le bord d'un mot non-vide~\(y\) est un préfixe propre de~\(y\) qui est
aussi un suffixe. Par exemple, le mot \word{abacaba} a trois bords:
\(\varepsilon\), \word{a}~et~\word{aba}. Ce dernier est le bord (de
longueur) maximum et nous écrivons
\(\Border{}{\underline{\word{aba}}\word{c}\underline{\word{aba}}} =
\word{aba}\). Un autre exemple est \(\Border{}{\word{abac}} =
\varepsilon\), car \(\word{abac} =
\underline{\varepsilon}\word{abac}\underline{\varepsilon}\). Les bords
maximums peuvent se recouvrir partiellement, par exemple
\(\Border{}{\underline{\word{aaa}}\word{a}} =
\Border{}{\word{a}\underline{\word{aaa}}} = \word{aaa}\).

L'accélération apportée par Morris et Pratt à la recherche na\"{\i}ve
est illustrée à la \fig~\vref{fig:mp}.
\begin{figure}[b]
\centering
\includegraphics[bb=74 621 367 717]{mp}
\caption{L'algorithme de Morris et Pratt (échec grisé)
\label{fig:mp}}
\end{figure}
Notons que, contrairement à la recherche na\"{\i}ve, les lettres du
texte sont comparées en ordre strictement croissant (on ne rebrousse
jamais chemin). Considérons l'exemple de la \fig~\vref{fig:mp_ex} où,
à la fin,
\begin{figure}[t]
\centering
\includegraphics{mp_ex}
\caption{L'algorithme de Morris et Pratt au travail
\label{fig:mp_ex}}
\end{figure}
\(p\)~est n'est pas trouvé dans~\(t\). Comme d'habitude, les lettres
sur fond gris correspondent à des échecs de comparaisons. Il est clair
que \(\Border{}{a} = \varepsilon\), pour toute lettre~\(a\).

Nous pouvons chercher soit \(\Border{}{ay}\), soit \(\Border{}{ya}\),
où \(y\)~est un mot non-vide. Puisque que nous souhaitons trouver le
bord maximum de chaque préfixe d'un motif donné, le second choix est
plus adéquat (\(y \pref ya\)). L'idée est de considérer récursivement
\(\Border{}{y} \cdot a\): si c'est un préfixe de~\(y\), alors
\(\Border{}{ya} = \Border{}{y} \cdot a\); sinon, nous cherchons le
bord maximum du bord maximum de~\(y\), c'est-à-dire \(\Border{2}{y}
\cdot a\) etc. jusqu'à ce que \(\Border{q}{y} \cdot a\) soit un
préfixe de~\(y\) ou \(\Border{q}{y} = \varepsilon\). Par exemple,
\(\Border{}{y \cdot \word{a}} = \Border{3}{y} \cdot \word{a}\) à la
\fig~\vref{fig:max_B}.
\begin{figure}[b]
\centering
\includegraphics{max_B}
\caption{\(\Border{}{y \cdot \word{a}}
   = \Border{}{\Border{}{y} \cdot \word{a}}
   = \Border{}{\Border{2}{y} \cdot \word{a}}
   = \Border{3}{y} \cdot \word{a}\).
\label{fig:max_B}}
\end{figure}

Mathématiquement, pour tout mot \({y \neq \varepsilon}\) et toute lettre~\(a\),
\begin{equation}
  \Border{}{a}         := \varepsilon;\qquad
  \Border{}{y \cdot a} := \left\{
    \begin{aligned}
      & \Border{}{y} \cdot a,
      && \text{si \(\Border{}{y} \cdot a \prefeq y\)};\\
      & \Border{}{\Border{}{y} \cdot a},
      && \text{sinon.}
    \end{aligned}
  \right.
\label{eq:Border}
\end{equation}
Considérons les exemples suivants où \(y\)~et~\(\Border{}{y}\) sont donnés:
\begin{align*}
  y             &= \word{abaabb},
& \Border{}{y}  &= \varepsilon,
& \Border{}{y \cdot \word{b}}
                &= \Border{}{\Border{}{y} \cdot \word{b}}
                 = \Border{}{\word{b}} = \varepsilon;\\
  y             &= \word{baaaba},
& \Border{}{y}  &= \word{ba},
& \Border{}{y \cdot \word{a}}
                &= \Border{}{y} \cdot \word{a}
                 = \word{baa};\\
  y             &= \word{abbbab},
& \Border{}{y}  &= \word{ab},
& \Border{}{y \cdot \word{a}}
                &= \Border{}{\Border{}{y} \cdot \word{a}}
                 = \Border{2}{y} \cdot \word{a}
                 = \word{a}.
\end{align*}
\index{recherche de motifs!bord|)}

\paragraph{Fonction de suppléance}
\index{recherche de motifs!fonction de suppléance|(}

Notons~\(\wlen{y}\) la longueur du mot~\(y\). Pour un mot~\(x\) donné,
définissons une fonction \(\MPfailureName_{x}\), pour tous ses
préfixes:
\begin{equation}
  \MPfailure{x}{}{\wlen{y}}
:= \wlen{\Border{}{y}},\,\; \text{pour tout
    \(x\) et \(y \neq \varepsilon\) tels
    que \(y \prefeq x\)}.\label{eq:MP_failure}
\end{equation}
Pour des raisons qui deviendront claires bientôt, cette fonction est
appelée la \emph{fonction de suppléance} de~\(x\). Une définition
équivalente est
\begin{equation*}
  \MPfailure{x}{}{i}
= \wlen{\Border{}{\ind{x}{0,i-1}}}, \,\;
\text{pour tout \(x\) et \(i\) tels que
\(0 < i \leqslant \wlen{x}\)}.
\end{equation*}
Par exemple, à la \fig~\vref{fig:mp_fail}, nous trouvons la table des
bords maximums des préfixes du mot \word{abacabac}.
\begin{figure}
\centering
\includegraphics{mp_fail}
\caption{Fonction de suppléance de \word{abacabac}
\label{fig:mp_fail}}
\end{figure}
À la \fig~\vref{fig:mp}, la longueur du bord maximum est~\(k\), donc
\(k=\MPfailure{p}{}{i}\) et \(\smash{\Ind{p}{\MPfailure{p}{}{i}}}\)
est la première lettre à être comparée avec~\(\ind{t}{j}\) après le
décalage. Par ailleurs, la figure présuppose que \({i>0}\), donc le
bord en question est défini. Les équations \eqref{eq:Border}
\vpageref{eq:Border} qui définissent le bord maximum peuvent être
déroulées de la manière suivante:
\begin{align*}
   \Border{}{ya}
&= \Border{}{\Border{}{y} \cdot a},
& \Border{}{y} \cdot a &\nprefeq y;\\
   \Border{}{\Border{}{y} \cdot a}
&= \Border{}{\Border{2}{y} \cdot a},
&  \Border{2}{y} \cdot a &\nprefeq \Border{}{y};\\
&\;\;\smash{\vdots} & &\;\;\smash{\vdots}\\
   \smash{\Border{}{\Border{p-1}{y} \cdot a}}
&= \smash{\Border{}{\Border{p}{y} \cdot a}},
& \smash{\Border{p}{y} \cdot a} &\nprefeq \smash{\Border{p-1}{y}};
\end{align*}
et \(\varepsilon \not\in \{y,\Border{}{y}, \dots,
\Border{p-1}{y}\}\). Par transitivité, les équations impliquent
\(\Border{}{ya} = \Border{}{\Border{p}{y} \cdot a}\). Deux cas sont
possibles: soit \(\Border{p}{y} = \varepsilon\), soit \(\Border{}{ya}
= \Border{}{a} = \varepsilon\), ou bien la recherche continue jusqu'à
ce que nous trouvions le plus petit \(q > p\) tel que
\(\Border{}{\Border{q-1}{y} \cdot a} = \Border{}{\Border{q}{y} \cdot
  a}\) avec \(\Border{q}{y} \cdot a \prefeq \Border{q-1}{y}\). Étant
donné qu'un bord est un préfixe propre, c'est-à-dire que
\(\Border{}{y} \pref y\), nous avons \(\Border{2}{y} =
\Border{}{\Border{}{y}} \pref \Border{}{y}\), ce qui entraîne
\(\Border{q}{y} \cdot a \prefeq \Border{q-1}{y} \pref \dots \pref
\Border{}{y} \pref y\). Par conséquent \(\Border{q}{y} \cdot a \prefeq
y\), car \({q > 0}\), et \(\Border{}{ya} = \Border{q}{y} \cdot a\). Ce
raisonnement établit que
\begin{equation*}
\Border{}{ya} =
\left\{
  \begin{aligned}
   & \Border{q}{y} \cdot a,
   && \text{si \(\Border{q}{y} \cdot a \prefeq y\)};\\
   & \varepsilon,
   && \text{sinon;}
  \end{aligned}
\right.
\end{equation*}
avec la contrainte additionnelle que \(q\)~doit être aussi petit que
possible. Cette forme de la définition de~\(\BorderName\) est plus
simple parce qu'elle ne contient pas un appel imbriqué comme
\(\Border{}{\Border{}{y} \cdot a}\). Nous pouvons maintenant prendre
la longueur de chaque côté des équations:
\begin{equation*}
\wlen{\Border{}{ya}} =
\left\{
  \begin{aligned}
   & \wlen{\Border{q}{y} \cdot a} = 1 + \wlen{\Border{q}{y}},
   && \text{si \(\Border{q}{y} \cdot a \prefeq y\)};\\
   & \wlen{\varepsilon} = 0,
   && \text{sinon.}
  \end{aligned}
\right.
\end{equation*}
Si \({ya \prefeq x}\), alors \(\wlen{\Border{}{ya}} =
\MPfailure{x}{}{\wlen{ya}} = \MPfailure{x}{}{\wlen{y}+1}\). Posons \(i
:= \wlen{y} > 0\).
\begin{equation*}
\MPfailure{x}{}{i+1} =
\left\{
  \begin{aligned}
   & 1 + \wlen{\Border{q}{y}},
   && \text{si \(\Border{q}{y} \cdot a \prefeq y\)};\\
   & 0,
   && \text{sinon.}
  \end{aligned}
\right.
\end{equation*}
Nous avons à présent besoin de travailler
sur~\(\wlen{\Border{q}{y}}\). De la définition de~\(\MPfailureName\)
par l'équation~\eqref{eq:MP_failure} \vpageref{eq:MP_failure}, nous
déduisons
\begin{equation}
\MPfailure{x}{q}{\wlen{y}} = \wlen{\Border{q}{y}},
\,\; \text{avec \(y \prefeq x\)},
\label{eq:MP_failure_p}
\end{equation}
que nous prouvons par induction sur~\(q\). Appelons cette propriété
\(\pred{P}{q}\). Trivialement, \(\pred{P}{0}\) est vraie. Supposons
donc \(\pred{P}{n}\) pour tout \(n \leqslant q\): ce sera l'hypothèse
d'induction. Supposons \(y \prefeq x\) et prouvons alors
\(\pred{P}{q+1}\):
\begin{equation*}
  \MPfailure{x}{q+1}{\wlen{y}}
= \MPfailure{x}{q}{\MPfailure{x}{}{\wlen{y}}}
= \MPfailure{x}{q}{\wlen{\Border{}{y}}}
\doteq \wlen{\Border{q}{\Border{}{y}}}
= \wlen{\Border{q+1}{y}},
\end{equation*}
où \((\smash{\doteq})\)~est une application valide de l'hypothèse
d'induction parce que \(\Border{}{y} \pref y \prefeq x\). Ceci prouve
\(\pred{P}{q+1}\) et le principe d'induction entraîne la validité de
\(\pred{P}{n}\) pour tout \(n \geqslant 0\). Par conséquent,
l'équation \eqref{eq:MP_failure_p} nous permet de raffiner notre
définition de \(\MPfailure{x}{}{i+1}\) de la manière suivante, avec
\(i>0\):
\begin{equation*}
\MPfailure{x}{}{i+1} =
\left\{
  \begin{aligned}
   & 1 + \MPfailure{x}{q}{i},
   && \text{si \(\Border{q}{y} \cdot a \prefeq y\)};\\
   & 0,
   && \text{sinon.}
  \end{aligned}
\right.
\end{equation*}
Nous n'avons pas mis à profit une partie de la
définition~\eqref{eq:Border} \vpageref{eq:Border}: \({\Border{}{a} :=
  \varepsilon}\). Elle entraîne \(\MPfailure{x}{}{1} =
\MPfailure{x}{}{\wlen{a}} = \wlen{\Border{}{a}} = \wlen{\varepsilon} =
0\) et, puisque la définition de~\(\MPfailureName\) implique
\(\MPfailure{x}{}{1} = 1 + \MPfailure{x}{}{0}\), on a
\(\MPfailure{x}{}{0} = -1\). La propriété «~\(\Border{q}{y} \cdot a
\prefeq y\) et \({ya \prefeq x}\) et \({\wlen{y} = i}\)~» implique les
égalités \(\smash{\Ind{y}{\wlen{\Border{q}{y}}}} = a \Leftrightarrow
\smash{\Ind{y}{\MPfailure{x}{q}{i}}} = a \Leftrightarrow
\smash{\Ind{x}{\MPfailure{x}{q}{i}}} = \smash{\Ind{x}{\wlen{y}}}
\Leftrightarrow \smash{\Ind{x}{\MPfailure{x}{q}{i}}} =
\ind{x}{i}\). Nous savons maintenant que
\begin{equation*}
 \MPfailure{x}{}{0}   = -1\quad\text{et}\quad
 \MPfailure{x}{}{i+1} =
   \left\{
     \begin{aligned}
       & 1 + \MPfailure{x}{q}{i},
       && \text{si \(\Ind{x}{\MPfailure{x}{q}{i}} = \ind{x}{i}\)};\\
       & 0,
       && \text{sinon;}
     \end{aligned}
   \right.
\end{equation*}
où \(q\)~est le plus petit entier naturel non nul satisfaisant la
condition. Nous pouvons simplifier davantage:
\begin{equation*}
\MPfailure{x}{}{0} = -1
\quad \text{et} \quad
\MPfailure{x}{}{i+1} = 1 + \MPfailure{x}{q}{i},
\end{equation*}
où~\(i \geqslant 0\) et \(q>0\) est le plus petit entier naturel tel
que \(\MPfailure{x}{q}{i} = -1\) ou
\(\smash{\Ind{x}{\MPfailure{x}{q}{i}}} = \ind{x}{i}\).

\mypar{Prétraitement}
\index{recherche de motifs!prétraitement de Morris et Pratt!définition|(}

L'appel de fonction
\(\fun{fail}_0(x,i)\)\index{fail0@\fun{fail\(_0\)/2}}, définie à la
\fig~\vref{fig:fail0}, réalise \(\MPfailure{x}{}{i}\).
\begin{figure}[t]
\abovedisplayskip=0pt
\belowdisplayskip=0pt
\framebox[\columnwidth]{\vbox{%
\begin{gather*}
\fun{fail}_0(x,0) \rightarrow -1;\quad
\fun{fail}_0(x,i) \rightarrow
          1 + \fun{fp}(x,\fun{nth}(x,i-1),\fun{fail}_0(x,i-1)).\\
\fun{nth}(\cons{a}{x},0) \rightarrow a;\quad
\fun{nth}(\cons{a}{x},i) \rightarrow \fun{nth}(x,i-1).
\\
\inferrule*{}{\fun{fp}(x,a,-1) \rightarrow -1;}
\quad
\inferrule
  {\fun{nth}(x,k) \twoheadrightarrow a}
  {\fun{fp}(x,a,k) \twoheadrightarrow k}
\,;\quad
\fun{fp}(x,a,k) \rightarrow \fun{fp}(x,a,\fun{fail}_0(x,k)).
\end{gather*}
}}
\caption{La fonction de suppléance \(\MPfailureName\) par \fun{fail\(_0\)/2}
\label{fig:fail0}}
\end{figure}
La fonction \fun{fp/3}\index{fp@\fun{fp/3}} (anglais, \emph{fixed
  point}) calcule \(\MPfailure{x}{q}{i-1}\), en commençant avec
\(\fun{fail}_0(x,i-1)\) et
\(\fun{nth}(x,i-1)\)\index{nth@\fun{nth/2}}, qui dénote
\(\ind{x}{i-1}\) et est nécessaire pour vérifier la condition
\(\smash{\Ind{x}{\MPfailure{x}{q}{i-1}}} = \ind{x}{i-1}\). Le test
d'égalité \(\MPfailure{x}{q}{i-1} = -1\) est effectué par la première
règle de \fun{fp/3}.\index{recherche de motifs!fonction de
  suppléance|)}

L'algorithme de Morris et Pratt nécessite le calcul de
\(\MPfailure{x}{}{i}\) pour tous les index~\(i\) du motif~\(x\) et,
puisqu'il dépend des valeurs de certains appels
\(\MPfailure{x}{}{j}\), où \(j < i\), il est plus efficace de calculer
\(\MPfailure{x}{}{i}\) pour des valeurs croissantes de la
variable~\(i\) et de les mémoriser, de telle sorte qu'elles peuvent
être réutilisées au lieu d'être recalculées. Cette technique est
appelée \emph{mémoïsation}\index{mémoïsation} (à ne pas confondre avec
mémorisation, au sens plus général). Dans ce cas, l'évaluation de
\(\MPfailure{x}{}{i}\) repose sur le mémo
\begin{equation*}
[\pair{\ind{x}{i-1}}{\MPfailure{x}{}{i-1}},
\pair{\ind{x}{i-2}}{\MPfailure{x}{}{i-2}}, \dots,
\pair{\ind{x}{0}}{\MPfailure{x}{}{0}}].
\end{equation*}
La version avec mémoïsation de \fun{fail\(_0\)/2} est appelée
\fun{fail/2}\index{fail@\fun{fail/2}} à la \fig~\ref{fig:fail}.
\begin{figure}[b]
\centering
\abovedisplayskip=0pt
\belowdisplayskip=0pt
\framebox[0.85\columnwidth]{\vbox{%
\begin{gather*}
\fun{fail}(p,0) \rightarrow -1;
\quad
\fun{fail}(\cons{\pair{a}{k}}{p},i) \rightarrow
\fun{fp}(p,a,k,i-1).
\\
\inferrule*{}{\fun{fp}(p,a,-1,i) \rightarrow 0;}\quad
\inferrule
  {\fun{suf}(p,i-k-1) \twoheadrightarrow \cons{\pair{a}{k'}}{p'}}
  {\fun{fp}(p,a,k,i) \twoheadrightarrow k+1}\,;\\
\inferrule
  {\fun{suf}(p,i-k-1) \twoheadrightarrow \cons{\pair{b}{k'}}{p'}}
  {\fun{fp}(p,a,k,i) \twoheadrightarrow \fun{fp}(p',a,k',k)}\,.
\\
\fun{suf}(p,0) \rightarrow p;\quad
\fun{suf}(\cons{a}{p},i) \rightarrow \fun{suf}(p,i-1).
\end{gather*}
}}
\caption{Fonction de suppléance avec mémoïsation\label{fig:fail}}
\end{figure}
Ici, nous travaillons avec le mémo~\(p\), qui est un préfixe retourné,
au lieu de~\(x\), donc nous devons connaître sa longueur~\(i\) pour
savoir combien de lettres doivent être ignorées par
\fun{suf/2}\index{suf@\fun{suf/2}} (\emph{suffixe}):
\(\fun{suf}(x,i-k-1)\) au lieu de \(\fun{fail}_0(x,k)\). Grâce au
mémo, \fun{fp/4} n'a pas besoin d'appeler \fun{fail/2}, seulement
d'examiner~\(p\) avec \fun{suf/2}. Remarquons que nous avons apporté
une petite amélioration cosmétique en déplaçant l'incrément: au lieu
de \(1 + \fun{fp}(\dots)\) et \(\dots \twoheadrightarrow k\), nous
avons maintenant \(\fun{fp}(\dots)\) et \(\dots \twoheadrightarrow
k+1\).

Nommons \fun{pp/1}\index{pp@\fun{pp/1}} (anglais,
\emph{preprocessing}) la fonction calculant la pile
\begin{equation*}
[\pair{\ind{x}{0}}{\MPfailure{x}{}{0}},
\pair{\ind{x}{1}}{\MPfailure{x}{}{1}}, \dots,
\pair{\ind{x}{m-1}}{\MPfailure{x}{}{m-1}}]
\end{equation*}
pour un motif~\(x\) de taille~\(m\). Sa définition est montrée à la
\fig~\vref{fig:pp},
\begin{figure}[t]
\begin{equation*}
\boxed{%
\begin{array}{r@{\;}l@{\;}l}
\fun{pp}(x) & \rightarrow & \fun{pp}(x,\el,0).\\
\\
\fun{pp}(\el,p,i) & \rightarrow & \fun{rev}(p);\\
\fun{pp}(\cons{a}{x},p,i)
  & \rightarrow
  & \fun{pp}(x,\cons{\pair{a}{\fun{fail}(p,i)}}{p},i+1).
\end{array}}
\end{equation*}
\caption{Prétraitement d'un motif~\(y\) par \fun{pp/1}
\label{fig:pp}}
\end{figure}
où \fun{rev/1}\index{rev@\fun{rev/1}} est la fonction de retournement
(définition~\eqref{def:rev} \vpageref{def:rev}), et
\fun{pp/1}\index{pp@\fun{pp/1}} simplement appelle la fonction de
suppléance \fun{fail/2} pour chaque nouvel index~\(i\) du mémo
courant~\(p\) et créé un nouveau mémo en accouplant l'index de l'échec
avec la lettre courante puis empile le mémo courant
(\(\cons{\pair{a}{\fun{fail}(p,i)}}{p}\)). Le retournement de pile à
la fin est nécessaire parce que le mémo contient les lettres en ordre
inverse par rapport au motif. L'exemple à la \fig~\vref{fig:mp_fail}
mène à l'évaluation suivante:
\begin{equation*}
\fun{pp}(x) \twoheadrightarrow
[\pair{\word{a}}{-1}, \pair{\word{b}}{0},
\pair{\word{a}}{0}, \pair{\word{c}}{1},
\pair{\word{a}}{0}, \pair{\word{b}}{1},
\pair{\word{a}}{2}, \pair{\word{c}}{3}],
\end{equation*}
où \(x=\word{abacabac}\). Si \(x=\word{ababaca}\), alors
\begin{equation*}
\abovedisplayskip=2pt
\belowdisplayskip=0pt
\fun{pp}(x) \twoheadrightarrow
[\pair{\word{a}}{-1}, \pair{\word{b}}{0},
\pair{\word{a}}{0}, \pair{\word{b}}{1},
\pair{\word{a}}{2}, \pair{\word{c}}{3},
\pair{\word{a}}{0}].
\end{equation*}
\index{recherche de motifs!prétraitement de Morris et Pratt!définition|)}

%\addcontentsline{toc}{subsection}{Cost}
\paragraph{Coût minimum}
\index{recherche de motifs!prétraitement de Morris et Pratt!coût minimum|(}

Il est clair que, d'après la définition~\eqref{eq:Border}
\vpageref{eq:Border}, la détermination du bord maximum d'un mot
non-vide requiert le bord maximum de tout ou partie de ses préfixes
propres, donc, si le mot contient \(n\)~lettres, au moins \(n-1\)
comparaisons sont nécessaires, car le bord de la première lettre seule
n'a besoin d'aucune comparaison. Cette borne inférieure peut être
atteinte, comme le raisonnement suivant le démontre. Appelons
\emph{comparaison positive}\index{recherche de
  motifs!comparaison!$\sim$ positive} le succès d'un test de préfixe,
tel que nous le trouvons dans la définition de~\(\BorderName\), soit
\(\Border{}{y} \cdot a \prefeq y\). Par dualité, une \emph{comparaison
  négative}\index{recherche de motifs!comparaison!$\sim$ positive} est
l'échec d'un test de préfixe. De façon à minimiser le nombre d'appels
pour évaluer \(\Border{}{ya}\), nous pourrions remarquer qu'une
comparaison positive n'entraîne que l'évaluation de~\(\Border{}{y}\),
alors qu'une comparaison négative demande deux appels:
\(\Border{}{\Border{}{y} \cdot a}\). Par conséquent, une première idée
serait de supposer que nous n'avons que des comparaisons positives:
\begin{equation*}
\Border{}{x} \!\eqn{n-2}\!
\Border{}{\ind{x}{0,n-2}} \cdot \ind{x}{n-1}
\!\eqn{n-1}\!\! \dots
\eqn{0}\! \Border{}{\ind{x}{0}} \cdot \ind{x}{1,n-1}
\!=\! \ind{x}{1,n\!-\!1},
\end{equation*}
où \((\smash{\eqn{i}})\)~implique \(\Border{}{\ind{x}{0,i}} \cdot
\ind{x}{i+1} \prefeq \ind{x}{0,i}\), pour \(0 \leqslant i \leqslant
n-2\). D'abord, \(i=0\) et la comparaison positive correspondante
entraîne \(\ind{x}{0} = \ind{x}{1}\). Les autres comparaisons
entraînent \(\ind{x}{0} = \ind{x}{1} = \dots = \ind{x}{n-1}\), donc un
cas parmi les meilleurs est \(x=a^n\), pour toute lettre~\(a\).

Mais il existe un autre cas, parce que l'appel le plus externe
à~\(\BorderName\) après une comparaison négative n'implique pas de
comparaisons si son argument est une unique lettre:
\begin{align*}
\abovedisplayskip=0pt
\belowdisplayskip=0pt
\Border{}{x}
&\eqn{n-2} \Border{}{\Border{}{\ind{x}{0,n-2}} \cdot \ind{x}{n-1}}\\
&\eqn{n-3} \Border{}{\Border{}{\Border{}{\ind{x}{0,n-3}} \cdot
    \ind{x}{n-2}} \cdot \ind{x}{n-1}}\\
&\;\;\;\smash{\vdots}\\
&\;\mathrel{\stackrel{\smash[t]{0}}{=}}
  \Border{}{\Border{}{\ldots \Border{}{\Border{}{\ind{x}{0}} \cdot
        \ind{x}{1}} \dots} \cdot \ind{x}{n-1}}\\
&\,\,\doteq
  \Border{}{\Border{}{\ldots \Border{}{\Border{}{\ind{x}{1}} \cdot
            \ind{x}{2}} \dots} \cdot \ind{x}{n-1}}\\
&\;\;\;\smash{\vdots}\\
&\,\,\doteq \Border{}{\ind{x}{n-1}} = \varepsilon.
\end{align*}
où \((\smash{\eqn{i}})\)~implique \(\Border{}{\ind{x}{0,i}} \cdot
\ind{x}{i+1} \nprefeq \ind{x}{0,i}\), pour \(0 \leqslant i \leqslant
n-2\) et \((\smash{\doteq})\) ne contient aucune
comparaison. Commencer avec \(i=0\) nous amène à \(\ind{x}{1} \neq
\ind{x}{0}\), donc \(i=1\) mène à \(\ind{x}{2} \neq \ind{x}{0}\)
etc. ainsi les effets de toutes ces comparaisons négatives sont
\(\ind{x}{0} \neq \ind{x}{i}\), pour \(1 \leqslant i \leqslant
n-2\). Le nombre de comparaisons négatives est \(n-1\), donc minimum,
mais la structure du mot est différente du cas précédent, car la
première lettre doit différer de toutes les suivantes. Soit
\(\OB{\fun{pp}}{n}\)\index{pp@$\OB{\fun{pp}}{n}$} le nombre minimum de
comparaisons impliquées dans l'évaluation de
\(\fun{pp}(x)\)\index{pp@\fun{pp/1}}, où la longueur du motif~\(x\)
est~\(n\). C'est le même nombre que le nombre de comparaisons pour
évaluer \(\Border{}{\ind{x}{0,n-2}}\) quand \(\ind{x}{0,n-2}\) est un
cas parmi les meilleurs. Par conséquent,
%\begin{equation*}
\(\OB{\fun{pp}}{n} = n - 2\).
%\end{equation*}
\index{recherche de motifs!prétraitement de Morris et Pratt!coût minimum|)}

\paragraph{Coût maximum}
\index{recherche de motifs!prétraitement de Morris et Pratt!coût maximum|(}

%Trouver le bord maximum d'un mot nécessite le bord maximum de tous ses
%préfixes propres, donc, p

Pour maximiser le nombre de comparaisons, nous
devons calculer le plus de bords possibles. Pour ce faire,
l'évaluation de \(\Border{}{x}\) amènerait à chercher le bord maximum
d'un facteur de longueur \(n-1\), où \(n\)~est la longueur
de~\(x\). Le meilleur des cas \(x=a^{n}\) montre que \(\Border{}{x} =
\ind{x}{1,n-1}\), ce qui sied à notre but, sauf que nous voudrions
\(\Border{}{\ind{x}{1,n-1}}\). En d'autres termes, nous ajoutons la
contrainte d'une première comparaison négative:
\begin{align*}
\Border{}{x}
&\eqn{n-1} \Border{}{\Border{}{\ind{x}{0,n-2}} \cdot \ind{x}{n-1}}\\
&\eqn{n-2} \Border{}{\Border{}{\ind{x}{0,n-3}} \cdot \ind{x}{n-2,n-1}}\\
&\;\;\;\smash{\vdots}\\
&\;\mathrel{\stackrel{\smash[t]{1}}{=}}
  \Border{}{\Border{}{\ind{x}{0}} \cdot \ind{x}{1,n-1}}
= \Border{}{\ind{x}{1,n-1}},
\end{align*}
où \((\smash{\eqn{n-1}})\) suppose \(\Border{}{\ind{x}{0,n-2}} \cdot
\ind{x}{n-1} \nprefeq \ind{x}{0,n-2}\), et \((\smash{\eqn{i}})\), avec
\(1 \leqslant i \leqslant n-2\), correspond à
\(\Border{}{\ind{x}{0,i}} \cdot \ind{x}{i+1} \prefeq
\ind{x}{0,i}\). Ces contraintes impliquent \(\ind{x}{0} = \ind{x}{1} =
\dots = \ind{x}{n-2} \neq \ind{x}{n-1}\), c'est-à-dire un motif
\(x=a^{n-1}b\), avec \(a \neq b\). Jusqu'à présent, le nombre de
comparaisons est \(n-1\), comme dans le cas minimal, mais l'évaluation
se poursuit ainsi:
\begin{equation*}
\Border{}{a^{i}b}
\eqn{i} \Border{}{\Border{}{a^{i}} \cdot b}\\
\doteq \Border{}{\Border{}{a^{i-1}} \cdot ab}\\
\doteq \dots \doteq \Border{}{a^{i-1}b},
\end{equation*}
où \(1 \leqslant i \leqslant n-2\) et \((\smash{\eqn{i}})\) causent
les comparaisons négatives \({\Border{}{a^{i}} \cdot b \nprefeq a^{i}}\)
et les comparaisons positives \((\smash{\doteq})\), que nous ne
comptons pas car nous avons pour objectif de trouver
\(\OW{\fun{pp}}{n}\)\index{pp@$\OW{\fun{pp}}{n}$}, donc des
évaluations répétées du même bord n'entraînent pas de comparaisons
répétées grâce à la mémoïsation\index{mémoïsation}. Par conséquent,
nous avons \(n-2\) comparaisons négatives jusqu'à ce que
\(\Border{}{b} = \varepsilon\), qui, avec les \(n-1\) comparaisons
positives précédentes, font un total de \(2n-3\). Puisque
\(\OW{\fun{pp}}{n}\) est le nombre de comparaisons pour calculer
\(\Border{}{\ind{x}{0,n-2}}\) sans répétitions, nous avons
\begin{equation*}
\belowdisplayskip=0pt
\OW{\fun{pp}}{n} = 2(n-1)-3 = 2n - 5.
\end{equation*}
\index{recherche de motifs!prétraitement de Morris et Pratt!coût maximum|)}

\vspace*{-20pt}

\mypar{Recherche}
\index{recherche de motifs!algorithme de Morris et Pratt!recherche|(}

Nous avons trouvé ci-dessus: \(n-2 \leqslant \OC{\fun{pp}}{n}
\leqslant 2n - 5\), où les bornes sont atteignables si \(n \geqslant
3\). Pour employer la valeur de \(\fun{pp}(p)\), nous pourrions
commencer par modifier la recherche linéaire de la
section~\ref{sec:naive_factoring}, en particulier le programme à la
\fig~\ref{fig:loc} \vpageref{fig:loc}, tout en gardant un {\oe}il sur
la \fig~\vref{fig:mp}. Le résultat est montré à la
\fig~\vref{fig:mp_def}.
\begin{figure}[b]
\centering
\abovedisplayskip=0pt
\belowdisplayskip=0pt
\framebox[0.8\columnwidth]{\vbox{%
\begin{gather*}
\inferrule
  {\fun{pp}(p) \twoheadrightarrow p'}
  {\fun{mp}(p,t) \twoheadrightarrow \fun{mp}(p',t,p',0,0)}.
\\
\begin{array}{r@{\;}l@{\;}l}
\fun{mp}(\el,t,p',i,j)   & \rightarrow & \fun{factor}(j-i);\\
\fun{mp}(p,\el,p',i,j)   & \rightarrow & \fun{absent}();\\
\fun{mp}(\cons{\pair{a}{k}}{p},\cons{a}{t},p',i,j)
                         & \rightarrow & \fun{mp}(p,t,p',i+1,j+1);\\
\fun{mp}(\cons{\pair{a}{-1}}{p},\cons{b}{t},p',0,j)
                         & \rightarrow & \fun{mp}(p',t,p',0,j+1);\\
\fun{mp}(\cons{\pair{a}{k}}{p},t,p',i,j)
                         & \rightarrow
                         & \fun{mp}(\fun{suf}(p',k),t,p',k,j).
\end{array}
\end{gather*}}}
\caption{L'algorithme de Morris et Pratt (recherche)\label{fig:mp_def}}
\end{figure}
Notons comment le premier argument de \fun{mp/5}\index{mp@\fun{mp/5}},
\(p\), est la copie de travail et le troisième, \(p'\), est l'original
qui demeure invariant (il est utilisé pour restaurer~\(p\) après
l'échec d'une comparaison). Les index~\(i\), \(j\)~et~\(k\) sont les
mêmes qu'à la \fig~\vref{fig:mp}. Le dernier n'est autre que la valeur
calculée par la fonction de suppléance; les variables \(i\)~et~\(j\)
sont incrémentées à chaque fois qu'une lettre du motif est identique à
une lettre du texte (troisième règle de \fun{mp/5}) et \(j\)~est aussi
incrémentée à chaque inégalité de la première lettre du motif
(quatrième règle de \fun{mp/5}).\index{recherche de
  motifs!algorithme de Morris et Pratt!recherche|)}


\addcontentsline{toc}{subsection}{Coût}
\paragraph{Coût minimum}
\index{recherche de motifs!algorithme de Morris et Pratt!coût minimum|(}

Soit
\(\smash[t]{\OB{\fun{mp/5}}{m,n}}\)\index{mp@$\OB{\fun{mp/5}}{m,n}$}
le nombre minimum de comparaisons effectuées durant une évaluation de
\fun{mp/5}, où \(m\)~est la longueur du motif et \(n\)~est la longueur
du texte. Tout comme avec la recherche na\"{\i}ve, le meilleur des cas
est quand le motif est un préfixe du texte, donc
\(\OB{\fun{mp/5}}{m,n} = m\). En prenant en compte le prétraitement,
le nombre de comparaisons
\(\OB{\fun{mp}}{m,n}\)\index{mp@$\OB{\fun{mp}}{m,n}$} de
\fun{mp/2}\index{mp@\fun{mp/2}} est
\begin{equation*}
\belowdisplayskip=0pt
\OB{\fun{mp}}{m,n} = \OB{\fun{pp}}{m} + \OB{\fun{mp/5}}{m,n} = (m-2)
+ m = 2m - 2.
\end{equation*}
\index{recherche de motifs!algorithme de Morris et Pratt!coût minimum|)}

\vspace*{-10pt}

\paragraph{Coût maximum}
\index{recherche de motifs!algorithme de Morris et Pratt!coût maximum|(}

Puisque l'algorithme de Morris et Pratt seulement lit le texte vers
l'avant, le pire des cas doit maximiser le nombre de fois que les
lettres du texte~\(t\) sont comparées avec une lettre du
motif~\(p\). Par conséquent, la première lettre du motif ne peut
différer de toutes les lettres du texte, sinon chaque lettre du texte
serait comparée exactement une fois. Supposons l'exact opposé:
\(\ind{p}{0} = \ind{t}{i}\), avec \(i \geqslant 0\). Mais ceci
impliquerait aussi une comparaison par lettre du texte. Pour forcer le
motif à se décaler le moins possible, nous imposons en plus
\(\ind{p}{1} \neq \ind{t}{i}\), pour \(i>0\). En bref, cela signifie
que \(ab \prefeq p\), avec des lettres \(a\)~et~\(b\) telles que \(a
\neq b\) et \(t=a^n\). Un simple diagramme suffit pour révéler que
cette configuration produit le nombre maximum de comparaisons
\(\OW{\fun{mp/5}}{m,n} = 2n - 1\)\index{mp@$\OW{\fun{mp/5}}{m,n}$},
car chaque lettre du texte est comparée deux fois, sauf la première,
qui n'est comparée qu'une fois. En tenant compte du prétraitement, le
nombre maximum de comparaisons \(\OW{\fun{mp}}{m,n}\)
\index{mp@$\OW{\fun{mp}}{m,n}$} de \fun{mp/2}\index{mp@\fun{mp/2}} est
\begin{equation*}
\belowdisplayskip=0pt
\OW{\fun{mp}}{m,n} = \OW{\fun{pp}}{m} + \OB{\fun{mp/5}}{m,n} = (2m-5)
+ (2n-1) = 2(n + m - 3).
\end{equation*}
\index{recherche de motifs!algorithme de Morris et Pratt!coût maximum|)}

\vspace*{-10pt}

\mypar{Métaprogrammation}
\index{recherche de motifs!algorithme de Morris et Pratt!métaprogrammation|(}

L'étude précédente mène à des programmes pour le prétraitement et la
recherche qui obscurcissent plutôt l'idée principale derrière
l'algorithme de Morris et Pratt, à savoir, le recours aux bords
maximums des préfixes propres du motif et la lecture unidirectionnelle
du texte. La raison de cette infortune est que, pour des impératifs
d'efficacité, nous devons mémoïser\index{mémoïsation} les valeurs de
la fonction de suppléance et, au lieu de travailler avec le motif
originel, nous en traitons une version qui a été augmentée avec ces
valeurs. Par ailleurs, l'emploi de piles pour modéliser le motif
ralentit et obscurcit la lecture des lettres et les décalages.

Si le motif est fixe, une approche plus lisible est possible,
consistant en la modification du prétraitement de telle sorte qu'un
programme dédié est produit. Ce type de méthode sur mesure, où un
programme est le résultat de l'exécution d'un autre, est appelé
\emph{métaprogrammation}\index{métaprogrammation}. Bien entendu, il
n'est envisageable que si le temps nécessaire pour émettre, compiler
et exécuter le programme est amorti à long terme, ce qui implique pour
le problème traité que le motif et le texte sont assez longs ou que la
recherche est répétée avec le même motif sur d'autres textes (ou le
reste du même texte, après qu'une occurrence du motif a été trouvée).

% Wrapping figure better declared before a paragraph
%
\setlength{\intextsep}{0pt}
\begin{wrapfigure}[30]{r}[0pt]{0pt}
% 29 vertical lines
% {r} mandatory right placement
% [0pt] of margin overhang
\centering
\includegraphics[bb=71 360 238 723]{metaprog}
\caption{Recherche de \word{abacabac} dans~\(t\)% with a metaprogram
\label{fig:metaprog}}
\end{wrapfigure}
Il existe une convention graphique commode pour le contenu de la table
de suppléance à la \fig~\vref{fig:mp_fail}, appelée \emph{automate
  fini déterministe}\index{recherche de motifs!algorithme de Morris et
  Pratt!automate}\index{automate fini!$\sim$ déterministe} et montrée
à la \fig~\vref{fig:abacabac}.
\begin{figure}[b]
\centering
\includegraphics[bb=80 651 390 710]{abacabac}
\caption{Automate de Morris et Pratt du motif \word{abacabac}
\label{fig:abacabac}}
\end{figure}
Nous ne décrirons ici que de manière sommaire les automates; pour un
traitement mathématique et très approfondi, voir
\cite{VanLeeuwen_1990c, HopcroftMotwaniUllman_2003} et
\cite{Sakarovitch_2003}. Imaginons que les cercles, appelés
\emph{états}\index{état}, contiennent les valeurs de~\(i\) d'après la
table. Les arcs, appelés \emph{transitions}\index{transition}, entre
deux états sont de deux sortes: soit continus et annotés par une
lettre appelée \emph{label}\index{label}, soit en pointillés et
orientés à gauche. La succession des états à travers des arcs continus
forme le mot \(x=\word{abacabac}\). L'état le plus à droite est
distingué par deux cercles concentriques parce qu'il marque la fin
de~\(x\). Par définition, si \(\MPfailure{x}{}{i} = j\), il y a une
transition rétrograde entre l'état~\(i\) et~\(j\). L'état le plus à
gauche est la cible d'un arc continu sans source et est la source d'un
arc pointillé sans cible. Le premier dénote simplement le début du mot
et le dernier correspond à la valeur spéciale \(\MPfailure{x}{}{0} =
-1\).

Ce qui est important pour nous est que le support intuitif qu'apporte
un automate mène aussi à une réalisation intuitive, où chaque état
correspond à une fonction et les transitions sont associées aux
différentes règles de la définition de la fonction pour l'état d'où
elles sortent.

L'exemple \word{abacabac} est montré à la \fig~\ref{fig:metaprog}. Les
états de l'automate, \(0\), \(1\), jusqu'à \(7\), correspondent aux
fonctions \fun{zero/2}, \fun{one/2} etc. jusqu'à \fun{seven/2}. Notons
comment \fun{mp\(_0\)/1}\index{mp0@\fun{mp\(_0\)/1}} met l'index
à~\(0\) quand elle initialise le premier état, c'est-à-dire, par
l'appel à \fun{zero/2}\index{zero@\fun{zero/2}}. L'index~\(j\) joue le
même rôle que dans la \fig~\ref{fig:mp} \vpageref{fig:mp}. La première
règle de chaque fonction correspond à une transition vers la droite
dans l'automate de la \fig~\vref{fig:abacabac} et la deuxième règle
est une transition rétrograde, donc un échec, sauf dans \fun{zero/2},
où elle signifie que le motif est décalé d'une lettre. La fonction
\fun{zero/2} possède une troisième règle traitant le cas où le motif
n'est pas un facteur du texte. Nous pourrions ajouter une règle
similaire aux autres définitions, pour les accélérer, mais nous
choisissons la brièveté et nous laissons les règles pour les échecs
successifs nous ramener à \fun{zero/2}\index{zero@\fun{zero/2}}. La
première règle de \fun{seven/2}\index{seven@\fun{seven/2}} est
spéciale aussi, parce qu'elle est utilisée quand le motif a été
trouvé. Finalement, remarquons l'index \(j-6\), qui montre clairement
que la longueur du motif fait partie du programme, qui est donc un
métaprogramme.\index{recherche de motifs!algorithme de Morris et
  Pratt!métaprogrammation|)} \index{recherche de motifs!algorithme de
  Morris et Pratt|)}

\mypar{Variante de Knuth}
\index{recherche de motifs!algorithme de Knuth, Morris et Pratt|(}

Un coup d'{\oe}il à la \fig~\vref{fig:mp}, devrait nous fournir
matière à réflexion. Que se passerait-il si \(a = \word{a}\)?  Alors
le décalage mènerait immédiatement à une comparaison négative. Donc,
comparons \(\smash{\Ind{p}{\MPfailure{p}{}{i}}}\) à~\(\ind{t}{j}\)
seulement si \(\smash{\Ind{p}{\MPfailure{p}{}{i}}} \neq
\ind{p}{i}\). Sinon, nous prenons le bord maximum du bord maximum
etc. jusqu'à ce que nous trouvions le plus petit~\(q\) tel que
\(\smash{\Ind{p}{\MPfailure{p}{q}{i}}} \neq \ind{p}{i}\). Ceci est une
amélioration proposée par \cite{KnuthMorrisPratt_1977}. Une édition
mise à jour a été publiée par \cite{Knuth_2010} et un traitement fondé
sur la théorie des automates par \cite{CrochemoreHancartLecroq_2007},
à la section~2.6. Voir aussi une dérivation du programme par
raffinements algébriques dans le livre de \cite{Bird_2010}.

Dans l'automate de recherche, quand un échec se produit à l'état~\(i\)
sur la lettre~\(a\), nous suivons une transition arrière vers l'état
\(\MPfailure{x}{}{i}\), mais, si la transition normale est~\(a\) à
nouveau, nous suivons à nouveau la transition rétrograde etc. jusqu'à
ce que nous trouvions une transition normale dont le label n'est
pas~\(a\) ou alors nous devons décaler le motif. L'amélioration
proposée par Knuth consiste à remplacer toutes ces transitions
d'échecs successifs par une seule.

Par exemple, l'automate de la \fig~\vref{fig:abacabac} est amélioré de
cette façon à la \fig~\vref{fig:kmp_abac}.
\begin{figure}
\centering
\includegraphics[bb=80 645 390 735]{kmp_abac}
\caption{Automate de Knuth, Morris et Pratt pour \word{abacabac}
\label{fig:kmp_abac}}
\end{figure}
\index{recherche de motifs!algorithme de Knuth, Morris et Pratt|)}
\index{recherche de motifs|)}

\paragraph{Exercices}

\begin{enumerate}

  \item Cherchez \(\M{\fun{loc}}{m,n}\).

  \item Prouvez que \fun{pp/1} et \fun{mp/2} terminent.

  \item Prouvez \(\fun{loc/2} = \fun{mp/2}\) (correction de
    l'algorithme de Morris et Pratt).

  \item Trouvez \(\B{\fun{pp}}{m}\) et \(\W{\fun{pp}}{m}\). (Attention
    au coût de \fun{suf/2}.)

  \item Cherchez \(\B{\fun{mp}}{m,n}\) et \(\W{\fun{mp}}{m,n}\).

  \item Proposez une simple modification pour éviter l'appel à
    \fun{rev/1} à la \fig~\vref{fig:pp}.\label{factoring_trick}

  \item Modifiez \fun{fail/2} de telle manière que \fun{mp/2} réalise
    l'algorithme de Knuth, Morris et Pratt. Étudiez les cas meilleurs
    et pires de cette variante et montrez que \(\OW{\fun{pp}}{m} =
    2m-6\), pour \(m \geqslant 3\).

  \item Écrivez le métaprogramme correspondant à l'automate à la
    \fig~\vref{fig:kmp_abac}.

  \item Écrivez une fonction \fun{rlw/2}\index{rlw@\fun{rlw/2}}
    (anglais, \emph{remove the last word}) telle que l'appel
    \(\fun{rlw}(w,t)\) est réécrit en le texte~\(t\) si le mot~\(w\)
    est absent, sinon en~\(t\) sans la dernière occurrence de~\(t\).

\end{enumerate}
